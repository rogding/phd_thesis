\chapter{Spectroscopy of $\Sr{87}$ Triplet Rydberg states}

We didn't set out to map out the triplet series of $\Sr{87}$ Rydberg states, but when we first started looking for $\Sr{87}$ $\SLJ{3}{S}{}$ and $\SLJ{3}{D}{}$ Rydberg line, their positions didn't line up with locations predicted using the known quantum defects.
It turns out that the hyperfine interaction in $\Sr{87}$ is not negligible where, as $n$ increases, the core electron couples more strongly to the $\Srion{87}{+}$ with $I=9/2$ and couples less strongly to the Rydberg electron.
Therefore the Rydberg states approach the $\Srion{87}{+}$ hyperfine split core. 
We then rediscovered some papers from the 1980's which explained the shifted $\SLJ{1}{S}{0}$ spectra in $\Sr{87}$ by including the mixing of states with the same $F$. 
Previous spectroscopic work was mostly through the $\nSLJ{5s5p}{1}{P}{1}$ state or from the metastable $\nSLJ{5s5p}{3}{P}{2}$ states so our work appears to be the first measurements of the $\Sr{87}$ hyperfine Rydberg states through the $\nSLJ{5s5p}{3}{P}{1}$ intermediate state. 

We measured $\Sr{87}$ hyperfine Rydberg spectra using a two-photon excitation through the intermediate $\nSLJ{5s5p}{3}{P}{1}$ states and presented the work in \cite{Ding2018.PRA.98.042505}.
This thesis contains a bit of a reanalysis from my point of view but the results remain largely unchanged from the published paper. 
To save some space, the tables of measured \Sr{87} Rydberg energies are placed in the appendix (see \cref{tab:data-s-state,tab:data-d-state}).

\section{Ionization limit of $\Sr{87}$ Rydberg States}

The first indication that the Rydberg levels of $\Sr{87}$ were not as simple as for the bosonic isotopes was that the reported ionization limit of \SI{45932.2861(10)}{\per\cm} \cite{Beigang1982.OC.42.19, Sansonetti2010.JPCRD.39.033103} for $\Sr{87}$ is well above that of all the bosonic isotopes (see \cref{tab:ionization_limits}). 
This suggests that the reported ionization limit is for the $\nSLJf{5s}{2}{S}{1/2}{4}$ state of $\Srion{87}{+}$.
Similar to the case for alkali atoms, the nonzero nuclear spin of $\Srion{87}{+}$ interacts with the spin of the inner $5s$ electron and splits the $\nSLJ{5s}{2}{S}{1/2}$ into $F=4,5$ components.
The magnetic dipole interaction can be written as
\begin{equation}
	\ev{W} = A \ev{\hat{\bm{I}} \vdot \hat{\bm{S}}} = \frac{A}{2} \qty(F\qty(F+1) - I\qty(I+1) - S\qty(S+1))
\end{equation}
where $\bm{F} = \bm{J} + \bm{I}$ and $S=s=1/2$ since core electrons of \Srion{87}{+} form a closed shell. 
A measurement of the $F=4$ and $F=5$ hyperfine splitting of the \Srion{87}{+} $\nSLJ{5s}{2}{S}{1/2}$ ground state found $\Delta E = \SI{5002368.363(57)}{\kHz}$, meaning $A = \SI{-1000473.673+-0.011}{\kHz}$ \cite{Sunaoshi1993.HI.78.241}.
Since the hyperfine shift of the $F=4$ state is $\Delta E_{F=4} = \flatfrac{A}{2}(\flatfrac{-11}{2})= \SI{45932.1943+-0.0010}{\per\cm}$.
This value is used for the analysis below.
There should be no quadrupole hyperfine interaction for $\nSLJ{ms}{2}{S}{1/2}$ states, i.e., $B=0$ ** cite? **.

** Keep section about mass-scaled ionization limit? **
As a check, we can use the ionization limits for the bosons and the atomic masses of the isotopes to obtain a mass-scaled ionization limit for \Sr{87}.
The value estimated using this method is **\SI{45932.19870+-0.00035}{\per\cm}**. 

\begin{figure}[htbp]
	\centering
	\includesvg[keepaspectratio, width=4in]{fermion_spectroscopy/calculations/Eion-mass-scaled.svg}
	\caption{
		\label{fig:Eion-mass-scaled}
		Mass-scaled $E_\text{ion}$.}
\end{figure}

\section{Rydberg States of Hyperfine Split cores}

A modern paper is \cite{Robicheaux2018.PRA.97.022508}. 

To calculate the effects of hyperfine interactions on the Rydberg spectra of $\Sr{87}$, we follow previous works by **cite Esherick, Beigang, K T Lu, Robicheaux?** where we mass-scale the Rydberg levels from an isotope without nuclear spin (i.e., $\Sr{88}$) and then apply the hyperfine interaction to obtain the lines for $\Sr{87}$.
The theoretical approach presented below follows the one Shuhei Yoshida presents in \cite{Ding2018.PRA.98.042505}.

\subsection{Incorporating hyperfine interaction in Rydberg levels}
\label{sec:rydberg_hyperfine_mixing}

**Work on this section about getting from 2-electron Hamiltonian to effective Rydberg Hamiltonian for 5sns states.**

The Hamiltonian for a two-electron system in an $\ket{(ms)(nl)}$ Rydberg state can be written as \cite{Beigang1983.PRL.51.771}
\begin{equation}
	H	= H_{0,1} + H_{0,2} + \frac{1}{r_{12}} + \xi_{nl} \vec{s} \cdot \vec{l} + a_{ms} \vec{s} \cdot \vec{I}
\end{equation}
where $H_{0,i}$ is the Coulomb attraction for the {$i$th} electron to the \Sr{++} core, $\flatfrac{1}{r_{12}}$ is the Coulomb repulsion between the two electrons, $\xi_{nl}$ is the magnitude of the spin-orbit interaction, and $a_{ms}$ is the magnetic dipole interaction of the inner $s$ electron.
For \Sr{88}, $I=0$ so we can write the Hamiltonian as
\begin{equation}
	H_{0}(88) = H
\end{equation}
where the appropriate mass-scaled values are used and includes the spin-orbit and spin-other-orbit interactions. 
The hyperfine interaction for \Sr{87} can be included as
\begin{equation}
	H(87) = H_{0}(88,87) + V_\text{hf}
\end{equation}

For two-electron systems, the hyperfine interaction can be expressed as
\begin{equation}
	\hat{V}_\text{hf} = a_{ms} \left(\hat{\bm{s}}_{1} + \hat{\bm{s}}_{2}\right) \cdot \hat{\bm{I}} \simeq a_{ms} \hat{\bm{s}}_{\text{in}} \cdot \hat{\bm{I}}
\end{equation}
where $\hat{\bm{s}}_{i}$ is the spin of the {$i$th} electron. 

For singly-excited states, the hyperfine interaction of the ``outer'' electron with the core should scale as **$\sim 1/n^3$** and should be negligible for high-lying states. 
We make the assumption that $\hat{V}_\text{hf} \simeq a_{ms} \hat{\bm{s}}_{\text{in}} \cdot \hat{\bm{I}}$ for singly-excited Rydberg states with inner electron $\ket{ms}$ and Rydberg electron $\ket{nl}$.
Expanding $\hat{\bm{s}}_\text{in}$ and $\hat{\bm{I}}$ in terms of ladder operators gives
\begin{equation}
	\label{eq:V_hf}
	\hat{V}_\text{hf} \simeq \frac{a_{ms}}{2} \left(\hat{s}_{\text{in},+} \hat{I}_{-} + \hat{s}_{\text{in},-} \hat{I}_{+} + 2\hat{s}_{\text{in},z} \hat{I}_{z}\right)
\end{equation}
with the usual ladder operators $\hat{J}_{\pm}=\hat{J}_{x} \pm \hat{J}_{y}$. 
Note that $\hat{V}_\text{hf}$ leaves the total spin $m_F$ unchanged, meaning it couples states of the same $m_F$. 

The energy scale of the hyperfine interaction is given by $a_{5s}$ and, as $n$ increases this energy scale becomes comparable.
As seen in \cref{fig:n-scaling}, it first becomes comparable to the fine structure splitting, then the singlet-triplet splitting, then the Coulomb splitting.
\begin{figure}[h]
	\centering
	\includesvg[keepaspectratio, width=\textwidth, height=\textheight]{fermion_spectroscopy/calculations/n-scaling.svg}
	\caption{
		\label{fig:n-scaling}
		The $n$ scaling of various splittings in \Sr{88} calculated using the most recently reported quantum defects and ionization energy \cite{Vaillant2012.JPB.45.135004, Couturier2019.PRA.99.022503}.
		The energy scale of the hyperfine interaction in \Sr{87} is approximately constant and is $\abs{a_{5s}} \approx \SI{1}{\GHz}$ \cite{Sunaoshi1993.HI.78.241}.
		The splittings between various states are represented by $\Delta E^{\qty(0)}_{X}$ between the given states where $X=S$ is the singlet-triplet splitting, $X=J$ is the fine structure splitting, and $X=n$ is the Coulomb splitting.}
\end{figure}
From \cref{fig:n-scaling}, one would also expect hyperfine mixing to occur between $n \sim 10$ and $n \sim 20$ which was indeed observed \cite{Beigang1981.PRL.47.326, Beigang1982.PRL.48.290, Vaillant2014.JPB.47.155001}.
Hyperfine mixing was also observed between the $\nSLJ{5s19d}{1}{D}{2}$ and $\nSLJ{5s19d}{3}{D}{3}$ states \cite{Beigang1982.JPB.15.L201}.

\subsection{Singly Excited $S$ states}

In the below, we follow the derivation outlined in \cite{Beigang1982.PRA.25.1496,Beigang1983.PRL.51.771}.
A more detailed derivation is provided in \cref{sec:hyperfine_mixing_derivation} where it is shown that $\hat{V}_\text{hf}$ only mixes states of the same $F$.
** I need to actually do the derivation.... **

Assuming there are no perturbing states, a reasonable assumption for $n \gtrsim 12$ for the both the $\nSLJ{5sns}{1}{S}{0}$ and $\nSLJ{5sns}{3}{S}{1}$ Rydberg series \cite{Vaillant2014.JPB.47.155001}, we start with the pure singly-excited singlet $\ket{\SLJf{1}{S}{0}{I}}$ and triplet $\ket{\SLJf{3}{S}{1}{I+1}}$, $\ket{\SLJf{3}{S}{1}{I}}$, and $\ket{\SLJf{3}{S}{1}{I-1}}$ hyperfine states.
Applying $\hat{V}_\text{hf}$ to these basis states gives
\begin{align}
	&	\hat{V}_\text{hf} \ket{\SLJf{3}{S}{1}{I+1}}	=	\frac{1}{2} a_{5s} I \ket{\SLJf{3}{S}{1}{I+1}}	\\
	&	\hat{V}_\text{hf} \ket{\SLJf{1}{S}{0}{I}}	=	\frac{1}{2} a_{5s} \sqrt{I\qty(I+1)} \ket{\SLJf{3}{S}{1}{I}}	\\
	&	\hat{V}_\text{hf} \ket{\SLJf{3}{S}{1}{I}}	=	\frac{1}{2} a_{5s} \qty(\sqrt{I\qty(I+1)} \ket{\SLJf{1}{S}{0}{I}} - \ket{\SLJf{3}{S}{1}{I}})	\\
	&	\hat{V}_\text{hf} \ket{\SLJf{3}{S}{1}{I-1}}	=	-\frac{1}{2} a_{5s} \qty(I+1) \ket{\SLJf{3}{S}{1}{I-1}}
\end{align}
From the above, it's apparent that the unmixed states have energy shifts
\begin{equation}
	\bra{\SLJf{3}{S}{1}{I+1}} \hat{V}_\text{hf} \ket{\SLJf{3}{S}{1}{I+1}}	=	\frac{1}{2} a_{5s} I			\approx	\SI{-2.25}{\GHz}
\end{equation}
\begin{equation}
	\bra{\SLJf{3}{S}{1}{I-1}} \hat{V}_\text{hf} \ket{\SLJf{3}{S}{1}{I-1}}	=	-\frac{1}{2} a_{5s} \qty(I+1)	\approx	\SI{2.75}{\GHz}
\end{equation}
The matrix elements of the mixed $F=I$ states are
\begin{equation}
	\bra{\SLJf{1}{S}{0}{I}} \hat{V}_\text{hf} \ket{\SLJf{1}{S}{0}{I}}	=	0
\end{equation}
\begin{equation}
	\bra{\SLJf{3}{S}{1}{I}} \hat{V}_\text{hf} \ket{\SLJf{3}{S}{1}{I}}	=	-\frac{1}{2} a_{5s}
\end{equation}
\begin{equation}
	\bra{\SLJf{3}{S}{1}{I}} \hat{V}_\text{hf} \ket{\SLJf{1}{S}{0}{I}}	=	\frac{1}{2} a_{5s} \sqrt{I\qty(I+1)}
\end{equation}
Therefore $F=I$ submatrix of the block-diagonal $\hat{V}_\text{hf}$ matrix can be written as
\begin{equation}
	\hat{V}_\text{hf}	=	\frac{1}{2} a_{5s}
							\begin{pmatrix}
								0							& \sqrt{I\left(I+1\right)}	\\ 
								\sqrt{I\left(I+1\right)}	& -1						\\ 
							\end{pmatrix}
\end{equation}
Diagonalizing the total Hamiltonian $\hat{H} = \hat{H}_0 + \hat{V}_\text{hf}$ gives the eigenenergies
\begin{equation}
	E^{\pm}	=	\frac{E_{1}+E_{3}}{2} - \frac{1}{4} a_{5s} \pm \frac{1}{2} \sqrt{\qty(E_{1}-E_{3})^2 + a_{5s} \qty(E_{1}-E_{3}) + a_{5s}^{2} \qty(I + \frac{1}{2})^2}
\end{equation}
Taking the limit $a_{5s} \rightarrow 0$ (i.e., no hyperfine interaction), we observe $E^{+} \rightarrow E\qty(\SLJ{1}{S}{0})$ and $E^{-} \rightarrow E\qty(\SLJ{3}{S}{1})$ meaning $E^{+}$ corresponds to the energy of $\ket{\nSLJf{5sns}{1}{S}{0}{9/2}}$ and $E^{+}$ corresponds to energy of $\ket{\nSLJf{5sns}{3}{S}{1}{9/2}}$.

\subsubsection{$n$-mixing}

At higher $n$, the hyperfine shift becomes comparable to the splitting between $n$ and $n^{\prime} = n \pm 1$ as shown in \cref{fig:n-scaling}. 
Indeed, hyperfine-induced $n$-mixing was experimentally observed around $n = \num{113}$ between the $\nSLJf{5sns}{1}{S}{0}{9/2}$ and $\nSLJf{5s\qty(n+1)s}{3}{S}{1}{9/2}$ states \cite{Beigang1983.PRL.51.771}.

It's not too difficult to extend the derivation above to incorporate $n$-mixing. 
Since $\hat{V}_\text{hf}$ only couples states with the same $F$ and considering the, states with different $F$ remain uncoupled. 
Considering the orthogonality of the radial wave functions for states of different $n$, the diagnoal matrix elements of $\hat{V}_\text{hf}$ for $\nSLJf{5sns}{3}{S}{1}{I+1}$ and $\nSLJf{5sns}{3}{S}{1}{I-1}$ are
\begin{equation}
	\mel{\nSLJf{5sn^{\prime}s}{3}{S}{1}{I+1}}{\hat{V}_\text{hf}}{\nSLJf{5sns}{3}{S}{1}{I+1}}	=	\frac{1}{2} a_{5s} I \delta_{n,n^{\prime}}
\end{equation}
\begin{equation}
	\mel{\nSLJf{5sn^{\prime}s}{3}{S}{1}{I-1}}{\hat{V}_\text{hf}}{\nSLJf{5sns}{3}{S}{1}{I-1}}	=	-\frac{1}{2} a_{5s} \qty(I+1) \delta_{n,n^{\prime}}
\end{equation}
The off-diagonal elements vanish between these states because they remain uncoupled.
On the other hand, because $\hat{V}_\text{hf}$ mixes the $\nSLJf{5sns}{1}{S}{0}{I}$ and $\nSLJf{5sns}{3}{S}{1}{I}$ states, we can write
\begin{align}
	\ket{\nSLJf{5sns}{1}{S}{0}{I}}	{}={}&	\ket{\nSLJf{n}{1}{S}{0}{I}} + \sum_{n^{\prime}}{\ket{\nSLJf{n^{\prime}}{3}{S}{1}{I}}}	\\
	\ket{\nSLJf{5sns}{3}{S}{1}{I}}	{}={}&	\ket{\nSLJf{n}{3}{S}{1}{I}} + \sum_{n^{\prime}}{\ket{\nSLJf{n^{\prime}}{1}{S}{0}{I}}}
\end{align}
where $\nSLJf{n}{1}{S}{0}{I}$ and $\nSLJf{n}{3}{S}{1}{I}$ represent pure singly-excited singlet and triplet $F=I$ states (recalling that the radial wave functions are orthogonal for $n \neq n^{\prime}$ for states in the same spin multiplet).
Calculating the matrix elements of $\hat{V}_\text{hf}$ between $\nSLJf{5sns}{1}{S}{0}{I}$ and $\nSLJf{5sns}{3}{S}{1}{I}$ gives
\begin{equation}
	\mel{\nSLJf{5s{n^\prime}s}{1}{S}{0}{I}}{\hat{V}_\text{hf}}{\nSLJf{5sns}{1}{S}{0}{I}}	=	0
\end{equation}
\begin{equation}
	\mel{\nSLJf{5s{n^\prime}s}{3}{S}{1}{I}}{\hat{V}_\text{hf}}{\nSLJf{5sns}{3}{S}{1}{I}}	=	-\frac{1}{2} a_{5s} \delta_{n,n^{\prime}}
\end{equation}
The off-diagonal elements can be expressed as
\begin{align}
	\mel{\nSLJf{5s{n^\prime}s}{1}{S}{0}{I}}{\hat{V}_\text{hf}}{\nSLJf{5sns}{3}{S}{1}{I}}
		{}={}&	-\frac{1}{2} a_{5s} (stuff to finish deriving)	\notag	\\
		{}={}&	-\frac{1}{2} a_{5s} \sqrt{I \qty(I+1)} O_{n,n^{\prime}}
\end{align}
where $O_{n,n^{\prime}}$ is the overlap between the singlet and triplet radial wave functions which can be estimated semiclassically \cite{Bhatti1981.PRA.24.161, Beigang1983.PRL.51.771}
\begin{equation}
	O_{n_{1},n_{2}}
		=	\braket{n_{1}^{*}}{n_{2}^{*}}
		=	\qty(-1)^{n_{2}-n_{1}} \frac{2 \sqrt{n_{1}^{*} n_{2}^{*}}}{n_{1}^{*} + n_{2}^{*}} \frac{\sin[\pi \qty(n_{2}^{*} - n_{1}^{*})]}{\pi \qty(n_{2}^{*} - n_{1}^{*})}
\end{equation}

\begin{figure}[htbp]
	\centering
	\includesvg[keepaspectratio, width=5in, height=\textheight]{fermion_spectroscopy/calculations/s-states/semiclassical_overlap.svg}
	\caption{
		\label{fig:semiclassical_overlap}
		** FIX FIGURE Y-AXIS LABELS **.
		Semiclassical estimation of the radial wave function overlap \cite{Bhatti1981.PRA.24.161} between $\nSLJ{n}{1}{S}{0}$ and $\nSLJ{n^{\prime}}{3}{S}{1}$ states ($O_{n,n^{\prime}}$) for $n^{\prime}=n, n+1, n+2$.}
\end{figure}

Between adjacent $n$ (i.e., for $n_{2} = n_{1} + 1$) and $\delta_{2} \neq \delta_{1}$, the approximation reduces to
\begin{equation}
	O_{n_{1},n_{1}+1}
		=	\braket{n_{1}^{*}}{\qty(n_{1}+1)^{*}}
		=	-\frac{2 \sqrt{n_{1}^{*} \qty(n_{1}+1)^{*}}}{n_{1}^{*} + \qty(n_{1}+1)^{*}} \frac{\sin[\pi \qty(\qty(n_{1}+1)^{*} - n_{1}^{*})]}{\pi \qty(\qty(n_{1}+1)^{*} - n_{1}^{*})}
\end{equation}
Since the $\nSLJf{5sns}{1}{S}{0}{9/2}$ state is always higher in energy than the $\nSLJf{5sns}{3}{S}{1}{9/2}$, they should cross when $n_{1}$ corresponds to the singlet state and $n_{2} = n_{1} + 1$ corresponds to the triplet state. 
Using the quantum defects from \cite{Vaillant2012.JPB.45.135004}, we expect $\braket{n_{1}^{*}}{\qty(n_{1}+1)^{*}} \approx \num{0.1}$ and $\braket{n_{1}^{*}}{\qty(n_{1}+1)^{*}} \approx \num{0.05}$ so we only include $n$ and $n+1$ mixing.

As see in \cref{fig:semiclassical_overlap}, the strongest $n$-mixing occurs for $\Delta n = \pm 1$ with $O_{n,n^{\prime}} \approx \num{0.1}$ and quickly drops off for larger $\Delta n$. 

************************
\begin{align}
	\mel{\nSLJ{n}{1}{S}{0}}{\hat{V}_\text{hf}}{\nSLJ{n+1}{3}{S}{1}}
		&{}={}	\qty(\bra{n^{*}_{\SLJ{1}{S}{0}}}\bra{\SLJ{1}{S}{0}}) \hat{V}_\text{hf} \qty(\ket{\SLJ{3}{S}{1}}\ket{\qty(n+1)^{*}_{\SLJ{3}{S}{1}}})	\\
		&{}={}	\braket{n^{*}_{\SLJ{1}{S}{0}}}{\qty(n+1)^{*}_{\SLJ{3}{S}{1}}} \bra{\SLJ{1}{S}{0}}\hat{V}_\text{hf}\ket{\SLJ{3}{S}{1}}
\end{align}

\begin{align}
	\mel{\nSLJ{n}{1}{S}{0}}{\hat{V}_\text{hf}}{\nSLJ{n+1}{3}{S}{1}}
		{}={}	&	\bra{\nSLJ{n}{1}{S}{0}} \qty[\hat{V}_\text{hf} \ket{\nSLJ{n+1}{3}{S}{1}}]	\\
		{}\neq{}&	\frac{1}{2} a_{5s} \bra{\nSLJ{n}{1}{S}{0}} \qty[\sqrt{I \qty(I+1)} \ket{\nSLJ{n+1}{1}{S}{0}} - \ket{\nSLJ{n+1}{3}{S}{1}}]	\\
		{}={}	&	\frac{1}{2} a_{5s} \sqrt{I \qty(I+1)} \braket{\nSLJ{n}{1}{S}{0}}{\nSLJ{n+1}{1}{S}{0}}	\notag	\\
				&	- \frac{1}{2} a_{5s} \braket{\nSLJ{n}{1}{S}{0}}{\nSLJ{n+1}{3}{S}{1}}	\\
		{}={}	&	\frac{1}{2} a_{5s} \sqrt{I \qty(I+1)} \braket{n^{*}_{\SLJ{1}{S}{0}}}{\qty(n+1)^{*}_{\SLJ{1}{S}{0}}} \braket{\SLJ{1}{S}{0}}{\SLJ{1}{S}{0}}		\notag	\\
				&	- \frac{1}{2} a_{5s} \braket{n^{*}_{\SLJ{1}{S}{0}}}{\qty(n+1)^{*}_{\SLJ{3}{S}{1}}} \braket{\SLJ{1}{S}{0}}{\SLJ{3}{S}{1}}	\\
		{}={}	&	0
\end{align}

\begin{equation}
	\bra{\nSLJ{n}{1}{S}{0}} \hat{V}_\text{hf} \qty(\sum_{n^\prime}\ket{\nSLJ{n^\prime}{3}{S}{1}})	=	...
\end{equation}

************************

\subsection{Singly Excited $D$ states}

Even before including hyperfine interactions (i.e., in the bosons), the spin-orbit interactions leads to a breakdown of $LS$ coupling and mixes the $\SLJ{1}{D}{2}$ and $\SLJ{3}{D}{2}$ states and is especially strong around $n = \num{15}$ where they swap singlet and triplet character \cite{Esherick1977.PRA.15.1920, Vaillant2014.JPB.47.155001}.
To account for this mixing, the eigenstates $\nSLJ{5sns}{1}{D}{2}$ and $\nSLJ{5sns}{3}{D}{2}$ of the $H_{0}\qty(88, m_{87})$ are expanded as
\begin{align}
	\ket{\nSLJ{5sns}{1}{D}{2}}	&{}={}	\cos\theta\ket{\nSLJ{n}{1}{D}{2}} + \sin\theta\ket{\nSLJ{n}{3}{D}{2}} \\
	\ket{\nSLJ{5sns}{3}{D}{2}}	&{}={}	-\sin\theta\ket{\nSLJ{n}{1}{D}{2}} + \cos\theta\ket{\nSLJ{n}{3}{D}{2}}
\end{align}
where $\theta$ is the mixing angle \cite{Sun1989.JPB.22.2887} of the pure singlet $\ket{\nSLJ{n}{1}{D}{2}}$ and triplet $\ket{\nSLJ{n}{3}{D}{2}}$ states.
*** For higher $n$, the singlet-triplet mixing of the $\SLJ{1}{D}{2}$ and $\SLJ{3}{D}{2}$ states becomes nearly $n$ independent as evidenced by their quantum defects \cite{Vaillant2012.JPB.45.135004, Couturier2019.PRA.99.022503}. **
Calculations using a two-active-electron (TAE) model \cite{Ye2013.PRA.88.043430, Fields2018.PRA.97.013429} estimates that converges towards $\theta \sim \num{-0.16}$ for high $n$.
** Where does $\theta \sim \num{-0.14}$ come from? **

** See \cite{Gudde1993.PRA.47.4725} for a calculation of mixing coefficients? **

***************************

The $\nSLJ{5sns}{1}{D}{2}$ and $\nSLJ{5sns}{3}{D}{J}$

** Stuff about D-states. **

** Following similar derivations (outlined by Shuhei Yoshida). **

Similar to the $S$-states, we find the matrix elements of $\hat{V}_\text{hf}$ starting with pure states. 
A further complication of the $D$-states is that spin-orbit interactions lead to a breakdown of $LS$ coupling even in the bosons (with $I=0$), leading to mixing between the $\SLJ{1}{D}{2}$ and $\SLJ{3}{D}{2}$ states ** \cite{Esherick1977.PRA.15.1920, Sun1989.JPB.22.2887} - check that these make sense **. 
To account for this mixing, the $D$ states for $I=0$ (i.e., the eigenstates of $H_0\left(88, m_{87}\right)$) are expanded in terms of pure states
\begin{align}
	\ket{\nSLJ{5snd}{1}{D}{2}}	&{}={}	\cos\theta \ket{\nSLJ{n^{*}_{1}}{1}{D}{2}} + \sin\theta \ket{\nSLJ{n^{*}_{1}}{3}{D}{2}}	\\
	\ket{\nSLJ{5snd}{3}{D}{2}}	&{}={}	-\sin\theta \ket{\nSLJ{n^{*}_{3}}{1}{D}{2}} + \cos\theta \ket{\nSLJ{n^{*}_{3}}{3}{D}{2}}
\end{align}
The singlet-triplet mixing is known to be strong around $n=15$ ** \cite{Esherick1977.PRA.15.1920, Beigang1981.PRL.47.326, Beigang1982.PRL.48.290} - also check these**. 
For higher $n$, on the other hand, the singlet-triplet mixing becomes nearly $n$ independent and $\theta$ is estimated to converge towards $\theta \sim \num{-0.14}$.
(The TAE calculation yields a similar value, $\theta \sim \num{-0.16}$.)
As will be shown later, the current experimental data can be well reproduced when $\theta$ is set to \num{-0.14} and this value is used in the following calculations. 

\subsubsection{${F=I+3}$}

There is a single state:
\begin{equation}
	\hat{V}_\text{hf}	=	\frac{1}{2} a_{ms} I
\end{equation}

\subsubsection{${F=I+2}$}

\begin{equation}
	\hat{V}_\text{hf}	=	\frac{1}{2} a_{ms}
							\begin{pmatrix}
								0									& \sqrt{\frac{2}{3}}I					& -\sqrt{\frac{I\left(I+3\right)}{3}}	\\
								\sqrt{\frac{2}{3}}I					& \frac{I}{3}							& \frac{\sqrt{2I\left(I+3\right)}}{3}	\\
								-\sqrt{\frac{I\left(I+3\right)}{3}}	& \frac{\sqrt{2I\left(I+3\right)}}{3}	& \frac{2}{3}I-1
							\end{pmatrix}
\end{equation}

\subsubsection{${F=I+1}$}

\begin{equation}
	\hat{V}_\text{hf}	=	\frac{1}{2} a_{ms}
							\begin{pmatrix}
								0										& \sqrt{\frac{3I\left(I+2\right)}{10}}I			& \frac{I-2}{\sqrt{6}}												& -\sqrt{\frac{2\left(2I-1\right)\left(2I+5\right)}{15}}			\\
								\sqrt{\frac{3I\left(I+2\right)}{10}}I	& -\frac{I}{2}									& \frac{3}{2}\sqrt{\frac{I\left(I+2\right)}{5}}						& 0																	\\
								\frac{I-2}{\sqrt{6}}					& \frac{3}{2}\sqrt{\frac{I\left(I+2\right)}{5}}	& \frac{I-2}{6}														& \frac{2}{3}\sqrt{\frac{\left(2I-1\right)\left(2I+5\right)}{5}}	\\
								0										& 0												& \frac{2}{3}\sqrt{\frac{\left(2I-1\right)\left(2I+5\right)}{5}}	& \frac{I-5}{3}
							\end{pmatrix}
\end{equation}

\subsubsection{${F=I}$}

\begin{equation}
	\hat{V}_\text{hf}	=	\frac{1}{2} a_{ms}
							\begin{pmatrix}
								0														& \sqrt{\frac{\left(2I-1\right)\left(2I+3\right)}{10}}				& -\sqrt{\frac{2}{3}}												& -\sqrt{\frac{3\left(I-1\right)\left(I+2\right)}{5}}	\\
								\sqrt{\frac{\left(2I-1\right)\left(2I+3\right)}{10}}	& \frac{1}{2}														& \frac{1}{2}\sqrt{\frac{3\left(2I-1\right)\left(2I+3\right)}{5}}	& 0														\\
								-\sqrt{\frac{3}{2}}										& \frac{1}{2}\sqrt{\frac{3\left(2I-1\right)\left(2I+3\right)}{5}}	& 0																	& \sqrt{\frac{2\left(I-1\right)\left(I+2\right)}{5}}	\\
								-\sqrt{\frac{3\left(I-1\right)\left(I+2\right)}{5}}		& 0																	& \sqrt{\frac{2\left(I-1\right)\left(I+2\right)}{5}}				& -2
							\end{pmatrix}
\end{equation}

\subsubsection{${F=I-1}$}

\begin{equation}
	\hat{V}_\text{hf}	=	\frac{1}{2} a_{ms}
							\begin{pmatrix}
								0														& \sqrt{\frac{2\left(I^2-1\right)}{10}}				& -\frac{I+3}{\sqrt{6}}												& -\sqrt{\frac{2\left(2I-3\right)\left(2I+3\right)}{15}}			\\
								\sqrt{\frac{2\left(I^2-1\right)}{10}}					& \frac{I+1}{2}										& \frac{3}{2}\sqrt{\frac{I^2-1}{5}}									& 0																	\\
								-\frac{I+3}{\sqrt{6}}									& \frac{3}{2}\sqrt{\frac{I^2-1}{5}}					& -\frac{I+3}{6}													& \frac{2}{3}\sqrt{\frac{\left(2I-3\right)\left(2I+3\right)}{5}}	\\
								-\sqrt{\frac{2\left(2I-3\right)\left(2I+3\right)}{15}}	& 0													& \frac{2}{3}\sqrt{\frac{\left(2I-3\right)\left(2I+3\right)}{5}}	& -\frac{I+6}{3}
							\end{pmatrix}
\end{equation}

\subsubsection{${F=I-2}$}

\begin{equation}
	\hat{V}_\text{hf}	=	\frac{1}{2} a_{ms}
							\begin{pmatrix}
								0													& \sqrt{\frac{2}{3}}\left(I+1\right)	& -\sqrt{\frac{\left(I+1\right)\left(I-2\right)}{3}}	\\
								\sqrt{\frac{2}{3}}\left(I+1\right)					& -\frac{1}{3}\left(I+1\right)			& \frac{\sqrt{2I\left(I+3\right)}}{3}	\\
								-\sqrt{\frac{\left(I+1\right)\left(I-2\right)}{3}}	& \frac{\sqrt{2I\left(I+3\right)}}{3}	& -\frac{1}{3}\left(2I+5\right)
							\end{pmatrix}
\end{equation}

\subsubsection{${F=I-3}$}

There is a single state:
\begin{equation}
	\hat{V}_\text{hf}	=	-\frac{1}{2} a_{ms} \left(I+1\right)
\end{equation}

\section{Experimental Method}

(** Take from PRA paper? **)

A schematic diagram of the present experimental arrangement is presented in \cref{fig:spectroscopy_exp_setup}.
The cooling and trapping of strontium is described in detail elsewhere \cite{Xu2003.JOSAB.20.968, Nagel2003.PRA.67.011401, Mukaiyama2003.PRL.90.113002, DeSalvo2010.PRL.105.030402, Stellmer2013.PRA.87.013611}.
Briefly, starting from a Zeeman slowed atomic beam, \Sr{87} atoms are first cooled and trapped using the blue MOT.
The atoms are then further cooled in a narrow-line red MOT.
Approximately \SI{1E6}{atoms} at \SI{\sim 2}{\mu\K} are captured before turning off all trapping fields for spectroscopy measurements.

\begin{figure}[htbp]
	\centering
	\includesvg[keepaspectratio, width=\textwidth, height=4in]{fermion_spectroscopy/experiment/exp_setup.svg}
	\caption{
		\label{fig:spectroscopy_exp_setup}
		(Color online.) {(a)} Diagram of the experimental arrangement showing the \SI{461}{\nm} cooling beams and the counterpropagating \SI{689}{\nm} and \SI{319}{\nm} Rydberg excitation lasers.
		{(b)} Two-photon excitation scheme utilizing either the {(\romannumeral 1)~${\nSLJF{5s5p}{3}{P}{1}{11/2}}$} or {(\romannumeral 2)~${\nSLJF{5s5p}{3}{P}{1}{9/2}}$} intermediate states. 
		The detunings ${\Delta_{11/2} \sim \SI{12}{\MHz}}$ and ${\Delta_{9/2} \sim \SI{36}{\MHz}}$ remain fixed. 
		{(c)} Arrangement of the electrodes used for ionizing Rydberg atoms and guiding the electrons towards the MCP detector.}
\end{figure}

Rydberg atoms are created by two-photon excitation using counterpropagating cross-linearly polarized \SI{689}{\nm} and \SI{319}{\nm} excitation laser beams which drive transitions to the $\nSLJ{5sns}{3}{S}{1}$ and $\nSLJ{5snd}{3}{D}{1,2,3}$ Rydberg levels via the intermediate $\nSLJf{5s5p}{3}{P}{1}{9/2\text{ or }11/2}$ states.
These intermediate states were selected to take advantage of selection rules to aid in identifying the Rydberg hyperfine states populated.
The typical detunings of the \SI{689}{\nm} laser, with respect to the intermediate states, were $\Delta_{9/2} \sim \SI{36}{MHz}$ and $\Delta_{11/2} \sim \SI{12}{\MHz}$, respectively. 
The \SI{689}{\nm} laser was chopped into {\SIrange[range-phrase=--]{10}{20}{\us}-long} pulses to generate temporally localized groups of Rydberg atoms. 
The number of Rydberg atoms produced by each pulse was determined by using the electrodes in \cref{fig:spectroscopy_exp_setup_c} to generate a ramped electric field sufficient to ionize the Rydberg atoms. 
The resulting electrons were directed towards, and detected by, a microchannel plate (MCP) whose output was fed into a multichannel scalar (MCS).
Typically \numrange[range-phrase=--]{100}{500} measurement cycles were performed before loading a new sample and changing the \SI{319}{\nm} laser frequency. 
Spectroscopic measurements at high~$n$ using \Sr{84} showed that the stray fields in the trapping region were less than \SI{10}{\mV\per\cm}. 
Any resultant Stark shifts should therefore be at most a few megahertz even at $n \sim 90$.

The \SI{319}{\nm} radiation was generated by frequency doubling the output of a \SI{638}{\nm} optical parametric oscillator (OPO). 
A sample of the output is sent though a broadband fiber electro-optic modulator (fEOM) from which one of the sidebands was locked to a transfer cavity, allowing the \SI{319}{\nm} laser to be scanned over multiple gigahertz.
The transfer cavity was stabilized using a \SI{689}{\nm} master laser locked to the $\nSLJ{5s^2}{1}{S}{0} \rightarrow \nSLJ{5s5p}{3}{P}{1}$ transition in \Sr{88}. 
The linewidth of the \SI{319}{\nm} laser is estimated to be \SI[input-comparators=\lesssim]{\lesssim500}{\kHz} based on the narrowest observed spectroscopic features.
The wavelength of the \SI{638}{\nm} laser was measured with an EXFO WA-1500 wavemeter. 

\subsection{Calibrating the EXFO WA-1500 Wavemeter}

Since we rely on the wavemeter to determine the absolute energy of the \SI{638}{\nm} photon, we want to characterize potential systematic offsets. 
We initially attempted to calibrate it by looking for a Doppler-free spectrum of molecular iodine ({\tsup{127}I\tsub{2}}) which has the ${P65~\left(7–4\right)}$ line at \SI{15672.517398(25)}{\per\cm} \cite{Sansonetti1997.JOSAB.14.1913}, near the $n=39$ and $n=40$ Rydberg lines.
We didn't have much luck finding narrow Doppler-free signals scanning the \SI{638}{\nm} laser through a our iodine cell\footnote{This matches similar experiences according to Jason Nguyen and Henry Luo in Randy Hulet's lab when they were looking to use an iodine cell to lock their \SI{646}{\nm} laser for a \Li{6} UV MOT \cite{Duarte2011.PRA.84.061406}.}.
A longer cell and a more sensitive detection method may have provided the signal we needed since most {\tsup{127}I\tsub{2}} spectroscopy setups in this wavelength range used cells longer than \SI{30}{\cm} and lock-in detection\cite{Sansonetti1997.JOSAB.14.1913, Huang2018.AO.57.2102}.

Although the iodine spectroscopy didn't work out, we were able to obtain a calibration of our wavemeter's systematics by measuring wavelengths of lasers lock to atomic transitions in \Sr{88} (\SI{689}{\nm}) and \Li{6} (\SI{671}{\nm} and \SI{646}{\nm}). 
As hinted at above, the \SI{646}{\nm} source is used by the Hulet lab for narrow line cooling of \Li{6} on the \SI{323}{\nm} transition \cite{Duarte2011.PRA.84.061406}.
Ya-Ting Chang, Danyel Cavazos, and Dr. Randy Hulet were kind enough to let us run a fiber between the Killian and Hulet labs and borrow some light from their system in order to calibrate our wavemeter. 
The wavemeter measurements of those three wavelengths were performed within about \SI{2}{\hour} to reduce sensitivity to day-to-day environmental changes and are presented in \cref{tab:wavemeter_calibration}.
All measurements of the Rydberg lines are accompanied by a \SI{689}{\nm} calibration measurement under the assumption that the differences between the \SI{689}{\nm} and \SI{638}{\nm} wavelengths are less sensitive to the environment than the absolute values reported by the wavemeter.
\begin{table}[htbp]
	\caption{
		\label{tab:wavemeter_calibration}
		Values used to calibrate the WA-1500 on a single day (2018/03/09) within about \SI{2}{\hour}. 
		The value of the \SI{689}{\nm} transition in \Sr{88} \cite{Ferrari2003.PRL.91.243002, Sansonetti2010.JPCRD.39.033103}.
		The values of the \SI{671}{\nm} and \SI{323}{\nm} (\SI{646}{\nm}) transitions in \Li{6} were taken from \cite{Sansonetti2011.PRL.107.023001, Sansonetti2012.PRL.109.259901, Radziemski1995.PRA.52.4462}.}
	\centering
	\makebox[\textwidth][c]{
	\begin{tabular}{@{}ccccc@{}}
		\toprule
		$\lambda$ [\si{\nm}]	& Atom						& Transition																		& Reference [\si{\per\cm}]			& Measured [\si{\per\cm}]		\\
		\midrule
		\num{689}				& \Sr{88}					& $\nSLJ{5s^2}{1}{S}{0} \rightarrow \nSLJ{5s5p}{3}{P}{1}$							& \num{14504.33824159+-0.00000033}	& \num{14504.34224+-0.00035}	\\
		\num{671}				& \Li{6}					& $\nSLJf{2s}{2}{S}{1/2}{3/2} \rightarrow \nSLJ{2p}{2}{P}{3/2}$						& \num{14903.6295242+-0.0000007}	& \num{14903.63391+-0.00033}	\\
		\num{323}				& \multirow{2}{*}{\Li{6}}	& \multirow{2}{*}{$\nSLJf{2s}{2}{S}{1/2}{3/2} \rightarrow \nSLJ{3p}{2}{P}{3/2}$}	& \num{30925.1703+-0.0010}			& \num{30925.1792+-0.0010}		\\
		(\num{646})				&							&																					& (\num{15462.5851+-0.0005})		& (\num{15462.5896+-0.0005})	\\
		\bottomrule
	\end{tabular}
	}
\end{table}
The values in \cref{tab:wavemeter_calibration} were used to calculate the offset $\delta = \nu_{\text{measured}} - \nu_{\text{reference}}$ between our measured values and published values and is plotted in \cref{fig:wm_fit}.
A linear fit yields a correction of \SI{\approx 140}{\MHz} at \SI{638}{\nm}. 
A Monte Carlo sampling was adopted in which linear fits to points drawn at random from the Gaussian uncertainty distributions appropriate to each point in the calibration were repeated, resulting in a systematic uncertainty ($\sigma_\mathrm{sys}$) of about \SI{\pm 25}{\MHz} ($\SI{\pm 50}{\MHz}$) at \SI{638}{\nm} (\SI{319}{\nm}). 
\begin{figure}[htbp]
	\centering
	\includesvg[keepaspectratio, width=4in]{fermion_spectroscopy/wavemeter_calibration/wm_fit.svg}
	\caption{
		\label{fig:wm_fit}
		Wavelength dependence of the offset $\delta$ between the measured and published transition frequencies used to calibrate the wavemeter. 
		The black line shows the linear fit used to obtain the offset at \SI{638}{\nm} and the shaded region the uncertainty in the wavemeter calibration obtained from Monte Carlo simulations.}
\end{figure}

To check for drifts in the wavemeter calibration, each \SI{638}{\nm} wavelength measurement was followed by a reference measurement of the \SI{689}{\nm} master laser. 
As shown in the inset in \cref{fig:wm_cal}, the day-to-day variations were relatively small compared to the wavemeter's systematic uncertainty. 
Whereas our wavemeter limits the measurements of individual term energies to \SI{\sim 60}{\MHz}, line separations can be measured to kilohertz-level accuracies when scanning within a single free spectral range (FSR) of the transfer cavity, and to megahertz-level accuracies when piecing together scans over successive FSRs.
\begin{figure}[htbp]
	\centering
	\includesvg[keepaspectratio, width=5in]{fermion_spectroscopy/wavemeter_calibration/wm_drift.svg}
	\caption{
		\label{fig:wm_drift}
		Compiled measurements of the wavemeter offset of the \SI{689}{\nm} transition.
		The data comes from the \SI{689}{\nm} reference measurements paired with each \SI{638}{\nm} measurement.}
\end{figure}

**Add footnote: at the time, the best reference for the \SI{323}{\nm} transition we could find was by \citeauthor{Radziemski1995.PRA.52.4462} although there should be some upcoming measurements using frequency combs (cite papers mentioning doing this)**

\subsection{Two-photon Rydberg excitation spectrum}

While taking the Rydberg spectra, we observed spurious electron counts when the UV laser detuning compensated for the \SI{689}{\nm} detuning. 
Due to this coincidence, we believe these ``ghost'' lines are due to atoms in the $\nSLJF{5s5p}{3}{P}{1}{9/2,11/2}$ state which were off-resonantly excited by a \SI{689}{\nm} photon before a \SI{319}{\nm} photon excites them to a Rydberg state. 
To check for this effect, we use the density matrix method **Lindblad operator** to simulate the time-dependent dynamics with parameter values estimated from the measured excitation laser powers and beam waists. 

**Show example spectra with ``ghost'' on-resonance counts.**

%\begin{figure}[htbp]
%	\centering
%	\includesvg[keepaspectratio, height=2in]{fermion_spectroscopy/three-level_system.svg}
%	\caption{\label{fig:three-level_system}Three-level diagram representing a two-photon excitation to a Rydberg state with $\ket{g}$, $\ket{e}$, and $\ket{r}$ representing the ground, intermediate, and Rydberg states, respectively. The detunings from the intermediate (Rydberg) state is $\Delta_e$ ($\Delta_r$). Couplings between states are represented by $\Omega_1 \exp(-\iu \omega_1 t)$ ($\Omega_2 \exp(-\iu \omega_2 t)$) for the \SI{689}{\nm} (\SI{319}{\nm}) lasers.}
%\end{figure}

The bare atomic Hamiltonian for this system can be expressed as 
\begin{equation}
	H_\text{A}
		= \omega_r \dyad{r} + \hbar \omega_e \dyad{e} + \hbar \omega_g \dyad{g}
		= \hbar	\begin{pmatrix}
					\omega_r	& 0			& 0	\\
					0			& \omega_e	& 0	\\
					0			& 0			& \omega_g
				\end{pmatrix}
\end{equation}
The atom-field interaction can be written as 
\begin{equation}
	H_\text{AF}
		= \frac{\hbar}{2}\left(\Omega_1 \operatorname{e}^{-\iu \omega_1 t} \dyad{g}{e} + \Omega_2 \operatorname{e}^{-\iu \omega_2 t} \dyad{e}{r} \right)+ \text{h.c}
		=	\frac{\hbar}{2}	\begin{pmatrix}
				0												& \Omega_2 \operatorname{e}^{-\iu \omega_2 t}		& 0	\\
				\Omega_2^\ast \operatorname{e}^{\iu \omega_2 t}	& 0													& \Omega_1 \operatorname{e}^{-\iu \omega_1 t}	\\
				0												& \Omega_1^\ast \operatorname{e}^{\iu \omega_1 t}	& 0
			\end{pmatrix}
\end{equation}
Using the unitary (?) transformation
\begin{equation}
	U	=	\begin{pmatrix}
				\operatorname{e}^{-\iu (\Omega_1 + \Omega_2) t}	& 0													& 0	\\
				0												& \operatorname{e}^{-\iu \Omega_1 t}					& 0	\\
				0												& 0													& 1
			\end{pmatrix}
\end{equation}
the total Hamiltonian $H = H_\text{A} + H_\text{AF}$ can be written as
\begin{equation}
	\widetilde{H}
		= U^\dag H U - \iu \hbar U^\dag \dv{U}{t}
		= \hbar	\begin{pmatrix}
					-\Delta_r						& \flatfrac{\Omega_2}{2}		& 0			\\
					\flatfrac{\Omega_2^\ast}{2}		& -\Delta_e						& \flatfrac{\Omega_1}{2}	\\
					0								& \flatfrac{\Omega_1^\ast}{2}	& \omega_g
				\end{pmatrix}
\end{equation}
where $\Delta_e = \omega_1 - \omega_e$ is the detuning from the intermediate state and $\Delta_r = \omega_r - (\omega_1 + \omega_2)$ is the total detuning from the Rydberg state. 
To simplify things, we can take $\omega_g = 0$.

**Hopefully the time-dynamics calculations explain the observed ``ghost'' lines**.

\begin{figure}[htbp]
	\centering
	\includesvg[keepaspectratio, width=4in]{fermion_spectroscopy/three-level_system/three-level_calculation.svg}
	\caption{\label{fig:three-level_calculation}Three-level system calculation.}
\end{figure}

\section{Results and Discussion}

The measured state energies of the \Sr{87} Rydberg states are presented in \cref{tab:data-s-state} and \cref{tab:data-d-state} which include our best estimates of both statistical and systematic uncertainties.
Although our state energies have an uncertainty of about **xxx MHz**, we can measure splittings of up to a few GHz with kHz level accuracies limited by the synthesizer. 

\Cref{tab:sr_qd_list} shows how the quantum defects extracted from the \Sr{87} spectra compare to quantum defects extracted from the bosonic isotopes.

\begin{table}[htbp]
	\caption{
		\label{tab:sr_qd_list}
		Values of the parameters $\delta_0$, $\delta_2$, and $\delta_4$ for the Rydberg-Ritz formula.}
	\centering
	\makebox[\textwidth][c]{
	\begin{tabular}{@{}cccccc@{}}
		\toprule
		Series	& Term				& $\delta_0$		& $\delta_2$		& $\delta_4$			& Ref.										\\
		\midrule
		$5sns$	& $\SLJ{1}{S}{0}$	& \num{3.26896(2)}	& \num{-0.138(7)}	& \num{0.9(6)}			& \cite{Vaillant2012.JPB.45.135004}			\\
		\midrule
		$5sns$	& $\SLJ{3}{S}{1}$	& \num{3.371(2)}	& \num{0.5(2)}		& \num{-1(2)e1}			& \cite{Vaillant2012.JPB.45.135004}			\\
				&					& \num{3.37065}		& \num{0.443}		& \num{-0.553}			& \cite{Ding2018.PRA.98.042505} (our work)	\\
				&					& \num{3.36924(2)}	& \num{0.52(2)}		& \num{-0.1(3)}			& \cite{Jackson2018.PhD}					\\
				&					& \num{3.370778(4)}	& \num{0.418(1)}	& \num{-0.3(1)}			& \cite{Couturier2019.PRA.99.022503}		\\
		\midrule
		$5snd$	& $\SLJ{1}{D}{2}$	& \num{2.3807(2)}	& \num{-39.41 (6)}	& \num{-109(2)e1}		& \cite{Vaillant2012.JPB.45.135004}			\\
		\midrule
		$5snd$	& $\SLJ{3}{D}{1}$	& \num{2.658(6)}	& \num{3(2)}		& \num{-8.8(7)e3}		& \cite{Vaillant2012.JPB.45.135004}			\\
				&					& \num{2.673}		& \num{-5.4}		& \num{-8166}			& \cite{Ding2018.PRA.98.042505} (our work)	\\
				&					& \num{2.664(7)}	& \num{-1(9)}		& \num{-8(2)e3}			& \cite{Jackson2018.PhD}					\\
				&					& \num{2.67517(20)}	& \num{-13.15(26)}	& \num{-4.444(91)e3}	& \cite{Couturier2019.PRA.99.022503}		\\
		\midrule
		$5snd$	& $\SLJ{3}{D}{2}$	& \num{2.636(5)}	& \num{-1(2)}		& \num{-9.8(9)e3}		& \cite{Vaillant2012.JPB.45.135004}			\\
				&					& \num{2.662}		& \num{-15.4}		& \num{-9804}			& \cite{Ding2018.PRA.98.042505} (our work)	\\
				&					& \num{2.66142(30)}	& \num{-16.77(38)}	& \num{-6.656(134)e3}	& \cite{Couturier2019.PRA.99.022503}		\\
		\midrule
		$5snd$	& $\SLJ{3}{D}{3}$	& \num{2.63(1)}		& \num{-42.3(3)}	& \num{-18(1)e3}		& \cite{Vaillant2012.JPB.45.135004}			\\
				&					& \num{2.612}		& \num{-41.4}		& \num{-15363}			& \cite{Ding2018.PRA.98.042505} (our work)	\\
		\bottomrule
	\end{tabular}
	}
\end{table}

\subsection{$S$-State Results}

Plotting our quantum defects compared to previously published data.

\begin{figure}[htbp]
	\centering
	\includesvg[keepaspectratio, width=\textwidth, height=\textheight]{fermion_spectroscopy/results/results-qd-1S0.svg}
	\caption{
		\label{fig:results-qd-1S0}
		Results for $\nSLJ{5sns}{1}{S}{0}$ quantum defects.}
\end{figure}

\begin{figure}[htbp]
	\centering
	\includesvg[keepaspectratio, width=\textwidth, height=\textheight]{fermion_spectroscopy/results/results-qd-3S1.svg}
	\caption{
		\label{fig:results-spectra-D}
		Results for $\nSLJ{5sns}{3}{S}{1}$ quantum defects.}
\end{figure}

\subsection{$D$-State Results}

\Cref{fig:spectra-n50,fig:spectra-n60,fig:spectra-n98} show our experimentally observed spectra and their theoretically calculated positions which agree quite well.

\begin{figure}[htbp]
	\centering
	\includesvg[keepaspectratio, width=\textwidth, height=\textheight]{fermion_spectroscopy/results/spectra/spectra-n50.svg}
	\caption{
		\label{fig:spectra-n50}
		Measured and predicted spectra for {$D$~states} near ${n=50}$ plotted relative to the $\nSLJf{5s50s}{3}{S}{1}{11/2}$ state.
		All the data, was obtained by following scheme {(\romannumeral 1)}.}
\end{figure}

\begin{figure}[htbp]
	\centering
	\includesvg[keepaspectratio, width=\textwidth, height=\textheight]{fermion_spectroscopy/results/spectra/spectra-n60.svg}
	\caption{
		\label{fig:spectra-n60}
		Measured and predicted spectra for {$D$~states} near ${n=60}$ plotted relative to the $\nSLJf{5s60s}{3}{S}{1}{11/2}$ state.
		All the data, except for the $\nSLJf{5s60s}{3}{S}{1}{11/2}$ spectra, was obtained by following scheme {(\romannumeral 1)}.
		The $\nSLJf{5s60s}{3}{S}{1}{11/2}$ data was taken following scheme {(\romannumeral 2)} and was scaled by $\num{1/5}$.}
\end{figure}

\begin{figure}[htbp]
	\centering
	\includesvg[keepaspectratio, width=\textwidth, height=\textheight]{fermion_spectroscopy/results/spectra/spectra-n98.svg}
	\caption{
		\label{fig:spectra-n98}
		Measured and predicted spectra for {$D$~states} near ${n=98}$ plotted relative to the $\nSLJf{5s98s}{3}{S}{1}{11/2}$ state.
		All the data, was obtained by following scheme {(\romannumeral 2)}.}
\end{figure}


Plotting our quantum defects compared to previously published data.

\begin{figure}[htbp]
	\centering
	\includesvg[keepaspectratio, width=\textwidth, height=\textheight]{fermion_spectroscopy/results/results-qd-1D2.svg}
	\caption{
		\label{fig:results-qd-1D2}
		Results for $\nSLJ{5snd}{1}{D}{2}$ quantum defects.}
\end{figure}

\begin{figure}[htbp]
	\centering
	\includesvg[keepaspectratio, width=\textwidth, height=\textheight]{fermion_spectroscopy/results/results-qd-3D1.svg}
	\caption{
		\label{fig:results-qd-3D1}
		Results for $\nSLJ{5snd}{3}{D}{1}$ quantum defects.}
\end{figure}

\begin{figure}[htbp]
	\centering
	\includesvg[keepaspectratio, width=\textwidth, height=\textheight]{fermion_spectroscopy/results/results-qd-3D2.svg}
	\caption{
		\label{fig:results-qd-3D2}
		Results for $\nSLJ{5snd}{3}{D}{2}$ quantum defects.}
\end{figure}

\begin{figure}[htbp]
	\centering
	\includesvg[keepaspectratio, width=\textwidth, height=\textheight]{fermion_spectroscopy/results/results-qd-3D3.svg}
	\caption{
		\label{fig:results-qd-3D3}
		Results for $\nSLJ{5snd}{3}{D}{3}$ quantum defects.}
\end{figure}

\section{Summary}

** Update and reword the summary from PRA !! **

The present work demonstrates that the energies of {high-$n$} \Sr{87} Rydberg states can be accurately determined by diagonalizing an isotope-rescaled Hamiltonian.
This Hamiltonian is constructed using spectral information for the bosonic isotope (\Sr{88}) which has vanishing nuclear spin combined with the hyperfine interaction present in \Sr{87}.
The present approach can be implemented for fermionic atoms whenever the energy levels for an isotope with vanishing nuclear spin are available.
The method can also be applied in reverse, allowing determination of spectroscopic information, in particular quantum defects, for bosonic isotopes from the hyperfine-resolved spectrum of the fermionic isotope.
The major limitation on the accuracy of the present analysis is the uncertainty in the hyperfine-resolved ionization threshold.
This uncertainty can be removed by focusing on energy differences to a reference level whereupon accuracies of the order of a few megahertz can be achieved. 