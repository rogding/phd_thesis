\chapter{Spectroscopy of $\Sr{87}$ Triplet Rydberg states}

We didn't set out to map out the triplet series of $\Sr{87}$ Rydberg states, but when we first started looking for $\Sr{87}$ $\SLJ{3}{S}{}$ and $\SLJ{3}{D}{}$ Rydberg states, we observed spectra which didn't line up with their predicted locations using the known quantum defects. 
It turns out that the hyperfine interaction in $\Sr{87}$ is not negligible as the Rydberg states approach the $\Srion{87}{+}$ hyperfine split core. 
We then rediscovered some papers from the 1980's which explained the shifted $\SLJ{1}{S}{0}$ spectra in $\Sr{87}$ by including the mixing of states with the same $F$. 
Previous work was accomplished mostly through the $\nSLJ{5s5p}{1}{P}{1}$ state or from the metastable $\nSLJ{5s5p}{3}{P}{2}$ states, our work appears to be the first measurements of the $\Sr{87}$ hyperfine Rydberg states through the $\nSLJ{5s5p}{3}{P}{1}$ intermediate state. 
**Should probably state the previous sentence as something like ``the first direct measurement of 87Sr triplet Rydberg states''.**

We measured $\Sr{87}$ hyperfine Rydberg spectra using a two-photon excitation through the intermediate $\nSLJ{5s5p}{3}{P}{1}$ states and presented the work in \cite{Ding2018.PRA.98.042505}.
This thesis contains a bit of a reanalysis from my point of view but the results remain largely unchanged from the published paper. 

To save some space, the tables of measured \Sr{87} Rydberg energies are placed in the appendix: see \Cref{tab:data-s-state} and \Cref{tab:data-d-state}.

\section{Rydberg States of Hyperfine Cplit cores}

A modern paper is \cite{Robicheaux2018.PRA.97.022508}. 

To calculate the effects of hyperfine interactions on the Rydberg spectra of $\Sr{87}$, we follow previous works by **cite Esherick, Beigang, K T Lu, Robicheaux?** where we mass-scale the Rydberg levels from an isotope without nuclear spin (i.e., $\Sr{88}$) and then apply the hyperfine interaction to obtain the lines for $\Sr{87}$.
The theoretical approach presented below follows the one Shuhei Yoshida presents in \cite{Ding2018.PRA.98.042505}.

\section{Ionization limit of $\Sr{87}$}

The first indication that the Rydberg levels of $\Sr{87}$ were not as simple as for the bosonic isotopes was that the reported ionization limit of \SI{45932.2861(10)}{\per\cm} \cite{Beigang1982.OC.42.19, Sansonetti2010.JPCRD.39.033103} for $\Sr{87}$ is well above that of all the bosonic isotopes (see \cref{tab:sr_nuc_prop}). 
This suggests that the reported ionization limit is for the $\nSLJF{5s}{2}{S}{1/2}{4}$ state of $\Srion{87}{+}$.
Like the case for alkali atoms, the nonzero nuclear spin of $\Srion{87}{+}$ interacts with the spin of the remaining $5s$ electron and splits the $\nSLJ{5s}{2}{S}{1/2}$ into $F=4,5$ components.
The magnetic dipole interaction can be written as
\begin{equation}
	\ev{W} = A \ev{\hat{\bm{I}} \cdot \hat{\bm{S}}} = \frac{A}{2} \left(F\left(F+1\right) - I\left(I+1\right) - S\left(S+1\right)\right)
\end{equation}
where $\bm{F} = \bm{J} + \bm{I}$ and $S=s=1/2$ since core electrons of \Srion{87}{+} form a closed shell. 
\citeauthor{sun1993.HI.78.241} measured the $F=4,5$ hyperfine splitting in \Srion{87}{+} to be $\Delta E = \SI{5002368.363(57)}{\kHz}$, meaning $A = \SI{-1000473.673+-0.011}{\kHz}$. 
Since the hyperfine shift of the $F=4$ state is $\Delta E_{F=4} = \flatfrac{A}{2}(\flatfrac{-11}{2})= \SI{45932.1943+-0.0010}{\per\cm}$.
This value is used for the analysis below.

As a check, we can use the ionization limits for the bosons and the atomic masses of the isotopes to obtain a mass-scaled ionization limit for \Sr{87}.
The value estimated using this method is **\SI{45932.19870+-0.00035}{\per\cm}**. 

\begin{figure}[htbp]
	\centering
	\includesvg[keepaspectratio, width=4in]{fermion_spectroscopy/calculations/Eion-mass-scaled.svg}
	\caption{
		\label{fig:Eion-mass-scaled}
		Mass-scaled $E_\text{ion}$.}
\end{figure}

\subsection{Incorporating hyperfine interaction in Rydberg levels}

**Work on this section about getting from 2-electron Hamiltonian to effective Rydberg Hamiltonian for 5sns states.**

The Hamiltonian for a two-electron system in an $\ket{(ms)(nl)}$ Rydberg state can be written as \cite{Beigang1983.PRL.51.771}
\begin{equation}
	H	= H_{0,1} + H_{0,2} + \frac{1}{r_{12}} + \xi_{nl} \vec{s} \cdot \vec{l} + a_{ms} \vec{s} \cdot \vec{I}
\end{equation}
where $H_{0,i}$ is the Coulomb attraction for the {$i$th} electron to the \Sr{++} core, $\flatfrac{1}{r_{12}}$ is the Coulomb repulsion between the two electrons, $\xi_{nl}$ is the magnitude of the spin-orbit interaction, and $a_{ms}$ is the magnetic dipole interaction of the inner $s$ electron.
For \Sr{88}, $I=0$ so we can write the Hamiltonian as
\begin{equation}
	H_{0}(88) = H
\end{equation}
where the appropriate mass-scaled values are used and includes the spin-orbit and spin-other-orbit interactions. 
The hyperfine interaction for \Sr{87} can be included as
\begin{equation}
	H(87) = H_{0}(88,87) + V_\text{hf}
\end{equation}

For two-electron systems, the hyperfine interaction can be expressed as
\begin{equation}
	V_\text{hf} = a_{ms} \left(\hat{\bm{s}}_{1} + \hat{\bm{s}}_{2}\right) \cdot \hat{\bm{I}} \simeq a_{ms} \hat{\bm{s}}_{\text{in}} \cdot \hat{\bm{I}}
\end{equation}
where $\hat{\bm{s}}_{i}$ is the spin of the {$i$th} electron. 

For singly-excited states, the hyperfine interaction of the ``outer'' electron with the core should scale as **$\sim 1/n^3$** and should be negligible for high-lying states. 
We make the assumption that $\hat{V}_\text{hf} \simeq a_{ms} \hat{\bm{s}}_{\text{in}} \cdot \hat{\bm{I}}$ for singly-excited Rydberg states with inner electron $\ket{ms}$ and Rydberg electron $\ket{nl}$.
Expanding $\hat{\bm{s}}_\text{in}$ and $\hat{\bm{I}}$ in terms of ladder operators gives
\begin{equation}
	\hat{V}_\text{hf} \simeq \frac{a_{ms}}{2} \left(\hat{s}_{\text{in},+} \hat{I}_{-} + \hat{s}_{\text{in},-} \hat{I}_{+} + 2\hat{s}_{\text{in},z} \hat{I}_{z}\right)
\end{equation}
with the usual ladder operators $\hat{J}_{\pm}=\hat{J}_{x} \pm \hat{J}_{y}$. 
Note that $\hat{V}_\text{hf}$ leaves the total spin $m_F$ unchanged, meaning it couples states of the same $m_F$. 

\subsection{Singly-Excited {$S$-states}}

Assuming there are no perturbers (** see \cite{Vaillant2014.JPB.47.155001} ? **), we make the approximation that the inner electron is in the $5s$-state whereas the other electron is in the {$ns$-state}. 

For $S$ states, the pure singlet states can be written as
\begin{equation}
	\ket{\nSLJm{5sns}{1}{S}{0}{0}}_{0} = \frac{1}{2} \left(\ket{5s;ns} + \ket{ns;5s}\right) \left(\ket{\uparrow;\downarrow} - \ket{\downarrow;\uparrow}\right)
\end{equation}
Similarly, the pure triplet $S$-states can be written as
\begin{align}
	&\ket{\nSLJm{5sns}{3}{S}{1}{1}}_{0} = \frac{1}{\sqrt{2}} \left(\ket{5s;ns} - \ket{ns;5s}\right) \ket{\uparrow;\uparrow}	\\
	&\ket{\nSLJm{5sns}{3}{S}{1}{0}}_{0} = \frac{1}{2} \left(\ket{5s;ns} - \ket{ns;5s}\right) \left(\ket{\uparrow;\downarrow} + \ket{\downarrow;\uparrow}\right)	\\
	&\ket{\nSLJm{5sns}{3}{S}{1}{-1}}_{0} = \frac{1}{\sqrt{2}} \left(\ket{5s;ns} - \ket{ns;5s}\right) \ket{\downarrow;\downarrow}
\end{align}
To incorporate the nuclear spin, we multiply the above states by $\ket{I, m_I}$.
Due to the hyperfine interaction, $J$ and $I$ are no longer good quantum numbers so we use $\vec{F} = \vec{J} + \vec{I}$.


*************

*************

*************


Due to the hyperfine interaction, $J$ and $I$ are no longer good quantum numbers so we use $F = J + I$.
Therefore, we can use the Clebsch–Gordan coefficients to determine basis states in terms of $\ket{F, m_F}$.
\begin{align}
	\ket{\nSLJFm{5sns}{3}{S}{1}{7/2}{m_F}}	&= \sum\limits_{-m_J}^{m_J} \sum\limits_{-m_I}^{m_I} \ket{\nSLJm{5sns}{3}{S}{1}{m_J}} \ket{I, m_I}
\end{align}

\subsubsection{{$n$-mixing}}

At higher $n$, the hyperfine shift becomes comparable to the splitting between $n$ and $n \pm 1$. 
Since the radial matrix elements are only orthogonal for the same $n$, there's mixing between states of different $n$ \cite{Beigang1983.PRL.51.771}.
For \Sr{87}, this occurs around $n=113$ (** CHECK **). 

The overlap integrals between adjacent $n$ can be approximated by \cite{Beigang1983.PRL.51.771, Bhatti1981.PRA.24.161}
\begin{equation}
	\braket{n^\star}{\left(n+1\right)^\star} = -\frac{2\left[n^\star\left(n+1\right)^\star\right]}{n^\star + \left(n+1\right)^\star} ....
\end{equation}

(** SHOW FIGURE **)

As see in Fig. xx, the strongest {$n$-mixing} occurs for $\Delta n = \pm 1$ with $O_{n,n^{\prime}} \sim \num{0.1}$ and which quickly drops off for larger $\Delta n$. 

\subsection{$D$-states}

** Stuff about D-states. **

** Following similar derivations (outlined by Shuhei Yoshida). **

\section{Experimental Method}

(** Take from PRA paper? **)

A schematic diagram of the present experimental arrangement is presented in \cref{fig:spectroscopy_exp_setup}.
The cooling and trapping of strontium is described in detail elsewhere \cite{Xu2003.JOSAB.20.968, Nagel2003.PRA.67.011401, Mukaiyama2003.PRL.90.113002, DeSalvo2010.PRL.105.030402, Stellmer2013.PRA.87.013611}.
Briefly, starting from a Zeeman slowed atomic beam, \Sr{87} atoms are first cooled and trapped using the blue MOT.
The atoms are then further cooled in a narrow-line red MOT.
Approximately \num{1E6} atoms at \SI{\sim 2}{\mu\K} are captured before turning off all trapping fields for spectroscopy measurements

Rydberg atoms are created by two-photon excitation using counterpropagating cross-linearly polarized \SI{689}{\nm} and \SI{319}{\nm} excitation laser beams.
Depending on which intermediate $F^{\prime}$ state we use, this scheme allows us to produce Rydberg atoms in the $F^{\prime\prime} = F^{\prime} \pm \left\{0,1\right\}$
\begin{equation*}
	\nSLJF{5s^2}{1}{S}{0}{9/2}
		\rightarrow {\nSLJ{5s5p}{3}{P}{1}{,\,}{F^{\prime}=\flatfrac{9}{2}{,} \flatfrac{11}{2}}}
		\rightarrow	\begin{cases}
						{\nSLJ{5sns}{3}{S}{1}{,\,}{F^{\prime\prime} = F^{\prime}\pm 0,1}}	\\
						{\nSLJ{5snd}{3}{D}{1,2,3}{,\,}{F^{\prime\prime} = F^{\prime}\pm 0,1}}
					\end{cases}
\end{equation*}
These intermediate states were selected to take advantage of selection rules to aid in identifying the Rydberg hyperfine states populated.
The typical detunings of the \SI{689}{\nm} laser were $\Delta_{9/2} \sim \SI{36}{MHz}$ and $\Delta_{11/2} \sim \SI{12}{\MHz}$. 
The \SI{689}{\nm} laser was chopped into **(10–20)-us-long** pulses to generate temporally localized groups of Rydberg atoms. 
The number of Rydberg atoms produced by each pulse was determined by using the electrodes in Fig. 4(c) to generate a ramped electric field sufficient to ionize the Rydberg atoms. 
The resulting electrons were directed towards, and detected by, a microchannel plate (MCP) whose output was fed into a multichannel scalar (MCS).
Typically 100–500 measurement cycles were performed before loading a new sample and changing the \SI{319}{\nm} laser frequency. 
Spectroscopic measurements at high $n$ using \Sr{84} showed that the stray fields in the trapping region were less than \SI{10}{\mV\per\cm}. 
Any resultant Stark shifts should therefore be at most a few megahertz even at $n \sim 90$.

\begin{figure}[htbp]
	\centering
	\includesvg[keepaspectratio, width=\textwidth, height=4in]{fermion_spectroscopy/experiment/exp_setup.svg}
	\caption{
		\label{fig:spectroscopy_exp_setup}
		(Color online.) {(a)} Diagram of the experimental arrangement showing the \SI{461}{\nm} cooling beams and the counterpropagating \SI{689}{\nm} and \SI{319}{\nm} Rydberg excitation lasers.
		{(b)} Two-photon excitation scheme utilizing either the {(\romannumeral 1)~${\nSLJF{5s5p}{3}{P}{1}{11/2}}$} or {(\romannumeral 2)~${\nSLJF{5s5p}{3}{P}{1}{9/2}}$} intermediate states. 
		The detunings ${\Delta_{11/2} \sim \SI{12}{\MHz}}$ and ${\Delta_{9/2} \sim \SI{36}{\MHz}}$ remain fixed. 
		{(c)} Arrangement of the electrodes used for ionizing Rydberg atoms and guiding the electrons towards the MCP detector.}
\end{figure}

The \SI{319}{\nm} radiation was generated by frequency doubling the output of a \SI{638}{\nm} optical parametric oscillator (OPO). 
A sample of the output is sent though a broadband fiber electro-optic modulator (fEOM) from which one of the sidebands was locked to a transfer cavity, allowing the \SI{319}{\nm} laser to be scanned over multiple gigahertz.
The transfer cavity was stabilized using a \SI{689}{\nm} master laser locked to the ${\nSLJ{5s^2}{1}{S}{0} \rightarrow \nSLJ{5s5p}{3}{P}{1}}$ transition in \Sr{88}. 
The linewidth of the \SI{319}{\nm} laser is estimated to be \SI[input-comparators=\lesssim]{\lesssim500}{\kHz} based on the narrowest observed spectroscopic features.
Additional details about how we scanned the UV laser frequency is given in \Cref{ap:scanning_uv_laser}.

\subsection{Calibrating the WA-1500}

Since we rely on the wavemeter to determine the absolute energy of the \SI{638}{\nm} photon, we want to characterize potential systematic offsets. 
We initially attempted to calibrate the wavemeter by looking for a Doppler-free spectrum of molecular iodine ({\tsup{127}I\tsub{2}}) which has the {P65 (7–4)} line at \SI{15672.517398(25)}{\per\cm} \cite{Sansonetti1997.JOSAB.14.1913}, close to the **$n=40$ Rydberg line in \Sr{88}**.
We didn't have much luck finding narrow Doppler-free signals scanning the \SI{638}{\nm} laser through a our iodine cell\footnote{This matches similar experiences according to Jason and Henry in Randy Hulet's lab (they were looking to use the iodine cell to lock their \SI{646}{\nm} laser for a **\Li{7}** UV MOT \cite{dua2011.PRA.84.061406}).}.
A longer cell and a more sensitive detection scheme may have provided the signal we needed since \citeauthor{Sansonetti1997.JOSAB.14.1913} used a \SI{30}{\cm} long cell and lock-in detection. A setup similar to the one in \cite{Huang2018.AO.57.2102} may work as well.

Although the iodine spectroscopy didn't work out, we were able to obtain a calibration of our wavemeter's systematics by measuring wavelengths of lasers lock to atomic transitions in \Sr{88} (\SI{689}{\nm}) and \Li{6} (\SI{671}{\nm} \cite{Sansonetti2011.PRL.107.023001, Sansonetti2012.PRL.109.259901} and \SI{646}{\nm} \cite{Radziemski1995.PRA.52.4462}). 
As hinted at above, the \SI{646}{\nm} source is used by the Hulet lab for narrow line cooling of \Li{6} on the \SI{323}{\nm} transition.
Ya-Ting Chang, Danyel Cavazos, and Dr. Randy Hulet were kind enough to let us run a fiber between the Killian and Hulet labs and borrow some light from their system in order to calibrate our wavemeter. 

A wavemeter (EXFO WA-1500) was used to measure the wavelength of the \SI{638}{\nm} output from the OPO and hence determine the Rydberg state energies with a resolution-limited statistical uncertainty ($\sigma_\mathrm{stat}$) of about \SI{\pm 15}{\MHz} ($\SI{\pm 30}{\MHz}$) at \SI{638}{\nm} (\SI{319}{\nm}). 
In order to estimate systematic offsets in the wavemeter, the frequencies of lasers locked to atomic transitions in \Sr{88} (${\nSLJ{5s^2}{1}{S}{0} \rightarrow \nSLJ{5s5p}{3}{P}{1}}$ at \SI{689}{\nm} \cite{Ferrari2003.PRL.91.243002, Sansonetti2010.JPCRD.39.033103}) and in \Li{6} (${\nSLJF{2s}{2}{S}{1/2}{3/2} \rightarrow \nSLJ{2p}{2}{P}{3/2}}$ at \SI{671}{\nm} and ${\nSLJF{2s}{2}{S}{1/2}{3/2} \rightarrow \nSLJ{3p}{2}{P}{3/2}}$ at $\SI{646}{\nm}/2=\SI{323}{\nm}$ \cite{Sansonetti2011.PRL.107.023001, Sansonetti2012.PRL.109.259901, Radziemski1995.PRA.52.4462}). 
The measured wavelengths were then compared to the published values for the same transitions and the differences, $\delta$, between the measured and published frequencies are shown in Fig.~\ref{fig:wm_cal}. 
A linear fit yields a correction of \SI{\approx 140}{\MHz} at \SI{638}{\nm}. 
In an attempt to estimate the systematic uncertainty in this calibration factor, a Monte Carlo sampling was adopted in which linear fits to points drawn at random from the Gaussian uncertainty distributions appropriate to each point in the calibration were repeated, resulting in a systematic uncertainty ($\sigma_\mathrm{sys}$) of about \SI{\pm 25}{\MHz} ($\SI{\pm 50}{\MHz}$) at \SI{638}{\nm} (\SI{319}{\nm}). 
To check for drifts in the wavemeter calibration, each \SI{638}{\nm} wavelength measurement was followed by a reference measurement of the \SI{689}{\nm} master laser. 
As shown in the inset in Fig.~\ref{fig:wm_cal}, the day-to-day variations were relatively small compared to the wavemeter's systematic uncertainty. 
Whereas our wavemeter limits the measurements of individual term energies to \SI{\sim 60}{\MHz}, line separations can be measured to kilohertz-level accuracies when scanning within a single free spectral range (FSR) of the transfer cavity, and to megahertz-level accuracies when piecing together scans over successive FSRs.

\begin{table}[htbp]
	\caption{
		\label{tab:wavemeter}Values used to calibrate the WA-1500 on a single day (2018/03/09) within ~2 hours to reduce day-to-day environmental changes)**. Reported values for \Li{6} were given with respect to the center-of-gravity of the lower $\nSLJ{5s}{2}{S}{1/2}$ state so I included the shift to the $F=3/2$ state.
		**Still need to calculate HF shifted values for \Li{6}.**}
	\centering
	\makebox[\textwidth][c]{
	\begin{tabular}{@{}ccccccc@{}}
		\toprule
		$\lambda$ [\si{\nm}]	& Atom						& Lower state									& Upper state								& Reported [\si{\per\cm}]			& Measured [\si{\per\cm}]	& Ref																							\\
		\midrule
		\num{323}				& \multirow{2}{*}{\Li{6}}	& \multirow{2}{*}{$\nSLJF{2s}{2}{S}{1/2}{3/2}$}	& \multirow{2}{*}{$\nSLJ{3p}{2}{P}{3/2}$}	& \num{30925.1703(10)}				& \num{30925.1792(9)}		& \multirow{2}{*}{\cite{Radziemski1995.PRA.52.4462, Sansonetti2011.PRL.107.023001, Sansonetti2012.PRL.109.259901}}	\\
		(\num{646})				&							&												&											& (\num{15462.5851(5)})				& (\num{15462.5896(5)})		&                                                       										\\
		\num{671}				& \Li{6}					& $\nSLJF{2s}{2}{S}{1/2}{3/2}$					& $\nSLJ{2p}{2}{P}{3/2}$					& \num{14903.6320617+-0.0000007}	& \num{14903.63391(33)}		& \cite{Sansonetti2011.PRL.107.023001, Sansonetti2012.PRL.109.259901}											\\
		\num{689}				& \Sr{88}					& $\nSLJ{5s^2}{1}{S}{0}$						& $\nSLJ{5s5p}{3}{P}{1}$					& \num{14504.33824159+-0.00000033}	& \num{14504.34224(35)}		& \cite{Ferrari2003.PRL.91.243002, Sansonetti2010.JPCRD.39.033103}											\\
		\bottomrule
	\end{tabular}
	}
\end{table}

\Cref{fig:wm_fit} shows the difference between our measured energies and the reported energies. 
To check for slow drifts in the wavemeter (e.g., due to environmental changes), we also measured the wavelength of the \SI{689}{\nm} for each measurement of the \SI{638}{\nm} photon energy.
\Cref{fig:wm_drift} shows the deviation of the \SI{689}{\nm} measurements overtime, showing relatively little long-term drifts. 

\begin{figure}[htbp]
	\centering
	\includesvg[keepaspectratio, width=4in]{fermion_spectroscopy/wavemeter_calibration/wm_fit.svg}
	\caption{
		\label{fig:wm_fit}
		Linear fit to find wavemeter systematic offset.}
\end{figure}

\begin{figure}[htbp]
	\centering
	\includesvg[keepaspectratio, width=4in]{fermion_spectroscopy/wavemeter_calibration/wm_drift.svg}
	\caption{
		\label{fig:wm_drift}
		Long-term wavemeter drifts.}
\end{figure}

**Add footnote: at the time, the best reference for the \SI{323}{\nm} transition we could find was by \citeauthor{Radziemski1995.PRA.52.4462} although there should be some upcoming measurements using frequency combs (cite papers mentioning doing this)**

\subsection{Two-photon Rydberg excitation spectrum}

While taking the Rydberg spectra, we observed spurious electron counts when the UV laser detuning compensated for the \SI{689}{\nm} detuning. 
Due to this coincidence, we believe these ``ghost'' lines are due to atoms in the $\nSLJF{5s5p}{3}{P}{1}{9/2,11/2}$ state which were off-resonantly excited by a \SI{689}{\nm} photon before a \SI{319}{\nm} photon excites them to a Rydberg state. 
To check for this effect, we use the density matrix method **Lindblad operator** to simulate the time-dependent dynamics with parameter values estimated from the measured excitation laser powers and beam waists. 

**Show example spectra with ``ghost'' on-resonance counts.**

%\begin{figure}[htbp]
%	\centering
%	\includesvg[keepaspectratio, height=2in]{fermion_spectroscopy/three-level_system.svg}
%	\caption{\label{fig:three-level_system}Three-level diagram representing a two-photon excitation to a Rydberg state with $\ket{g}$, $\ket{e}$, and $\ket{r}$ representing the ground, intermediate, and Rydberg states, respectively. The detunings from the intermediate (Rydberg) state is $\Delta_e$ ($\Delta_r$). Couplings between states are represented by $\Omega_1 \exp(-\iu \omega_1 t)$ ($\Omega_2 \exp(-\iu \omega_2 t)$) for the \SI{689}{\nm} (\SI{319}{\nm}) lasers.}
%\end{figure}

The bare atomic Hamiltonian for this system can be expressed as 
\begin{equation}
	H_\text{A}
		= \omega_r \dyad{r} + \hbar \omega_e \dyad{e} + \hbar \omega_g \dyad{g}
		= \hbar	\begin{pmatrix}
					\omega_r	& 0			& 0	\\
					0			& \omega_e	& 0	\\
					0			& 0			& \omega_g
				\end{pmatrix}
\end{equation}
The atom-field interaction can be written as 
\begin{equation}
	H_\text{AF}
		= \frac{\hbar}{2}\left(\Omega_1 \operatorname{e}^{-\iu \omega_1 t} \dyad{g}{e} + \Omega_2 \operatorname{e}^{-\iu \omega_2 t} \dyad{e}{r} \right)+ \text{h.c}
		=	\frac{\hbar}{2}	\begin{pmatrix}
				0												& \Omega_2 \operatorname{e}^{-\iu \omega_2 t}		& 0	\\
				\Omega_2^\ast \operatorname{e}^{\iu \omega_2 t}	& 0													& \Omega_1 \operatorname{e}^{-\iu \omega_1 t}	\\
				0												& \Omega_1^\ast \operatorname{e}^{\iu \omega_1 t}	& 0
			\end{pmatrix}
\end{equation}
Using the unitary (?) transformation
\begin{equation}
	U	=	\begin{pmatrix}
				\operatorname{e}^{-\iu (\Omega_1 + \Omega_2) t}	& 0													& 0	\\
				0												& \operatorname{e}^{-\iu \Omega_1 t}					& 0	\\
				0												& 0													& 1
			\end{pmatrix}
\end{equation}
the total Hamiltonian $H = H_\text{A} + H_\text{AF}$ can be written as
\begin{equation}
	\widetilde{H}
		= U^\dag H U - \iu \hbar U^\dag \dv{U}{t}
		= \hbar	\begin{pmatrix}
					-\Delta_r						& \flatfrac{\Omega_2}{2}		& 0			\\
					\flatfrac{\Omega_2^\ast}{2}		& -\Delta_e						& \flatfrac{\Omega_1}{2}	\\
					0								& \flatfrac{\Omega_1^\ast}{2}	& \omega_g
				\end{pmatrix}
\end{equation}
where $\Delta_e = \omega_1 - \omega_e$ is the detuning from the intermediate state and $\Delta_r = \omega_r - (\omega_1 + \omega_2)$ is the total detuning from the Rydberg state. 
To simplify things, we can take $\omega_g = 0$.

**Hopefully the time-dynamics calculations explain the observed ``ghost'' lines**.

\begin{figure}[htbp]
	\centering
	\includesvg[keepaspectratio, width=4in]{fermion_spectroscopy/three-level_system/three-level_calculation.svg}
	\caption{\label{fig:three-level_calculation}Three-level system calculation.}
\end{figure}

\section{Results}

The measured state energies of the \Sr{87} Rydberg states are presented in **\cref{tab:data-s-state} and \cref{tab:data-d-state}** which include our best estimates of both statistical and systematic uncertainties.
Although our state energies have an uncertainty of about **xxx MHz**, we can measure splittings of up to a few GHz with kHz level accuracies limited by the synthesizer. 

\subsection{$S$-State Results}

Plotting our quantum defects compared to previously published data.

\begin{figure}[htbp]
	\centering
	\includesvg[keepaspectratio, width=\textwidth, height=\textheight]{fermion_spectroscopy/results/results-qd-1S0.svg}
	\caption{
		\label{fig:results-qd-1S0}
		Results for $\nSLJ{5sns}{1}{S}{0}$ quantum defects.}
\end{figure}

\begin{figure}[htbp]
	\centering
	\includesvg[keepaspectratio, width=\textwidth, height=\textheight]{fermion_spectroscopy/results/results-qd-3S1.svg}
	\caption{
		\label{fig:results-spectra-D}
		Results for $\nSLJ{5sns}{3}{S}{1}$ quantum defects.}
\end{figure}

\subsection{$D$-State Results}

\begin{figure}[htbp]
	\centering
	\includesvg[keepaspectratio, width=\textwidth, height=\textheight]{fermion_spectroscopy/results/spectra/spectra-n50.svg}
	\caption{
		\label{fig:results-qd-1D2}
		Spectra for $\nSLJ{5snd}{1,3}{D}{$J$}$ {$n=50, 60, 98$} showing the predicted and observed locations.
		The figures above are plotted relative to the $\nSLJF{5s50s}{3}{S}{1}{11/2}$, $\nSLJF{5s60s}{3}{S}{1}{11/2}$, and $\nSLJF{5s98s}{3}{S}{1}{11/2}$ states, respectively.}
\end{figure}

\begin{figure}[htbp]
	\centering
	\includesvg[keepaspectratio, width=\textwidth, height=\textheight]{fermion_spectroscopy/results/spectra/spectra-n60.svg}
	\caption{
		\label{fig:results-qd-1D2}
		Spectra for $\nSLJ{5snd}{1,3}{D}{$J$}$ {$n=50, 60, 98$} showing the predicted and observed locations.
		The figures above are plotted relative to the $\nSLJF{5s50s}{3}{S}{1}{11/2}$, $\nSLJF{5s60s}{3}{S}{1}{11/2}$, and $\nSLJF{5s98s}{3}{S}{1}{11/2}$ states, respectively.}
\end{figure}

\begin{figure}[htbp]
	\centering
	\includesvg[keepaspectratio, width=\textwidth, height=\textheight]{fermion_spectroscopy/results/spectra/spectra-n98.svg}
	\caption{
		\label{fig:results-qd-1D2}
		Spectra for $\nSLJ{5snd}{1,3}{D}{$J$}$ {$n=50, 60, 98$} showing the predicted and observed locations.
		The figures above are plotted relative to the $\nSLJF{5s50s}{3}{S}{1}{11/2}$, $\nSLJF{5s60s}{3}{S}{1}{11/2}$, and $\nSLJF{5s98s}{3}{S}{1}{11/2}$ states, respectively.}
\end{figure}


Plotting our quantum defects compared to previously published data.

\begin{figure}[htbp]
	\centering
	\includesvg[keepaspectratio, width=\textwidth, height=\textheight]{fermion_spectroscopy/results/results-qd-1D2.svg}
	\caption{
		\label{fig:results-qd-1D2}
		Results for $\nSLJ{5snd}{1}{D}{2}$ quantum defects.}
\end{figure}

\begin{figure}[htbp]
	\centering
	\includesvg[keepaspectratio, width=\textwidth, height=\textheight]{fermion_spectroscopy/results/results-qd-3D1.svg}
	\caption{
		\label{fig:results-qd-3D1}
		Results for $\nSLJ{5snd}{3}{D}{1}$ quantum defects.}
\end{figure}

\begin{figure}[htbp]
	\centering
	\includesvg[keepaspectratio, width=\textwidth, height=\textheight]{fermion_spectroscopy/results/results-qd-3D2.svg}
	\caption{
		\label{fig:results-qd-3D2}
		Results for $\nSLJ{5snd}{3}{D}{2}$ quantum defects.}
\end{figure}

\begin{figure}[htbp]
	\centering
	\includesvg[keepaspectratio, width=\textwidth, height=\textheight]{fermion_spectroscopy/results/results-qd-3D3.svg}
	\caption{
		\label{fig:results-qd-3D3}
		Results for $\nSLJ{5snd}{3}{D}{3}$ quantum defects.}
\end{figure}