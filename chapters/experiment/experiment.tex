\chapter{Strontium Rydberg Experiment}
\label{ch:experiment}

This chapter describes the Rydberg apparatus used in this work.
It provides documentation of the current state of various systems for producing ultracold and quantum degenerate gases of strontium. 
Being the third ultracold strontium at Rice, we were able to leverage lessons learned on the previous experiments (generally referred to as ``Neutral'' and ``Plasma'') to build build quite a capable and flexible system.

Although most of our system looks very similar to other ultracold strontium systems, what makes our system different is the inclusion of systems for creating, manipulating, and detecting charged particles. 
In particular, we have in-vacuum electric field plates to cancel stray electric fields as well as ramp to ionizing fields which direct electrons or ions towards the microchannel plate (MCP) detector.

Our old-school ``metal can'' setup can't compete with the optical access provided by many recently started alkaline-earth Rydberg tweezer systems using glass cells \cite{Norcia2018.PRX.8.041054, Cooper2018.PRX.8.041055, Saskin2019.PRL.122.143002}.
There's still a place for our system which likely provides capabilities for better shielding from stray electric fields, performing SFI, and the ability to detect charged particles which, without going to significantly custom systems, are somewhat limiting on the standard glass cell setups.

\section{Vacuum System}

An overview of our vacuum system is shown in \cref{fig:vacuum_system}. 
\begin{figure}[htbp]
	\centering
	\includesvg[keepaspectratio, width=\textwidth, height=\textheight]{experiment/vacuum/rydberg_vacuum_system.svg}
	\caption{
		\label{fig:vacuum_system}
		Overview of the strontium Rydberg apparatus highlighting the main sections of the vacuum system and various components.}
\end{figure}
We follow a conventional setup for producing cold and ultracold strontium gases (e.g., see \cite{Boyd2007.PhD, Stellmer2013.PhD, Barker2016.PhD}).
More detailed descriptions of each are provided in their respective sections below.
Perhaps the most noticeable aspect of our vacuum system is that it was designed to have the center of the main chamber at $\SI{3.5}{\inch}=\SI{8.89}{\cm}$ from the surface of the optical table.
Keeping the working height fixed at \SI{3.5}{\inch} significantly reduced the need to build rigid supporting structures to support both the vacuum system above the table and various elevated platforms for optics.
This also reduced the need for periscopes for getting the laser beams to the height of an elevated vacuum chamber.
Fiber-coupling all the laser beams is a potential alternative to avoiding periscopes but that includes some unavoidable loss in power and some wavelengths (i.e., the \SI{320}{\nm} UV laser) would be difficult fiber couple at the present time.

To facilitate vertically propagating beams (e.g., MOT and dipole trap beams), a \SI{10}{\inch} hole was cut through the table under the main chamber.
We had originally planned on mounting vertical optics directly to the main chamber but decided to mount them to a single elevated platform above the main chamber and the underside of the table instead. 
This should reduce sensitivity to vibrations as the main chamber is (alomst) ``free-floating'' from the optics on the table.
Adding a breadboard under the table required tapping {1/4''-20} threads on the underside of the table so that we could mount an optical breadboard.
The major disadvantage of our current setup is the long distance between the undertable breadboard and the top of the main chamber which is about \SI{12}{\inch}, i.e., the thickness of the table, which makes it difficult to design upward-propgating dipole trap beams so we prefer to send beams downward instead.

\subsection{Source-side Vacuum System}

This section is where strontium is converted from its solid metallic state in to a gaseous form. 
Due to strontium's extremely low vapor pressure, an oven is needed to increase the vapor pressure enough to form an atomic beam.
Although there are now various designs for strontium ovens \cite{Senaratne2015.RSI.86.023105, Schioppo2012.RSI.83.103101}, and even commercial systems from AO Sense, we used a design which has shown success in the Killian lab. 
It's a simple design with a Watlow FIREROD cartridge heater\footnote{These are neat little devices which come with leads for both the resistive heating element and an integrated thermocouple.} providing most of the heating to the atom oven.
Capillary tubes are used to provide some degree of collimation, forming a nozzle, and a heater wire is wrapped around the nozzle to reduce the effects of strontium building up in the capillary tubes and clogging the oven. 
We typically run our main FIREROD oven heater around \SI{425}{\celsius} with the nozzle heater wire around \SI{390}{\celsius}. Note that the FIREROD temperature sensor is inside the ceramic body whereas the nozzle thermocouple is attached to the nozzle itself (i.e., the temperature at the atoms is likely lower). 
\Cref{fig:neutral_sr_oven} shows a similar oven made to replace the Neutral's oven.
The design is very similar to the one on our setup but it was mounted on the much smaller \SI{2.75}{\inch} flange whereas ours is mounted on a \SI{3.375}{\inch} flange which provides some extra room for various feedthroughs without being too cramped.

\begin{figure}[htbp]
	\centering
	\includesvg[keepaspectratio, width=\textwidth, height=3in]{experiment/atomic_oven/neutral_sr_oven.svg}
	\caption{
		\label{fig:neutral_sr_oven}
		Neutral's new strontium oven during construction before thermocouple and nozzle heating wire have been attached. 
		The design is very similar to our oven but is mounted oo a 2-3/4 on CF flange instead of a 3-3/8 in CF flange as on our experiment.
		(Left) Oven without heat shield.
		(Right) After adding heat shield.}
\end{figure}

After atoms exit the oven, they pass through a two-dimensional (2D) collimator which applies transverse optical molasses \cite{Sheehy1990.CP.145.317, Joffe1993.JOSAB.10.2257} to increase the flux of atoms down the Zeeman slower. 
We've observed an increase in blue MOT atom number by about eight times when the 2D collimator is well aligned to the beam.
The long arms of the 2D collimator help prevent strontium buildup on the AR coated viewports.

Before the atoms leave this section of the vacuum system and enter the Zeeman slower, the pass through a small differential pumping tube ($C \approx \SI{0.85}{\liter\per\second}$) which helps maintain the pressure differential between the source and experiment sides of the system. 
The pressure in the source side is typically higher than the experiment side and is around \SI{7.6E-9}{\Torr}\footnote{MiniVac ion pump controller outputting $\SI{7.9}{\mV}=\SI{7.9E-6}{\A}$ for the  rebuilt VacIon Plus 75 Starcell on 2019/06/13.}.
We suspect the ion pump on the source side is being choked off by the zero-length reducer which adapts the \SI{4.5}{\inch} flange on the source-side vacuum system to the \SI{6}{\inch} flange on the Varian ion pump.

A shutter and a gate valve was also installed between the source-side vacuum and the Zeeman slower. 
The shutter allows us to physically block the atomic beam (as we don't have 2D deflection MOT) and extend the time between needing to change the Zeeman slower entrance window. 
The gate valve, in principle, should allow us to isolate the source-side vacuum system from the rest of the chamber but we found it to be somewhat leaky when we initially pumped out our system\footnote{Something similar was noticed by the Weld group at UCSB and they used two gate valves to separate their atomic source from their UHV main chamber \cite{Senaratne2018.PhD}.}.

\subsection{Zeeman Slower}

Since an oven is required to produce enough strontium vapor for trapping, a Zeeman slower is required to slow the atoms down so that they can be captured by the MOT. 
The Zeeman slower was designed and built by Francisco Camargo and is detailed in his master's thesis \cite{Camargo2015.MS} so it will only be briefly covered here. 
A Zeeman slower uses a spatially-varying magnetic field and circularly-polarized light to keep the atoms resonant with a counterpropagating light so that they continuiously scatter photons as they slow \cite{Phillips1982.PRL.48.596}.
Our particular design is a ``spin-flip'' Zeeman slower where the amplitude of the axial magnetic field crosses zero between the ends.
Similar designs are presented in \cite{Bell2010.RSI.81.013105, Hopkins2016.RSI.87.043109}.

Our Zeeman slower was custom made with a double wall so that cooling water can be run between the inner wall (which separates the water from UHV) and the outer wall where the magnetic field coils are wrapped. 
Due to space constraints, the source-side has a \SI{2.75}{\inch} CF flange whereas it's connected to the main chamber by a \SI{2.125}{\inch} CF flange.
These flanges are also where cooling water enters and exits the jacket.
It's possible to avoid water-cooling by building a permanent magnet Zeeman slower \cite{Lebedev2014.JPB.47.155003, Cheiney2011.RSI.82.063115, Reinaudi2012.JOSAB.29.729, Hill2014.JPB.47.075006} but the magnetic field would likely require more precise cancellation as it cannot be turned off.

The Zeeman slower was constructed in multiple layers, tweaking subsequent layers so the resulting field was tweaked to match the desired profile.
Each layer being attached to the Zeeman slower tube with thermally-conductive epoxy\footnote{Emerson and Cuming STYCAST 2762 or STYCAST 2762 FT with Catalyst 14, 17, or 17M-1.}.
Due to our relatively long Zeeman slower  ($C \approx \SI{1.4}{\liter\per\second}$), it should provide some additional differential pumping between the source and experiment sides.

\subsection{Experiment-side Vacuum System}

The experiment-side vacuum system, as the name suggests, is where the experiments occur and consists of the main chamber, the ``pumping tower'', and the Zeeman slower entry viewport. 
Francisco designed the main chamber\footnote{Fabricated by Huntington Mechanical Labs.} and features:
\begin{itemize}
	\item Recessed top and bottom flanges --
		The recessed flanges move the top and bottom \SI{2.75}{\inch} viewports closer to the center of the chamber while providing room for the MOT coils to fit around the viewport.
		The in-vacuum electric field plates are mounted to the bottom flange which has eleven \SI{10}{\kV} SHV feedthroughs for connections.
	\item Six \SI{2.75}{\inch} horizontal viewports --
		These viewports procide optical access for the various MOT and optical dipole trap beams.
	\item A **tube** to house the MCP --
		The MCP fits inside a ** xxx inch ** tube.
	\item Three \SI{1.33}{\inch} horizontal viewports --
		The mini \SI{1.33}{\inch} viewports are used for the various \SI{689}{\nm} and \SI{320}{\nm} excitations beams and are oriented perpendicular or parallel to the MCP.
	\item A conical expansion to a \SI{6}{\inch} flange --
		The conical expansion to a \SI{6}{\inch} flange allows for increased conductance to the pumps.
		The Zeeman slower beam also passes through this flange. 
\end{itemize}
** The \SI{2.75}{\inch} viewports were antireflection (AR) coated for \SI{461}{\nm}, \SI{689}{\nm}, and \SI{1064}{\nm} by Reynard Corporation. **

Vacuum pressure is maintained by a \SI{75}{\liter\per\second} Gamma Vacuum 75S-CVX-6S-SC-N-N ion pump located at the top of the pumping tower.
The ion pump current suggests the vacuum on this side is about \SI{3.4E-9}{\Torr}\footnote{MiniVac ion pump controller outputting $\SI{-4.8}{\mV}=\SI{4.8E-6}{\A}$ for the Gamma Vacuum TiTan 75S on 2019/06/13.}.
Additional pumping is provided by a non-evaporable getter (NEG)\footnote{SAES CapaciTorr D 200.} located about halfway up the pumping tower. 
In retrospect, we should have also included a titanium sublimation pump (TSP) for additional pumping capability.

We also added a gate valve between the Zeeman slower window and main chamber to allow us to change the viewport once it gets coated\footnote{The lab has tried using Plasma's pulsed Nd:YAG laser to perform ablation on Neutral's coated Zeeman window with moderate levels of success.}.

\section{Magnetic Field Trim Coils}

The trim coils are used to cancel out stray magnetic fields as well as apply bias fields. 
The two horizontal pairs of rectangular coils are oriented such that the ``X-axis'' produce a field towards or away from the MCP with the ``Y-axis'' perpendicular. 
The ``Z-axis'' coil pair produces a field perpendicular to the table surface. 

Due to the choice of keeping the center of the chamber at \SI{3.5}{\inch} above the table surface, this required the horizontal (X-axis and Y-axis) trim coils to be centered at this height. 

\begin{figure}[htbp]
	\centering
	\includesvg[keepaspectratio, width=\textwidth, height=\textheight]{experiment/trim_coils/trim_coils.svg}
	\caption{
		\label{fig:trim_coils}
		Magnetic field trim coils around main chamber. (Blue) X-axis, (green) Y-axis, and (red) Z-axis.}
\end{figure}

The vertical (Z-axis) coils were would on forms around the outside of the top and bottom flanges of the main chamber which allowed us to put many turns on them. 
The circular U-channels used as the forms also feature a radial slice, filled with epoxy, to mitigate the generation of eddy currents. 
In retrospect, we should made the horizontal coils beefier (especially along the X-axis) in order generate larger bias fields along the MCP axis and increase detection efficiency. 
We found that having bias fields along other directions (i.e., not towards/away from the MCP) reduced our detection efficiency, likely due to the Lorentz force steering electrons away from the MCP. 

\section{Dual MOT Coils}
\label{sec:dual_mot_coils}

One of the things we did differently from the previous strontium experiments is that we have two pairs of MOT coils: the larger pair for the blue MOT and the smaller pair for the red MOT.
This allows us to use two different current supplies without needing a controller which is has the dynamic range to provide the \SI{\sim 40}{\A} to generate the \SI{\sim 40}{\gauss\per\cm} field for the blue MOT while also generating the \SI{\sim 1}{\gauss\per\cm} low-noise field for the red MOT.

\begin{figure}[htbp]
	\includesvg[keepaspectratio, width=\textwidth, height=\textheight]{experiment/mot_coils/dual_mot_coils.svg}
	\caption{\label{fig:dual_mot_coils}
		(Left) Picture of our assembled dual MOT coils on a test setup. 
		(Right) CAD rendering of how they are centered around the buket windows of the main chamber.
		** Redo with 2D CAD drawing instead of rendered image. **}
\end{figure}

The blue MOT coils are made from xxx\footnote{S{\&}W Wire 125SQ DPG/BARE (NEMA 46-C).} hollow inside with approximately ** ** outer dimensions with a hollow ** ** through the middle for water. 
We typically run the blue MOT coils with around ** 40-50 amps ** of current from a fixed power supply ** GIVE MODEL **. 
A ** MOSFET ** is used to quickly switch off the current to the coils. 
** We so far haven't included a flyback diode although the MOSFET we use seems to include an internal flyback diode. **
These coils produce a magnetic field gradient of about \SI{1.25}{\gauss\per\cm\per\ampere}.

The red MOT coils are constructed from square {\num{13} AWG} copper wire\footnote{MWS Wire Industries 13 SQ HML-240 (NEMA MW 20-C).} and are epoxied to the undersides of the blue MOT coils, placing them closer to the center of the vacuum chamber. 
We currently use a homebuilt voltage-controlled current driver\footnote{**Powered by an APEX PA12**.} but we should be able to use any current driver capable of supplying up to about ** 5 A **. 
Our current geometry for the red MOT coils produce a magnetic field gradient of about \SI{1}{\gauss\per\cm\per\ampere}.

**
It would have been nice to have an extra MOSFET switch on the blue MOT coils so that we'd be able to apply extremely large fields along the Z-axis such as e.g. Pedro's experiment Feshbach coils.
We're not too worried about not having this capability since ground-state strontium has no magnetic Feshbach resonances and the fact that we've observed a decrease MCP detection efficiencly we believe is due to the Lorentz force as the MCP is oriented along the X-axis.
**

\section{Laser Systems for Producing Cold and Ultracold Strontium Gases}

We follow the standard procedure for laser cooling strontium so the procedure will only be outlined here with more details are available from various sources (e.g., \cite{Boyd2007.PhD, Stellmer2013.PRA.87.013611, Mickelson2010.PhD, Martinez2012.PhD, Stellmer2013.PhD, Barker2016.PhD}).
\Cref{fig:trapping_beams} gives an overview of the orientation of our various laser cooling and trapping beams.
\begin{sidewaysfigure}[htbp]
	\includesvg[keepaspectratio, width=\paperwidth, height=\paperheight]{experiment/laser_beams/trapping_beams.svg}
	\caption{\label{fig:trapping_beams}
		Simplified schematic of the various trapping and cooling lasers and the Rydberg excitation beams.}
\end{sidewaysfigure}

As mentioned above, strontium's low vapor pressure requires an oven to liberate atoms from the solid. 
With our oven typically operating around $\SIrange[range-phrase=-]{400}{500}{\celsius}$, the atoms in the beam are moving at about $\overline{v} = \SIrange[range-phrase=-]{474}{508}{\meter\per\second}$\footnote{For an atomic beam at $T = \SIrange[range-phrase=-]{400}{500}{\celsius}$: $v_{\text{mp}} = \SIrange[range-phrase=-]{437}{468}{\meter\per\second}$, $v_{\text{avg}} = \SIrange[range-phrase=-]{474}{508}{\meter\per\second}$, and $v_{\text{rms}} = \SIrange[range-phrase=-]{504}{541}{\meter\per\second}$ \cite{Metcalf1999.LCT}.}.
Therefore the strong ($\Gamma/2\pi = \SI{30.5}{\MHz}$) dipole-allowed $\nSLJ{5s^2}{1}{S}{0} \rightarrow \nSLJ{5s5p}{1}{P}{1}$ transition at \SI{461}{\nm} is used for both the 2D collimator and Zeeman slower.
The 2D collimator has light at $\Gamma/2$ \cite{Lett1989.JOSAB.6.2084}. 
** ZEEMAN SLOWER FREQUENCY? **

% Blue MOT
Once the atoms reach exit the Zeeman slower and enter the main chamber, a standard six-beam \SI{461}{\nm} ``blue'' MOT captures and cools the atoms to about $\SIrange[range-phrase=-]{1}{2}{\milli\kelvin}$.
The quadrupole magnetic field gradient are produced by running by the larger, water-cooled coils described in \cref{sec:dual_mot_coils}.
The Zeeman slower beam passes through the blue MOT which leads to some distortion of the MOT shape and shifting of it's position compared to when the Zeeman slower light is extinguished.
** MOT FREQUENCY? **

% Magnetic trap
The \SI{461}{\nm} transition is not completely closed with about $\num{1}\colon\num{50000}$ decays following the $\nSLJ{5s5p}{1}{P}{1} \rightarrow \nSLJ{5s4d}{1}{D}{2}$ decay path.
About $\flatfrac{2}{3}$ then decay to the $\nSLJ{5s5p}{3}{P}{1}$ which then decays to the $\nSLJ{5s^2}{1}{S}{0}$ and return to the \SI{461}{\nm} cooling cycle.
The remaining $\flatfrac{1}{3}$ decay to the long-lived metastable $\nSLJ{5s5p}{3}{P}{2}$ states with lifetimes of about \SI{520}{\second} \cite{Yasuda2004.PRL.92.153004}, effectively removing them from the blue MOT cooling cycle.
Of the $\nSLJ{5s5p}{3}{P}{2}$ states, the low-field seeing ${m_J = 1,2}$ states can become trapped in the quadrupole magnetic field of the blue MOT.
This decay path, initially seen as a loss from the blue MOT, ends up being a powerful tool for accumulating atoms in metastable reservoir \cite{Nagel2003.PRA.67.011401} at roughly the same temperature as the blue MOT. 
The magnetic trap was an essential tool in overcoming the abysmal \SI{0.56(2)}{\percent} natural abundance of \Sr{84} to produce the first strontium BECs \cite{Stellmer2009.PRL.103.200401, Martinez2009.PRL.103.200402}.
** For details of our magnetic trap, see \cite{Camargo2015.MS}. **
The magnetic trap is also important when loading multiple isotopes where we can store one isotope in the magnetic trap, move the \SI{461}{\nm} laser to trap another isotope.

% Repumping
Atoms in the magnetic trap are recovered by repumping on the $\nSLJ{5s5p}{3}{P}{2} \rightarrow \nSLJ{5p^2}{3}{P}{2}$ transition at \SI{481}{\nm}.

% Red MOT
We further cool the sample driving the narrow ($\Gamma/2\pi = \SI{7.5}{\kHz}$) $\nSLJ{5s^2}{1}{S}{0} \rightarrow \nSLJ{5s5p}{3}{P}{1}$ in the red MOT.
Unlike ytterbium which in which a Zeeman-slowed atomic beam is capable of being loaded directly in to a narrow line MOT \cite{Guttridge2016.JPB.49.145006}, strontium's linewidth is a bit too weak.
As a result, the blue MOT is necessary for precooling the atoms for the red MOT.
After laser cooling on the \SI{689}{\nm} transition, our sample temperatures are typically $\sim\SIrange[range-phrase=-]{1}{2}{\micro\kelvin}$ and ready for loading in to an optical dipole trap (ODT).

% ODT
The ODT rely on the light-shift produced by far-off-resonance beams and are conservative as a result.
Being far-off-resonance means the heating rate is greatly suppressed.
Our system uses a high-power single-mode \SI{1064}{\nm} fiber amplifier which amplifies the the power from a narrow line seed laser \footnote{We are currently borrowing the Hulet lab's multimode YLR-100-1070-LP as our laser was undergoing repair service.} which are readily available.

\subsection{\SI{461}{\nm} ``Blue'' Laser System}

We currently have three blue diode-based lasers which generate all the \SI{461}{\nm} power available for our experiment.
A single external cavity diode laser (ECDL) ``master'' laser\footnote{New Focus TLB-6802: a Littman-Metcalf external cavity diode laser (ECDL).} outputs about \SI{40}{\mW} and is locked to a homebuilt strontium saturated absorption cell.
Although this laser tunes relatively well and remains single-mode when tuning between the various isotopes, we have very disappointed with output quality of the laser which has what appears to be a {Hermite-Gaussian TEM\tsub{00}} output mode **see Fig. XX). 
** INCLUDE BEAM PROFILER FIGURE **
Due to this poor output profile, used a D-mirror to split the beam in to two nearly-Gaussian modes and directed each half to a slave laser for injection locking.
After using the laser for about six years, the piezo tuning element died\footnote{The laser died after powering down and keying off the laser driver...} and the laser had to be sent back to New Focus for repairs.
During the repairs, we changed our setup so that the two slave lasers are injection locked from the output of a fiber so that we can easily change master lasers should this laser die again.
We also noticed some weird grounding issues with the TLB-6700 drivers New Focus sent us with the lasers\footnote{Similar issues were noticed by Jun Ye's group when I asked them if they have observed issues with the New Focus lasers.}.
Overall, we were not satisfied with New Focus' ECDL and their repair process and costs so much so that they only reduced the quoted repair cost\footnote{The piezo repair was going to cost 7500 with an additional 4000 for a new diode...} after we were ready to purchase a new \SI{461}{\nm} ECDL from either Toptica or MOGLabs\footnote{We ended up purchasing a \SI{461}{\nm} ECDL from Toptica but both companies provided quotes for NEW lasers, about 21k from MOGLabs and about 25k from Toptica but with about \SI{100}{\mW} of output power.}.

\begin{sidewaysfigure}[htbp]
	\centering
	\includesvg[keepaspectratio, width=\paperwidth, height=\paperheight]{experiment/blue_system/blue_system.svg}
	\caption{
		\label{fig:blue_system}
		Simplified diagram of the \SI{461}{\nm} laser system.}
\end{sidewaysfigure}

To increase the amount of \SI{461}{\nm} power available, we use a master-slave\cite{Shimada2013.RSI.84.063101, Hosoya2015.RSI.86.073110} setups where the master laser injection locks two slave lasers\footnote{New Focus TLB-6802-IJ-D: a gratingless version of the TLB-6802 likely with a higher-power diode as well.}.
We have previously tried homebuilding our own \SI{461}{\nm} slave laser with can-opened blue diode\footnote{Nichia NDB4216E - a non-AR coated diode with about \SI{100}{\mW} of output power.} but found it to be unsatisfactory. 
We ended up going with two New Focus TLB-6802-IJ\footnote{From what we can tell, these are simply their standard TLB-6802 without the grating and probably using the Nichia NDB4216E non-AR coated diode which is specified to output about \SI{100}{\mW}.} both of which output about \SI{100}{\mW} of power.
The ``MOT'' slave is injection locked at the same frequency as the master laser and its power is split between the blue MOT, 2D collimator, and the imaging beam.
The injection beam for the ``Zeeman'' slave is first shifted by \SI{-535}{\MHz} relative to the master laser before injection locking the laser. This allows all the power out from the slave to be used for operating the Zeeman slower instead of taking a hit going through an AOM.

\begin{figure}[htbp]
	\includesvg[keepaspectratio, width=\textwidth, height=\textheight]{experiment/blue_system/old_blue_MOT.svg}
	\caption{\label{fig:old_blue_MOT}
		An early picture of our \Sr{88} blue MOT taken on 2014/08/29 through a \SI{2.75}{\inch} viewport with the atom source and Zeeman slower are off to the right.
		Considering the electric field plates are separated by about \SI{1}{\inch}, this blue MOT is likely has a diameter of about \SI{0.5}{\inch}. 
		A video is available at {https://youtu.be/ENDIizrlqMA}.}
\end{figure}

The entire \SI{461}{\nm} system is housed at far end of our table from the vacuum system and boxed in an opaque enclosure to contain the blue light, even a little of which can lead to heating our samples. 
Mechanical shutters\footnote{We use hard drive shutters as we've found them to be the most reliable.} are used to provide \SI{100}{\percent} extinction.
** Some details about our blue cooling system can be found in \cite{Camargo2015.MS, Ding2016.MS}. **

\subsection{\SI{481}{\nm} Repumping System}

Before moving on to the narrow line ``red'' MOT, the atoms in the metastable resevoir are dark to both the \SI{461}{\nm} and \SI{689}{\nm} cooling light and must be brought back to the ground state.
Several transitions have been explored for repumping atoms out of the metastable resevoir over the years:
** I don't know if I want to keep the list below since it's very similar to one in Simon Stellmer's thesis **
\begin{itemize}
	\item {$\nSLJ{5s6s}{3}{S}{1}$}, $\lambda = \SI{707}{\nm}$ --
		this is a popular transition and is used by a lot of past and current strontium experiments \cite{Courtillot2005.EPJD.33.161, Boyd2007.PhD, Ludlow2008.PhD, Barker2015.PRA.92.043418, Campbell2017.PhD, Cooper2018.PRX.8.041055, Norcia2018.PRX.8.041054} since it's well characterized and accessible with easily available laser diodes.
		The drawback is that this upper state is $J=1$ meaning $\nSLJ{5s6s}{3}{S}{1}$ can decay to the metastable $\nSLJ{5s5p}{3}{P}{0}$ state as well, requiring another laser at \SI{679}{\nm} to repump out of this state. 
	\item {$\nSLJ{5s4d}{3}{D}{2}$} --
		at \SI{3011.84}{\nm} driving the $\nSLJ{5s5p}{3}{P}{2} \rightarrow \nSLJ{5s4d}{3}{D}{2}$ \cite{Mickelson2009.JPB.42.235001} transition.
		Being a $J=2$ state means a single laser can be used to clear out the $\nSLJ{5s5p}{3}{P}{2}$ reservoir but the \SI{3}{\um} wavelength is difficult to produce.
	\item {$\nSLJ{5s5d}{3}{D}{2}$} --
		at \SI{496.93}{\nm} driving the $\nSLJ{5s5p}{3}{P}{2} \rightarrow \nSLJ{5s5d}{3}{D}{2}$ transition \cite{Stellmer2013.PhD, Stellmer2014.PRA.90.022512, Moriya2018.JPC.2.125008}.
		The change of using this transition is the difficulty in producing the green repumping light.
	\item {$\nSLJ{5s6d}{3}{D}{1,2}$} --
		at \SI{403.35}{\nm} driving the $\nSLJ{5s5p}{3}{P}{2} \rightarrow \nSLJ{5s6d}{3}{D}{2}$ transition \cite{Stellmer2014.PRA.90.022512, Moriya2018.JPC.2.125008}.
		Similar attributes as using the $\nSLJ{5s5d}{3}{D}{2}$ with the added advantage of being accessible with Blu-ray laser diodes.
	\item {$\nSLJ{5p^2}{3}{P}{1,2}$} --
		at \SI{481.323}{\nm} driving the $\nSLJ{5s5p}{3}{P}{2} \rightarrow \nSLJ{5p^2}{3}{P}{2}$ transition \cite{Wongwaitayakornkul2013.BS, Camargo2015.MS, Ding2016.MS, Couturier2018.RSI.89.043103, Hu2019.PRA.99.033422}.
\end{itemize}

Our current \SI{481}{\nm} repumper system is shown in \cref{fig:repumper_system}. 
The \SI{481}{\nm} light is provided by a Toptica DL 100\footnote{The laser is equipped with a {LD-0488-0060-1} diode.} which outputs about ** \SI{10}{\mW} ** of power. 
For now, this laser is stabilized to a Doppler-broadened \Te{130}{2} line \cite{Wongwaitayakornkul2013.BS} which has a transition at \SI{20776.0886}{\per\cm} \cite{Cariou1980.Te2_Atlas} which is close to the $\nSLJ{5s5p}{3}{P}{2} \rightarrow \nSLJ{5p^2}{3}{P}{2}$ transition at \SI{20776.087}{\per\cm} \cite{Sansonetti2010.JPCRD.39.033103}.
Originally, when the lab was working with a single isotope, simply tuning the lock point to maximize repumping efficiency of the isotope of interest was good enough (typically \Sr{84} since it's the least abundant).
Now that we routinely work with multiple isotopes and mixtures, a free-space electro-optic modulator (EOM) was introducted to to add sidebands to the \SI{481}{\nm} light in order to address multiple isotopes and the hyperfine shift of \Sr{87}.
The hyperfine shift of the lower $\nSLJ{5s5p}{3}{P}{2}$ state is known \cite{Heider1977.PRA.16.1371} and cover the hyperfine shift of the upper $\nSLJ{5p^2}{3}{P}{2}$ doubly-excited is expected to be small **because of the lack of an {$s$-electron}\footnote{I.e., the magnetic dipole hyperfine contribution comes from the $\delta(r)$....} CHECK **.

\begin{figure}[htbp]
	\centering
	\includesvg[keepaspectratio, width=\textwidth, height=\textheight]{experiment/repumper_system/repumper_system.svg}
	\caption{
		\label{fig:repumper_system}
		Simplified diagram of our current \SI{481}{\nm} repumper laser system.
		A portion of the light form the \SI{481}{\nm} external cavity diode laser (ECDL) is used to lock to a Doppler-broadened \Te{130}{2} line in a cell heated to about \SI{555}{\celsius}.
		The rest of the light is sent through a free-space electro-optic modulator (EOM) at about \SI{565}{\MHz} which is used to apply sidebands to the light in order to address the various isotopes and hyperfine states.
		\SI{481}{\nm} light is delivered to the various experiments by multimode fibers.}
\end{figure}

In principle, it should be possible to lock to a hollow cathode lamp (HCL) due to collisions populating enough metastable atoms to perform spectroscopy on \cite{Norcia2016.RSI.87.023110}.
Instead of using an HCL, we plan on using a ``super lock'' \cite{Lindsay1991.RSI.62.1656, Jaffe1993.RSI.64.2475, Zhao1998.RSI.69.3737, Subhankar2019.RSI.90.043115} (see Appendix xxxx for future upgrade plans) since we already have a stable \SI{689}{\nm} reference laser (although we could also lock to a stabilized HeNe).
We also considered a lock to the wavemeter as in \cite{Couturier2018.RSI.89.043103} but ultimately decided against it as our wavemeter's accuracy is lacking, its sampling rate a bit too slow, and we needed the wavemeter for other tasks (e.g., finding Rydberg lines). 

\subsection{\SI{689}{\nm} ``Red'' Laser System}

Some details about our narrow line cooling system can be found in \cite{Ding2016.MS}.
The operation of the narrow line \SI{689}{\nm} ``red'' MOT can be found in \cite{Stellmer2013.PhD, Boyd2007.PhD, Ding2016.MS, Barker2016.PhD, Campbell2017.PhD}. 
Details of the \Sr{87} red MOT is particularly well described in \cite{Stellmer2013.PhD} and will not be reproduced here. 

\subsubsection{\SI{689}{\nm} Master Laser}

A single \SI{689}{\nm} Toptica DL pro ECDL serves as our master laser and is shared between the Rydberg and Neutral experiments. 
The laser is first locked to a homebuilt high-finesse Fabry-P{\'{e}}rot (FP) cavity\footnote{$F \gtrsim \num{2040}$ . Additional details can be found in \cite{Martinez2012.PhD}.} which is then locked to a heated strontium saturated absorption cell.
Since the $\SLJ{1}{S}{0} \rightarrow \SLJ{3}{P}{1}$ is very weak, the cell is relatively long to allow enough absorption to occur.
For most of the experiments described in this thesis, the red master laser was found to have a linewidth of about \SI{30}{\kHz}.
** More details of the \SI{689}{\nm} master laser system is provided in Jim's Ph.D. thesis. **
We recently purchased a ULE cavity from Stable Laser Systems (SLS). 
Some additional details in \cref{ssec:ULE_cavity}.

Due to the ease of building \SI{689}{\nm} slave lasers \cite{Ding2016.MS}, we only need a single \SI{689}{\nm} stabilized master laser system to run both the Rydberg and Neutral experiments. 
From the red master table, two fibers run to each experiment with light at the following frequencies:
\begin{itemize}
	\item $f_\text{master}$ --
		This light is locked to \SI{-82}{\MHz} of the $\nSLJ{5s^2}{1}{S}{0} \rightarrow \nSLJ{5s5p}{3}{P}{1}$ transition in \Sr{88}. 
	\item $f_\text{master} - \SI{1440.440}{\MHz}$ --
		This light is generated by passing the light directly out of the \SI{689}{\nm} master laser though a freespace gigahertz AOM before being sent to fibers to the Rydberg and Neutral experiments (instead of each experiment operating inefficient gigahertz AOMs to generate their own light for trapping \Sr{87}).
		This light is used to injection lock slave lasers for proudicing the \Sr{87} ``trap'' red MOT light.
\end{itemize}

We currently do not implement a fiber phase noise cancellation system \cite{Ma1994.OL.19.1777, Rauf2018.RSI.89.033103} so it's possible that the light at the slave lasers is broadened to ** kilohertz-levels ** over the \SI{25}{\meter} fibers delivering master light to the Rydberg experiment but it shouldn't be too difficult to implement in the future when necessary. 

\subsubsection{Rydberg Red Laser System}

Once the red light arrives on the Rydberg table, typically only about \SI{1}{\mW} is available so we use the master light to injection lock slave lasers to increase the available \SI{689}{\nm} power. 
\SI{689}{\nm} slave lasers are relatively easy to build and are detailed in \cite{Ding2016.MS}.
We currently use a rejected beam from the slave laser isolator to monitor the mode of the slave laser. 
In the future, I would instead use a stray reflection or beam sampler after the isolator as the current arrangement could lead to a small amount of light from the spectrum analyzer cavity getting back to the slave lasers (especially since we share the spectrum analyzer cavity with multiple slaves), possibly leading to instability issues. 

\Cref{fig:red_system} shows what the red MOT system looks like when set up for trapping any combination of isotopes. 
In practice, we don't always have all the red MOT AOMs lined up, typically only keeping the AOMs necessary for trapping the isotope(s) for a particular experiment while the other red MOT AOMs are used to generate various other beams (e.g., spectroscopy, ``blow away'', etc.).
\begin{sidewaysfigure}[htbp]
	\centering
	\includesvg[keepaspectratio, width=\paperwidth, height=\paperheight]{experiment/red_system/red_system.svg}
	\caption{\label{fig:red_system}Simplified diagram of the red MOT system (when everything is put together) for multi-isotope trapping. The master laser frequency is referenced to the $\nSLJ{5s^2}{1}{S}{0} \rightarrow \nSLJ{5s5p}{3}{P}{1}$ in \Sr{88}.}
\end{sidewaysfigure}
Without using a fiber combiner, we're able to devote a lot of power to the red MOT for each isotope, enabling us to use large red MOT beams. 
The drawbacks is that we're much more sensitive to misalignments and the ugly intensity profiles out of the diodes leads to strange red MOT shapes. 

We currently have three slave lasers which amplify the light from the red master. 
We found that fiber-coupling the injection beams made them much less sensitive to mechanical drifts and greatly increased the ease of switching the frequency of the slave laser by changing the source of the injection beam.

****
Image of \Sr{84} and \Sr{87} red MOTs.
****

Since the red MOT is only operated for a few hundred milliseconds, the zeroth-order from the red MOT AOMs also provide the light for various other purposes (e.g., spectroscopy, loss, etc.). 

**
Mention SWAP cooling? \cite{Norcia2018.NJP.20.023021, Bartolotta2018.PRA.98.023404, Muniz2018.arXiv.1806.00838, Snigirev2019.arXiv.1903.06435}.
**


\subsection{Optical Dipole Trap}

Strontium in the ground $\nSLJ{5s^2}{1}{S}{0}$ state have a negligible magnetic moment meaning we cannot magnetically trap ground state atoms necessitating the use of optical traps. 
Our current high-power \SI{1064}{\nm} laser system is powered by a Nufern NuAMP SUB-1382 which amplifies a few milliwatts of \SI{1064}{\nm} seed light to about \SI{50}{\W}.
There are apparently ``tricks'' for making these lasers operate very quietly as used in quantum gas microscopes \cite{Mazurenko2017.PhD, Mazurenko2019.RSI.90.033101} but we have not done such modifications. 
Due to the high IR powers, we used fused silica optics whenever possible although we still have BK7 Brewster plates in the Thorlabs IO-5-1064-VHP isolator \cite{Mazurenko2017.PhD, Mazurenko2019.RSI.90.033101}.

We have used both free-space and fibered optical dipole traps but due to having to replace and realign the high-power IR lasers multiple times, we ended up sticking with the fibered systems due to them not losing alignment.
The drawback to the fibered ODT systems is that it's difficult to couple more than about \SI{10}{\W} through continuously without thermal effects degrading the coupling efficiency.
We get around this issue by only having the ODT on at high powers during the transfer from the red MOT before quickly moving on to the evaporation stage. 

We started with a free-spaced crossed-beam optical dipole trap (XODT) with waists of about $\SI{60}{\um} \times \SI{60}{\um}$. 
This trap was a bit tight but worked well for making \num{100000} BECs of \Sr{84}. 
We found that loading was improved if the trap was artificially widened by driving the AOM with three slightly different frequencies before turning the sideband frequencies down and evaporating \cite{Camargo2017.PhD}.
After going through three ODT lasers, we stuck with the sheet trap and vertical crossing trap because they were fiber coupled. 
Having the fiber-coulped sheet trap saved our ass as we've had to switch out the high-power laser multiple times (from the Coherent Verdi-IR, Nufern NuAmp, to IPG, back to Nufern NuAmp. 

For the high-power \SI{1064}{\nm} fibers, we've had good experiences using PMJ-A3AHPC,A3AHPC-1064-6/125-3AS-\textit{L}-1\footnote{These fibers are air-gapped and have an endcap on the ends which allows the beam to expand before entering/exiting the fiber, reducing hte power density. The adjustable fiber tip positions allow for optimization of coupling efficiency in to the fiber.} from OZ Optics. 

\section{Making and detecting Rydberg atoms}

\Cref{fig:rydberg_beams} shows a diagram of the various Rydberg beams we have available.

\begin{sidewaysfigure}[htbp]
	\includesvg[keepaspectratio, width=\paperwidth, height=\paperheight]{experiment/laser_beams/rydberg_beams.svg}
	\caption{\label{fig:rydberg_beams}
		Diagram of the various Rydberg excitation beams.}
\end{sidewaysfigure}

\subsection{UV laser system}

This section details the laser system we use to produce \SI{319}{\nm} radiation for two-photon excitation to a Rydberg state. 
Our system for generating the \SI{638}{\nm} radiation is similar to the one described in \cite{mie2014.OE.22.011182}.

Depending on the experiment, whether we want the ability to tune over multiple gigahertz with \SI{\sim 500}{\kHz} linewidth or a laser with a narrower linewidth (\SI{250}{\kHz}), we use either a Koheras Basik BMY10PztSPm or NP Photonics RFLM-50-3-1064.53-1-S-0 ``Rock''. 
The \SI{1064}{\nm} seed is then amplified by an IPG YAR-15K-1064-LP-SF fiber amplifier to about \SI{7.5}{\W}. 

The amplified \SI{1064}{\nm} radiation pumps an Argos (now Lockheed Martin) 2400 CW optical parametric oscillator (OPO) which uses a periodically-poled lithium niobate (PPLN) crystal. 
The crystal first converts $\lambda_\text{pump}=\SI{1064}{\nm}$ to \SI{1.6}{\um} and \SI{3.2}{\um} photons. 
Sum frequency generation then combines a \SI{1064}{\nm} photon with a \SI{1.6}{\um} photon to produce the \SI{638}{\nm} output. 

\begin{figure}[htbp]
	\centering
	\includesvg[keepaspectratio, width=\textwidth, height=\textheight]{experiment/uv_system/uv_system.svg}
	\caption{\label{fig:uv_system}Diagram of the laser system used to produce \SI{318}{\nm} for exciting strontium Rydberg atoms. Diagram of the Argos internals based on the one in \cite{mor2013.RSI.84.013102}.}
\end{figure}

The \SI{7.5}{\W} of \SI{1064}{\nm} gets converted to about **\SI{1.5}{\W}** of \SI{638}{\nm}. 
The Toptica SHG Pro frequency doubles the \SI{638}{\nm} to produce about **\SI{\sim 100}{\mW}** of \SI{319}{\nm} depending how well we optimize the doubling efficiency. 

Depending on the seed we use, we observe linewidths of about **\SI{500}{\kHz} (\SI{< 200}{\kHz})** with the Koheras (Rock).

\begin{align}
	\frac{1}{\SI{1064}{\nm}} &= \frac{1}{\SI{1.6}{\um}} + \frac{1}{\SI{3.2}{\um}} \\
	\frac{1}{\SI{638}{\nm}} &= \frac{1}{\SI{1064}{\nm}} + \frac{1}{\SI{1.6}{\um}}
\end{align}

\subsection{Electric Field System}

Due to the extreme ${n}^{7}$ scaling of polarizability \cite{Gallagher1988.RPP.51.143, Gallagher1994.RydbergAtoms}, Rydberg atoms are extremely sensitive to electric fields. 
To mitigate stray electric fields at the location of the atoms as well as to provide the capability for trimming residual electric fields and perform selective field ionization (SFI), we installed in-vacuum electric field plates. 
The electric field plate system was designed by Francisco Camargo and additional details his Ph.D. thesis \cite{Camargo2017.PhD} and the design follows similar ones in \cite{Millen2011.PhD, Low2012.JPB.45.113001} using a split-ring electrode geometry with four quadrant electrodes above and four quadrant electrodes below the center of the chamber (i.e., where the atoms are). 
The split-ring electrodes and in vacuum wires are made from oxygen-free high conductivity (OFHC) copper with stainless steel making up the supporting scaffold, Einzel lens, and guiding plates. 
All parts were polished by hand before being sent off to be electropolished. 

** Stuff about the scaffold and copper electric field plates. SIMION simulations? **

\begin{figure}[htbp]
	\centering
	\includesvg[keepaspectratio, width=\textwidth, height=\textheight]{experiment/electric_field_system/electric_field_system-CAD.svg}
	\caption{
		\label{fig:electric_field_system-CAD}
		CAD renderings of the electric field plates with part of the supporting structure removed for clarity. 
		The eight electric field plates are split in to upper and lower quadrants to provide electric field trimming and ionization. 
		An Einzel lens and two guiding plates help steer electrons towards the MCP. 
		Blue spheres represent the sapphire spacers used to isolate the high-voltage components from the grounded support structure.
		(Right) Electrodes labeled according to the feedthrough they're connected to.}
\end{figure}

\begin{figure}[htbp]
	\centering
	\includesvg[keepaspectratio, width=\textwidth, height=\textheight]{experiment/electric_field_system/electric_field_system-real.svg}
	\caption{
		\label{fig:electric_field_system-real}
		(Left) Overview of the actual electric field plates, Einzel lens, and guiding plates. 
		Fish spine beads are used to isolate the wires connecting the feedthroughs to the various electric field plates. 
		(Right) Close up of of the Einzel lens and guiding plates.}
\end{figure}

Working at high-$n$, Soumya was able to trim electric fields to better than ** xxxx V/cm ** when taking spectra at $n=160$.
We should be able to improve our electric field zeroing a bit more if we're more careful with biases on the SFI HV switches but extremely accurate switching would likely require going to a different design (** e.g. Tilman Pfau's box **).
For now, the electric field system is sufficient for our purposes. 

\subsection{Charged Particle Detection}

** Joe will be detailing the electric field ionization and detection system in his thesis but some details are included here for completeness. **

To detect the creation of Rydberg atoms, we implemented a system which ramps the electric field plates and MCP to perform selective field ionization (SFI). 
The MCP's output is first amplified by an SRS xxx preamplifier before being sent to an multichannel scaler (MCS). 

The MCP is an Advanced Performance Detector (APD) 2 miniTOF made by Photonis. 
This detector was selected becuse it fit inside the ** xxx in ** diameter tube off the main chamber. 

\begin{figure}[htbp]
	\centering
	\includesvg[keepaspectratio, width=\textwidth, height=\textheight]{experiment/mcp/mcp.svg}
	\caption{\label{fig:mcp}MCP for detecting charged particles. ** Ceramic spacers were used to help keep the feedthroughs from getting too close. The entire MCP assembly is mounted on a xxxxx feedthrough.**}
\end{figure}

Over the years, we have used the MCP to detect both electrons and ions, but we primarily detect electrons as they arrive much quicker than ions. 