\chapter{Strontium Rydberg Experiment}
\label{ch:experiment}

Lorem ipsum dolor sit amet, consectetur adipiscing elit. Mauris volutpat bibendum condimentum. Nunc non placerat quam. Nulla ullamcorper diam elit, id hendrerit erat faucibus at. In tincidunt, velit non egestas lacinia, nulla arcu ornare tortor, et lobortis nulla eros at odio. Aliquam vel aliquet tortor, eu semper diam. Nulla vestibulum dui vel mauris scelerisque, ultrices dapibus ex condimentum. Phasellus quam justo, molestie sed porta ac, posuere pellentesque est. In viverra, nibh non ullamcorper blandit, elit ex porttitor nisl, et commodo elit magna nec magna. Lorem ipsum dolor sit amet, consectetur adipiscing elit. Quisque maximus, urna nec tempus ultrices, quam magna blandit augue, nec vestibulum sapien felis non sem. Ut et tincidunt ante.

\section{Versatile red MOT system}
\label{sec:redmotsystem}

Stuff about versatile red MOT system.

\Cref{fig:red_MOT_system} shows what the red MOT system looks like when set up for trapping any combination of isotopes. 
In practice, we don't always have all the red MOT AOMs lined up, typically only keeping the AOMs necessary for trapping the isotope(s) for a particular experiment while the other red MOT AOMs are used to generate various other beams (e.g., spectroscopy, ``blow away'', etc.).

\begin{sidewaysfigure}[htbp]
	\centering
	\includesvg[keepaspectratio=true, width=\paperwidth, height=\paperheight]{red_MOT_system.svg}
	\caption{\label{fig:red_MOT_system}Simplified diagram of the red MOT system (when everything is put together) for multi-isotope trapping. The master laser frequency is referenced to the $\nSLJ{5s^2}{1}{S}{0} \rightarrow \nSLJ{5s5p}{3}{P}{1}$ in \Sr{88}.}
\end{sidewaysfigure}

Without using a fiber combiner, we're able to devote a lot of power to the red MOT for each isotope, enabling us to use large red MOT beams. 
The drawbacks is that we're much more sensitive to misalignments and the ugly intensity profiles out of the diodes leads to strange red MOT shapes. 