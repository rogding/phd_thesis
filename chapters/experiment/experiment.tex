\chapter{Strontium Rydberg Experiment}
\label{ch:experiment}

This chapter describes the Rydberg apparatus used in this work.
It provides documentation of the current state of various systems for producing ultracold and quantum degenerate gases of strontium. 
Being the third ultracold strontium at Rice, we were able to leverage lessons learned on the previous experiments (generally referred to as ``Neutral'' and ``Plasma'') and build quite a capable system.

Although most of our system looks very similar to other ultracold strontium systems, what makes our system different is the inclusion of systems for creating, manipulating, and detecting charged particles. 
In particular, we have in-vacuum electric field plates to null stray electric fields as well as ramp to ionizing fields which direct electrons or ions towards the microchannel plate (MCP) detector.

Our old-school ``metal can'' setup can't compete with the optical access provided by many recently started alkaline-earth Rydberg tweezer systems using glass cells \cite{Norcia2018.PRX.8.041054, Cooper2018.PRX.8.041055, Saskin2019.PRL.122.143002}.
There's still a place for our system which likely provides capabilities for better shielding from stray electric fields, performing SFI, and the ability to detect charged particles which, without going to significantly custom systems, are somewhat limiting on the standard glass cell setups.

\section{Vacuum System}

We follow a pretty conventional strontium laser cooling system which uses an oven to produce an atomic beam of strontium which is then slowed by a spin-flip Zeeman slower. 
Some information about the Rydberg vacuum system can be found in \cite{Camargo2015.MS, Ding2016.MS}. 
Similar setups are described elsewhere (e.g., \cite{Boyd2007.PhD, Stellmer2013.PhD, Barker2016.PhD, Campbell2017.PhD}).
The most noticeable aspect of our vacuum system is that it was designed to have the center of the main chamber at $\SI{3.5}{\inch}=\SI{8.89}{\cm}$ from the surface of the optical table.
Keeping the working height fixed at \SI{3.5}{\inch} significantly reduced the need to build a rigid supporting structure to hold the vacuum system above the table, elevated platforms for optics, as well as various periscopes necessary for getting the laser beams to the height of an elevated (although using fibers would mitigate this issue with the tradeoff of less laser power to the atoms). 
The downside is that it's much more difficult to set up optics below the vacuum chamber, e.g., for the vertical MOT or dipole trap beams. 

\begin{figure}[htbp]
	\centering
	\includesvg[keepaspectratio, width=\textwidth, height=\textheight]{experiment/vacuum/rydberg_vacuum_system.svg}
	\caption{
		\label{fig:vacuum_system}
		Rydberg vacuum system.}
\end{figure}

Our system maintains ultra-high vacuum pressure using two \SI{75}{\liter\per\second} ion pumps with one on the source side\footnote{Varian VacIon Plus 75 Starcell rebuilt by Duniway Stockroom Corporation.} and the second connected to the main chamber\footnote{Gamma Vacuum 75S-CVX-6S-SC-N-N.} by high conductance flanges. 
A non-evaporable getter (NEG)\footnote{SAES CapaciTorr D 200.} is also attached between the main chamber and ion pump.
We suspect the ion pump on the source side is being choked off by the zero-length reducer which adapts the ** \SI{4.5}{\inch} ** flange on the source-side vacuum system to the ** \SI{6}{\inch} ** flange on the Varian ion pump.
In retrospect, we should have included a titanium sublimation pump for additional pumping.
We should have also used a double gate valve between the Zeeman slower and the source-side vacuum as we found this valve to be leaky when we initially pumped out the system\footnote{Something similar was noticed by the Weld group at UCSB and they used two gate valves to separate their atomic source from their UHV main chamber \cite{Senaratne2018.PhD}.}.

We also added a gate valve between the Zeeman slower window and main chamber to allow us to change the viewport once it gets coated\footnote{The lab has tried using Plasma's pulsed Nd:YAG laser to perform ablation on Neutral's coated Zeeman window with moderate levels of success.}.

Although our vacuum wasn't great initially with a magnetic lifetime of about \SI{9}{\second}, more recent measurements have found lifetime to be about \SI{18}{\second} now. 
** Lifetime measurement figures? **

\subsection{Atomic Oven and Two-Dimensional Collimator}

Due to the ** extremely ** low vapor pressure of strontium, an oven is needed to increase the vapor pressure enough to form an atomic beam.
Although there are now various designs for strontium ovens \cite{Senaratne2015.RSI.86.023105, Schioppo2012.RSI.83.103101}, and even commercial systems from AO Sense, we used a design which has shown success in the Killian lab. 
It's a simple design with a ** FireRod ** providing most of the heating to the atom oven.
Capillary tubes are used to provide some degree of collimation, forming a nozzle, and a heater wire is wrapped around the nozzle to reduce the effects of strontium building up in the capillary tubes and clogging the oven. 
We typically run our main ** FireRod ** oven heater around \SI{425}{\celsius} with the nozzle heater wire around \SI{390}{\celsius}. Note that the ** FireRod ** tempearture sensor is inside the ceramic body whereas the nozzle thermocouple is attached to the nozzle itself (i.e., the FireRod remperature at the atoms is likely lower). 

\begin{figure}[htbp]
	\centering
	\includesvg[keepaspectratio, width=\textwidth, height=3in]{experiment/atomic_oven/neutral_sr_oven.svg}
	\caption{
		\label{fig:neutral_sr_oven}
		Neutral's new strontium oven during construction before thermocouple and nozzle heating wire have been attached. 
		The design is very similar to our oven but is mounted oo a 2-3/4 on CF flange instead of a 3-3/8 in CF flange as on our experiment.
		(Left) Oven without heat shield.
		(Right) After adding heat shield.}
\end{figure}

\Cref{fig:neutral_sr_oven} shows a similar oven made to replace the oven Neutral's setup. 
The design is very similar to the one on our setup but it was mounted on the much smaller 2-3/4 in flange whereas ours is mounted on a 3-3/8 in flange which provides some extra room for various feed throughs without being too cramped.

\begin{figure}[htbp]
	\centering
	\includesvg[keepaspectratio, width=\textwidth, height=3in]{experiment/atomic_oven/atomic_beam.svg}
	\caption{
		\label{fig:atomic_beam}
		Strontium atomic beam seen down an arm of the 2D collimator between the oven (to the right) and the Zeeman slower (to the left).
		A ``second'', shorter atomic beam is visible as a reflection off the bottom 2D collimator retro mirror.}
\end{figure}

After atoms exit the oven, they pass through a two-dimensional (2D) collimator which applies transverse optical molasses \cite{Sheehy1990.CP.145.317, Joffe1993.JOSAB.10.2257} to increase the flux of atoms down the Zeeman slower. 
When well tuned, we've seen an increase of blue MOT atoms by about ** 8 times ** when the 2D collimator is well aligned to the beam.
The long arms of the 2D collimator help prevent strontium buildup on the AR coated viewports.

\subsection{Zeeman Slower}

I included information about our Zeeman slower for completeness as Francisco had already made it by the time I joined the experiment so see \cite{Camargo2015.MS} for the gory details.

Since an oven is required to produce enough strontium vapor for trapping, a Zeeman slower is required to slow the atoms down so that they can be captured by the MOT. 
A Zeeman slower uses a spatially-varying magnetic field and circularly-polarized light to keep the atoms resonant with a counterpropagating light so that they continuiously scatter photons which slows their velocity \cite{Phillips1982.PRL.48.596}.
Our particular design uses a spin-flip Zeeman slower where the axial magnetic field crosses zero between the ends (similar designs are presented in \cite{Bell2010.RSI.81.013105, Hopkins2016.RSI.87.043109}).
The Zeeman slower was constructed in multiple layers as the resulting field was tweaked to match the desired profile with each layer being attached to the Zeeman slower tube with thermally-conduction epoxy ** GIVE MODEL **. 

Our Zeeman slower is water-cooled. 
We have run our Zeeman slower without water cooling without noticing significantly decreased performance but the slower does get a bit toasty. 

\subsection{Main Chamber}

** Section about the main chamber? **
Francisco designed the main chamber was custom made\footnote{Huntington Mechanical Labs} and features recessed top and bottom flanges which provide buckets for the MOT coils to fit around the 2-3/4 in top and bottom viewports. 
The bottom flange also has ** 11 ** \SI{10}{\kilo\volt} SHV? feedthroughs which are connected to the various electric field plates. 
Around the side are ** 7 ** 2-3/4 in viewports, ** 3 ** 1-1/3 in viewports for various excitation lasers, a 2-1/4 CF for attaching the Zeeman slower, and a conical expansion to a 6 in CF flange to the main pumps. 

\section{Trim Coils}

The trim coils are used to cancel out stray magnetic fields as well as apply bias fields. 
The two horizontal pairs of rectangular coils are oriented such that the ``X-axis'' produce a field towards or away from the MCP with the ``Y-axis'' perpendicular. 
The ``Z-axis'' coil pair produces a field perpendicular to the table surface. 

Due to the choice of keeping the center of the chamber at \SI{3.5}{\inch} above the table surface, this required the horizontal (X-axis and Y-axis) trim coils to be centered at this height. 

\begin{figure}[htbp]
	\centering
	\includesvg[keepaspectratio, width=\textwidth, height=\textheight]{experiment/trim_coils/trim_coils.svg}
	\caption{
		\label{fig:trim_coils}
		Magnetic field trim coils around main chamber. (Blue) X-axis, (green) Y-axis, and (red) Z-axis.}
\end{figure}

The vertical (Z-axis) coils were would on forms around the outside of the top and bottom flanges of the main chamber which allowed us to put many turns on them. 
The circular U-channels used as the forms also feature a radial slice, filled with epoxy, to mitigate the generation of eddy currents. 
In retrospect, we should made the horizontal coils beefier (especially along the X-axis) in order generate larger bias fields along the MCP axis and increase detection efficiency. 
We found that having bias fields along other directions (i.e., not towards/away from the MCP) reduced our detection efficiency, likely due to the Lorentz force steering electrons away from the MCP. 

\section{Dual MOT Coils}

One of the things we did differently from the Neutral experiment is that we have two pairs of MOT coils: the larger pair for the blue MOT and the smaller pair for the red MOT.
This allows us to use two different current supplies without needing a controller which is has the dynamic range to provide the \SI{\sim 40}{\A} to generate the \SI{\sim 40}{\gauss\per\cm} field for the blue MOT while also generating the \SI{\sim 1}{\gauss\per\cm} low-noise field for the red MOT.

\begin{figure}[htbp]
	\includesvg[keepaspectratio, width=\textwidth, height=\textheight]{experiment/mot_coils/dual_mot_coils.svg}
	\caption{\label{fig:dual_mot_coils}
		(Left) Picture of our assembled dual MOT coils on a test setup. 
		(Right) CAD rendering of how they are centered around the buket windows of the main chamber.}
\end{figure}

The blue MOT coils are hollow inside with approximately ** ** outer dimensions with a hollow ** ** through the middle for water. 
We typically run the blue MOT coils with around ** 40-50 amps ** of current from a fixed power supply ** GIVE MODEL **. 
A ** MOSFET ** is used to quickly switch off the current to the coils. 
** We so far haven't included a flyback diode although the MOSFET we use seems to include an internal flyback diode. **

The red MOT coils are constructed from ** 10 gauge ** copper wire and are epoxied to the undersides of the blue MOT coils, placing them closer to the center of the vacuum chamber. 
We currently use a homebuilt voltage-controlled current driver\footnote{**Powered by an APEX PA12**.} but we should be able to use any current driver capable of supplying up to about ** 5 A **. 
Our current geometry for the magnetic field coils provides about ** 1 G/cm/A ** conversion.

**
It would have been nice to have an extra MOSFET switch on the blue MOT coils so that we'd be able to apply extremely large fields along the Z-axis such as e.g. Pedro's experiment Feshbach coils.
We're not too worried about not having this capability since ground-state strontium has no magnetic Feshbach resonances and the fact that we've observed a decrease MCP detection efficiencly we believe is due to the Lorentz force as the MCP is oriented along the X-axis.
**

\section{Laser Systems Cooling Strontium}

This section covers the various laser systems used for laser cooling strontium and various spectroscopy beams.

\subsection{Versatile Blue MOT System}

Section about Zeeman slower and blue cooling setup. Some details about our blue cooling system can be found in \cite{Camargo2015.MS, Ding2016.MS}. 
After having to send back our Newport TLB-18xx for repairs, we decided to fiber couple the injection path for the Zeeman and MOT slave lasers. 
Fiber coupling allowed us to quickly fix injection lock issues by optimizing the reverse coupling of rejected light from the slave lasers (which counterpropagate the path of the injection light) through the fiber. 

\begin{sidewaysfigure}[htbp]
	\centering
	\includesvg[keepaspectratio, width=\paperwidth, height=\paperheight]{experiment/blue_system.svg}
	\caption{\label{fig:blue_system}**Placeholder**Simplified diagram of the blue MOT system.}
\end{sidewaysfigure}

\begin{figure}[htbp]
	\includesvg[keepaspectratio, width=\textwidth, height=\textheight]{experiment/blue_system/old_blue_MOT.svg}
	\caption{\label{fig:old_blue_MOT}
		An early picture of our \Sr{88} blue MOT taken on 2014/08/29 through a \SI{2.75}{\inch} viewport with the atom source and Zeeman slower are off to the right.
		Considering the electric field plates are separated by about \SI{1}{\inch}, this blue MOT is likely has a diameter of about \SI{0.5}{\inch}. 
		A video is available at {https://youtu.be/ENDIizrlqMA}.}
\end{figure}

\subsection{Magnetic Trap and Repumping}

As mentioned in the blue MOT section above, the $\nSLJ{5s^2}{1}{S}{0} \rightarrow \nSLJ{5s5p}{1}{P}{1}$ is not completely closed with about $1:\num{50000}$ decays following the $\nSLJ{5s5p}{1}{P}{1} \rightarrow \nSLJ{5s4d}{1}{D}{2}$ decay path.
About $\flatfrac{2}{3}$ then decay to the $\nSLJ{5s5p}{3}{P}{1}$ which then decays to the $\nSLJ{5s^2}{1}{S}{0}$ and return to the \SI{461}{\nm} cooling cycle.
The remaining $\flatfrac{1}{3}$ decay to the long-lived metastable $\nSLJ{5s5p}{3}{P}{2}$ states with lifetimes of about \SI{520}{\second} \cite{Yasuda2004.PRL.92.153004}\footnote{Theoretical calculations predict a lifetime of \SI{17}{\minute} \cite{Derevianko2001.PRL.87.023002}.}, effectively removing them from the blue MOT cooling cycle.
Of the $\nSLJ{5s5p}{3}{P}{2}$ states, the low-field seeing ${m_J = 1,2}$ states can become trapped in the quadrupole magnetic field of the blue MOT.
This decay path, initially seen as a ``loss'' ends up being a powerful tool for accumulating atoms in metastable reservoir \cite{Nagel2003.PRA.67.011401} at roughly the same temperature as the blue MOT. 
The magnetic trap was an essential tool in overcoming the abysmal \SI{0.56(2)}{\percent} natural abundance of \Sr{84} to produce the first strontium BECs \cite{Stellmer2009.PRL.103.200401, Martinez2009.PRL.103.200402}.
For details of our magnetic trap, see \cite{Camargo2015.MS}.

\subsection{\SI{481}{\nm} Repumping System}

In order to recover atoms from the metastable reservoir, several repumping transitions have been explored:
\begin{itemize}
		\item \SI{707}{\nm} and \SI{679}{\nm} lasers to drive ${\nSLJ{5s5p}{3}{P}{2} \rightarrow \nSLJ{5s6s}{3}{S}{1}}$ and ${\nSLJ{5s5p}{3}{P}{0} \rightarrow \nSLJ{5s6s}{3}{S}{1}}$ transitions, respectively \cite{Barker2015.PRA.92.043418, Campbell2017.PhD} ** probably also reference multiple theses**. 
		\item \SI{3011.8}{\nm} driving the ${\nSLJ{5s5p}{3}{P}{2} \rightarrow \nSLJ{5s4d}{3}{D}{2}}$ \cite{Mickelson2009.JPB.42.235001}.
		\item \SI{496.93}{\nm} driving the ${\nSLJ{5s5p}{3}{P}{2} \rightarrow \nSLJ{5s5d}{3}{D}{2}}$ transition \cite{Stellmer2014.PRA.90.022512, Moriya2018.JPC.2.125008}.
		\item \SI{403.35}{\nm} driving the ${\nSLJ{5s5p}{3}{P}{2} \rightarrow \nSLJ{5s6d}{3}{D}{2}}$ transition \cite{Stellmer2014.PRA.90.022512, Moriya2018.JPC.2.125008}.
		\item \SI{481.323}{\nm} driving the ${\nSLJ{5s5p}{3}{P}{2} \rightarrow \nSLJ{5p^2}{3}{P}{2}}$ transition \cite{Wongwaitayakornkul2013.BS, Camargo2015.MS, Ding2016.MS, Couturier2018.RSI.89.043103, Hu2019.PRA.99.033422}.
\end{itemize}

Our current \SI{481}{\nm} repumper system is shown in \cref{fig:repumper_system}. 
The \SI{481}{\nm} light is provided by a Toptica DL 100\footnote{The laser is equipped with a {LD-0488-0060-1} diode.} which outputs about ** \SI{10}{\mW} ** of power. 
For now, this laser is stabilized to a Doppler-broadened \Te{130}{2} line \cite{Wongwaitayakornkul2013.BS} which has a transition at \SI{20776.0886}{\per\cm} \cite{Cariou1980.Te2_Atlas}.
Originally, when the lab was working with a single isotope, simply tuning the lock point to maximize repumping efficiency of the isotope of interest was good enough (typically \Sr{84} since it's the least abundant).
Now that we routinely work with multiple isotopes and mixtures, a free-space electro-optic modulator (EOM) was added to to add sidebands to the \SI{481}{\nm} light in order to address multiple isotopes and the hyperfine shift of \Sr{87}.
The hyperfine shift of the lower $\nSLJ{5s5p}{3}{P}{2}$ state is known \cite{Heider1977.PRA.16.1371} and the hyperfine shift of the upper $\nSLJ{5p^2}{3}{P}{2}$ doubly-excited is expected to be small **because of the lack of an {$s$-electron}\footnote{I.e., the magnetic dipole hyperfine contribution comes from the $\delta(r)$....} CHECK **.

In principle, it should be possible to lock to a hollow cathode lamp (HCL) due to collisions populating enough metastable atoms to perform spectroscopy on \cite{Norcia2016.RSI.87.023110}.
Instead of using an HCL, we plan on using a ``super lock'' \cite{Lindsay1991.RSI.62.1656, Jaffe1993.RSI.64.2475, Zhao1998.RSI.69.3737, Subhankar2019.RSI.90.043115} (see Appendix xxxx for future upgrade plans) since we already have a stable \SI{689}{\nm} reference laser (although we could also lock to a stabilized HeNe).
We also considered a lock to the wavemeter as in \cite{Couturier2018.RSI.89.043103} but ultimately decided against it as our wavemeter's accuracy is lacking, its sampling rate a bit too slow, and we needed the wavemeter for other tasks (e.g., finding Rydberg lines). 

\begin{figure}[htbp]
	\centering
	\includesvg[keepaspectratio, width=\textwidth, height=\textheight]{experiment/repumper_system/repumper_system.svg}
	\caption{
		\label{fig:repumper_system}
		Simplified diagram of our current \SI{481}{\nm} repumper laser system.
		A portion of the light form the \SI{481}{\nm} external cavity diode laser (ECDL) is used to lock to a Doppler-broadened \Te{130}{2} line in a cell heated to about \SI{555}{\celsius}.
		The rest of the light is sent through a free-space electro-optic modulator (EOM) at about \SI{565}{\MHz} which is used to apply sidebands to the light in order to address the various isotopes and hyperfine states.
		\SI{481}{\nm} light is delivered to the various experiments by multimode fibers.}
\end{figure}

\subsection{\SI{689}{\nm} Laser System}

Some details about our narrow line cooling system can be found in \cite{Ding2016.MS}.
The details of the narrow line \SI{689}{\nm} ``red'' MOT can be found in \cite{Stellmer2013.PhD, Boyd2007.PhD, Ding2016.MS, Barker2016.PhD, Campbell2017.PhD}. 
The details of the \Sr{87} red MOT is particularly well described in \cite{Stellmer2013.PhD} and will not be reproduced here. 

\subsubsection{\SI{689}{\nm} Master Laser}

Due to the ease of building \SI{689}{\nm} slave lasers \cite{Ding2016.MS}, we only need a single \SI{689}{\nm} stabilized master laser system to run both the Rydberg and Neutral experiments. 
From the red master table, two fibers run to each experiment: one fiber with light at \SI{-82}{\MHz} of $\nSLJ{5s^2}{1}{S}{0} \rightarrow \nSLJ{5s5p}{3}{P}{1}$ transition and the other at ** \SI{-1440.440}{\MHz} ** for injection locking the \Sr{87} ``trap'' slave lasers. 
For most of the experiments described in this thesis, the red master laser was found to have a linewidth of about \SI{30}{\kHz}.
** More details of the \SI{689}{\nm} master laser system is provided in Jim's Ph.D. thesis. **

We currently do not implement a fiber phase noise cancellation system \cite{Ma1994.OL.19.1777, Rauf2018.RSI.89.033103} so it's possible that the light at the slave lasers is broadened to ** \si{\kHz}-levels ** but it shouldn't be too difficult to implement in the future when necessary. 

\subsubsection{Multi-Isotope Red MOT System}

\Cref{fig:red_system} shows what the red MOT system looks like when set up for trapping any combination of isotopes. 
In practice, we don't always have all the red MOT AOMs lined up, typically only keeping the AOMs necessary for trapping the isotope(s) for a particular experiment while the other red MOT AOMs are used to generate various other beams (e.g., spectroscopy, ``blow away'', etc.).

\begin{sidewaysfigure}[htbp]
	\centering
	\includesvg[keepaspectratio, width=\paperwidth, height=\paperheight]{experiment/red_system.svg}
	\caption{\label{fig:red_system}Simplified diagram of the red MOT system (when everything is put together) for multi-isotope trapping. The master laser frequency is referenced to the $\nSLJ{5s^2}{1}{S}{0} \rightarrow \nSLJ{5s5p}{3}{P}{1}$ in \Sr{88}.}
\end{sidewaysfigure}

Without using a fiber combiner, we're able to devote a lot of power to the red MOT for each isotope, enabling us to use large red MOT beams. 
The drawbacks is that we're much more sensitive to misalignments and the ugly intensity profiles out of the diodes leads to strange red MOT shapes. 

**

Mention SWAP cooling? \cite{Norcia2018.NJP.20.023021, Bartolotta2018.PRA.98.023404, Muniz2018.arXiv.1806.00838, Snigirev2019.arXiv.1903.06435}.

**

****
Image of \Sr{84} and \Sr{87} red MOTs.
****

*** NOTE ABOUT NOT USING REJECTED BEAM FOR MONITOR SLAVE MODE ***
We currently use a rejected beam from the slave laser isolator to monitor the mode of the slave laser. 
In the future, I would instead use a stray reflection or beam sampler after the isolator as the current arrangement could lead to a small amount of light from the spectrum analyzer cavity getting back to the slave lasers (especially since we share the spectrum analyzer cavity with multiple slaves), possibly leading to instability issues. 
***

\subsection{Optical dipole trap}

Optical dipole traps work by ...... \cite{Grimm1999.arXiv.9902072}.

Since the $\nSLJ{5s^2}{1}{S}{0}$ electrons are paired, it has a negligible magnetic moment meaning we cannot magnetically trap ground state strontium atoms.
As a result, we use optical traps to confine and evaporatively cool ground state strontium atoms. 

In our setup, we've used both free-space and fibered optical dipole traps but due to having to replace and realign the high-power IR lasers multiple times, we ended up sticking with the fibered systems due to them not losing alignment.
The drawback to the fibered ODT systems is that it's difficult to couple more than about \SI{5}{\W} through continuously without thermal effects degrading the coupling efficiency.
We get around this issue by only having the ODT on at high powers during the transfer from the red MOT before quickly moving in to the evaporation. 

We started with a free-spaced crossed-beam optical dipole trap (XODT) with waists of about $\SI{60}{\um} \times \SI{60}{\um}$. 
This trap was a bit tight but worked well for making \num{100000} BECs of \Sr{84}. 
We found that loading was improved if the trap was artificially widened by driving the AOM with three slightly different frequencies before turning the sideband frequencies down and evaporating \cite{Camargo2017.PhD}.
After going through three ODT lasers, we stuck with the sheet trap and vertical crossing trap because they were fiber coupled. 


\subsection{A tale of three lasers}

Our first ODT laser was a Coherent Verdi IR - essentially a \SI{1064}{\nm} version of the more common \SI{532}{\nm} variant. 
That worked ok until it died due to an electrical surge (?). 
At which point we moved to using a Nufern NuAmp \SI{50}{\W} fiber amplifier laser.

\subsection{Fiber-Coupled ODTs}

Having the fiber-coulped sheet trap saved our ass as we've had to switch out the high-power laser multiple times (from the Coherent Verdi-IR, Nufern NuAmp, to IPG, back to Nufern NuAmp. 

For the high-power \SI{1064}{\nm} fibers, we've had good experiences using PMJ-A3AHPC,A3AHPC-1064-6/125-3AS-\textit{L}-1\footnote{These fibers are air-gapped and have an endcap on the ends which allows the beam to expand before entering/exiting the fiber, reducing hte power density. The adjustable fiber tip positions allow for optimization of coupling efficiency in to the fiber.} from OZ Optics. 

\section{Making and detecting Rydberg atoms}

\subsection{UV laser system}

This section details the laser system we use to produce \SI{319}{\nm} radiation for two-photon excitation to a Rydberg state. 
Our system for generating the \SI{638}{\nm} radiation is similar to the one described in \cite{mie2014.OE.22.011182}.

Depending on the experiment, whether we want the ability to tune over multiple gigahertz with \SI{\sim 500}{\kHz} linewidth or a laser with a narrower linewidth (\SI{250}{\kHz}), we use either a Koheras Basik BMY10PztSPm or NP Photonics RFLM-50-3-1064.53-1-S-0 ``Rock''. 
The \SI{1064}{\nm} seed is then amplified by an IPG YAR-15K-1064-LP-SF fiber amplifier to about \SI{7.5}{\W}. 

The amplified \SI{1064}{\nm} radiation pumps an Argos (now Lockheed Martin) 2400 CW optical parametric oscillator (OPO) which uses a periodically-poled lithium niobate (PPLN) crystal. 
The crystal first converts $\lambda_\text{pump}=\SI{1064}{\nm}$ to \SI{1.6}{\um} and \SI{3.2}{\um} photons. 
Sum frequency generation then combines a \SI{1064}{\nm} photon with a \SI{1.6}{\um} photon to produce the \SI{638}{\nm} output. 

\begin{figure}[htbp]
	\centering
	\includesvg[keepaspectratio, width=\textwidth, height=\textheight]{experiment/uv_system.svg}
	\caption{\label{fig:uv_system}Diagram of the laser system used to produce \SI{318}{\nm} for exciting strontium Rydberg atoms. Diagram of the Argos internals based on the one in \cite{mor2013.RSI.84.013102}.}
\end{figure}

The \SI{7.5}{\W} of \SI{1064}{\nm} gets converted to about **\SI{1.5}{\W}** of \SI{638}{\nm}. 
The Toptica SHG Pro frequency doubles the \SI{638}{\nm} to produce about **\SI{\sim 100}{\mW}** of \SI{319}{\nm} depending how well we optimize the doubling efficiency. 

Depending on the seed we use, we observe linewidths of about **\SI{500}{\kHz} (\SI{< 200}{\kHz})** with the Koheras (Rock).

\begin{align}
	\frac{1}{\SI{1064}{\nm}} &= \frac{1}{\SI{1.6}{\um}} + \frac{1}{\SI{3.2}{\um}} \\
	\frac{1}{\SI{638}{\nm}} &= \frac{1}{\SI{1064}{\nm}} + \frac{1}{\SI{1.6}{\um}}
\end{align}

\subsection{Electric Field System}

Due to the extreme ${n}^{7}$ scaling of polarizability \cite{Gallagher1988.RPP.51.143, Gallagher1994.RydbergAtoms}, Rydberg atoms are extremely sensitive to electric fields. 
To mitigate stray electric fields at the location of the atoms as well as to provide the capability for trimming residual electric fields and perform selective field ionization (SFI), we installed in-vacuum electric field plates. 
The electric field plate system was designed by Francisco Camargo and additional details his Ph.D. thesis \cite{Camargo2017.PhD} and the design follows similar ones in \cite{Millen2011.PhD, Low2012.JPB.45.113001} using a split-ring electrode geometry with four quadrant electrodes above and four quadrant electrodes below the center of the chamber (i.e., where the atoms are). 
The split-ring electrodes and in vacuum wires are made from oxygen-free high conductivity (OFHC) copper with stainless steel making up the supporting scaffold, Einzel lens, and guiding plates. 
All parts were polished by hand before being sent off to be electropolished. 

** Stuff about the scaffold and copper electric field plates. SIMION simulations? **

\begin{figure}[htbp]
	\centering
	\includesvg[keepaspectratio, width=\textwidth, height=\textheight]{experiment/electric_field_system/electric_field_system-CAD.svg}
	\caption{
		\label{fig:electric_field_system-CAD}
		CAD renderings of the electric field plates with part of the supporting structure removed for clarity. 
		The eight electric field plates are split in to upper and lower quadrants to provide electric field trimming and ionization. 
		An Einzel lens and two guiding plates help steer electrons towards the MCP. 
		Blue spheres represent the sapphire spacers used to isolate the high-voltage components from the grounded support structure.
		(Right) Electrodes labeled according to the feedthrough they're connected to.}
\end{figure}

\begin{figure}[htbp]
	\centering
	\includesvg[keepaspectratio, width=\textwidth, height=\textheight]{experiment/electric_field_system/electric_field_system-real.svg}
	\caption{
		\label{fig:electric_field_system-real}
		(Left) Overview of the actual electric field plates, Einzel lens, and guiding plates. 
		Fish spine beads are used to isolate the wires connecting the feedthroughs to the various electric field plates. 
		(Right) Close up of of the Einzel lens and guiding plates.}
\end{figure}

Working at high-$n$, Soumya was able to trim electric fields to better than ** xxxx V/cm ** when taking spectra at $n=160$.
We should be able to improve our electric field zeroing a bit more if we're more careful with biases on the SFI HV switches but extremely accurate switching would likely require going to a different design (** e.g. Tilman Pfau's box **).
For now, the electric field system is sufficient for our purposes. 

\subsection{Charged Particle Detection}

** Joe will be detailing the electric field ionization and detection system in his thesis but some details are included here for completeness. **

To detect the creation of Rydberg atoms, we implemented a system which ramps the electric field plates and MCP to perform selective field ionization (SFI). 
The MCP's output is first amplified by an SRS xxx preamplifier before being sent to an multichannel scaler (MCS). 

The MCP is an Advanced Performance Detector (APD) 2 miniTOF made by Photonis. 
This detector was selected becuse it fit inside the ** xxx in ** diameter tube off the main chamber. 

\begin{figure}[htbp]
	\centering
	\includesvg[keepaspectratio, width=\textwidth, height=\textheight]{experiment/mcp/mcp.svg}
	\caption{\label{fig:mcp}MCP for detecting charged particles. ** Ceramic spacers were used to help keep the feedthroughs from getting too close. The entire MCP assembly is mounted on a xxxxx feedthrough.**}
\end{figure}

Over the years, we have used the MCP to detect both electrons and ions, but we primarily detect electrons as they arrive much quicker than ions. 

\section{Future Improvements}

Below I cover a few upgrades we are currently working towards or plan on implementing in the near future. 
Some of the improvements are focused around making systems we inherited from previous graduate students more stable and reliable throughout the lab (i.e., the \SI{481}{\nm} repumper and \SI{689}{\nm} system) while others are meant as strictly an upgrade to add improved capabilities to the Rydberg experiment. 

\subsection{Locking the \SI{481}{\nm} with a ``Super Lock''}

One of the upgrades currently underway is improving the \SI{481}{\nm} repumper lock.
The previous setup locked to the side of a Doppler-broadened line in ** \Te{130}{2} **, which has been very sensitive to background light and environmental changes. 
To improve the lock, we'll be implementing a scanning Fabry-Perot lock where the position of the \SI{481}{\nm} laser in a Fabry-Perot cavity is locked relative to the \SI{689}{\nm} master laser.
Instead of implementing a proper transfer cavity lock, we'll be using a ``superlock'' where a Fabry-Perot cavity is rapidly scanned and stabilized to one laser wavelength (our \SI{689}{\nm} reference) which allows another laser (the \SI{481}{\nm} laser) to be stabilized relative to the reference laser \cite{Lindsay1991.RSI.62.1656, Tonyushkin2007.RSI.78.123103, Subhankar2019.RSI.90.043115}. 

Since we should be able to lock the \SI{481}{\nm} repumper to sub-megahertz frequencies, we plan on mapping out the $\nSLJ{5s5p}{3}{P}{2} \rightarrow \nSLJ{5p^2}{3}{P}{2}$ transition for optimum multi-isotope repumping. 

\subsection{Upgrading to a ULE Cavity}

The lab recently purchased an ultra-low expansion (ULE) glass cavity from Stable Laser Systems (SLS) with mirrors coated for \SI{640}{\nm}, \SI{689}{\nm}, and \SI{698}{\nm}.
The mirrors have a specified finesse of ** $\mathcal{F} \approx \num{100000}$ ** for these wavelengths, which means we should be able to narrow our laser to ** sub-kilohertz ** levels. 
This ULE cavity will be used to stabilize and narrow both our workhorse \SI{689}{\nm} laser and \SI{640}{\nm} Rydberg laser (pre-doubling). 
We currently do not have a \SI{698}{\nm} clock laser but the option is available.
We'll be implementing a system similar to \cite{Gregory2015.NJP.17.055006} where two broadband fiber EOMs will be used to provide both the PDH modulation on tunable sidebands with gigahertz separations. 
Eventually, the ULE cavity will provide the frequency reference for the \SI{689}{\nm} instead of the atoms\footnote{Other groups have mentioned that they experienced frequency broadening when locking their \SI{689}{\nm} lasers to the atoms after using the ULE to narrow the laser.}. 
The \SI{689}{\nm} saturated absorption cell will remain to provide a check on the absolute frequency reference as the ULE cavity settles overtime (** ULE cavities are expected to drift at about \SI{100}{\Hz} per day \cite{Barker2016.PhD, Reschovsky2017.PhD} **).

In our current setup, we observe positional instability in the (bosonic) red MOT final (single-frequency) position which appears to be similar to the one reported in \cite{Hanley2018.PhD}. 
As written by \citeauthor{Hanley2018.PhD}, once they switched to lock to the ULE cavity, the noticed a significant improvement in the vertical positional stability of their red MOT which will hopefully also be the case for us. 

\subsection{Redoing the \SI{1064}{\nm} ODT system}

Since we've been borrowing Randy Hulet's \SI{1070}{\nm} laser for our ODT system, we need to eventually change back to our Nufern NuAMP \SI{50}{\W} single-mode amplifier. 
During this process, we should take the time to revise the optical setup so that it's both simpler and has the capability of monitoring the coupling efficiency through the ODT fibers (monitoring both the input and output powers). 
We will also be implementing a second sheet trap ** with similar dimensions as the one I built ** so that we have a ``crossed-pancake'' setup. 
This should be advantageous since the pancake geometry modematches well with the red MOT and crossing the sheet traps should provide significantly better longitudinal confinement over our current single-beam sheet.
The vertical dimple will remain as an option for producing tight optical trapping potentials when we need high densities (e.g., for \Sr{88} and \Sr{87}).
Although we haven't experienced significant issues with damaging the high-power \SI{1064}{\nm} ODT fibers (** find theses where they mention damaging these fibers? **), the capability to monitor (and implement an interlock) should add an additional layer of protection.

\subsection{Fiber-Coupling the Red MOT}

We've found that our current setup is susceptible to beam-pointing drifts. 
We haven't pinpointed the source of the drifts but it appears to be the slave laser housings we've been suing (Thorlabs' xxxxxxx). 
Fiber-coupling the red MOT beams should allow much greater performance consistency although we likely have to sacrifice power (i.e., we'll need to use smaller red MOT beams). 

We still haven't come across a good system for controlling the frequency of the red MOT lasers...