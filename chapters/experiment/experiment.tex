\chapter{Strontium Rydberg experiment}
\label{ch:experiment}

Section intro about the Rydberg experiment.

\section{Vacuum system}

We follow a pretty conventional strontium laser cooling system which uses an oven to produce an atomic beam of strontium which is then slowed by a spin-flip Zeeman slower. 
Similar setups are described in \cite{stellmer2013.phd, barker2016.phd}.
The most noticeable feature is the fact that it was designed to have the center of the main chamber at $\SI{3.5}{\inch}=\SI{8.89}{\cm}$ from the surface of the optical table.
Keeping the working height at this height significantly reduced the need to build a rigid supporting structure to hold the vacuum system above the table as well as the necessary periscopes for the laser beams (although using fibers would mitigate this issue). 
The downside is that it's much more difficult to set up optics below the vacuum chamber, e.g., for the vertical MOT or dipole trap beams. 

Some information about the Rydberg vacuum system can be found in \cite{camargo2015.ms, ding2016.ms}. 

\section{Magnetic field coils}

\subsection{Zeeman slower}

I included information about our Zeeman slower for completeness as Francisco had already made it by the time I joined the experiment so see \cite{camargo2015.ms} for the gory details.

\subsection{Dual MOT coils}

One of the things we did differently from the Neutral experiment is that we have two pairs of MOT coils: the larger pair for the blue MOT and the smaller pair for the red MOT.
This allows us to use two different current supplies without needing a controller which is has the dynamic range to provide the \SI{\sim 40}{\A} to generate the \SI{\sim 40}{\gauss\per\cm} field for the blue MOT while also generating the \SI{\sim 1}{\gauss\per\cm} low-noise field for the red MOT.

\subsection{Trim coils}

The trim coils are used to cancel out stray magnetic fields as well as apply bias fields. 
The two horizontal pairs of rectangular coils are oriented such that the ``X-axis'' produce a field towards or away from the MCP with the ``Y-axis'' perpendicular. 
The ``Z-axis'' coil pair produces a field perpendicular to the table surface. 

Due to the choice of keeping the center of the chamber at \SI{3.5}{\inch} above the table surface, this required the horizontal (X-axis and Y-axis) trim coils to be centered at this height. 

\begin{figure}[htbp]
	\centering
	\includesvg[keepaspectratio=true, width=\textwidth, height=\textheight]{experiment/magnetic_field_coils/trim_coils/trim_coils.svg}
	\caption{\label{fig:trim_coils}Magnetic field trim coils around main chamber. (Blue) X-axis, (green) Y-axis, and (red) Z-axis.}
\end{figure}

The vertical (Z-axis) coils were would on forms around the outside of the top and bottom flanges of the main chamber which allowed us to put many turns on them. 
The circular U-channels used as the forms also feature a radial slice, filled with epoxy, to mitigate the generation of eddy currents. 
In retrospect, we should made the horizontal coils beefier (especially along the X-axis) in order generate larger bias fields along the MCP axis and increase detection efficiency. 
We found that having bias fields along other directions (i.e., not towards/away from the MCP) reduced our detection efficiency, likely due to the Lorentz force steering electrons away from the MCP. 

\section{Laser systems}

\subsection{Versatile blue MOT system}

Section about Zeeman slower and blue cooling setup. Some details about our blue cooling system can be found in \cite{camargo2015.ms, ding2016.ms}. 
After having to send back our Newport TLB-18xx for repairs, we decided to fiber couple the injection path for the Zeeman and MOT slave lasers. 
Fiber coupling allowed us to quickly fix injection lock issues by optimizing the reverse coupling of rejected light from the slave lasers (which counterpropagate the path of the injection light) through the fiber. 

\begin{sidewaysfigure}[htbp]
	\centering
	\includesvg[keepaspectratio=true, width=\paperwidth, height=\paperheight]{experiment/blue_system.svg}
	\caption{\label{fig:blue_system}**Placeholder**Simplified diagram of the blue MOT system.}
\end{sidewaysfigure}

\subsection{Multi-isotope red MOT system}

Some details about our narrow line cooling system can be found in \cite{ding2016.ms}.

\Cref{fig:red_system} shows what the red MOT system looks like when set up for trapping any combination of isotopes. 
In practice, we don't always have all the red MOT AOMs lined up, typically only keeping the AOMs necessary for trapping the isotope(s) for a particular experiment while the other red MOT AOMs are used to generate various other beams (e.g., spectroscopy, ``blow away'', etc.).

\begin{sidewaysfigure}[htbp]
	\centering
	\includesvg[keepaspectratio=true, width=\paperwidth, height=\paperheight]{experiment/red_system.svg}
	\caption{\label{fig:red_system}Simplified diagram of the red MOT system (when everything is put together) for multi-isotope trapping. The master laser frequency is referenced to the $\nSLJ{5s^2}{1}{S}{0} \rightarrow \nSLJ{5s5p}{3}{P}{1}$ in \Sr{88}.}
\end{sidewaysfigure}

Without using a fiber combiner, we're able to devote a lot of power to the red MOT for each isotope, enabling us to use large red MOT beams. 
The drawbacks is that we're much more sensitive to misalignments and the ugly intensity profiles out of the diodes leads to strange red MOT shapes. 

****
Image of \Sr{84} and \Sr{87} red MOTs.
****

\subsection{Optical dipole trap}

Optical dipole traps work by ...... \cite{gri1999.odt}.

Since the $\nSLJ{5s^2}{1}{S}{0}$ electrons are paired, it has a negligible magnetic moment meaning we cannot magnetically trap ground state strontium atoms.
As a result, we use optical traps to confine and evaporatively cool ground state strontium atoms. 

In our setup, we've used both free-space and fibered optical dipole traps but due to having to replace and realign the high-power IR lasers multiple times, we ended up sticking with the fibered systems due to them not losing alignment.
The drawback to the fibered ODT systems is that it's difficult to couple more than about \SI{5}{\W} through continuously without thermal effects degrading the coupling efficiency.
We get around this issue by only having the ODT on at high powers during the transfer from the red MOT before quickly moving in to the evaporation. 

We started with a free-spaced crossed-beam optical dipole trap (XODT) with waists of about $\SI{60}{\um} \times \SI{60}{\um}$. 
This trap was a bit tight but worked well for making \num{100000} BECs of \Sr{84}. 
We found that loading was improved if the trap was artificially widened by driving the AOM with three slightly different frequencies before turning the sideband frequencies down and evaporating \cite{camargo2017.phd}.
After going through three ODT lasers, we stuck with the sheet trap and vertical crossing trap because they were fiber coupled. 


\subsection{A tale of three lasers}

Our first ODT laser was a Coherent Verdi IR - essentially a \SI{1064}{\nm} version of the more common \SI{532}{\nm} variant. 
That worked ok until it died due to an electrical surge (?). 
At which point we moved to using a Nufern NuAmp \SI{50}{\W} fiber amplifier laser.

\section{Making and detecting Rydberg atoms}

\subsection{UV laser system}

This section details the laser system we use to produce \SI{319}{\nm} radiation for two-photon excitation to a Rydberg state. 
Our system for generating the \SI{638}{\nm} radiation is similar to the one described in \cite{mie2014.OE.22.011182}.

Depending on the experiment, whether we want the ability to tune over multiple gigahertz with \SI{\sim 500}{\kHz} linewidth or a laser with a narrower linewidth (\SI{250}{\kHz}), we use either a Koheras Basik BMY10PztSPm or NP Photonics RFLM-50-3-1064.53-1-S-0 ``Rock''. 
The \SI{1064}{\nm} seed is then amplified by an IPG YAR-15K-1064-LP-SF fiber amplifier to about \SI{7.5}{\W}. 

The amplified \SI{1064}{\nm} radiation pumps an Argos (now Lockheed Martin) 2400 CW optical parametric oscillator (OPO) which uses a periodically-poled lithium niobate (PPLN) crystal. 
The crystal first converts $\lambda_\text{pump}=\SI{1064}{\nm}$ to \SI{1.6}{\um} and \SI{3.2}{\um} photons. 
Sum frequency generation then combines a \SI{1064}{\nm} photon with a \SI{1.6}{\um} photon to produce the \SI{638}{\nm} output. 

\begin{figure}[htbp]
	\centering
	\includesvg[keepaspectratio=true, width=\textwidth, height=\textheight]{experiment/uv_system.svg}
	\caption{\label{fig:uv_system}Diagram of the laser system used to produce \SI{318}{\nm} for exciting strontium Rydberg atoms. Diagram of the Argos internals based on the one in \cite{mor2013.RSI.84.013102}.}
\end{figure}

The \SI{7.5}{\W} of \SI{1064}{\nm} gets converted to about **\SI{1.5}{\W}** of \SI{638}{\nm}. 
The Toptica SHG Pro frequency doubles the \SI{638}{\nm} to produce about **\SI{\sim 100}{\mW}** of \SI{319}{\nm} depending how well we optimize the doubling efficiency. 

Depending on the seed we use, we observe linewidths of about **\SI{500}{\kHz} (\SI{< 200}{\kHz})** with the Koheras (Rock).

\begin{align}
	\frac{1}{\SI{1064}{\nm}} &= \frac{1}{\SI{1.6}{\um}} + \frac{1}{\SI{3.2}{\um}} \\
	\frac{1}{\SI{638}{\nm}} &= \frac{1}{\SI{1064}{\nm}} + \frac{1}{\SI{1.6}{\um}}
\end{align}

\subsection{Electric field ionization and detection}

Joe will be detailing the electric field ionization and detection system in his thesis but some details are included here for completeness. 

** picture of the electric field plates and mcp **

Stuff about the scaffold and copper electric field plates. SIMION simulations?

\subsection{Charged particle detection}

Stuff about MCP.

\begin{figure}[htbp]
	\centering
	\includesvg[keepaspectratio=true, width=\textwidth, height=\textheight]{experiment/mcp/mcp.svg}
	\caption{\label{fig:mcp}MCP for detecting charged particles. ** Ceramic spacers were used to help keep the feedthroughs from getting too close. The entire MCP assembly is mounted on a xxxxx feedthrough.**}
\end{figure}
