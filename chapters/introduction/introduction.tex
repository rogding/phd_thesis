\chapter{Introduction}
\label{ch:intro}

\section{Motivation}
\label{sec:motivation}


\section{Strontium}

The work in this thesis was done with the alkaline earth metal atom strontium. 
Being a Group~II element, it has two valence electrons around a closed shell which gives it some nice properties regarding regarding laser cooling and spectroscopy:
-Ground-state has no/very weak magnetic moment - We don't have Feshbach resonances like the alkalis do but it's also relatively insensitive to magnetic fields.
-\Sr{87} has a large ({$I=\flatfrac{9}{2}$}) nuclear spin, making it interesting for ** INSERT REASONS **.
-The narrow intercombination line at \SI{689}{\nm} provides a really nice transition for performing spectroscopy and laser cooling.
Simon Stellmer has a nice writeup of the history of strontium in his PhD thesis \cite{Stellmer2013.PhD}.

\begin{figure}[h]
	\centering
	\includesvg[keepaspectratio, width=\textwidth, height=\textheight]{introduction/sr_levels/sr_levels-boson.svg}
	\caption{
		\label{fig:sr_levels-boson}
		Energy levels of (bosonic) strontium (\Sr{88} in particular).}
\end{figure}

\subsection{Laser Cooling Strontium}

We follow the typical cooling scheme for making cold or quantum degenerate gases of strontim. 
Our apparatus will be described in more detail in \cref{ch:experiment}.