\chapter{Introduction}
\label{ch:intro}

\section{Motivation}
\label{sec:motivation}

Rydberg excitations have even been observed in solid state systems \cite{Kazimierczuk2014.Nature.514.343}. 

\section{Strontium}

The work in this thesis was done with the alkaline earth metal atom strontium. 
Being a Group~II element, it has two valence electrons around a closed shell which gives it some nice properties regarding regarding laser cooling and spectroscopy:
-Ground-state has no/very weak magnetic moment - We don't have Feshbach resonances like the alkalis do but it's also relatively insensitive to magnetic fields.
-\Sr{87} has a large ({$I=\flatfrac{9}{2}$}) nuclear spin, making it interesting for ** INSERT REASONS **.
-The narrow intercombination line at \SI{689}{\nm} provides a really nice transition for performing spectroscopy and laser cooling.
Simon Stellmer has a nice writeup of the history of strontium in his PhD thesis \cite{Stellmer2013.PhD}.

\begin{figure}[h]
	\centering
	\includesvg[keepaspectratio, width=\textwidth, height=\textheight]{introduction/sr_levels/sr_levels-boson.svg}
	\caption{
		\label{fig:sr_levels-boson}
		Grotrian diagram for (bosonic) strontium (\Sr{88} in particular).}
\end{figure}

\section{Rydberg Dressing}

Our experiment was originally built to pursue Rydberg dressing where atomic interactions are modified by incorporating strongly interacting Rydberg states \cite{Johnson2010.PRA.82.033412, Honer2010.PRL.105.160404}.
After work by our group \cite{Aman2016.PRA.93.043425, DeSalvo2016.PRA.93.022709, Gaul2016.PRL.116.243001}, the Pfau group \cite{Balewski2014.NJP.16.063012} ** CHECK IF CORRECT PFAU PAPER **, and the Porto group \cite{Goldschmidt2016.PRL.116.113001, Boulier2017.PRA.96.053409} found that Rydberg dressing was lossy, we moved on to other things. 

\subsection{Laser Cooling Strontium}

We follow the typical cooling scheme for making cold or quantum degenerate gases of strontium. 
Our apparatus will be described in more detail in \cref{ch:experiment}.	