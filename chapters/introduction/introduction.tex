\chapter{Introduction}

***************************************************

*********** NOTES TO SELF, REMOVE LATER ***********

(0) Long-range and anisotropic interactions

(0a) Few-/Many-body physics

(0b) Ultra-long-range Rydberg molecules

(1) Spectroscopy

(2) Effects of quantum statistics -- use Joe's figure?

Add in repumper spectrum.

***************************************************

***************************************************



The ability to produce ultracold samples paved the way towards achievement of quantum degenerate gases of atoms which are ideal systems for studying quantum interactions. 
From the observation of BEC-BCS crossover \cite{Chen2005.PR.412.1, Gurarie2007.JAOP.322.2, Bourdel2004.PRL.93.050401}, superfluid to Mott insulator transition in an optical lattice \cite{Greiner2002.Nature.415.39}, and the formation of matter-wave solitons \cite{Strecker2002.Nature.417.150, Khaykovich2002.Science.296.1290}, most of the work in these systems have been based on a short-range, isotropic contact interactions described by the pseudopotential \cite{Pethick.BEC}
\begin{equation}
	U\pqty{\va{r}}	=	\frac{4 \pi \hbar^2}{m} a_{s} \delta\pqty{\va{r}}
\end{equation}
where $a_{s}$ is the $s$-wave scattering length.
This effective interaction requires the atomic wave functions to overlap. 

An important recent trend in the field of ultracold atomic gases is the study of systems with long-range interactions which can typically be expanded as \cite{Pethick.BEC}
\begin{equation}
	U\pqty{\va{r}}	=	\sum^{\infty}_{k=1} \frac{C_{k} \pqty{\va{r}}}{r^{k}}
\end{equation}
with the ``long-range'' coming from the $U \sim {1}/{r^{k}}$ dependence. 
Depending on the type of interaction (e.g., Coulomb $k = 1$, dipole-dipole $k = 3$, van der Waals $k = 6$), the potentials can be isotropic or anisotropic, providing additional richness to quantum systems and the opportunity to study a wide variety of phenomena including applications towards quantum computing \cite{Jaksch2000.PRL.85.2208, DeMille2002.88.067901}, quantum simulation \cite{deLeseleuc2018.PRL.120.113602, Schauss2018.QST.3.023001}, and tests of theories **** SOURCES? ****
There are several methods for incorporating long-range interactions in to ultracold systems with some of the most prominent being \cite{Lahaye2009.RPP.72.126401}: 
\begin{description}
	\item[Polar molecules]
		These molecules can have large (permanent) electric dipole moments \cite{Aymar2005.JCP.122.204302} and can, ignoring chemical reactions, have extremely long lifetimes. 
		The drawback is that the additional complexity typically makes it very difficult to produce ultracold and quantum degenerate samples. 
		In spite of these challenges, a degenerate Fermi gas of {{\tsup{40}\text{K}}{\tsup{87}\text{Rb}}} was recently achieved by associating atoms from a BEC of {\tsup{87}\text{Rb}} and a degenerate Fermi gas of {\tsup{40}\text{K}} \cite{DeMarco2019.Science.363.853}. 
		Steady progress continues to be made in the direct laser cooling of molecules as well \cite{Anderegg2018.NatPhys.14.890}.
	\item[Magnetic atoms]
		Atoms themselves can have permanent dipole moments which can be exploited to study long-range interactions. 
		Quantum degeneracy has been reached with chromium \cite{Griesmaier2005.PRL.94.160401, Naylor2015.PRA.91.011603}, erbium \cite{Aikawa2012.PRL.108.210401, Aikawa2014.PRL.112.010404}, and dysprosium \cite{Lu2011.PRL.107.190401, Lu2012.PRL.108.215301}.
		Although simpler than molecules, their dipolar interactions are limited by the magnetic moment of atoms and typically requires tuning the short-range interaction strength to be weaker than the dipole-dipole interaction to observe their effects.
		There's also been a recent proposal for creating ultracold dysprosium atoms which exhibit an electric dipole moment in addition to the magnetic dipole moment \cite{Lepers2018.PRL.121.063201}.
	\item[Coupling atoms to a high-finesse cavity]
		An ingenious way of inducing long-range interactions in atomic systems is by placing the atoms in a high-finesse optical cavity such that they couple to the cavity mode which in turn causes the atoms to experience position-based forces \cite{Baumann2010.Nature.464.1301, Mottl2012.Science.336.1570}.
		Although the interactions are (essentially) infinite-range, they are likely to be restricted to only that case. 
		They are also constrained by the geometry of the experiment which could make it difficult to tune between isotropic and anisotropic interactions.
	\item[Rydberg states]
		Highly excited Rydberg states of atoms (and molecules) with principal quantum number $n \gg 1$ can also exhibit strong dipole interactions \cite{Gallagher1994.RydbergAtoms} and have even been observed in solid state systems \cite{Kazimierczuk2014.Nature.514.343}. 
		Since many of the properties of Rydberg atoms scale with $n$, the Rydberg-Rydberg interactions can be tuned in strength and anisotropy by simply exciting different Rydberg states. 
		Rydberg atoms can also leverage well-established techniques for producing quantum degenerate gases of atoms so that their interactions can be readily incorporated to existing setups. 
		The major disadvantage of Rydberg systems are their sensitivities to stray electric fields and the limited Rydberg lifetimes due to both blackbody and Rydberg-Rydberg interactions. 
\end{description}
This thesis will be focused on the latter by studying Rydberg systems in ultracold strontium gases. 
Most ultracold Rydberg experiments have predominantly focused on exciting alkali metal atoms (primarily rubidium and cesium with a growing interest in potassium).
With their hydrogen-like electronic structure, there are well-established methods for both calculating their properties (e.g., see \cite{Sibalic2017.CPC.220.319}) and techniques for producing ultracold and quantum degenerate gases.
But since the interactions are long-ranged, generally comparable to the inter-particle separation in the gas, the particular atomic species used should only affect the details of the interactions but not the overall form. 
As will be discussed below, strontium has some properties which make it a very advantageous system for studying Rydberg physics. 

\section{Strontium Rydberg Atoms}

Strontium offer several benefits over the alkali atoms with perhaps the most obvious, as seen in \cref{fig:sr_levels-boson}, being the singlet and triplet Rydberg series.
Starting from the $\nSLJ{5s^2}{1}{S}{0}$ ground state, a two-photon excitation can be used to access the myriad of $\SLJ{1}{S}{0}$, $\SLJ{3}{S}{1}$, $\SLJ{1}{D}{2}$, and $\SLJ{3}{D}{1,2,3}$ Rydberg levels. 
In addition to the anisotropies of a particular Rydberg orbital, the $C_{6}$ coefficients have been calculated and exhibit both attractive ($C_{6} < 0$) and repulsive ($C_{6} > 0$) interactions \cite{Vaillant2012.JPB.45.135004}. 

Another significant advantage of strontium is the narrow $\nSLJ{5s^2}{1}{S}{0} \rightarrow \nSLJ{5s5p}{3}{P}{1}$ transition with linewidth ${\Gamma}/{2 \pi} = \SI{7.5}{\kHz}$. 
Not only does this transition enable us to easily produce samples below about $\SI{2}{\micro\kelvin}$, it is also serves as the first leg of a two-photon transition to a Rydberg state. 
Considering the two-photon coupling between states in a (three-level system) $\Omega \sim {\Gamma}/{\Delta}$ and the photon scattering rate $\Gamma_{\text{sc}} \sim \pqty{{\Gamma}/{\Delta}}^2$ for a detuning $\Delta$ (e.g., see \cite{Metcalf1999.LCT, Bransden.Atoms, Foot2005.Atomic, Steck.QuantumAtomOptics, Gentile1989.PRA.40.5103}), it enables stronger coupling for comparable scattering rates. 
Active work is also progressing towards exciting strontium Rydberg atoms using the even narrower $\nSLJ{5s^2}{1}{S}{0} \rightarrow \nSLJ{5s5p}{3}{P}{0}$ clock transition \cite{Gil2014.PRL.112.103601}. 
Recently, a direct measurement of the $\nSLJ{5s^2}{1}{S}{0} \rightarrow \nSLJ{5s5p}{3}{P}{2}$ transition was performed \cite{Onishchenko2019.PRA.99.052503} so, in principle, this state could also be used.

* Nuclear spin - hyperfine (or lackthereof). 

The advantages and disadvantages discussed for strontium also generally apply to other alkaline earth-like atoms (e.g., calcium, barium, ytterbium) as well.








*********************


The work in this thesis was done with the alkaline earth metal atom strontium. 
Being a Group~II element, it has two valence electrons around a closed shell which gives it some nice properties regarding regarding laser cooling and spectroscopy:
-Ground-state has no/very weak magnetic moment - We don't have Feshbach resonances like the alkalis do but it's also relatively insensitive to magnetic fields.
-\Sr{87} has a large ({$I=\flatfrac{9}{2}$}) nuclear spin, making it interesting for ** INSERT REASONS **.
-The narrow intercombination line at \SI{689}{\nm} provides a really nice transition for performing spectroscopy and laser cooling.
Simon Stellmer has a nice writeup of the history of strontium in his PhD thesis \cite{Stellmer2013.PhD}.

\begin{figure}[h]
	\centering
	\includesvg[keepaspectratio, width=\textwidth, height=\textheight]{introduction/sr_levels/sr_levels-boson.svg}
	\caption{
		\label{fig:sr_levels-boson}
		Grotrian diagram for (bosonic) strontium (\Sr{88} in particular).}
\end{figure}


***************************

Isotope shift of the \SI{461}{\nm} transition.

\begin{table}[h]
	\caption[]{
		\label{tab:blue_isotope_shift}
		Isotope shifts of the $\SI{461}{\nm}$ $\nSLJ{5s^2}{1}{S}{0} \rightarrow \nSLJ{5s5p}{1}{P}{1}$ transition relative to \Sr{88} calculated from the values in \cite{Bushaw2000.SAPB.55.1679}.
		See \cref{tab:blue_isotope_shift_detailed,tab:hyperfine_coefficients_5s5p1P1} for more details.}
	\centering
	\begin{tabular}{@{}cccc@{}}
		\toprule
		Isotope						& Lower level										& Upper level					& $\nu_{A} - \nu_{88}$ [$\si{\MHz}$]	\\
		\midrule	
		\Sr{88}						& $\nSLJ{5s^2}{1}{S}{0}$							& $\nSLJ{5s5p}{1}{P}{1}$		& $\num{0}$								\\
		\midrule	
		\multirow{3}{*}{\Sr{87}}	& \multirow{3}{*}{$\nSLJf{5s^{2}}{1}{S}{0}{9/2}$}	& $\nSLJf{5s5p}{1}{P}{1}{7/2}$	& $\num{-7.8+-0.4}$					\\
									&													& $\nSLJf{5s5p}{1}{P}{1}{11/2}$	& $\num{-49.5+-0.4}$					\\
									&													& $\nSLJf{5s5p}{1}{P}{1}{9/2}$	& $\num{-68.1+-0.4}$					\\
		\midrule	
		\Sr{86}						& $\nSLJ{5s^2}{1}{S}{0}$							& $\nSLJ{5s5p}{1}{P}{1}$		& $\num{-126.3+-0.2}$					\\
		\midrule	
		\Sr{84}						& $\nSLJ{5s^2}{1}{S}{0}$							& $\nSLJ{5s5p}{1}{P}{1}$		& $\num{-273.2+-0.3}$					\\
		\bottomrule
	\end{tabular}
\end{table}

\subsection{Ultralong-Range Rydberg Molecules}

Perhaps one of the strangest consequences of exciting a Rydberg atom in a cold and dense gas is the formation of objects known as ``ultralong-range Rydberg molecules'' (ULRRMs). 
These molecules were theoretically predicted \cite{Omont1977.JPF.38.1343, Greene2000.PRL.85.2458}
They arise due to the scattering of the Rydberg electron off a nearby ground-state atom which can be described by the pseudopotential \cite{Fermi1934.NC.11.157, Omont1977.JPF.38.1343, DeSalvo2015.PRA.92.031403, Liebisch2016.JPB.49.182001}
\begin{equation}
	V\pqty*{\va{r}_{1}, \va{r}_{2}, \va{R}}
		=	\sum_{i=1}^{2} \frac{2 \pi \hbar^{2} A_{s}\pqty*{k\pqty*{\va{R}}}}{m_{e}} \delta\pqty*{\va{r}_i - \va{R}}
			+ \frac{6 \pi \hbar^{2} A_{p}^{3}\pqty*{k\pqty*{\va{R}}}}{m_{e}} \overleftarrow{\grad}{\delta\pqty*{\va{r}_i - \va{R}}}\overrightarrow{\grad}
\end{equation}

\begin{equation}
	V\pqty*{\va{r}_{1}, \va{r}_{2}, \va{R}}	\simeq	\frac{2 \pi \hbar^{2} A_{s}\pqty*{k}}{m_{e}} \psi + \frac{6 \pi \hbar^{2} A_{p}^{2}\pqty*{k}}{m_{e}} \psi
\end{equation}
** SEE \cite{Bendkowsky2010.PhD} FOR A DERIVATION? **
These Rydberg molecules were first experimentally observed in \Rb{87} \cite{Bendkowsky2009.Nature.458.1005} with subsequent observations in {Cs} \cite{Tallant2012.PRL.109.173202} and {Sr} \cite{DeSalvo2015.PRA.92.031403}. 
Since the binding energies are very weak and scale as $E_{b} \sim {1}/{n^6}$, these can only be studied in cold gases.

\Cref{fig:n34_rydberg_molecule_wave_functions_and_spectra} provides an example of the molecular potential and the resulting radial vibration wave functions along with the spectroscopic signal. 
\begin{figure}[htbp]
	\centering
	\includegraphics[keepaspectratio, width=\textwidth, height=\textheight]{introduction/rydberg_molecules/n34-wave_functions_and_spectra/n34-wave_functions_and_spectra.pdf}
	\caption[]{
		\label{fig:n34_rydberg_molecule_wave_functions_and_spectra}
		(Top) Calculated molecular potential ($V$) for a $\nSLJ{5s34s}{3}{S}{1} + \nSLJ{5s^2}{1}{S}{0}$ atom pair together with the radial vibrational wave functions for the $\nu = 0$ to $\nu = 4$ vibrational states.
		(Bottom) Rydberg molecule spectra in unpolarized gas of \Sr{87} and highlighting various dimer vibrational states.}
\end{figure}


***********

OFR with Rydberg molecules \cite{Thomas2018.NatComm.9.2238}.