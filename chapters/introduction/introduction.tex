\chapter{Introduction}
\label{ch:intro}

For the last

**********************

The ability to produce cold and ultracold samples paved the way towards achievement of quantum degenerate gases of atoms which are ideal systems for studying quantum interactions. 
From the observation of BEC-BCS crossover, to xxx, and yyyy, most of the work so far has been based on short-range, isotropic contact interactions which require the atomic wave functions to overlap.
One way to extend these systems to more complexity and to study other phenomena ** D-wave super conductors? ** is to incorporate long-range interactions.
Now, the atoms no longer need to have an overlap to interact.



\section{Motivation}
\label{sec:motivation}


************************
(0) Long-range and anisotropic interactions

(0a) Few-/Many-body physics

(0b) Ultra-long-range Rydberg molecules

(1) Spectroscopy

(2) Effects of quantum statistics -- use Joe's figure?

Add in repumper spectrum.


*** Need good intro to why study long-range interactions ***

************************

There are several methods for incorporating long-range interactions in to quantum systems are currently being pursued \cite{Lahaye2009.RPP.72.126401}: 
\begin{description}
	\item[Polar molecules]
		These molecules can have large (permanent) electric dipole moments \cite{Aymar2005.JCP.122.204302}. 
		The drawback is that the additional complexity typically makes it very difficult to produce ultracold and quantum degenerate samples.
		Steady progress has been made in the direct laser cooling of molecules \cite{Anderegg2018.NatPhys.14.890} as well as the recent producing of a degenerate Fermi gas of {{\tsup{40}\text{K}}{\tsup{87}\text{Rb}}} \cite{DeMarco2019.Science.363.853} produced by photoassociating ultracold atoms.
		These have the advantage of, ignoring chemical reactions for now, extremely long lifetimes. 
	\item[Magnetic atoms]
		Atoms themselves can have permanent dipole moments which can be used to study long-range interactions. 
		Quantum degeneracy has been reached with chromium \cite{Griesmaier2005.PRL.94.160401, Naylor2015.PRA.91.011603}, erbium \cite{Aikawa2012.PRL.108.210401, Aikawa2014.PRL.112.010404}, and dysprosium \cite{Lu2011.PRL.107.190401, Lu2012.PRL.108.215301}.
		Although simpler than molecules, their dipolar interactions are limited by the magnetic moment of atoms and typically requires tuning the short-range interactions to be weaker than the dipole interaction to observe their effects ** cite **.
		** There's also been a recent proposal for creating ultracold dysprosium atoms which exhibit an electric dipole moment in addition to the magnetic dipole moment \cite{Lepers2018.PRL.121.063201}. **
	\item[Coupling atoms to a high-finesse cavity]
		An interesting way of inducing long-range interactions by placing atoms in a high-finesse optical cavity such that they couple to the cavity mode which in turn causes the atoms to experience forces based on its position \cite{Baumann2010.Nature.464.1301, Mottl2012.Science.336.1570}.
		Although ingenious, the interactions are likely restricted to the infinitely long-range case along with the limitations posed by the cavity modes (e.g., it would likely be difficult to tune between isotropic and anisotropic interactions).
	\item[Rydberg states]
		Highly excited Rydberg states of atoms (and molecules) with principal quantum number $n \gg 1$ can also exhibit strong dipole interactions ** \cite{Gallagher1994.RydbergAtoms} **. 
		** Many of the relevant scales scale with $n$. **
		The advantage of Rydberg atoms is that they permit the use of well-established techniques for producing quantum degenerate gases of atoms so that their interactions can be readily incorporated to existing setups. 
		** (Justify/clarify) The Rydberg-Rydberg interactions can also be very strong, much stronger than the magnetic dipole-dipole interaction in dysprosium. 
		The Rydberg dipole-dipole interaction also provide tunability where both the strength and sign of the interaction can be chosen based $n$.
		The major drawbacks are the sensitivities to stray electric fields due to the high-$n$ necessary for achieving strong interactions and Rydberg lifetimes due to both blackbody and Rydberg-Rydberg interactions. 
		Rydberg interactions can be (nearly) isotropic with $S$-states ($L = 0$) or anisotropic with $L > 0$ states ** add refs **. 
		Rydberg excitations have even been observed in solid state systems \cite{Kazimierczuk2014.Nature.514.343}. 
\end{description}

Our experiment pursues the latter by studying strontium Rydberg atoms.
Building on the expertise at Rice for producing ultracold and quantum degenerate strontium of the Killian lab and the creation, control, and detection of Rydberg atoms in the Dunning lab, our experiment melds the expertise of both groups. 

\section{Strontium Rydberg Atoms}

Compared to other work in ultracold and quantum degenerate Rydberg systems (primarily Rb and Cs with a growing interest in K), strontium offers some nice advantages over the alkalis. 

*********************


The work in this thesis was done with the alkaline earth metal atom strontium. 
Being a Group~II element, it has two valence electrons around a closed shell which gives it some nice properties regarding regarding laser cooling and spectroscopy:
-Ground-state has no/very weak magnetic moment - We don't have Feshbach resonances like the alkalis do but it's also relatively insensitive to magnetic fields.
-\Sr{87} has a large ({$I=\flatfrac{9}{2}$}) nuclear spin, making it interesting for ** INSERT REASONS **.
-The narrow intercombination line at \SI{689}{\nm} provides a really nice transition for performing spectroscopy and laser cooling.
Simon Stellmer has a nice writeup of the history of strontium in his PhD thesis \cite{Stellmer2013.PhD}.

\begin{figure}[h]
	\centering
	\includesvg[keepaspectratio, width=\textwidth, height=\textheight]{introduction/sr_levels/sr_levels-boson.svg}
	\caption{
		\label{fig:sr_levels-boson}
		Grotrian diagram for (bosonic) strontium (\Sr{88} in particular).}
\end{figure}

\section{Rydberg Dressing}

Our experiment was originally built to pursue Rydberg dressing where atomic interactions are modified by incorporating strongly interacting Rydberg states \cite{Johnson2010.PRA.82.033412, Honer2010.PRL.105.160404}.
After work by our group \cite{Aman2016.PRA.93.043425, DeSalvo2016.PRA.93.022709, Gaul2016.PRL.116.243001}, the Pfau group \cite{Balewski2014.NJP.16.063012} ** CHECK IF CORRECT PFAU PAPER **, and the Porto group \cite{Goldschmidt2016.PRL.116.113001, Boulier2017.PRA.96.053409} found that Rydberg dressing was lossy, we moved on to other things. 

\section{Long-Range Interactions}

A lot of recent interest in Rydberg atoms (and long-range interactions in general) has been from quantum computing ** cite DeMille paper? ** where these can be used to entangle gates. 

Our experiment has a more fundamental focus on these studies. 
We don't use the dipole interaction, but instead use the ``orbiting'' Rydberg electron to probe correlations in quantum gases. 

\subsection{Ultralong-Range Rydberg Molecules}

Perhaps one of the strangest consequences of exciting a Rydberg atom in an ultracold, dense gas is the formation of objects known as ``ultralong-range Rydberg molecules''. 
These molecules were theoretically predicted \cite{Omont1977.JPF.38.1343, Greene2000.PRL.85.2458} and first experimentally observed in \Rb{87} \cite{Bendkowsky2009.Nature.458.1005}. 
They arise due to the scattering of the Rydberg electron off a nearby ground-state atom which results in a potential of the form ** CITE EQUATION SOURCE \cite{DeSalvo2015.PRA.92.031403} **
\begin{align}
	V\pqty*{\va{r}_{1}, \va{r}_{2}, \va{R}}
		&{}={}
			\sum_{i=1}^{2} \frac{2 \pi \hbar^{2} A_{s}\bqty*{k\pqty*{\va{R}}}}{m_{e}} \delta\pqty*{\va{r}_i - \va{R}}
			+ \frac{6 \pi \hbar^{2} A_{p}^{3}\bqty*{k\pqty*{\va{R}}}}{m_{e}} \overleftarrow{\grad}{\delta\pqty*{\va{r}_i - \va{R}}}\overrightarrow{\grad}	\\
		&{}\simeq{}
			\frac{2 \pi \hbar^{2} A_{s}\bqty*{k}}{m_{e}} \psi + \frac{6 \pi \hbar^{2} A_{p}^{2}\bqty*{k}}{m_{e}} \psi
\end{align}
Subsequent experiments have observed them in {Cs} \cite{Tallant2012.PRL.109.173202} and {Sr} \cite{DeSalvo2015.PRA.92.031403}.

** SEE \cite{Bendkowsky2010.PhD} FOR A DERIVATION? **

**************


These molecules are formed by the interaction of the Rydberg electron with nearby (ground-state) atoms such that, if the interaction is attractive, leads to them being bound together\footnote{An alternative system which can also be described as a ``Rydberg molecule'' is when the system starts as a molecule when is then promoted to a highly-excited electronic state. In this thesis, we stick with the first definition.}.