\chapter{Introduction}
\label{ch:introduction}

Advances in laser cooling have paved the way towards the creation of cold, ultracold, and quantum degenerate gases of atoms which are ideal for studying quantum systems with unprecedented levels of control. 
Compared to many other systems where the challenge is separating the influences of a particular interaction from a myriad of other effects, ultracold quantum gas systems follow a ``bottom-up'' approach where interactions are carefully (and often painstakingly) introduced.
So far, from the observation of the BEC-BCS crossover \cite{Chen2005.PR.412.1, Gurarie2007.JAOP.322.2, Bourdel2004.PRL.93.050401} to the superfluid-Mott insulator transition in an optical lattice \cite{Greiner2002.Nature.415.39} to the formation of matter-wave solitons \cite{Strecker2002.Nature.417.150, Khaykovich2002.Science.296.1290}, most of these effects arise from an isotropic, two-body short-range contact interaction that can be described by the pseudopotential
\begin{equation}
	\label{eq:contact_interactions}
	\hat{V}\pqty*{\vb{r}}  \psi\pqty*{\vb{r}}	=	\bqty{g \delta\pqty*{\vb{r}} \pdv{r}r}\psi\pqty*{\vb{r}}
\end{equation}
where $\vb{r}$ is the relative distance between two colliding particles, $g$ is the interaction strength\footnote{The effective interaction for two particles of equal mass is $g = {4 \pi \hbar^2 a_{s}}/{m}$ with $a_{s}$ is the $s$-wave scattering length \cite{Pethick.BEC}.}, and $\delta\pqty*{\vb{r}}$ is the Dirac delta function \cite{Pethick.BEC, Dalibard1998.Collisions}.
Due to the $\delta\pqty*{\vb{r}}$ term, this requires the particle wave functions to overlap for interactions to occur.

An recent trend in the field of ultracold gases has been the introduction of long-range interactions into systems that have been generally well-described by \cref{eq:contact_interactions}.
These long-range interactions can typically be expanded as
\begin{equation}
	\label{eq:long-range_interactions}
	\hat{V}\pqty*{\vb{r}} \psi\pqty*{\vb{r}}	=	\bqty{\sum_{k} \frac{C_{k} \pqty*{\vb{r}}}{\abs*{\vb{r}}^{k}}} \psi\pqty*{\vb{r}}
\end{equation}
with interaction strength $C_{k}\pqty*{\vb{r}}$ and the ``long-range'' coming from the $\hat{V} \sim {1}/{\abs*{\vb{r}}^{k}}$ dependence \cite{Pethick.BEC}.
Some common interactions are $k = 1$ for Coulomb, $k = 3$ for dipole-dipole, and $k = 6$ for van der Waals.
These can also have an angular dependence as well, which leads to isotropic and anisotropic interactions that adds additional ``richness'' to the system. 
The provides the opportunity to study a wide variety of phenomena including those with applications towards quantum computing \cite{Jaksch2000.PRL.85.2208, DeMille2002.88.067901}, quantum simulation \cite{deLeseleuc2018.PRL.120.113602, Schauss2018.QST.3.023001}, and tests of theory.
There are several methods for incorporating long-range interactions into ultracold systems, each of which has trade-offs between production, manipulation, and duration (i.e., how difficult is it to get cold enough to study the interactions of interest, how controllable are the samples and the interactions, and and how long can the effect be studied).
Some of the most prominent methods are (e.g., see \cite{Lahaye2009.RPP.72.126401}): 
\begin{description}
	\item[Polar molecules] \hfill \\
		Typically heteronuclear, these molecules can have both large (permanent) electric dipole moments \cite{Aymar2005.JCP.122.204302} and, ignoring chemical reactions, extremely long lifetimes. 
		The major disadvantage is that their additional complexity generally makes it very difficult to produce ultracold and quantum degenerate samples and control their interactions.
		In spite of these challenges, a degenerate Fermi gas of {{\tsup{40}\text{K}}{\tsup{87}\text{Rb}}} was recently achieved by combining atoms from a BEC of {\tsup{87}\text{Rb}} and a degenerate Fermi gas of {\tsup{40}\text{K}} \cite{DeMarco2019.Science.363.853}. 
		Additionally, steady progress continues to be made towards direct laser cooling of molecules \cite{McCarron2018.JPB.51.212001, Tarbutt2018.arXiv.1902.05628, Anderegg2018.NatPhys.14.890}.
	\item[Magnetic atoms] \hfill \\
		Atoms themselves can have permanent dipole moments that can be exploited to study long-range interactions with quantum degeneracy having been achieved in chromium \cite{Griesmaier2005.PRL.94.160401, Naylor2015.PRA.91.011603}, erbium \cite{Aikawa2012.PRL.108.210401, Aikawa2014.PRL.112.010404}, and dysprosium \cite{Lu2011.PRL.107.190401, Lu2012.PRL.108.215301}.
		Although simpler than molecules and have (essentially) infinite lifetimes, their dipolar interactions are limited by the magnetic moment of the atoms and typically requires tuning of the short-range interaction strength to be weaker than the dipole-dipole interaction in order to observe their effects.
		There is also a recent proposal for creating ultracold dysprosium atoms which exhibits both an electric and a magnetic dipole moment \cite{Lepers2018.PRL.121.063201}.
	\item[Coupling atoms to a high-finesse cavity] \hfill \\
		An ingenious way of inducing long-range interactions in atomic systems is by placing the atoms in a high-finesse optical cavity such that they couple to the cavity mode which in turn causes the atoms to experience position-dependent forces \cite{Baumann2010.Nature.464.1301, Mottl2012.Science.336.1570}.
		Although the interactions are (essentially) infinite-range, they are likely restricted to only that case. 
		They are also constrained by the geometry of the experiment which could make it difficult to tune between isotropic and anisotropic interactions.
	\item[Rydberg states] \hfill \\
		Highly excited Rydberg states of atoms (and molecules) with principal quantum numbers $n \gg 1$ can also exhibit strong dipole-dipole interactions \cite{Gallagher1994.RydbergAtoms}. 
		(Rydberg states have also been observed in solid state systems \cite{Kazimierczuk2014.Nature.514.343}, opening another avenue for studying long-range interactions.)
		Since many of the properties of Rydberg atoms scale with $n$, their interactions can be tuned in strength and anisotropy simply by exciting to different  states. 
		Rydberg atoms can also leverage well-established techniques for producing quantum degenerate gases of atoms so that their interactions can be readily incorporated into existing experimental setups. 
		The major disadvantage of Rydberg systems is their sensitivity to stray electric fields and their limited lifetime due to spontaneous decay and to both blackbody and Rydberg-Rydberg interactions. 
\end{description}
This thesis will be focused on the latter by studying Rydberg systems in ultracold strontium gases. 
Most ultracold Rydberg experiments have predominantly focused on the alkali metal atoms rubidium, cesium, and potassium.
With their hydrogen-like electronic structure, there are well-established methods for both calculating their properties (e.g., see \cite{Sibalic2017.CPC.220.319}) and techniques for producing ultracold and quantum degenerate gases, making them a staple of ultracold gas experiments.
But since Rydberg interactions are long-ranged, generally comparable to the interparticle separation in the gas, the particular atomic species used should only affect the details of the interactions but not their overall form. 
As will be discussed below, strontium has some properties which make it a very attractive system for studying Rydberg physics. 

\section{Rydberg Atoms}

A Rydberg atom can be loosely defined as an atom in an excited state with very large principal quantum number $n$ such that $n \gg 1$. 
Starting in the late {1800's}, it was observed that the wavenumbers of atomic levels ($\nu_{n,l}$) could be described by the formula
\begin{equation}
	\label{eq:Rydberg_formula}
	\nu_{n,j,l}	=	\nu_{\infty,j,l} - \frac{R}{\pqty*{n-\delta_{n,j,l}}^2}	=	\nu_{\infty,j,l} - \frac{R}{{n^*}^2}
\end{equation}
where $\nu_{\infty,j,l}$ is the series limit, $\delta_{n,j,l}$ is the ``quantum defect'', and $R$ is the Rydberg constant \cite{Rydberg1890.Recherches, Rydberg1890.PM.29.331, Gallagher1994.RydbergAtoms}. 
(The subscripts in \cref{eq:Rydberg_formula} indicate a common dependency of the values on $n$, $j$, and $l$.)
Oftentimes, the principal quantum number and quantum defect are combined into an ``effective quantum number'' ${n^*} = n-\delta_{n,j,l}$. 
This is the Rydberg formula.
Conceptually, a Rydberg atom can be thought of as being comprised of a core ion and a highly-excited valence electron which spends the majority of its time far away from the core.
As a result, the Rydberg electron has only a small probability of being found near the core which allows its effects to be characterized by a (nearly) constant parameter $\delta_{n,j,l}$ due to it primarily experiencing the long-range Coulomb interaction of the core ion.
While the details of the core ion and electrons are important, causing $\delta_{n,j,l}$ to vary from atom-to-atom and from orbital-to-orbital, interactions with the core generally do not have a large effect on the overall properties of the Rydberg atoms.

While the effects due to the core are usually small, the principal quantum number does greatly affect the atomic wave function. 
As a result, $n$ sets the energy scale for many interactions and leads to the properties of Rydberg atoms becoming exaggerated when $n$ becomes large.
\Cref{tab:Rydberg_n-scaling} gives the $n$ dependence for a few properties of Rydberg atoms.
\begin{table}[!htbp]
	\centering
	\caption[]{
		\label{tab:Rydberg_n-scaling}
		Some properties of rubidium and strontium Rydberg atoms \cite{Gallagher1988.RPP.51.143, Gallagher1994.RydbergAtoms, Low2012.JPB.45.113001}.
		Unless otherwise noted, the rubidium and strontium values in the table are calculated from the ionization energies, Rydberg constants, and quantum defects presented in \cite{Lorenzen1983.PS.27.300} and \cite{Couturier2019.PRA.99.022503}, respectively.}
	\begin{tabular}{@{}cccc@{}}
		\toprule
		Property											& $n^*$ scaling			& {Rb}: {$\nSLJ{38s}{2}{S}{}$}								& {Sr}: {$\nSLJ{5s38s}{3}{S}{1}$}						\\
		\midrule
		Binding energy ($E_{n^*}$)							& $\pqty*{n^*}^{-2}$	& $\SI{11.19}{\meV}$										& $\SI{11.35}{\meV}$									\\
		Coulomb splitting ($E_{n^*} - E_{\pqty*{n+1}^*}$)	& $\pqty*{n^*}^{-3}$	& $\SI{148.77}{\GHz}$										& $\SI{151.84}{\GHz}$									\\
		Orbital radius ($\ev*{r}$)							& $\pqty*{n^*}^{2}$		& $\SI{1823.74}{\bohr}$										& $\SI{1798.74}{\bohr}$									\\
		Lifetime ($\tau$)									& $\pqty*{n^*}^{3}$		& $\SI{28.6+-1.1}{\us}$ \cite{Branden2009.JPB.43.015002}	& $\SI{21+-1}{\us}$ \cite{Camargo2016.PRA.93.022702}	\\
		\bottomrule
	\end{tabular}
\end{table}

\subsection{Ultra-Long-Range Rydberg Molecules (ULRRMs)}

One key aspect to focus on from \cref{tab:Rydberg_n-scaling} is that size of a Rydberg atom increases very rapidly with $n^*$. 
In a dense enough gas, the $\ev*{r} \sim \pqty*{n^*}^{2}$ scaling means the Rydberg orbital radius can quickly become comparable to or larger than the interparticle separations.
It was in this context, when an atom (or multiple atoms) could be located inside the Rydberg electron's orbit, that a density-dependent shift was first observed in the spectral lines of alkali atoms with the direction of the shift, relative to the unperturbed atomic line, also dependent on the buffer gas used \cite{Amaldi1934.NC.11.145, Amaldi1934.Nature.133.141, Liebisch2016.JPB.49.182001, Shaffer2018.NatComm.9.1965}.
\begin{figure}[!htbp]
	\centering
	\includegraphics[keepaspectratio, width=5in, height=\textheight]{introduction/rydberg_molecules/orbital_radius-density_n-scaling.pdf}
	\caption[]{
		\label{fig:orbital_radius-density_n-scaling}
		Orbital radius $\ev{r}$ for a strontium Rydberg atom in the $\nSLJ{5sns}{3}{S}{1}$ state and the corresponding density ${1}/{\ev{r}^3}$.
		The shaded region represents the range of densities typically achievable in ultracold strontium gas experiments with a maximum of about $\SI{1E15}{\per\cm\cubed}$ occurring in a \Sr{88} BEC \cite{Yan2013.PRL.110.123201, Stellmer2014.arXiv.1307.0601}.}
\end{figure}
\Cref{fig:orbital_radius-density_n-scaling} shows $\ev{r}$ and ${1}/{\ev{r}^3}$ for a strontium atom in the $\nSLJ{5sns}{3}{S}{1}$ state and the typical densities achievable in ultracold gas experiments.
Notice that, even for modest principal quantum numbers and densities, it's easily possible to excite Rydberg atoms much larger than the interparticle separations.

This shift was first explained by Fermi \cite{Fermi1934.NC.11.157}, and later generalized by Omont \cite{Omont1977.JPF.38.1343}, to be the result of low-energy scattering of the Rydberg electron off nearby neutral atoms, describing the electron-atom interaction with the pseudopotential
\begin{equation}
	\hat{V}\pqty*{\vb{R}}
		=	\sum_{i} \frac{2 \pi \hbar^{2} A_{s}\bqty*{k\pqty*{\vb{R}}}}{m_{e}} \delta\pqty*{\vb{r}_i - \vb{R}}
			+ \frac{6 \pi \hbar^{2} A_{p}^{3}\bqty*{k\pqty*{\vb{R}}}}{m_{e}} \overleftarrow{\grad}{\delta\pqty*{\vb{r}_i - \vb{R}}}\overrightarrow{\grad}
\end{equation}
where $\vb{r}_{i}$ is the position of the Rydberg valence electron(s) and $\vb{R}$ is the position of the ground-state atom relative to the Rydberg core \cite{DeSalvo2015.PRA.92.031403, Liebisch2016.JPB.49.182001, Shaffer2018.NatComm.9.1965}. 
On the right hand side, the first term is the $s$-wave contribution and the second term is the $p$-wave contribution with momentum-dependent scattering lengths $A_{s}\pqty*{k}$ and $A_{p}\pqty*{k}$, respectively.
The semiclassical Rydberg electron momentum is given by $\hbar {k \pqty*{\vb{r}}} = \sqrt{2 m_{e} \bqty*{{e^2}/{\pqty*{4 \pi \epsilon_0 \vb{r}}} - E_{b}}}$ for an unperturbed Rydberg electron with binding energy $E_{b}$ \cite{DeSalvo2015.PRA.92.031403}.
Evaluating the interaction with the Rydberg electron wave function $\ket*{\Psi_{nl}}$ leads to an effective molecular potential
\begin{equation}
	\ev*{\hat{V}\pqty*{\vb{R}}}
		\simeq	\sum_{i} \frac{2 \pi \hbar^{2} A_{s}\pqty*{k}}{m_{e}} {\abs*{\Psi_{nl}\pqty*{\vb{r}_{i} = \vb{R}}}^2}
				+ \frac{6 \pi \hbar^{2} A_{p}^{2}\pqty*{k}}{m_{e}} {\abs*{\grad \Psi_{nl}\pqty*{\vb{r}_{i} = \vb{R}}}^2}
\end{equation}
experienced by a ground-state atom \cite{DeSalvo2015.PRA.92.031403}.
A very nice derivation of the interaction is presented in \cite{Bendkowsky2010.PhD}\footnote{She references \cite{Balewski2009.Diploma} which has more details of Fermi's approach.} along with review articles \cite{Shaffer2018.NatComm.9.1965, Fey2019.arXiv.1904.08372, Eiles2019.JPB.52.113001}.

We are particularly interested in the case when, in the low-energy scattering regime, $A_{s} < 0$ because the interaction is attractive and can lead to the formation of bound states involving nearby ground state atoms located in the Rydberg electron wave function.
These objects are known as ``ultra-long-range Rydberg molecules''\footnote{Other objects can also be called ``Rydberg molecules'' (e.g., a molecule electronically excited to a high principal quantum number) but for the purposes of this thesis, a Rydberg molecule refers to the binding of neutral atoms to a Rydberg atom due to Rydberg electron-neutral atom scattering.} (ULRRMs) and were theoretically predicted by Greene et al., \cite{Greene2000.PRL.85.2458} in {2000}. 
Instead of the more conventional binding mechanisms (e.g., ionic or covalent), the ground state atoms are bound to the Rydberg atom by low-energy Rydberg electron-neutral atom scattering. 
The first experimental detection of ULRRMs occurred in \Rb{87} \cite{Bendkowsky2009.Nature.458.1005} with subsequent observations in {Cs} \cite{Tallant2012.PRL.109.173202} and {Sr} \cite{DeSalvo2015.PRA.92.031403}.
\Cref{fig:n34_rydberg_molecule_wave_functions_and_spectra} provides an illustrative example of a molecular potential together with the resulting radial molecular vibrational wave functions and a typical Rydberg excitation spectrum seen experimentally.
\begin{figure}[!htbp]
	\centering
	\includegraphics[keepaspectratio, width=\textwidth, height=\textheight]{introduction/rydberg_molecules/n34-wave_functions_and_spectra/n34-wave_functions_and_spectra.pdf}
	\caption[]{
		\label{fig:n34_rydberg_molecule_wave_functions_and_spectra}
		(Top) Calculated molecular potential $\ev*{\hat{V}\pqty*{R}}$ for a $\nSLJ{5s34s}{3}{S}{1} + \nSLJ{5s^2}{1}{S}{0}$ atom pair together with the radial wave functions for the {$\nu = 0$, $1$, $2$, $3$, and $4$} vibrational states.
		(Bottom) Rydberg excitation signal for an unpolarized gas of \Sr{87} with the various dimer vibrational states as indicated.}
\end{figure}
Starting from the atomic line and tuning the laser below resonance, additional peaks appear which correspond to the various radial vibrational states (labeled by $\nu$) of the ULRRM. 
Note that the binding energies of ULRRMs are typically a few to a few-tens of megahertz and scale as $E_{b} \sim {1}/{\pqty*{n^*}^6}$\footnote{Since $\ev*{\hat{V}\pqty*{\vb{R}}} \propto {\abs*{\Psi_{nl}\pqty*{\vb{R}}}^2}$, the ${1}/{\pqty*{n^*}^6}$ scaling can be understood as the electron probability being ``spread out'' over a volume which scale as $V \sim \ev*{r}^{3} \sim \pqty*{n^*}^6$.}, meaning that ULRRMs can only be created in ultracold, dense samples.

Although \cref{fig:n34_rydberg_molecule_wave_functions_and_spectra} only shows ``dimer'' data with one Rydberg atom in the $\nSLJ{5s34s}{3}{S}{1}$ state bound to a single nearby $\nSLJ{5s^2}{1}{S}{0}$ atom, this effect is generalizable to multiple bodies bound to the Rydberg atom. 
A ``trimer'' is a Rydberg atom containing two bound ground-state atoms, a ``tetramer'' is a Rydberg atom with three embedded ground-state atoms, and so on (e.g., see \cite{Gaj2014.NatComm.5.4546}).
Once a macroscopic occupation of the bound molecular states occurs, it makes more sense to describe the system in the quasiparticle picture with the Rydberg atom and the bound ground-state atoms forming a ``polaron'' \cite{Camargo2018.PRL.120.083401, Schmidt2018.PRA.97.022707}.

\section{Strontium Rydberg Atoms}
\label{sec:sr_rydberg_atoms}

Strontium offers several benefits over the alkali atoms with perhaps the most obvious, as seen in \cref{fig:sr_levels-boson}, being the existence of singlet and triplet Rydberg series due to the presence of two valence electrons.
\begin{figure}[h]
	\centering
	\includesvg[keepaspectratio, width=\textwidth, height=\textheight]{introduction/sr_levels/strontium_level_diagram.svg}
	\caption{
		\label{fig:sr_levels-boson}
		Energy level diagram for (bosonic) strontium with preference given to levels relevant to our experimental setup\protect\footnotemark.
		Driven transitions are indicated by double arrows. 
		Spontaneous decays are indicated dashed single arrows.}
\end{figure}
\footnotetext{Based on the energy level diagrams presented in \cite{Stellmer2013.PhD} and by the Weidem{\"{u}}ller strontium Rydberg experiment at University of Science and Technology of China (USTC) Shanghai Institute for Advanced Studies.}
Starting from the $\nSLJ{5s^2}{1}{S}{0}$ ground state, two-photon excitation can be used to access the myriad of $\SLJ{1}{S}{}$, $\SLJ{3}{S}{}$, $\SLJ{1}{D}{}$, and $\SLJ{3}{D}{}$ Rydberg levels.
In addition to the anisotropies associated with a particular Rydberg orbital, the $C_{6}$ coefficients have been calculated for these states and exhibit both attractive ($C_{6} < 0$) and repulsive ($C_{6} > 0$) interactions \cite{Vaillant2012.JPB.45.135004}. 
The second valence electron also introduces additional complexity due to there being multiple Rydberg series which can interact and perturb one another.
The treatment of such perturbations typically requires multichannel quantum defect theory (MQDT) \cite{Gallagher1994.RydbergAtoms, Vaillant2014.JPB.47.155001, Robicheaux2018.PRA.97.022508} and/or a two-active-electron (TAE) model \cite{Ye2013.PRA.88.043430}.
There also exists the opportunity to produce doubly-excited and autoionizing states \cite{Gallagher1994.RydbergAtoms} which, besides being interesting in their own right (e.g., see \cite{Eichmann1992.PRL.68.21}), have proven to be a sensitive tool for detecting Rydberg atoms \cite{Millen2010.PRL.105.213004, Millen2011.PhD, Lochead2013.PRA.87.053409, Lochead2016.PhD}. 

Another significant advantage of strontium is the narrow $\nSLJ{5s^2}{1}{S}{0} \rightarrow \nSLJ{5s5p}{3}{P}{1}$ transition with a ${\Gamma}/{2 \pi} = \SI{7.5}{\kHz}$ ($\tau = \SI{21}{\us}$) linewidth. 
Not only does this transition enable us to easily produce samples with temperatures below $\SI{2}{\micro\kelvin}$, it also provides a convenient first step for two-photon excitation to Rydberg states. 
Considering the two-photon coupling between states (in a three-level system) $\Omega \sim {\Gamma}/{\Delta}$ and the photon scattering rate $\Gamma_{\text{sc}} \sim \pqty{{\Gamma}/{\Delta}}^2$ for a detuning $\Delta$ (e.g., see \cite{Metcalf1999.LCT, Bransden.Atoms, Foot2005.Atomic, Steck.QuantumAtomOptics, Gentile1989.PRA.40.5103}), the narrow linewidth enables stronger coupling for comparable scattering rates. 
Active research is also progressing towards exciting strontium Rydberg atoms using the even narrower $\nSLJ{5s^2}{1}{S}{0} \rightarrow \nSLJ{5s5p}{3}{P}{0}$ clock transition \cite{Gil2014.PRL.112.103601}. 
In principle, the ultranarrow $\nSLJ{5s^2}{1}{S}{0} \rightarrow \nSLJ{5s5p}{3}{P}{2}$ transition could also be used to excite Rydberg states and a direct measurement of the transition frequency was recently reported \cite{Onishchenko2019.PRA.99.052503}.

A third, more subtle, aspect of strontium is the range of stable isotopes available, both bosonic ($I=0$) and fermionic ($I={9}/{2}$). 
Since all the bosonic isotopes (\Sr{88}, \Sr{86}, and \Sr{84}) have been Bose condensed \cite{Stellmer2009.PRL.103.200401, Martinez2009.PRL.103.200402, Mickelson2010.PRA.81.051601, Stellmer2010.82.041602, Stellmer2013.PRA.87.013611} and a degenerate Fermi gas \Sr{87} has also been produced \cite{DeSalvo2010.PRL.105.030402, Stellmer2013.PRA.87.013611}, strontium provides a unique opportunity for incorporating Rydberg interactions into both ultracold Bose and Fermi gases (as well as Bose-Bose and Bose-Fermi mixtures). 
For the bosons, the lack of nuclear spin means there is no hyperfine structure which greatly simplifies their Rydberg spectra whereas the hyperfine structure of \Sr{87} makes its spectra significantly more complicated. 
However, \Sr{87} provides a unique opportunity for approximating an ultracold classical gas since, in an unpolarized gas of \Sr{87}, the atomic population is distributed among the $\pqty{2 I + 1} = \num{10}$ degenerate $\nSLJFM{5s^2}{1}{S}{0}{F}{m_F}$ spin states.

Many of the advantages (and challenges) discussed above for strontium generally apply to other alkaline earth-like atoms as well (e.g., calcium, barium, ytterbium).
One other advantage of strontium in particular\footnote{At the time of writing, ULRRMs of other alkaline earth-like atoms have not yet been reported.} is the lack of a scattering resonance between the Rydberg electron and nearby perturbing atoms. 
It was observed that the lifetimes of rubidium ULRRMs were significantly shorter than the lifetime of the parent Rydberg atom with the difference attributed to the occurrence of a $p$-wave scattering resonance \cite{Butscher2011.JPB.44.184004, Hamilton2002.JPB.35.L199}.
In contrast, strontium ULRRMs exhibit lifetimes similar to the parent Rydberg state, indicating a much weaker (or absent) $p$-wave scattering resonance \cite{Camargo2016.PRA.93.022702}.
It should be noted that in very dense gases (e.g., in a BEC), even strontium ULRRMs exhibit reduced lifetimes with the likely cause being collisions between ground-state atoms and the Rydberg core ion \cite{Whalen2017.PRA.96.042702}. 

\section{Thesis Outline}

The goal of this thesis is to document experiments towards utilizing the unusual characteristics of ULRRMs as probes of long-range spatial correlations in ultracold gases.
At ultracold temperatures, the atoms in the gas are expected to exhibit the effects of quantum statistics that affect the likelihood of finding two (or more) strontium atoms within a certain radius (i.e., a ``spatial correlation function'').
This ``bunching'' or ``antibunching'' of atoms should then be reflected in the excitation rates to ULRRM states.
Towards that end, \cref{ch:experiment} outlines the techniques used to produce ultracold gases of strontium as well as documenting the current state of the Rydberg experiment. 
It also aims to provide a (small) collection of references regarding laser cooling strontium in general, noting several methods that are not implemented on our current system but may be of interest for future upgrades.
\Cref{ch:fermion_spectroscopy} includes a reproduction of the paper \cite{Ding2018.PRA.98.042505} which details efforts in understanding the hyperfine Rydberg states of \Sr{87}.
Understanding the Rydberg states of \Sr{87} is of high importance because it is used to represent both a Fermi gas and a (nearly) classical gas.
Additional material and details are provided that were not included in the published paper.
Current work exploring the effects of spatial correlations on ULRRMs is presented in \cref{ch:correlations}, describing experiments examining the effects of spatial correlations on production of vibrationally-excited dimer ULRRMs.
Lastly, \cref{ch:summary} provides a summary of this thesis and some potential avenues for future exploration.
Hopefully this thesis serve as a valuable resource for future studies of strontium Rydberg atoms and molecules.