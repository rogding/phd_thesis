\chapter{Spectroscopy of \Sr{87} triplet Rydberg states}

We didn't set out to map out the triplet series of \Sr{87} Rydberg states, but when we first started looking for \Sr{87} \SLJ{3}{S}{} and \SLJ{3}{D}{} Rydberg states, we observed spectra which didn't line up with their predicted locations using the known quantum defects. 
It turns out that the hyperfine interaction in \Sr{87} is not negligible as the Rydberg states approach the \Srion{87}{+} hyperfine split core. 
We then rediscovered some papers from the 1980's which explained the shifted \SLJ{1}{S}{0} spectra in \Sr{87} by including the mixing of states with the same $F$. 
Previous work was accomplished mostly through the \nSLJ{5s5p}{1}{P}{1} state or from the metastable \nSLJ{5s5p}{3}{P}{2} states, our work appears to be the first measurements of the \Sr{87} hyperfine Rydberg states through the \nSLJ{5s5p}{3}{P}{1} intermediate state. 

We measured \Sr{87} hyperfine Rydberg spectra using a two-photon excitation through the intermediate $\nSLJ{5s5p}{3}{P}{1}$ states and presented the work in \cite{din2018.PRA.98.042505}.

\section{Rydberg states of hyperfine split cores}

A modern paper is \cite{rob2018.PRA.97.022508}. 

To calculate the effects of hyperfine interactions on the Rydberg spectra of \Sr{87}, we follow previous works by **cite Esherick, Beigang, K T Lu, Robicheaux?** where we mass-scale the Rydberg levels from an isotope without nuclear spin (i.e., \Sr{88}) and then apply the hyperfine interaction to obtain the lines for \Sr{87}.
The theoretical approach presented below follows the one Shuhei Yoshida presents in \cite{din2018.PRA.98.042505}.

\section{Ionization limit of \Sr{87}}

Bosonic isotopes of strontium have no nuclear spin (i.e., $I=0$) meaning their ionization limits converge to a single $\nSLJ{5s^2}{2}{S}{1/2}$ state of \Srion{}{+}.
This is not the case for \Sr{87} which has $I=9/2$. 
Much like the case for alkali atoms, the nonzero nuclear spin of \Srion{87}{+} interacts with the spin of the remaining $5s$ electron and splits the $\nSLJ{5s}{2}{S}{1/2}$ into $F=4,5$ components.
The magnetic dipole interaction can be written as
\begin{equation}
	\ev{W} = A \ev{\hat{\bm{I}} \cdot \hat{\bm{S}}} = \frac{A}{2} \left(F\left(F+1\right) - I\left(I+1\right) - S\left(S+1\right)\right)
\end{equation}
where $\vec{\bm{F}} = \vec{\bm{J}} + \vec{\bm{I}}$ and $S=s=1/2$ since core electrons of \Srion{87}{+} form a closed shell. 
\citeauthor{sun1993.HI.78.241} measured the $F=4,5$ hyperfine splitting in \Srion{87}{+} to be $\Delta E = \SI{5002368.363(57)}{\kHz}$, meaning $A = \SI{-1000473.673+-0.011}{\kHz}$. 

Considering the reported \Sr{87} ionization limit of \SI{45932.2861(10)}{\per\cm} from \cite{bei1982.OC.42.19,san2010.JPCRD.39.033103} is well above that of \Sr{88}, this suggests that it's likely the ionization limit corresponding $\nSLJF{5s}{2}{S}{1/2}{4}$ state of \Srion{87}{+}.
Since the hyperfine shift of the $F=4$ state is $\Delta E_{F=4} = \flatfrac{A}{2}(\flatfrac{-11}{2})= \SI{45932.1943+-0.0010}{\per\cm}$.
This value is used for the analysis below.

\subsection{Mass-scaled ionization limit of \Sr{87}}

Alternatively, an ionization limit for \Sr{87} assuming $I=0$ can be obtained by mass-scaling the ionization limits of \Sr{88}, \Sr{86}, and \Sr{84}. 
Using the ionization values for the bosonic isotopes given in **cite appendix**, mass-scaling gives an ionization limit for \Sr{87} (assuming $I=0$) of **\SI{45932.19497+-0.00022}{\per\cm}** for \Sr{87} (assuming $I=0$) (see \cref{fig:Eion-mass-scaled}).
This value agrees well with the calculated value above for an assumed \Sr{87} with no nuclear spin.
\begin{figure}[htbp]
	\centering
	\includesvg[keepaspectratio=true, width=4in]{spectroscopy/Eion-mass-scaled.svg}
	\caption{\label{fig:Eion-mass-scaled}Mass-scaled $E_\text{ion}$.}
\end{figure}

\subsection{Incorporating hyperfine interaction}


**************
**************


**************
**************
**Work on this section about getting from 2-electron Hamiltonian to effective Rydberg Hamiltonian for 5sns states.**

The Hamiltonian for a two-electron system in an $\ket{(ms)(nl)}$ Rydberg state can be written as \cite{bei1983.PRL.51.771}
\begin{equation}
	H	= H_{0,1} + H_{0,2} + \frac{1}{r_{12}} + \xi_{nl} \vec{s} \cdot \vec{l} + a_{ms} \vec{s} \cdot \vec{I}
\end{equation}
where $H_{0,i}$ is the Coulomb attraction for the {$i$th} electron to the \Sr{++} core, $\flatfrac{1}{r_{12}}$ is the Coulomb repulsion between the two electrons, $\xi_{nl}$ is the magnitude of the spin-orbit interaction, and $a_{ms}$ is the magnetic dipole interaction of the inner $s$ electron.
For \Sr{88}, $I=0$ so we can write the Hamiltonian as
\begin{equation}
	H_{0}(88) = H
\end{equation}
where the appropriate mass-scaled values are used and includes the spin-orbit and spin-other-orbit interactions. 
The hyperfine interaction for \Sr{87} can be included as
\begin{equation}
	H(87) = H_{0}(88,87) + V_\text{hf}
\end{equation}

For two-electron systems, the hyperfine interaction can be expressed as
\begin{equation}
	V_\text{hf} = a_{ms} \left(\hat{\bm{s}}_{1} + \hat{\bm{s}}_{2}\right) \cdot \hat{\bm{I}} \simeq a_{ms} \hat{\bm{s}}_{\text{in}} \cdot \hat{\bm{I}}
\end{equation}
where $\hat{\bm{s}}_{i}$ is the spin of the {$i$th} electron. 

For singly-excited states, the hyperfine interaction of the ``outer'' electron with the core should scale as **$\sim 1/n^3$** and should be negligible for high-lying states. 
We make the assumption that $\hat{V}_\text{hf} \simeq a_{ms} \hat{\bm{s}}_{\text{in}} \cdot \hat{\bm{I}}$ for singly-excited Rydberg states with inner electron $\ket{ms}$ and Rydberg electron $\ket{nl}$.
Expanding $\hat{\bm{s}}_\text{in}$ and $\hat{\bm{I}}$ in terms of ladder operators gives
\begin{equation}
	\hat{V}_\text{hf} \simeq \frac{a_{ms}}{2} \left(\hat{s}_{\text{in},+} \hat{I}_{-} + \hat{s}_{\text{in},-} \hat{I}_{+} + 2\hat{s}_{\text{in},z} \hat{I}_{z}\right)
\end{equation}
with the usual ladder operators $J_{\pm}=J_{x} \pm J_{y}$. 

\subsection{$S$-states}

For $S$ states, the singlet states can be written as
\begin{equation}
	\ket{\nSLJ{5sns}{1}{S}{0}} = \frac{1}{2} \left(\ket{5s;ns} + \ket{ns;5s}\right) \left(\ket{\uparrow;\downarrow} - \ket{\downarrow;\uparrow}\right)
\end{equation}
Similarly, the three triplet $S$-states can be written as
\begin{align}
	\ket{\nSLJm{5sns}{3}{S}{1}{+1}}	&= \frac{1}{\sqrt{2}} \left(\ket{5s;ns} - \ket{ns;5s}\right) \ket{\uparrow;\uparrow}	\\
	\ket{\nSLJm{5sns}{3}{S}{1}{0}}	&= \frac{1}{2} \left(\ket{5s;ns} - \ket{ns;5s}\right) \left(\ket{\uparrow;\downarrow} + \ket{\downarrow;\uparrow}\right)	\\
	\ket{\nSLJm{5sns}{3}{S}{1}{-1}}	&= \frac{1}{\sqrt{2}} \left(\ket{5s;ns} - \ket{ns;5s}\right) \ket{\downarrow;\downarrow}
\end{align}
Due to the hyperfine interaction, $J$ and $I$ are no longer good quantum numbers so we use $F = J + I$.
Therefore, we can use the Clebsch–Gordan coefficients to determine basis states in terms of $\ket{F, m_F}$.
\begin{align}
	\ket{\nSLJFm{5sns}{3}{S}{1}{7/2}{m_F}}	&= \sum\limits_{-m_J}^{m_J} \sum\limits_{-m_I}^{m_I} \ket{\nSLJm{5sns}{3}{S}{1}{m_J}} \ket{I, m_I}
\end{align}

\subsection{$D$-states}

Stuff about D-states. 


Following similar derivations by.

\section{Experimental method}

We follow the usual techniques for producing cold gases of strontium **add references**. 
Briefly, starting from a Zeeman slowed atomic beam, \Sr{87} atoms are first cooled and trapped using the blue MOT.
The atoms are then further cooled in a narrow-line red MOT.
Approximately \num{1E6} atoms at \SI{\sim 2}{\mu\K} are captured before turning off all trapping fields for spectroscopy measurements

Rydberg atoms are created by two-photon excitation using counterpropagating cross-linearly polarized \SI{689}{\nm} and \SI{319}{\nm} excitation laser beams.
Depending on which intermediate $F^{\prime}$ state we use, this scheme allows us to produce Rydberg atoms in the $F^{\prime\prime} = F^{\prime} \pm \left\{0,1\right\}$
\begin{equation}
	\nSLJF{5s^2}{1}{S}{0}{9/2}
		\rightarrow \nSLJF{5s5p}{3}{P}{1}{F^{\prime}}
		\rightarrow	\begin{cases}
						\nSLJF{5sns}{3}{S}{1}{F^{\prime} \pm 0,1}	\\
						\nSLJF{5snd}{3}{D}{1,2,3}{F^{\prime} \pm 0,1}
					\end{cases}
\end{equation}
These intermediate states were selected to take advantage of selection rules to aid in identifying the Rydberg hyperfine states populated.
The typical detunings of the \SI{689}{\nm} laser were $\Delta_{9/2} \sim \SI{36}{MHz}$ and $\Delta_{11/2} \sim \SI{12}{\MHz}$. 
The \SI{689}{\nm} laser was chopped into **(10–20)-us-long** pulses to generate temporally localized groups of Rydberg atoms. 
The number of Rydberg atoms produced by each pulse was determined by using the electrodes in Fig. 4(c) to generate a ramped electric field sufficient to ionize the Rydberg atoms. 
The resulting electrons were directed towards, and detected by, a microchannel plate (MCP) whose output was fed into a multichannel scalar (MCS).
Typically 100–500 measurement cycles were performed before loading a new sample and changing the \SI{319}{\nm} laser frequency. 
Spectroscopic measurements at high $n$ using \Sr{84} showed that the stray fields in the trapping region were less than \SI{10}{\mV\per\cm}. 
Any resultant Stark shifts should therefore be at most a few megahertz even at $n \sim 90$.

\begin{figure}[htbp]
	\centering
	\includesvg[keepaspectratio=true, width=4in]{spectroscopy/exp_setup.svg}
	\caption{\label{fig:exp_setup}Experimental setup for performing \Sr{87} triplet Rydberg spectroscopy.}
\end{figure}

Additional details about how we scanned the UV laser frequency is given in \Cref{ap:scanning_uv_laser}.

\subsection{Calibrating the WA-1500}

Since we rely on the wavemeter to determine the absolute energy of the \SI{638}{\nm} photon, we want to characterize potential systematic offsets. 
We initially attempted to calibrate the wavemeter by looking for a Doppler-free spectrum of molecular iodine ({\tsup{127}I}) which has the {P65 (7–4)} line at \SI{15 672.517398 (25)}{\per\cm} \cite{san1997.JOSAB.14.1913}, close to the **$n=40$ Rydberg line in \Sr{88}**.
We didn't have much luck finding narrow Doppler-free signals scanning the \SI{638}{\nm} laser through a our iodine cell, matching similar experiences according to Jason and Henry in Randy Hulet's lab (they were looking to use the iodine cell to lock their \SI{646}{\nm} laser for a **\Li{7}** UV MOT \cite{dua2011.PRA.84.061406}). 
A longer cell and a more advanced detection scheme may have provided the signal we needed since \citeauthor{san1997.JOSAB.14.1913} used a \SI{30}{\cm} long cell and lock-in detection. A setup similar to the one in \cite{hua2018.AO.57.2102} may work as well.

Although the iodine spectroscopy didn't work out, we were able to obtain a calibration of our wavemeter's systematics by measuring wavelengths of lasers lock to atomic transitions in \Sr{88} (\SI{689}{\nm}) and \Li{6} (\SI{671}{\nm} \cite{san2011.PRL.107.023001,san2012.PRL.109.259901E} and \SI{646}{\nm} \cite{rad1995.PRA.52.4462}). 
As hinted at above, the \SI{646}{\nm} source is used by the Hulet lab for narrow line cooling of \Li{6} on the \SI{323}{\nm} transition.
Ya-Ting Chang, Danyel Cavazos, and Dr. Randy Hulet were kind enough to let us run a fiber between the Killian and Hulet labs and borrow some light from their system in order to calibrate our wavemeter. 

\begin{table}[htbp]
	\centering
	\makebox[\textwidth][c]{
	\begin{tabular}{@{}ccccccc@{}}
		\toprule
		$\lambda$ [\si{\nm}]	& Atom						& Lower state									& Upper state								& Measured [\si{\per\cm}]	& Reported [\si{\per\cm}]			& Ref																							\\
		\midrule
		\num{323}				& \multirow{2}{*}{\Li{6}}	& \multirow{2}{*}{$\nSLJF{2s}{2}{S}{1/2}{3/2}$}	& \multirow{2}{*}{$\nSLJ{3p}{2}{P}{3/2}$}	& \num{30925.1792(9)}		& \num{30925.1703(10)}				& \multirow{2}{*}{\cite{rad1995.PRA.52.4462,san2011.PRL.107.023001,san2012.PRL.109.259901E}}	\\
		(\num{646})				&							&												&											& (\num{15462.5896(5)})		& (\num{15462.5851(5)})				&                                                       										\\
		\num{671}				& \Li{6}					& $\nSLJF{2s}{2}{S}{1/2}{3/2}$					& $\nSLJ{2p}{2}{P}{3/2}$					& \num{14903.63391(33)}		& \num{14903.6320617+-0.0000007}	& \cite{san2011.PRL.107.023001,san2012.PRL.109.259901E}											\\
		\num{689}				& \Sr{88}					& $\nSLJ{5s^2}{1}{S}{0}$						& $\nSLJ{5s5p}{3}{P}{1}$					& \num{14504.34224(35)}		& \num{14504.33824159+-0.00000033}	& \cite{fer2003.PRL.91.243002,san2010.JPCRD.39.033103}											\\
		\bottomrule
	\end{tabular}
	}
	\caption{\label{tab:wavemeter}Values used to calibrate the WA-1500 on a single day (2018/03/09), within ~2 hours, to reduce day-to-day environmental changes)**. Reported values for \Li{6} were given with respect to the center-of-gravity of the lower $\nSLJ{5s}{2}{S}{1/2}$ state so I included the shift to the $F=3/2$ state. **Still need to calculate HF shifted values for \Li{6}.**}
\end{table}

\Cref{fig:wm_fit} shows the difference between our measured energies and the reported energies. 
To check for slow drifts in the wavemeter (e.g., due to environmental changes), we also measured the wavelength of the \SI{689}{\nm} for each measurement of the \SI{638}{\nm} photon energy.
\Cref{fig:wm_drift} shows the deviation of the \SI{689}{\nm} measurements overtime, showing relatively little long-term drifts. 

\begin{figure}[htbp]
	\centering
	\includesvg[keepaspectratio=true, width=4in]{spectroscopy/wavemeter_calibration/wm_fit.svg}
	\caption{\label{fig:wm_fit}Linear fit to find wavemeter systematic offset.}
\end{figure}

\begin{figure}[htbp]
	\centering
	\includesvg[keepaspectratio=true, width=4in]{spectroscopy/wavemeter_calibration/wm_drift.svg}
	\caption{\label{fig:wm_drift}Long-term wavemeter drifts.}
\end{figure}

**Add footnote: at the time, the best reference for the \SI{323}{\nm} transition we could find was by \citeauthor{rad1995.PRA.52.4462} although there should be some upcoming measurements using frequency combs (cite papers mentioning doing this)**

\subsection{Two-photon Rydberg excitation spectrum}

While taking the Rydberg spectra, we observed spurious electron counts when the UV laser detuning compensated for the \SI{689}{\nm} detuning. 
Due to this coincidence, we believe these ``ghost'' lines are due to atoms in the $\nSLJF{5s5p}{3}{P}{1}{9/2,11/2}$ state which were off-resonantly excited by a \SI{689}{\nm} photon before a \SI{319}{\nm} photon excites them to a Rydberg state. 
To check for this effect, we use the density matrix method **Lindblad operator** to simulate the time-dependent dynamics with parameter values estimated from the measured excitation laser powers and beam waists. 

**Show example spectra with ``ghost'' on-resonance counts.**

\begin{figure}[htbp]
	\centering
	\includesvg[keepaspectratio=true, height=2in]{spectroscopy/three-level_system.svg}
	\caption{\label{fig:three-level_system}Three-level diagram representing a two-photon excitation to a Rydberg state with $\ket{g}$, $\ket{e}$, and $\ket{r}$ representing the ground, intermediate, and Rydberg states, respectively. The detunings from the intermediate (Rydberg) state is $\Delta_e$ ($\Delta_r$). Couplings between states are represented by $\Omega_1 \exp(-\iu \omega_1 t)$ ($\Omega_2 \exp(-\iu \omega_2 t)$) for the \SI{689}{\nm} (\SI{319}{\nm}) lasers.}
\end{figure}

The bare atomic Hamiltonian for this system can be expressed as 
\begin{equation}
	H_\text{A}
		= \omega_r \dyad{r} + \hbar \omega_e \dyad{e} + \hbar \omega_g \dyad{g}
		= \hbar	\begin{pmatrix}
					\omega_r	& 0			& 0	\\
					0			& \omega_e	& 0	\\
					0			& 0			& \omega_g
				\end{pmatrix}
\end{equation}
The atom-field interaction can be written as 
\begin{equation}
	H_\text{AF}
		= \frac{\hbar}{2}\left(\Omega_1 \operatorname{e}^{-\iu \omega_1 t} \dyad{g}{e} + \Omega_2 \operatorname{e}^{-\iu \omega_2 t} \dyad{e}{r} \right)+ \text{h.c}
		=	\frac{\hbar}{2}	\begin{pmatrix}
				0												& \Omega_2 \operatorname{e}^{-\iu \omega_2 t}		& 0	\\
				\Omega_2^\ast \operatorname{e}^{\iu \omega_2 t}	& 0													& \Omega_1 \operatorname{e}^{-\iu \omega_1 t}	\\
				0												& \Omega_1^\ast \operatorname{e}^{\iu \omega_1 t}	& 0
			\end{pmatrix}
\end{equation}
Using the unitary (?) transformation
\begin{equation}
	U	=	\begin{pmatrix}
				\operatorname{e}^{-\iu (\Omega_1 + \Omega_2) t}	& 0													& 0	\\
				0												& \operatorname{e}^{-\iu \Omega_1 t}					& 0	\\
				0												& 0													& 1
			\end{pmatrix}
\end{equation}
the total Hamiltonian $H = H_\text{A} + H_\text{AF}$ can be written as
\begin{equation}
	\widetilde{H}
		= U^\dag H U - \iu \hbar U^\dag \dv{U}{t}
		= \hbar	\begin{pmatrix}
					-\Delta_r						& \flatfrac{\Omega_2}{2}		& 0			\\
					\flatfrac{\Omega_2^\ast}{2}		& -\Delta_e						& \flatfrac{\Omega_1}{2}	\\
					0								& \flatfrac{\Omega_1^\ast}{2}	& \omega_g
				\end{pmatrix}
\end{equation}
where $\Delta_e = \omega_1 - \omega_e$ is the detuning from the intermediate state and $\Delta_r = \omega_r - (\omega_1 + \omega_2)$ is the total detuning from the Rydberg state. 
To simplify things, we can take $\omega_g = 0$.

**Hopefully the time-dynamics calculations explain the observed ``ghost'' lines**.

\section{Results}

The measured state energies of the \Sr{87} Rydberg states are presented in **tables xx, yy** which include the systematic uncertainties in the wavemeter calibration.
Although our state energies have an uncertainty of about **xxx MHz**, we can measure splittings of up to a few GHz with kHz level accuracies limited by the synthesizer. 

\begin{center}
	\begin{longtable}[c]{@{}ccccccccc@{}}
		\toprule
		Series	& $n$					& Term				& $F$		& $E_\text{exp}$ [\si{\per\cm}]	& $\Delta E_\text{exp}$ [\si{\GHz}]	& $E_\text{th}$	[\si{\per\cm}]	& $\Delta E_\text{th}$ [\si{\GHz}]	\\
		\midrule
		$5sns$	& $40$					& $\SLJ{1}{S}{0}$	& $9/2$		& \num{45850.8762+-0.0021}		& \num{16.35+-0.08}					& \num{45850.8702}				& \num{16.22}						\\
				& $60$					& 					&			& \num{45898.1444+-0.0022}		& \num{7.28+-0.09}					& \num{45898.1421}				& \num{7.26}						\\
				& $72$					& 					&			& \num{45909.0252+-0.0020}		& \num{6.10+-0.09}					& \num{45909.0240}				& \num{6.1}							\\
				& $74$					& 					&			& \num{45910.3230+-0.0021}		& \num{5.98+-0.09}					& \num{45910.3211}				& \num{5.99}						\\
				& $76$					& 					&			& \num{45911.5148+-0.0020}		& \num{5.91+-0.08}					& \num{45911.5127}				& \num{5.89}						\\
				& $77$					& 					&			& \num{45912.0738+-0.0020}		& \num{5.84+-0.09}					& \num{45912.0725}				& \num{5.85}						\\
				& $78$					& 					&			& \num{45912.6114+-0.0020}		&									& \num{45912.6100}				& \num{5.81}						\\
				& $82$					& 					&			& \num{45914.5606+-0.0022}		& \num{5.66+-0.09}					& \num{45914.5589}				& \num{5.67}						\\
				& $86$					& 					&			& \num{45916.2336+-0.0021}		& \num{5.56+-0.08}					& \num{45916.2321}				& \num{5.56}						\\
				& $90$					& 					&			& \num{45917.6802+-0.0019}		& \num{5.46+-0.08}					& \num{45917.6791}				& \num{5.47}						\\
				& $94$					& 					&			& \num{45918.9402+-0.0019}		& \num{5.40+-0.08}					& \num{45918.9388}				& \num{5.39}						\\
				& $98$					& 					&			& \num{45920.0438+-0.0022}		& \num{5.325+-0.005}				& \num{45920.0423}				& \num{5.327}						\\
		\midrule
		$5sns$	& $40$					& $\SLJ{3}{S}{1}$	& $7/2$		& \num{45850.4974+-0.0021}		& \num{4.99+-0.08}					& \num{45850.4960}				& \num{5.0}							\\
				& $60$					& 					&			& \num{45898.0688+-0.0021}		& \num{5.02+-0.08}					& \num{45898.0668}				& \num{5.0}							\\
		\midrule
		$5sns$	& $40$					& $\SLJ{3}{S}{1}$	& $9/2$		& \num{45850.4078+-0.0021}		& \num{2.31+-0.08}					& \num{45850.4061}				& \num{2.31}						\\
				& $50$					& 					&			& \num{45881.7138+-0.0022}		& \num{1.88+-0.09}					& \num{45881.7119}				& \num{1.89}						\\
				& $72$					& 					&			& \num{45908.8546+-0.0021}		& \num{0.99+-0.09}					& \num{45908.8528}				& \num{0.97}						\\
				& $74$					& 					&			& \num{45910.1518+-0.0022}		& \num{0.85+-0.09}					& \num{45910.1516}				& \num{0.91}						\\
				& $76$					& 					&			& \num{45911.3460+-0.0019}		& \num{0.85+-0.08}					& \num{45911.3445}				& \num{0.85}						\\
				& $77$					& 					&			& \num{45911.9068+-0.0021}		& \num{0.83+-0.09}					& \num{45911.9049}				& \num{0.83}						\\
				& $78$					& 					&			& \num{45912.4444+-0.0019}		&									& \num{45912.4429}				& \num{0.8}							\\
				& $82$					& 					&			& \num{45914.3958+-0.0021}		& \num{0.72+-0.09}					& \num{45914.3935}				& \num{0.71}						\\
				& $86$					& 					&			& \num{45916.0696+-0.0021}		& \num{0.64+-0.08}					& \num{45916.0677}				& \num{0.63}						\\
				& $90$					& 					&			& \num{45917.5172+-0.0021}		& \num{0.57+-0.08}					& \num{45917.5155}				& \num{0.56}						\\
				& $94$					& 					&			& \num{45918.7774+-0.0022}		& \num{0.52+-0.09}					& \num{45918.7759}				& \num{0.51}						\\
				& $98$					& 					&			& \num{45919.8816+-0.0022}		& \num{0.46302+-0.00007}			& \num{45919.8800}				& \num{0.46164}						\\
		\midrule
		$5sns$	& $30$					& $\SLJ{3}{S}{1}$	& $11/2$	& \num{45777.3637+-0.0020}		&									& \num{45777.3621}				& 									\\
				& $31$					& 					&			& \num{45788.3644+-0.0021}		&									& \num{45788.3624}				& 									\\
				& $32$					& 					&			& \num{45798.2325+-0.0022}		&									& \num{45798.2302}				& 									\\
				& $33$					& 					&			& \num{45807.1179+-0.0019}		&									& \num{45807.1158}				& 									\\
				& $34$					& 					&			& \num{45815.1469+-0.0021}		&									& \num{45815.1452}				& 									\\
				& $35$					& 					&			& \num{45822.4253+-0.0021}		&									& \num{45822.4252}				& 									\\
				& $36$					& 					&			& \num{45829.0469+-0.0020}		&									& \num{45829.0460}				& 									\\
				& $37$					& 					&			& \num{45835.0865+-0.0021}		&									& \num{45835.0851}				& 									\\
				& $38$					& 					&			& \num{45840.6098+-0.0014}		&									& \num{45840.6085}				& 									\\
				& $39$					& 					&			& \num{45845.6759+-0.0022}		&									& \num{45845.6734}				& 									\\
				& $40$					& 					&			& \num{45850.3308+-0.0015}		&									& \num{45850.3291}				& 									\\
				& $42$					& 					&			& \num{45858.5807+-0.0021}		&									& \num{45858.5793}				& 									\\
				& $43$					& 					&			& \num{45862.2455+-0.0020}		&									& \num{45862.2439}				& 									\\
				& $44$					& 					&			& \num{45865.6435+-0.0021}		&									& \num{45865.6413}				& 									\\
				& $45$					& 					&			& \num{45868.7988+-0.0015}		&									& \num{45868.7968}				& 									\\
				& $49$					& 					&			& \num{45879.4140+-0.0019}		&									& \num{45879.4124}				& 									\\
				& $50$					& 					&			& \num{45881.6510+-0.0021}		&									& \num{45881.6488}				& 									\\
				& $55$					& 					&			& \num{45890.9526+-0.0020}		&									& \num{45890.9511}				& 									\\
				& $60$					& 					&			& \num{45897.9014+-0.0019}		&									& \num{45897.9000}				& 									\\
				& $65$					& 					&			& \num{45903.2294+-0.0019}		&									& \num{45903.2272}				& 									\\
				& $72$					& 					&			& \num{45908.8216+-0.0022}		&									& \num{45908.8205}				& 									\\
				& $74$					& 					&			& \num{45910.1236+-0.0022}		&									& \num{45910.1213}				& 									\\
				& $76$					& 					&			& \num{45911.3178+-0.0019}		&									& \num{45911.3161}				& 									\\
				& $77$					& 					&			& \num{45911.8790+-0.0021}		&									& \num{45911.8774}				& 									\\
				& $82$					& 					&			& \num{45914.3718+-0.0022}		&									& \num{45914.3699}				& 									\\
				& $86$					& 					&			& \num{45916.0482+-0.0019}		&									& \num{45916.0467}				& 									\\
				& $90$					& 					&			& \num{45917.4982+-0.0019}		&									& \num{45917.4967}				& 									\\
				& $94$					& 					&			& \num{45918.7600+-0.0021}		&									& \num{45918.7590}				& 									\\
				& $98$					& 					&			& \num{45919.8662+-0.0022}		&									& \num{45919.8646}				& 									\\
				& $99$					& 					&			& \num{45920.1210+-0.0022}		&									& \num{45920.1196}				& 									\\
		\bottomrule
		\caption{\label{tab:data-s-state}Experimentally measured and calculated energies of selected $\nSLJ{5sns}{1}{S}{0}$ and $\nSLJ{5sns}{3}{S}{1}$ states in \Sr{87}. $\Delta E_\text{exp}$ and $\Delta E_\text{th}$ are the measured and predicted separations from the $\nSLJF{5sns}{3}{S}{1}{11/2}$ state of the same $n$ which is used as a reference. The uncertainties shown include both the statistical and systematic uncertainties in the wavemeter calibration.}
	\end{longtable}
\end{center}

**Table 2 of splittings 5snd states**