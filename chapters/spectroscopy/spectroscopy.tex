\chapter{Spectroscopy of \Sr{87} triplet Rydberg states}

We didn't set out to map out the triplet series of \Sr{87} Rydberg states, but when we first started looking for \Sr{87} \SLJ{3}{S}{} and \SLJ{3}{D}{} Rydberg states, we observed spectra which didn't line up with their predicted locations using the known quantum defects. 
It turns out that the hyperfine interaction in \Sr{87} is not negligible as the Rydberg states approach the \Srion{87}{+} hyperfine split core. 
We then rediscovered some papers from the 1980's which explained the shifted \SLJ{1}{S}{0} spectra in \Sr{87} by including the mixing of states with the same $F$. 
Previous work was accomplished mostly through the \nSLJ{5s5p}{1}{P}{1} state or from the metastable \nSLJ{5s5p}{3}{P}{2} states, our work appears to be the first measurements of the \Sr{87} hyperfine Rydberg states through the \nSLJ{5s5p}{3}{P}{1} intermediate state. 

We measured \Sr{87} hyperfine Rydberg spectra using a two-photon excitation through the intermediate $\nSLJ{5s5p}{3}{P}{1}$ states and presented the work in \cite{din2018.PRA.98.042505}.

\section{Ionization limit of \Sr{87}}

\section{Rydberg states of hyperfine split cores}

A modern paper is \cite{rob2018.PRA.97.022508}. 

To calculate the effects of hyperfine interactions on the Rydberg spectra of \Sr{87}, we follow previous works by **cite Esherick, Beigang, K T Lu, Robicheaux?** where we mass-scale the Rydberg levels from an isotope without nuclear spin (i.e., \Sr{88}) and then apply the hyperfine interaction to obtain the lines for \Sr{87}.
The theoretical approach presented below follows the one Shuhei Yoshida presents in \cite{din2018.PRA.98.042505}.

\subsection{Incorporating hyperfine interaction}

**Work on this section about getting from 2-electron Hamiltonian to effective Rydberg Hamiltonian for 5sns states.**

The Hamiltonian for a two-electron system in an $\ket{(ms)(nl)}$ Rydberg state can be written as \cite{bei1983.PRL.51.771}
\begin{equation}
	H	= H_{0,1} + H_{0,2} + \frac{1}{r_{12}} + \xi_{nl} \vec{s} \cdot \vec{l} + a_{ms} \vec{s} \cdot \vec{I}
\end{equation}
where $H_{0,i}$ is the Coulomb attraction for the {$i$th} electron to the \Sr{++} core, $\flatfrac{1}{r_{12}}$ is the Coulomb repulsion between the two electrons, $\xi_{nl}$ is the magnitude of the spin-orbit interaction, and $a_{ms}$ is the magnetic dipole interaction of the inner $s$ electron.
For \Sr{88}, $I=0$ so we can write the Hamiltonian as
\begin{equation}
	H_{0}(88) = H
\end{equation}
where the appropriate mass-scaled values are used and includes the spin-orbit and spin-other-orbit interactions. 
The hyperfine interaction for \Sr{87} can be included as
\begin{equation}
	H(87) = H_{0}(88,87) + V_\text{hf}
\end{equation}

For two-electron systems, the hyperfine interaction can be expressed as
\begin{equation}
	V_\text{hf} = a_{ms} \left(\hat{\bm{s}}_{1} + \hat{\bm{s}}_{2}\right) \cdot \hat{\bm{I}} \simeq a_{ms} \hat{\bm{s}}_{\text{in}} \cdot \hat{\bm{I}}
\end{equation}
where $\hat{\bm{s}}_{i}$ is the spin of the {$i$th} electron. 

For singly-excited states, the hyperfine interaction of the ``outer'' electron with the core should scale as **$\sim 1/n^3$** and should be negligible for high-lying states. 
We make the assumption that $\hat{V}_\text{hf} \simeq a_{ms} \hat{\bm{s}}_{\text{in}} \cdot \hat{\bm{I}}$ for singly-excited Rydberg states with inner electron $\ket{ms}$ and Rydberg electron $\ket{nl}$.
Expanding $\hat{\bm{s}}_\text{in}$ and $\hat{\bm{I}}$ in terms of ladder operators gives
\begin{equation}
	\hat{V}_\text{hf} \simeq \frac{a_{ms}}{2} \left(\hat{s}_{\text{in},+} \hat{I}_{-} + \hat{s}_{\text{in},-} \hat{I}_{+} + 2\hat{s}_{\text{in},z} \hat{I}_{z}\right)
\end{equation}
with the usual ladder operators $J_{\pm}=J_{x} \pm J_{y}$. 

\subsection{$S$-states}

For $S$ states, the singlet states can be written as
\begin{equation}
	\ket{\nSLJ{5sns}{1}{S}{0}} = \frac{1}{2} \left(\ket{5s;ns} + \ket{ns;5s}\right) \left(\ket{\uparrow;\downarrow} - \ket{\downarrow;\uparrow}\right)
\end{equation}
Similarly, the three triplet $S$-states can be written as
\begin{align}
	\ket{\nSLJm{5sns}{3}{S}{1}{+1}}	&= \frac{1}{\sqrt{2}} \left(\ket{5s;ns} - \ket{ns;5s}\right) \ket{\uparrow;\uparrow}	\\
	\ket{\nSLJm{5sns}{3}{S}{1}{0}}	&= \frac{1}{2} \left(\ket{5s;ns} - \ket{ns;5s}\right) \left(\ket{\uparrow;\downarrow} + \ket{\downarrow;\uparrow}\right)	\\
	\ket{\nSLJm{5sns}{3}{S}{1}{-1}}	&= \frac{1}{\sqrt{2}} \left(\ket{5s;ns} - \ket{ns;5s}\right) \ket{\downarrow;\downarrow}
\end{align}
Due to the hyperfine interaction, $J$ and $I$ are no longer good quantum numbers so we use $F = J + I$.
Therefore, we can use the Clebsch–Gordan coefficients to determine basis states in terms of $\ket{F, m_F}$.
\begin{align}
	\ket{\nSLJFm{5sns}{3}{S}{1}{7/2}{m_F}}	&= \sum\limits_{-m_J}^{m_J} \sum\limits_{-m_I}^{m_I} \ket{\nSLJm{5sns}{3}{S}{1}{m_J}} \ket{I, m_I}
\end{align}

\subsection{$D$-states}

Stuff about D-states. 


Following similar derivations by.

\section{Experiment}

\subsection{Two-photon Rydberg excitation}

We excite atoms to the Rydberg state using a two-photon transition.

$\nSLJF{5s^2}{1}{S}{0}{9/2}$

\begin{figure}[htbp]
	\centering
	\includesvg[keepaspectratio=true, width=1in]{spectroscopy/three-level_system.svg}
	\caption{\label{fig:three-level_system}Three-level diagram representing a two-photon excitation to a Rydberg state.}
\end{figure}

The bare atomic Hamiltonian for this system can be expressed as 
\begin{equation}
	H_\text{A}
		= \omega_r \dyad{r} + \hbar \omega_e \dyad{e} + \hbar \omega_g \dyad{g}
		= \hbar	\begin{pmatrix}
					\omega_r	& 0			& 0	\\
					0			& \omega_e	& 0	\\
					0			& 0			& \omega_g
				\end{pmatrix}
\end{equation}
The atom-field interaction can be written as 
\begin{equation}
	H_\text{AF}
		= \frac{\hbar}{2}\left(\Omega_1 \operatorname{e}^{-\iu \omega_1 t} \dyad{g}{e} + \Omega_2 \operatorname{e}^{-\iu \omega_2 t} \dyad{e}{r} \right)+ \text{h.c}
		=	\frac{\hbar}{2}	\begin{pmatrix}
				0												& \Omega_2 \operatorname{e}^{-\iu \omega_2 t}		& 0	\\
				\Omega_2^\ast \operatorname{e}^{\iu \omega_2 t}	& 0													& \Omega_1 \operatorname{e}^{-\iu \omega_1 t}	\\
				0												& \Omega_1^\ast \operatorname{e}^{\iu \omega_1 t}	& 0
			\end{pmatrix}
\end{equation}
Using the unitary (?) transformation
\begin{equation}
	U	=	\begin{pmatrix}
				\operatorname{e}^{-\iu (\Omega_1 + \Omega_2) t}	& 0													& 0	\\
				0												& \operatorname{e}^{-\iu \Omega_1 t}					& 0	\\
				0												& 0													& 1
			\end{pmatrix}
\end{equation}
the total Hamiltonian $H = H_\text{A} + H_\text{AF}$ can be written as
\begin{equation}
	\widetilde{H}
		= U^\dag H U - \iu \hbar U^\dag \dv{U}{t}
		= \hbar	\begin{pmatrix}
					-\Delta_r						& \flatfrac{\Omega_2}{2}		& 0			\\
					\flatfrac{\Omega_2^\ast}{2}		& -\Delta_e						& \flatfrac{\Omega_1}{2}	\\
					0								& \flatfrac{\Omega_1^\ast}{2}	& \omega_g
				\end{pmatrix}
\end{equation}
where $\Delta_e = \omega_1 - \omega_e$ is the detuning from the intermediate state and $\Delta_r = \omega_r - (\omega_1 + \omega_2)$ is the total detuning from the Rydberg state. 
To simplify things, we can take $\omega_g = 0$.

\section{Experimental setup}

Stuff about doing the experiment. 

\subsection{Calibrating the WA-1500}

Stuff about calibrating the EXFO WA-1500 wavemeter. 

\section{Results}

Show results here. 