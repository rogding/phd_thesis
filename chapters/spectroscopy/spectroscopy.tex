\chapter{Spectroscopy of \Sr{87} triplet Rydberg states}

We didn't set out to map out the triplet series of \Sr{87} Rydberg states, but when we first started looking for \Sr{87} \SLJ{3}{S}{} and \SLJ{3}{D}{} Rydberg states, we observed spectra which didn't line up with their predicted locations using the known quantum defects. 
It turns out that the hyperfine interaction in \Sr{87} is not negligible as the Rydberg states approach the \Srion{87}{+} hyperfine split core. 
We then rediscovered some papers from the 1980's which explained the shifted \SLJ{1}{S}{0} spectra in \Sr{87} by including the mixing of states with the same $F$. 
Previous work was accomplished mostly through the \nSLJ{5s5p}{1}{P}{1} state or from the metastable \nSLJ{5s5p}{3}{P}{2} states, our work appears to be the first measurements of the \Sr{87} hyperfine Rydberg states through the \nSLJ{5s5p}{3}{P}{1} intermediate state. 

\section{Rydberg states of hyperfine split cores}

A modern paper is \cite{rob2018.PRA.97.022508}. 




Following similar derivations by.

\section{Experiment}

\subsection{Two-photon Rydberg excitation}

We excite atoms to the Rydberg state using a two-photon transition.

$\nSLJF{5s^2}{1}{S}{0}{9/2}$

\begin{figure}[htbp]
	\centering
	\includesvg[keepaspectratio=true, width=1in]{spectroscopy/three-level_system.svg}
	\caption{\label{fig:three-level_system}Three-level diagram representing a two-photon excitation to a Rydberg state.}
\end{figure}

The bare atomic Hamiltonian for this system can be expressed as 
\begin{equation}
	H_\text{A}
		= \omega_r \dyad{r} + \hbar \omega_e \dyad{e} + \hbar \omega_g \dyad{g}
		= \hbar	\begin{pmatrix}
					\omega_r	& 0			& 0	\\
					0			& \omega_e	& 0	\\
					0			& 0			& \omega_g
				\end{pmatrix}
\end{equation}
The atom-field interaction can be written as 
\begin{equation}
	H_\text{AF}
		= \frac{\hbar}{2}\left(\Omega_1 \operatorname{e}^{-\iu \omega_1 t} \dyad{g}{e} + \Omega_2 \operatorname{e}^{-\iu \omega_2 t} \dyad{e}{r} \right)+ \text{h.c}
		=	\frac{\hbar}{2}	\begin{pmatrix}
				0												& \Omega_2 \operatorname{e}^{-\iu \omega_2 t}		& 0	\\
				\Omega_2^\ast \operatorname{e}^{\iu \omega_2 t}	& 0													& \Omega_1 \operatorname{e}^{-\iu \omega_1 t}	\\
				0												& \Omega_1^\ast \operatorname{e}^{\iu \omega_1 t}	& 0
			\end{pmatrix}
\end{equation}
Using the unitary (?) transformation
\begin{equation}
	U	=	\begin{pmatrix}
				\operatorname{e}^{-\iu (\Omega_1 + \Omega_2) t}	& 0													& 0	\\
				0												& \operatorname{e}^{-\iu \Omega_1 t}					& 0	\\
				0												& 0													& 1
			\end{pmatrix}
\end{equation}
the total Hamiltonian $H = H_\text{A} + H_\text{AF}$ can be written as
\begin{equation}
	\widetilde{H}
		= U^\dag H U - \iu \hbar U^\dag \dv{U}{t}
		= \hbar	\begin{pmatrix}
					-\Delta_r						& \flatfrac{\Omega_2}{2}		& 0			\\
					\flatfrac{\Omega_2^\ast}{2}		& -\Delta_e						& \flatfrac{\Omega_1}{2}	\\
					0								& \flatfrac{\Omega_1^\ast}{2}	& \omega_g
				\end{pmatrix}
\end{equation}
where $\Delta_e = \omega_1 - \omega_e$ is the detuning from the intermediate state and $\Delta_r = \omega_r - (\omega_1 + \omega_2)$ is the total detuning from the Rydberg state. 
To simplify things, we can take $\omega_g = 0$.
