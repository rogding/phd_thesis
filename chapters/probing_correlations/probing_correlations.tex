\chapter{Vibrationally-Excited Rydberg Molecules in Cold Gases of \Sr{87}}
\label{ch:correlations}

Many interesting phenomena in quantum systems have length scales associated with their underlying interactions.
Perhaps one of the most fundamental results from either Bose-Einstein or Fermi-Dirac statistics which lead to the bunching or antibunching of particles \cite{Naraschewski1999.PRA.59.4595}.
Another interesting phenomena is the ``fermionization'' of strongly-interacting bosons in a one-dimensional gas \cite{Kinoshita2004.Science.305.1125}.
In these examples, the interactions affect the likelihood of finding one particle within a fixed distance of another, i.e., they become spatially correlated.
As a result, tools that provide information about the spatial correlations in such systems are extremely valuable and can furnish new insights into the behavior of quantum gases.

Possibly one of the most ingenious ways of probing correlations due to quantum statistics was demonstrated with metastable helium atoms.
By dropping both ultracold and quantum degenerate Bose and Fermi gases onto a time-resolved position-sensitive MCP detector, the two-body and three-body spatial correlations of the dropped cloud (or the momentum correlations of the initial cloud) were able to be reconstructed \cite{Schellekens2005.Science.310.648, Jeltes2007.Nature.445.402, Hodgman2011.Science.331.1046}.
Although powerful, this techniques requires atoms to be in a metastable state with enough energy to trigger the MCP channels. 
It is also not \textit{in situ}, needing the samples to be at equilibrium prior to dropping in order for the reconstruction to make sense, thus limiting the possibility of observing the temporal development of spatial correlations. 

Alternative, less destructive, \textit{in situ} techniques have also been developed and employed but these are generally restricted by the dimensionality of a system or the length scales accessible.
For the shortest length scales (i.e., $R = 0$ contact interactions), inelastic loss due to three-body recombination has been shown to be sensitive to the effects of $g^{\pqty*{3}}\pqty*{R}$ as evidenced by the different loss rates observed in nondegenerate compared to degenerate Bose gases \cite{Burt1997.PRL.79.337}. 
For length scales comparable to the sizes of molecular bonds (i.e., $R \sim R_{\text{vdW}} \lesssim \SI{10}{\nm}$), photoassociative spectroscopy (PAS) can be used 
since efficient photoassociation requires the constituent atoms to have a separation distance comparable to the range of the molecular potential (the van der Waals length $R_{\text{vdW}}$) \cite{Jones2006.RMP.78.483, Jones2006.RMP.78.1041}.
If the effect can be observed in a two- (or one-) dimensional lattice structure, quantum gas microscopes (QGM) have the capability of directly detecting the presence or absence of atoms on a particular lattice site and, therefore, are able to directly measure spatial correlations \cite{Bakr2010.Science.329.547, Omran2015.PRL.115.263001, Mazurenko2017.Nature.545.462}.
Since QGMs require an optical lattice, their accessible length scales are generally restricted by the wavelength of light used to form the lattice ($R \gtrsim \SI{266}{\nm}$).
\Cref{fig:correlation_tools} illustrates some of the tools and their typical effective length scales.
\begin{figure}[!htbp]
	\centering
	\includegraphics[keepaspectratio, width=5in, height=\textheight]{probing_correlations/correlation_tools/correlation_length_scales.pdf}
	\caption[]{
		\label{fig:correlation_tools}
		A (non-exhaustive) list of lengths scales in typical ultracold gas experiments.
		Black lines show the length scales associated with various interactions and red lines illustrate the effective ranges of various techniques.
		The blue line indicates the length scales accessible with ULRRMs at different values of principal quantum numbers ($n$).}
\end{figure}

As seen in \cref{fig:correlation_tools}, the techniques mentioned above cannot probe length scales $\SI{10}{\nm} \lesssim R \lesssim \SI{266}{\nm}$.
Such length scales, however, can be explored with ULRRMs. 
Analogous to how PAS can be used to probe correlations on the order of the molecular potential $R_{\text{vdW}}$, the formation rate of ULRRMs can be applied to probe spatial correlations on the length scale of the ULRRM molecular potential. 
The advantage of ULRRMs is that the ULRRM molecular potential is related to the size of the parent Rydberg atom, allowing many different interparticle separations to be explored by simply varying the principal quantum number.
This tunability gives ULRRMs a huge advantage, allowing them to probe particle correlations on length scales previously inaccessible using alternate techniques. 

This chapter contains work that is currently in progress.
As such, the presented data, analysis, and results should be considered preliminary.

%******* ADD MORE ??? *******

\section{Probing of Spatial Correlations with ULRRMs}

In an ideal noninteracting gas, the atoms are expected to exhibit quantum statistical effects in their density distributions depending on whether they are bosonic, fermionic, or classical.
For the two-body case, this effect on the density distribution is typically characterized by the pair correlation function ${g}^{\pqty*{2}}\pqty*{R}$ and can be written (for the ideal noninteracting case) as
\begin{equation}
	{g}^{\pqty{2}}\pqty{R}	=	1 + \epsilon {\me}^{-2 \pi R^{2}/{\lambda_{\text{dB}}}^{2}}
\end{equation}
where {$\epsilon = +1$, $-1$, and $0$} for bosons, fermions, and classical gases, respectively, and $\lambda_{\text{dB}} = {h}/{\sqrt{2 \pi m k_{B} T}}$ is the thermal de Broglie wavelength.
For a more detailed discussion, see \cite{Naraschewski1999.PRA.59.4595}\footnote{I found the derivations in \cite{Schellekens2007.PhD, Jeltes2008.PhD, Krachmalnicof2009.PhD} to be helpful.}.

It was shown in \cite{Whalen2019.PRA.100.011402} that the excitation rate of ULRRMs is sensitive to spatial correlations on length scales of the Rydberg electron orbital radius, i.e., $R \sim R_{n} \approx 2 \pqty*{n-\delta}^2$. 
In particular, these studies were performed using the most deeply-bound vibrational ($\nu = 0$) dimer state which is spatially well-localized in the outer lobe of the Rydberg electron wave function as seen in \cref{fig:vary_n-nu0}.
\begin{figure}[!htbp]
	\centering
	\includegraphics[keepaspectratio, width=\textwidth, height=\textheight]{probing_correlations/correlations/vary_n/vary_n-nu0.pdf}
	\caption[]{
		\label{fig:vary_n-nu0}
		Calculated radial molecular potentials of a $\nSLJ{5sns}{3}{S}{1}+\nSLJ{5s^2}{1}{S}{0}$ ULRRM and the $\nu = 0$ radial vibrational wave functions for {$n = 30$, $35$, and $45$}.
		As seen in the figure, the position of the well-localized $\nu = 0$ vibrational state depends on $n$.
		The binding energy of the $\nu = 0$ states is also shown with the ${\pqty*{n^*}^{-6}}$ scaling (dotted line).}
\end{figure}
Since the state is radially well-localized, by varying $n$ (and therefore $R_n$) the optimum interatomic pair separation between atoms for formation of $\nu = 0$ ULRRMs can be tuned.
Therefore, differences in the initial spatial correlations between bosonic, fermionic, and classical gases can be observed as changes in the ULRRM excitation rates.

The probability of exciting a Rydberg molecule in an ultracold gas can be modeled similarly to photoassociation processes \cite{Abraham1996.PRA.R3713, Jones2006.RMP.78.483, Jones2006.RMP.78.1041} and a schematic is outlined in \cref{fig:ulrrm_photoassociation}.
\begin{figure}[!htbp]
	\centering
	\includegraphics[keepaspectratio, width=4.5in, height=\textheight]{probing_correlations/ulrrm_photoassociation/g2_cartoon_wfn.pdf}
	\caption[]{
		\label{fig:ulrrm_photoassociation}
		Schematic illustration of the excitation process of two free atoms to a vibrational ULRRM state.
		Two colliding free atoms in the $\nSLJ{5s^2}{1}{S}{0}-\nSLJ{5s^2}{1}{S}{0}$ interaction potential (blue) have the energy-dependent scattering wave function $\ket*{\chi^{E}_{0}}$ (black) with free particle wave vector $k = \sqrt{{2 \mu E}/{\hbar^2}}$ for collision energy $E$ (a thermal average of the initial colliding wave functions is represented by the different $\ket*{\chi^{E}_{0}}$ curves).
		ULRRM potentials for $\nSLJ{5s^2}{1}{S}{0} + \nSLJ{5sns}{3}{S}{1}$ states are shown for $n$ and $n^{\prime}$ (orange) along with the wave functions for the most deeply-bound $\nu = 0$ states $\ket*{\chi_{n}^{\nu}}$ (green) which are spatially-localized at $R_{n}^{\nu = 0}$.
		Figure adapted from \cite{Whalen2019.PRA.100.011402}.}
\end{figure}
Starting from a pair of colliding atoms, their scattering wave function can be written as $\ket*{\chi^{E}_{0}}$ where, well-outside the interatomic potential, they can be described by free-particle wave functions wave vector $k = \sqrt{{2 \mu E}/{\hbar^2}}$ separated by $R$ in the center-of-mass frame.
It is this scattering wave function which contains information about the symmetry of the initial states.
The probability of photoexcitation to a particular ULRRM state with wave function $\ket*{\chi_{n}^{\nu}}$ can be thought of in terms of an effective Franck-Condon factor $\mathcal{F}_{n}^{\nu} = \bra*{\chi^{E}_{0}}\ket*{\chi_{n}^{\nu}}$
\begin{equation}
	\label{eq:effective_fc}
	\abs{\mathcal{F}_{n}^{\nu}}^{2}
		=	\abs{\bra{\chi^{E}_{0}}\ket{\chi_{n}^{\nu}}}^{2}
		=	\abs{\int_{0}^{\infty} \dd{R} {R^2} \chi^{E}_{0}\pqty*{R} \chi_{n}^{\nu}\pqty*{R}}^2
\end{equation}
Taking a thermal average over the initial collision energies of the two-particle states ($\ev*{\cdots}_{E}$) leads to
\begin{equation}
	\label{eq:effective_fc_th}
	\ev{\abs{\mathcal{F}_{n}^{\nu}}^2}_{E}
		=	\ev{\abs{\bra{\chi^{E}_{0}}\ket{\chi_{n}^{\nu}}}^{2}}_{E}
		=	\ev{\abs{\int_{0}^{\infty} \dd{R} {R^2} \chi^{E}_{0}\pqty*{R} \chi_{n}^{\nu}\pqty*{R}}^2}_{E}
\end{equation}

As described in \cite{Whalen2019.PRA.100.011402}, due to the $\nu = 0$ state being well-localized around $R_{n}$, it can be approximated as $\chi_{n}^{\nu=0}\pqty*{R} \approx \delta\pqty*{R - R_n}$.
Evaluating \cref{eq:effective_fc_th} under this approximation gives\footnote{When evaluating \cref{eq:effective_fc_nu0}, it helps to notice that $\int_{0}^{\infty} \dd{R} {R^2} \chi^{E}_{0}\pqty*{R} A \delta\pqty*{R - R_n} = {R_n^2} \chi^{E}_{0}\pqty*{R_n} A = \chi^{E}_{0}\pqty*{R_n} \int_{0}^{\infty} \dd{R} {R^2} A \delta\pqty*{R-R_n}$.}
\begin{align}
	\label{eq:effective_fc_nu0}
	\ev{\abs{\mathcal{F}_{n}^{\nu=0}}^2}_{E}
		{}&=		\ev{\abs{\int_{0}^{\infty} \dd{R} {R^2} \chi^{E}_{0}\pqty*{R} \chi_{n}^{\nu=0}\pqty*{R}}^2}_{E}	\notag	\\
		{}&\approx	\ev{\abs{\chi^{E}_{0}\pqty*{R_n}}^{2}}_{E} \abs{\int_{0}^{\infty} \dd{R} {R^2} \chi_{n}^{\nu=0}\pqty*{R}}^2	\notag	\\
		{}&=		{g^{\pqty*{2}}\pqty*{R_n}} \abs{\int_{0}^{\infty} \dd{R} {R^2} \chi_{n}^{\nu=0}\pqty*{R}}^2
\end{align}
where, in the last line, the relation $\ev*{\abs*{\chi^{E}_{0}\pqty*{R_n}}^{2}}_{E} = g^{\pqty*{2}}\pqty*{R_n}$, with $g^{\pqty*{2}}\pqty*{R_n}$ the pair-correlation function for separation $R_{n}$, was used. 
This replacement is nontrivial but can be made when the sample is thermal and $\chi_{n}^{\nu=0}\pqty*{R}$ is well-localized on the scale of the initial collisional state $\chi^{E}_{0}\pqty*{R}$ \cite{Whalen2019.PRA.100.011402, Ding2019.arXiv.1909.04634 }.
For the higher vibrational states $\nu > 0$, the approximation $\chi_{n}^{\nu=0}\pqty{R} \approx \delta\pqty*{R_n}$ cannot be made since they are delocalized and, therefore, \cref{eq:effective_fc_th} must be evaluated with the full $\ket*{\chi_{n}^{\nu}}$ ULRRM wave function.

\section{Experimental Procedure}

The experimental procedures used for studies of vibrationally-excited ULRRMs are similar to those presented in \cite{Whalen2019.PRA.100.011402} and so will only be briefly covered.
The production of cold gases of both spin-polarized and unpolarized \Sr{87} begins by loading a mixture of \Sr{87} and \Sr{84} in to the magnetic trap.
As described in \cref{ssec:magnetic_trap_and_repumping}, both isotopes are loaded by operating a blue $\SI{461}{\nm}$ MOT on the $\nSLJ{5s^2}{1}{S}{0} \rightarrow \nSLJ{5s5p}{1}{P}{1}$ transition tuned to the first isotope and then changing the laser frequency to trap the second isotope.
Laser cooling on the $\SI{461}{\nm}$ transition populates the magnetic trap with atoms following the weak $\nSLJ{5s5p}{1}{P}{1} \rightarrow \nSLJ{5s4d}{1}{D}{2} \rightarrow \nSLJ{5s5p}{3}{P}{2}$ decay path.
After the atoms are collected in the magnetic trap, both isotopes are repumped with a pulse of $\SI{481}{\nm}$ light resonant with the $\nSLJ{5s5p}{3}{P}{2} \rightarrow \nSLJ{5p^2}{3}{P}{2}$ transition and second-stage narrow line laser cooling proceeds simultaneously for both isotopes on the $\nSLJ{5s^2}{1}{S}{0} \rightarrow \nSLJ{5s5p}{3}{P}{1}$ transition at $\SI{689}{\nm}$.
Narrow line cooling produces samples at temperatures of about $\SI{2}{\micro\kelvin}$ which are then loaded in to the ODT before all laser cooling light is extinguished.

Once loaded in to the ODT, sample preparation for the spin-polarized and unpolarized gases differ slightly.
For the unpolarized sample, the atoms are held in the ODT for a short period of time to allow for free-evaporation before starting forced evaporation.
For the spin-polarized sample, an approximately $\SI{7}{\gauss}$ magnetic bias field is ramped on during the free-evaporation hold, providing both a quantization axis and Zeeman splitting the $\nSLJf{5s5p}{3}{P}{1}{{9}/{2}}$ states by approximately $\SI{600}{\kHz}$ between adjacent $m_F$ levels. 
Spin-polarization is achieved by optically pumping on the $\nSLJf{5s^2}{1}{S}{0}{{9}/{2}} \rightarrow \nSLJf{5s5p}{3}{P}{1}{{9}/{2}}$ transition, individually addressing each ${\Delta}m_{F} = +1$ transition with $\sigma^{+}$-polarized $\SI{689}{\nm}$ light. 
After spin-polarization, the bias magnetic field is ramped down to $B_{\text{pol}} \approx\SI{1}{\gauss}$ to maintain a quantization axis and forced evaporation is performed.

The effectiveness of spin-polarization is evaluated spectroscopically by exciting trapped atoms to a $\nSLJf{5sns}{3}{S}{1}{{11}/{2}}$ Rydberg state in the $B_{\text{pol}} \approx\SI{1}{\gauss}$ bias magnetic field.
For this setup, excitation is performed with counterpropagating $\SI{689}{\nm}$ and $\SI{320}{\nm}$ beams both with $\pi$-polarization to drive ${\Delta}m_F = 0$ transitions. 
The $\SI{689}{\nm}$ photon is ${\Delta}/{2\pi} = \SI{14.85+-1}{\MHz}$ blue-detuned of intermediate $\nSLJf{5s5p}{3}{P}{1}{{11}/{2}}$ state and the frequency of the $\SI{320}{\nm}$ laser is scanned.
During excitation, the ODT was turned off to eliminate AC Stark shifts and turned back on to recapture the remaining ground-state atoms after any Rydberg atoms produced are field ionized and detected with the MCP.
In order to increase signal, the $\pi$-$\pi$ excitation and measurement cycles are performed $\num{1000}$ times before before a new sample is loaded and the $\SI{320}{\nm}$ laser frequency changed.
A comparison of excitation in an unpolarized and spin-polarized sample is shown in \cref{fig:n33-check_spin-polarization}.
\begin{figure}[!htbp]
	\centering
	\includegraphics[keepaspectratio, width=5in, height=\textheight]{probing_correlations/spin-polarization/checking_spin-polarization/n33-check_spin-polarization.pdf}
	\caption[]{
		\label{fig:n33-check_spin-polarization}
		Experimental spectra and fit of excitation to the $\nSLJf{5s33s}{3}{S}{1}{{11}/{2}}$ state from both an unpolarized (green) and a spin-polarized (blue) samples.
		A bias magnetic field $B_{\text{pol}} \approx\SI{1}{\gauss}$ Zeeman splits the Rydberg state.
		Excitation is performed with $\pi$-polarized counterpropagating $\SI{689}{\nm}$ and $\SI{320}{\nm}$ beams.
		Ideally, only ${\Delta}m_{F} = 0$ transitions are driven with the $\pi$-$\pi$ excitation scheme so only $\num{10}$ peaks should be visible in the unpolarized spectra.
		The small features on the extreme left and right ($m_F = \pm {11}/{2}$) are likely due to imperfect polarization.}
\end{figure}
Model fits to the data show that the optical pumping transfers $\SI{> 90}{\percent}$ of the population into the $\nSLJfm{5s^2}{1}{S}{0}{{9}/{2}}{{9}/{2}}$ state.
It should be noted that the repeated $\pi$-$\pi$ excitations do cause some population redistribution among the $m_F$ states so the fit values place a lower bound on the effectiveness of optical pumping.

After forced evaporation, both spin-polarized and unpolarized samples are at about $\SI{900}{\nano\kelvin}$.
Prior to Rydberg excitation, a pulse of $\SI{689}{\nm}$ light resonant with the $\nSLJ{5s^2}{1}{S}{0} \rightarrow \nSLJ{5s5p}{3}{P}{1}$ transition in \Sr{84} is applied to remove the isotope from the ODT.
Due to the large isotope shifts, minimal heating was observed in \Sr{87}.

% \subsection{Spin-Polarization of \Sr{87}}

% In order to realize a fermionic sample, we spin-polarize the \Sr{87} atoms in to the $\nSLJfm{5s^2}{1}{S}{0}{9/2}{9/2}$ state after loading in to the ODT. 
% Once the unpolarized \Sr{87} atoms are loaded in to the ODT, we apply a bias magnetic field of about $\SI{7}{\gauss}$.
% Spin-polarization is achieved by driving the $\nSLJf{5s^2}{1}{S}{0}{9/2} \rightarrow \nSLJf{5s5p}{3}{S}{1}{9/2}$ transition with each $m_F$ state separated by about $\SI{600}{\kHz}$. 
% Using ${\sigma}^{+}$ light, we sequentially drive ${\Delta}m_F=+1$ transitions from $m_F=-{9}/{2}$ to $m_F={7}/{2}$ with $\SI{45}{\ms}$ pulses.
% A last $\SI{500}{\ms}$ ``clean-up'' pulse driving the $m_F={7}/{2} \rightarrow m_{F}^{\prime}={9}/{2}$ helps produce a sample with better than $\SI{90}{\percent}$ population in the $\nSLJfm{5s^2}{1}{S}{0}{9/2}{9/2}$ state. 

% \begin{figure}[!htbp]
	% \centering
	% \includegraphics[keepaspectratio, width=\textwidth, height=\textheight]{probing_correlations/spin-polarization/producing_spin-polarization/mF_population_vs_duration.pdf}
	% \caption[]{
		% \label{fig:lin-lin_vs_loops}
		% Checking that linear-linear spectroscopy does not significantly alter our population.}
% \end{figure}

% Before performing Rydberg excitation, we reduce the bias field to about $\SI{1.2}{\gauss}$. 

% \subsubsection{Evaluation of Spin-Polarization}

% The effectiveness of spin-polarization was evaluated by performing ``linear-linear'' two-photon Rydberg spectroscopy. 
% Crossed-polarized excitation beams travel perpendicular to the bias field with the UV beam driving $\pi$-transitions. 
% In this setup, the $\SI{689}{\nm}$ drives $\nSLJf{5s^2}{1}{S}{0}{9/2} \rightarrow \nSLJf{5s5p}{3}{P}{1}{9/2}$ and the $\SI{320}{\nm}$ drives $\nSLJf{5s5p}{3}{P}{1}{9/2} \rightarrow \nSLJf{5sns}{3}{S}{1}{11/2}$.

% \begin{figure}[!htbp]
	% \centering
	% \includegraphics[keepaspectratio, width=\textwidth, height=\textheight]{probing_correlations/spin-polarization/checking_spin-polarization/lin-lin_vs_loops.pdf}
	% \caption[]{
		% \label{fig:lin-lin_vs_loops}
		% Checking that linear-linear spectroscopy does not significantly alter our population.}
% \end{figure}

With the exception of the $\SI{\approx 1}{\gauss}$ magnetic bias field used to maintain the spin-polarization (no bias field is applied when studing an unpolarized sample), the same two-photon Rydberg excitation scheme is applied to both the spin-polarized and unpolarized gases.
The first photon is at $\SI{689}{\nm}$ with $\sigma^{+}$-polarization and is blue-detuned from the intermediate $\nSLJf{5s5p}{3}{P}{1}{{11}/{2}}$ state by about ${\Delta}/{2 \pi} = \SI{14.85 +- 1}{\MHz}$.
The second photon is at $\SI{320}{\nm}$ with $\pi$-polarization and is tuned to excite ULRRMs of the parent $\nSLJf{5sns}{3}{S}{1}{{11}/{2}}$ Rydberg state.
Excitation powers were kept low to avoid saturating the MCP so the signal-to-noise ratio was increased by performing $\num{1000}$ excitation and detection cycles on a sample before a new sample is prepared and the frequency of the $\SI{320}{\nm}$ laser changed.
Again, the ODT is turned off during Rydberg excitation to eliminate AC Stark shifts before being turned back on after the field ionization and detection to recapture remaining ground-state atoms.

\section{Spectroscopy of $\nu=1$ and $\nu = 2$ ULRRMs}

Spectroscopy was performed by exciting ULRRMs of the parent $\nSLJf{5sns}{3}{S}{1}{{11}/{2}}$ Rydberg state for $n = \numrange[range-phrase=-]{31}{41}$.
In the experiment, the photoexcitation signal of the ULRRM dimers can be modeled as
\begin{equation}
	\mathcal{S}_{n}^{\nu}\pqty*{f}	=	\alpha I_{\SI{689}{\nm}} I_{\SI{320}{\nm}} \mathcal{N}_2 \beta_{n} \mathcal{C} \ev{\abs{\mathcal{F}_{n}^{\nu}}^2}_{E} L\pqty*{f}
\end{equation}
where $\alpha$ is the detection efficiency, $I_{\SI{689}{\nm}}$ and $I_{\SI{320}{\nm}}$ are the intensities of the two-photon excitation lasers, $\beta_n$ is proportional to the square of the reduced two-photon electronic-transition matrix element, $\mathcal{N}_2 \equiv \int \dd[3]{r} \rho\pqty*{R}^2$ is the volume integral of the square of the density distribution, $\mathcal{C}$ is a Clebsch-Gordan factor, and $L\pqty*{f}$ is a normalized lineshape function.
To eliminate dependences on the specific lineshape, the quantity of interest is the integrals of the observed excitation signal
\begin{equation}
	\label{eq:ulrrm_signal}
	\mathcal{S}_{n}^{\nu}
		=	\int \dd{f} \mathcal{S}_{n}^{\nu}\pqty*{f}
		=	\alpha I_{\SI{689}{\nm}} I_{\SI{320}{\nm}} \mathcal{N}_2 \beta_{n} \mathcal{C} \ev{\abs{\mathcal{F}_{n}^{\nu}}^2}_{E}
\end{equation}
For a particular $n$, $\beta_{n}$ should remain the same for both spin-polarized and unpolarized samples and the Clebsch-Gordan factor $\mathcal{C}$ can be calculated.
By monitoring the atom number, temperature, $I_{\SI{689}{\nm}}$, and $I_{\SI{320}{\nm}}$, corrections can be applied to the observed signals to reduce potential systematic errors.

The first corrections applied to the ``raw'' MCP signal involves accounting for any undercounting due to MCP saturation using the model described in \cref{ssec:charged_particle_detection}.
Slow changes in the two-photon excitation rate are easily removed by dividing the signal by $I_{\SI{689}{\nm}}$ and $I_{\SI{320}{\nm}}$ recorded on a DAQ\footnote{We actually record voltages $V_{\SI{689}{\nm}}$ and $V_{\SI{320}{\nm}}$ from monitor photodiodes which are proportional to $I_{\SI{689}{\nm}}$ and $I_{\SI{320}{\nm}}$, respectively.}.
Applying these corrections results in a ``scaled'' MCP signal.
An example two-photon excitation spectra to the $\nSLJf{5s34s}{3}{S}{1}{{11}/{2}}$ parent Rydberg state from both a spin-polarized and an unpolarized \Sr{87} sample is shown in \cref{fig:processing_ULRRM_signal}, demonstrating the effects of removing dependences on $I_{\SI{689}{\nm}}$, $I_{\SI{320}{\nm}}$, and MCP saturation.
\begin{figure}[!htbp]
	\centering
	\includegraphics[keepaspectratio, width=\textwidth, height=\textheight]{probing_correlations/excited_dimer_signals/20190123-n34/n34-excited_dimer_signal.pdf}
	\caption[]{
		\label{fig:processing_ULRRM_signal}
		Upper figure shows the signals obtained directly from the MCP accumulated from $\num{1000}$ excitation and detection cycles when scanning the $\SI{640}{\nm}$ fiber EOM frequency (the ``raw'' MCP signal).
		Lower figure shows the ``scaled'' MCP signals after removing the Zeeman shift and accounting for MCP saturation as well as both $I_{\SI{689}{\nm}}$ and $I_{\SI{320}{\nm}}$.
		\Sr{87} atom number was approximately the same for both the spin-polarized and unpolarized samples.
		Data shown is for excitation to the $\nSLJf{5s34s}{3}{S}{1}{{11}/{2}}$ parent state.
		The theoretically calculated positions of the various vibrational dimer states $\nu$ are indicated by arrows (lower figure).}
\end{figure}
Fitting is performed on the scaled MCP signals to obtain the integral of the particular ULRRM line of interest (represented by $\mathcal{S}_{n}^{\nu}$ in \cref{eq:ulrrm_signal}). 
Fits to the spectra in \cref{fig:processing_ULRRM_signal} are shown in \cref{fig:n34_fitted_dimers}.
\begin{figure}[!htbp]
	\centering
	\includegraphics[keepaspectratio, width=\textwidth, height=\textheight]{probing_correlations/excited_dimer_signals/20190123-n34/n34-pol_unpol_fitted.pdf}
	\caption[]{
		\label{fig:n34_fitted_dimers}
		Fits to the spin-polarized (upper panel) and unpolarized (lower panel) $\nSLJf{5s34s}{3}{S}{1}{{11}/{2}}$ spectra shown in \cref{fig:processing_ULRRM_signal}.
		Solid lines show the fit results and the shaded regions represent the integrals used to extract $\mathcal{S}_{n=34}^{\nu}$.}
\end{figure}

Before extracting a value for the integral, we must still determine the two-body density-dependent correction term $\mathcal{N}_{2}$ to reduce the effects of atom number and temperature variations for the different samples.
To make a comparison between spin-polarized and unpolarized signals, it is also necessary to determine the ratio of their detection efficiencies ${\alpha_{\text{pol}}}/{\alpha_{\text{unpol}}}$.

\subsection{Two-Body Density Correction: $\mathcal{N}_2$}

During data collection, both the atom number and sample temperature are recorded with absorption imaging on the $\nSLJ{5s^2}{1}{S}{0} \rightarrow \nSLJ{5s5p}{1}{P}{1}$ transition.
All the samples have the same final ODT trapping parameters prior to Rydberg excitation.
The vertical trap frequency was measured to be $f_{z} \approx \SI{300}{\Hz}$ and radial trap frequency is about $f_{r} \approx \SI{130}{\Hz}$.
Knowing these parameters, the density-dependent $\mathcal{N}_{2}$ factor can be calculated and applied to the ULRRM signals. 

For a thermal gas in a three-dimensional harmonic trap, the density distribution is given by
\begin{equation}
	\rho\pqty{\vb{r}}	=	\frac{N}{{\pi}^{3/2} R_{x} R_{y} R_{z}} {\me}^{-{x^2}/{R_x^2}} {\me}^{-{y^2}/{R_y^2}} {\me}^{-{z^2}/{R_z^2}}
\end{equation}
where $N$ is the total atom number and $R_i^2 = {2 k_{B} T}/{m \omega_i^2}$ where $f_{i} = {\omega_i}/{2 \pi}$ is the trap frequency \cite{Pethick.BEC}. 
For the dimers, the total signal should be proportional to $\bqty*{\rho\pqty*{\vb{r}}}^{2}$ integrated over all space.
Performing this integration yields
\begin{equation}
	\mathcal{N}_2	=		\int \dd[3]{r} {\rho\pqty{\vb{r}}}^2
					=		\frac{N^2}{{\pqty{2 \pi}^{3/2}} R_x R_y R_z}
					\propto	\frac{N^2}{T^{3/2}}
\end{equation}
Therefore, we can scale the measured signal by ${N^2}/{T^{3/2}}$ in order to account for the signal dependence on atom number and temperature. 
Recall that the same trapping parameters was used for both the spin-polarized and unpolarized samples and therefore the dependences on the trap frequencies drop out.
This correction is applied to all the dimer data.

\subsection{Examining Polarized-Polarized and Unpolarized-Unpolarized Ratios}

In order to eliminate any dependency of $\mathcal{S}_{n}^{\nu}$ on the detection efficiency $\alpha$, spin-polarized signals were compared to spin-polarized signals and unpolarized signals were compared to unpolarized signals within a principal quantum number.
In particular, the ratios of the {$\nu = 1$ and $2$} states were compared to the $\nu = 0$ with the ratio defined as
\begin{equation}
	\label{eq:excited_dimer_ratios}
	\frac{\mathcal{S}_{n,x}^{\nu}}{\mathcal{S}_{n,x}^{\nu=0}}
		=	\frac{\alpha_{x} I_{\SI{689}{\nm}} I_{\SI{320}{\nm}} \mathcal{N}_2 \beta_{n} \mathcal{C}_{x} \ev{\abs{\mathcal{F}_{n,x}^{\nu}}^2}_{E}}{\alpha_{x} I_{\SI{689}{\nm}} I_{\SI{320}{\nm}} \mathcal{N}_2 \beta_{n} \mathcal{C}_{x} \ev{\abs{\mathcal{F}_{n,x}^{\nu=0}}^2}_{E}}
		=	\frac{I_{\SI{689}{\nm}} I_{\SI{320}{\nm}} \mathcal{N}_2 \ev{\abs{\mathcal{F}_{n,x}^{\nu}}^2}_{E}}{I_{\SI{689}{\nm}} I_{\SI{320}{\nm}} \mathcal{N}_2 \ev{\abs{\mathcal{F}_{n,x}^{\nu=0}}^2}_{E}}
\end{equation}
where $x$ represents the spin-polarized ($\mathcal{S}_{n,\text{pol}}^{\nu}$) or unpolarized ($\mathcal{S}_{n,\text{unpol}}^{\nu}$) signal.
The dependence on $\alpha_{x}$ is eliminated because the data come from either polarized or unpolarized samples only.
Monitoring the intensities $I_{\SI{689}{\nm}}$ and $I_{\SI{320}{\nm}}$ allows those factors to be scaled out of a particular measurement. 
The recorded atom number and temperature are used to calculate $\mathcal{N}_2$.
Therefore, the ${\ev*{\abs*{\mathcal{F}_{n,x}^{\nu}}^2}_{E}}/{\ev*{\abs*{\mathcal{F}_{n,x}^{\nu=0}}^2}_{E}}$ must be the dominant contributing factor to ${\mathcal{S}_{n,x}^{\nu}}/{\mathcal{S}_{n,x}^{\nu=0}}$.

The measured ratios when varying the principal quantum number over $n = \numrange[range-phrase=-]{31}{41}$ is shown in \cref{fig:excited_dimer_ratios}. 
\begin{figure}[!htbp]
	\centering
	\includegraphics[keepaspectratio, width=5in, height=\textheight]{probing_correlations/dimer_ratios/Fig6.pdf}
	\caption[]{
		\label{fig:excited_dimer_ratios}
		Experimentally measured ratios of the $\nu = 1$ (upper panel) and $\nu = 2$ (lower panel) vibrational states compared to the $\nu = 0$ state of the same $n$ for both spin-polarized (blue circles) and unpolarized (green squares) samples.}
\end{figure}
The $\mathcal{S}_{n,x}^{\nu=1}/\mathcal{S}_{n,x}^{\nu=0}$ and $\mathcal{S}_{n,x}^{\nu=2}/\mathcal{S}_{n,x}^{\nu=0}$ ratios exhibit very different $n$-dependencies common to both the spin-polarized and unpolarized samples.
While the $\mathcal{S}_{n,x}^{\nu=1}/\mathcal{S}_{n,x}^{\nu=0}$ ratio exhibits the expected excitation reduction for lower $n$, the $\mathcal{S}_{n,x}^{\nu=2}/\mathcal{S}_{n,x}^{\nu=0}$ ratio is counterintuitive and was found to be enhanced for lower $n$. 

This $n$-scaling of the ratios common to both the spin-polarized and unpolarized samples can be qualitatively understood by examining the radial vibrational wave functions\footnote{In analogy with standard photoassociation where varying the vibrational state $\nu$ provides information about both the initial free-free scattering state and the molecular state.}.
\Cref{fig:ulrrm_wave_functions_vs_n} shows ULRRM potentials and the {$\nu = 0$, $1$, and $2$} radial vibrational wave functions for a representative selection of $n$. 
\begin{figure}[!htbp]
	\centering
	\includegraphics[keepaspectratio, width=\textwidth, height=\textheight]{probing_correlations/ulrrm_wave_functions/ulrrm_wave_functions_vs_n.pdf}
	\caption[]{
		\label{fig:ulrrm_wave_functions_vs_n}
		Illustrative ULRRM potentials ($V$) radial vibrational wave functions for {$\nu = 0$, $1$, and $2$} for a $\nSLJ{5sns}{3}{S}{1} + \nSLJ{5s^2}{1}{S}{0}$ dimer.
		Notice that the binding energies span different ranges in the three figures.}
\end{figure}
The $\nu = 0$ state is localized in the outermost well of the ULRRM potential near $R_{n}^{\nu=0} \simeq 1.87 \pqty*{n-\delta}^{2} \si{\bohr}$\footnote{This scaling is obtained from fitting the theoretically calculated ULRRM radial vibrational wave functions.} meaning the contribution to $\ev*{\abs*{\mathcal{F}_{n,x}^{\nu = 0}}^2}_{E}$ comes exclusively from the region around $R_{n}^{\nu=0}$. 
As a result of this relatively simple $n$-dependence, it is straightforward to interpret the effects of the initial spatial distribution of atoms on the excitation rate to the $\nu = 0$ ULRRM state across different $n$.

This simple interpretation for the $\mathcal{S}_{n,x}^{\nu=1}/\mathcal{S}_{n,x}^{\nu=0}$ and $\mathcal{S}_{n,x}^{\nu=2}/\mathcal{S}_{n,x}^{\nu=0}$ ratios is not as easy due to the delocalized nature of the $\nu = 1$ and $\nu = 2$ wave functions. 
It can be seen in \cref{fig:ulrrm_wave_functions_vs_n} that the $\nu = 1$ state can be considered somewhat localized around a particular (or a few adjacent) well(s) of the ULRRM potential and therefore $\ev*{\abs*{\mathcal{F}_{n,x}^{\nu = 1}}^2}_{E}$ can be expected (and was observed) to follow a similar $n$-scaling behavior as the $\nu = 0$ state (i.e., decreasing $\mathcal{S}_{n,x}^{\nu=1}/\mathcal{S}_{n,x}^{\nu=0}$ at lower $n$). 
The $\nu = 2$ state is much more delocalized with no obvious $n$-dependence so the ``simple'' picture used to understand the $\mathcal{S}_{n,x}^{\nu=1}/\mathcal{S}_{n,x}^{\nu=0}$ ratio is not applicable for the $\mathcal{S}_{n,x}^{\nu=2}/\mathcal{S}_{n,x}^{\nu=0}$ ratio.
It should be noted that both the $\nu = 1$ and $\nu = 2$ wave functions have a node around $R_{n}^{\nu=0}$ meaning the contribution to $\ev*{\abs*{\mathcal{F}_{n,x}^{\nu = 1}}^2}_{E}$ and $\ev*{\abs*{\mathcal{F}_{n,x}^{\nu = 2}}^2}_{E}$ at this position is reduced. 

While the $n$-dependence of the ratios that are common to both the spin-polarized and unpolarized samples provides information about the initial colliding atomic wave function and the ULRRM wave function, the differences in $\mathcal{S}_{n,\text{pol}}^{\nu}/\mathcal{S}_{n,\text{pol}}^{\nu=0}$ and $\mathcal{S}_{n,\text{unpol}}^{\nu}/\mathcal{S}_{n,\text{unpol}}^{\nu=0}$ provides information regarding the effects of correlations. 
Most notably, it can be seen in \cref{fig:excited_dimer_ratios} that there is a weak but consistent systematic reduction of the polarized ratios compared to the unpolarized ratios in both $\mathcal{S}_{n}^{\nu=1}/\mathcal{S}_{n}^{\nu=0}$ and $\mathcal{S}_{n,x}^{\nu=2}/\mathcal{S}_{n,x}^{\nu=0}$ ratios across the range of $n$ explored. 
This is evidence suggesting that the spin-polarized sample is experiencing stronger effects of antibunching compared to the unpolarized sample.

\subsection{Electron Detection Efficiency Ratio: ${\alpha_{\text{pol}}}/{\alpha_{\text{unpol}}}$}

In order to compare the spin-polarized data to the unpolarized data, the electron detection efficiency $\alpha$ needs to be determined.
Although it would be extremely useful to know the exact value of $\alpha$, only the ratio ${\alpha_{\text{pol}}}/{\alpha_{\text{unpol}}}$ is needed since we are interested in how the spin-polarized samples differ from the unpolarized samples.
The ratio ${\alpha_{\text{pol}}}/{\alpha_{\text{unpol}}}$ accounts for the differences in detection efficiency between the spin-polarized sample and unpolarized samples with the former measurements taken in a $B_{\text{pol}} \approx \SI{1}{\gauss}$ bias magnetic field oriented towards the MCP (to maintain spin-polarization) and the latter in no bias field ($B_{\text{unpol}} = \SI{0}{\gauss}$). 
Na{\"{i}}vely, the spin-polarized samples are expected to exhibit a larger signal than the unpolarized sample simply due to the Lorentz force of the magnetic field, oriented towards the MCP, helping guide the electrons towards the detector.
Towards this end, we measured the two-photon excitation signals with \Sr{84} due to its simpler electronic structure in both $B_{\text{pol}}$ and $B_{\text{unpol}}$ configurations.

The measurements with \Sr{84} used the same beam paths and geometry as the ULRRM measurements in \Sr{87}.
Two-photon excitation following the $\nSLJ{5s^2}{1}{S}{0} \rightarrow \nSLJ{5s5p}{3}{P}{1} \rightarrow \nSLJ{5s37s}{3}{S}{1}$ path was used.
The first $\SI{689}{\nm}$ photon has $\sigma^+$-polarization and is blue-detuned from the intermediate state by ${\Delta}/{2 \pi} = \SI{14.93+-1}{\MHz}$.
The second $\SI{320}{\nm}$ photon has $\pi$-polarization.
\Cref{fig:detection_efficiency_vs_B} shows spectra obtained with and without the bias magnetic field when $\SI{689}{\nm}$ is held fixed while the $\SI{320}{\nm}$ frequency is scanned across the Rydberg atomic line. 
\begin{figure}[!htbp]
	\centering
	\includegraphics[keepaspectratio, width=5in, height=\textheight]{probing_correlations/detection_efficiency_vs_B/B-field-set_points/detection_efficiency_vs_B.pdf}
	\caption[]{
		\label{fig:detection_efficiency_vs_B}
		The \Sr{84} $\nSLJ{5s37s}{3}{S}{1}$ atomic Rydberg line with $B_{\text{pol}} \approx \SI{1}{\gauss}$ (blue circles) and $B_{\text{unpol}} \approx \SI{0}{\gauss}$ (green squares) bias magnetic fields. 
		The MCP signal was scaled accounting for both the $\SI{689}{\nm}$ and $\SI{320}{\nm}$ powers while the atom number was constant for all the measurements.
		Gaussian fits (lines) were used to extract the integrals, resulting with $\mathcal{S}\pqty*{B_{\text{pol}}} = \num{369+-9}$ (blue shaded region) and $\mathcal{S}\pqty*{B_{\text{unpol}}} = \num{278+-7}$ (green shaded region).}
\end{figure}
From the data shown in the figure, the values $\mathcal{S}\pqty*{B_{\text{pol}}} = \num{369+-9}$ and $\mathcal{S}\pqty*{B_{\text{unpol}}} = \num{278+-7}$ were extracted for fits to the Rydberg line.
Taking their ratio gives ${\alpha\pqty*{B_{\text{pol}}}}/{\alpha\pqty*{B_{\text{unpol}}}} = \num{1.33+-0.05}$.

Before reporting a detection efficiency difference for the $B_{\text{pol}}$ and $ B_{\text{unpol}}$ conditions, there are still a few things to check and consider.
A possible source of systematic error in this determination of the ratio could be due to poor polarization of either (or both) the $\SI{689}{\nm}$ and $\SI{320}{\nm}$ excitation lasers. 
This seems unlikely as there is very little signal at ${\Delta}/{2 \pi} = \SI{0}{\MHz}$ when the the magnetic field is applied (blue circles), suggesting that the polarizations are likely reasonably well-aligned to the quantization axis. 
This also matches our previous spin-polarization tests which found the $\sigma^+$ beam path to have relatively good circular polarization.
The $\SI{320}{\nm}$ path passes through a high-extinction ratio calcite polarizer\footnote{Thorlabs' GLB10 Glan-Laser alpha-BBO calcite polarizer.} that cleans up its polarization.

Another thing to consider is that the Zeeman effect could change the intermediate detuning with the effect scaling as $\Gamma_{\text{sc}} \sim {1}/{\Delta^2}$.
From \cref{fig:detection_efficiency_vs_B}, the observed shift of the $\nSLJm{5s37s}{3}{S}{1}{1}$ state with and without the magnetic field is $\SI{3.326+-0.006}{\MHz}$.
Since the $\nSLJ{5s37s}{3}{S}{1}$ state has $g_J = 2$, this corresponds to $B_{\text{pol}} = \SI{1.1882+-0.0020}{\gauss}$, which agrees with previous calibration measurements obtained by performing spectroscopy on the $\nSLJ{5s^2}{1}{S}{0} \rightarrow \nSLJ{5s5p}{3}{P}{1}$ transition.
Considering the detuning ${\Delta_{\text{unpol}}}/{2 \pi} = \SI{14.9+-1}{\MHz}$ in the $B_{\text{unpol}}$ case, the intermediate state detuning from the $\nSLJ{5s^2}{1}{S}{0} \rightarrow \nSLJm{5s5p}{3}{P}{1}{1}$ is expected to be reduced to ${\Delta_{\text{pol}}}/{2 \pi} = \SI{12.4+-1}{\MHz}$ in the $B_{\text{pol}}$ case. 
Therefore, the $\mathcal{S}\pqty*{B_{\text{pol}}}$ is expected to be about a factor of ${\Delta_{\text{unpol}}^2}/{\Delta_{\text{pol}}^2} = \num{1.44+-0.04}$ larger than $\mathcal{S}\pqty*{B_{\text{unpol}}}$ due to the reduced detuning.

Since the the value of ${\Delta_{\text{unpol}}^2}/{\Delta_{\text{pol}}^2}$ is close to the measured value ${\mathcal{S}\pqty*{B_{\text{pol}}}}/{\mathcal{S}\pqty*{B_{\text{unpol}}}}$, this would suggest that the enhancement of the signal obtained in $B_{\text{pol}}$ is due to the Zeeman shift reducing the intermediate state detuning instead of due to the Lorentz force guiding more electrons to the MCP.
Fortuitously, the experimental parameters used to determine ${\alpha_{\text{pol}}}/{\alpha_{\text{unpol}}}$ in \Sr{84} are almost identical to the conditions used for the ULRRM measurements in \Sr{87}.
In both measurements, the $\SI{689}{\nm}$ laser was blue-detuned from the intermediate state by approximately $\SI{14.9}{\MHz}$ in no magnetic field (i.e., ${\Delta_{\text{unpol}}}/{2\pi} \approx \SI{14.9}{\MHz}$).
With the Zeeman shifts of the intermediate states being the same, ${g_J}{m_J} = {3}/{2}$ for the $\nSLJm{5s5p}{3}{P}{1}{1}$ state in \Sr{84} and ${g_F}{m_F} = {3}/{2}$ for the $\nSLJfm{5s5p}{3}{P}{1}{{11}/{2}}{{11}/{2}}$ state in \Sr{87}, we can reasonably expect that the enhancement of spin-polarized signal (with $B = B_{\text{pol}}$) compared to the unpolarized signal (with $B = B_{\text{unpol}}$) to also be given by ${\alpha\pqty*{B_{\text{pol}}}}/{\alpha\pqty*{B_{\text{unpol}}}} = \num{1.33+-0.05}$.

\subsection{Preliminary Comparison of $\mathcal{S}_{n,\text{pol}}^{\nu}$ to $\mathcal{S}_{n,\text{unpol}}^{\nu}$}

A preliminary comparison of the spin-polarized to unpolarized ratios ${\mathcal{S}_{n,\text{pol}}^{\nu}}/{\mathcal{S}_{n,\text{unpol}}^{\nu}}$ for the {$\nu = 0$, $1$, and $2$} vibrational states is shown in \cref{fig:vibrational_pol_unpol_ratio}.
The spin-polarized to unpolarized ratios in the figure account for the ${\alpha\pqty*{B_{\text{pol}}}}/{\alpha\pqty*{B_{\text{unpol}}}} = \num{1.33+-0.05}$ factor and the difference in Clebsch-Gordan coupling strength to the Rydberg state\footnote{$\mathcal{C}_{\text{pol}} = {11}/{13}$ and $\mathcal{C}_{\text{unpol}} = {46}/{275}$.}.
\begin{figure}[!htbp]
	\centering
	\includegraphics[keepaspectratio, width=5.25in, height=\textheight]{probing_correlations/dimer_ratios/pol_unpol_ratio.pdf}
	\caption[]{
		\label{fig:vibrational_pol_unpol_ratio}
		Comparison of the ratio ${\mathcal{S}_{n,\text{pol}}^{\nu}}/{\mathcal{S}_{n,\text{unpol}}^{\nu}}$ for {$\nu = 0$, $1$, and $2$} across multiple principal quantum numbers ($n$).
		The extracted spin-polarized to unpolarized ratios accounts for both ${\alpha\pqty*{B_{\text{pol}}}}/{\alpha\pqty*{B_{\text{unpol}}}} = \num{1.33+-0.05}$ and for differences in the Clebsch-Gordan couplings.
		For the $\nu = 0$ ratios, $g^{\pqty*{2}}\pqty*{R}$ (solid line) and $0.8 g^{\pqty*{2}}\pqty*{R}$ (dotted line) for fermions is shown where $R_{n} \simeq 1.87 \pqty*{n-\delta}^{2} \si{\bohr}$ and $T = \SI{850 +- 100}{\nano\kelvin}$ (the shaded region corresponds to the expected uncertainty).}
\end{figure}

With the exception of the data taken at $n = 31$, both the ${\mathcal{S}_{n,\text{pol}}^{\nu=0}}/{\mathcal{S}_{n,\text{unpol}}^{\nu=0}}$ and the ${\mathcal{S}_{n,\text{pol}}^{\nu=1}}/{\mathcal{S}_{n,\text{unpol}}^{\nu=1}}$ ratios exhibit a pronounced decrease in their values at smaller $n$ (i.e., smaller interparticle separations $R$), which is additional evidence of antibunching. 
\Cref{fig:vibrational_pol_unpol_ratio} also shows the expected $g^{\pqty*{2}}\pqty*{R}$ for fermions at $T = \SI{850 +- 100}{\nano\kelvin}$ with $R_{n}^{\nu=0}$ given previously for the $\nu = 0$ state.
For comparison, $0.8 g^{\pqty*{2}}\pqty*{R}$ is also shown which has qualitatively better agreement with the observations. 
Currently, it is unclear why the $\nu = 0$ ratio appears to converge $\num{\approx 0.8}$ as $n$ increases instead of $\num{1}$ expected from $g^{\pqty*{2}}\pqty*{R}$.

% Again, insights can be gained by examining the shapes of the ULLRM radial vibrational wave functions shown in \cref{fig:ulrrm_wave_functions_vs_n}.
% \begin{figure}[!htbp]
	% \centering
	% \includegraphics[keepaspectratio, width=\textwidth, height=\textheight]{probing_correlations/ulrrm_wave_functions/ulrrm_wave_functions_vs_n.pdf}
	% \caption[]{
		% %\label{fig:ulrrm_wave_functions_vs_n}
		% Illustrative ULRRM potentials ($V$) radial vibrational wave functions for {$\nu = 0$, $1$, and $2$} for a $\nSLJ{5sns}{3}{S}{1} + \nSLJ{5s^2}{1}{S}{0}$ dimer.
		% Notice that the binding energies span different ranges in the three figures.}
% \end{figure}
% It is clear from this figure that, \textit{a priori}, the $\nu = 1$ and $\nu = 2$ should not be expected to follow the simple scaling with $n$ that the well-localized $\nu = 0$ state does due to their additional structure. 
% For the $\nu = 0$ state, the contribution to $\mathcal{F}_{n}^{\nu = 0}$ comes exclusively from the well-localized wave function centered around $R_{n}^{\nu=0}$ (i.e., the outermost lobe of the Rydberg electron wave function).
% Notice that the {$\nu = 1$ and $2$} vibrational states have a node near the outermost well meaning the contribution to $\mathcal{F}_{n}^{\nu}$ near $R_{n}^{\nu=0}$ is reduced. 
% Although the $\nu = 1$ state is delocalized, it can almost be considered as being (somewhat) localized in a particular (or a few adjacent) potential well(s) that should contribute the most to $\mathcal{F}_{n}^{\nu = 1}$.
% This ``intuition'' cannot be easily applied to the $\nu = 2$ ULRRM state as it is far more delocalized.

\section{Conclusion}

We have presented preliminary measurements of the photoexcitation rates of the dimer $\nu = 1$ and $\nu = 2$ vibrationally-excited ULRRM states.
By comparing ratios instead of signals, sensitivities to common factors can be reduced or eliminated.
Comparison of spin-polarized to spin-polarized samples and unpolarized to unpolarized samples, shown in \cref{fig:excited_dimer_ratios}, highlights common $n$-dependencies of the ULRRM excitation rate of a particular vibrational state in both spin-polarized and unpolarized samples. 
It also provided the first evidence of antibunching with the systematically lower spin-polarized ratios compared to the unpolarized ratios across $n$. 

Comparison spin-polarized to unpolarized signals is more difficult as the measurement requires more accurate knowledge of, and is sensitive to, experimental factors which are not common between the conditions under which data is obtained for the spin-polarized and unpolarized samples. 
\Cref{fig:vibrational_pol_unpol_ratio} does indicate that the $\nu = 1$ polarized-unpolarized ratio follows the general trend of the $\nu = 0$ ratios but slightly reduced, again, in agreement with intuition of the $\nu = 1$ state experiencing additional suppression.
The delocalized $\nu = 2$ state does not appear to show much of a consistent trend, likely resulting from the structure exhibited by the wave function. 

The presented measurements were very challenging due to long-term drifts in the $\SI{320}{\nm}$ Rydberg laser frequency and to a previously unnoticed changes in beam pointing when tuning to different $n$.
Additional work exploring the vibrational states ULRRM states is worthwhile since they do exhibit more complicated dependencies compared to the $\nu = 0$ state.
Once experimental challenges are overcome, these states can be used to test theoretical calculations of both the ULRRM radial vibrational wave functions and how the effects of quantum statistics affect their production rate. 


% Knowing these challenges, future work towards stabilizing both the frequency and beam pointing of the Rydberg laser can result in a more accurate spectroscopy.
% The updated data can then be passed to theorists so that they can revise the models to more accurately reflect the observed spectra.
% Future work should have promising results since it can not only provide a test of theory, but it can also provide a tool for probing spatial correlations on length scales inaccessible with the $\nu = 0$ ULRRM states.

% \subsection{Considerations for Future Experiments}

% **** We could check spin-polarization with 5s5p 3P1 F=9/2 state instead maybe to check photon polarizations.

% \subsection{Acknowledgments}

% We thank Dr. R. G. Hulet for loan of the ODT laser. 