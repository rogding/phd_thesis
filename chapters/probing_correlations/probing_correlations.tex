\chapter{Vibrationally-Excited Rydberg Molecules in Cold Gases of \Sr{87}}
\label{ch:correlations}

Many interesting phenomena in quantum systems have a length scale associated with their interaction.
Perhaps one of the most fundamental are the effects of either Bose-Einstein or Fermi-Dirac statistics leading to the bunching or antibunching of particles \cite{Naraschewski1999.PRA.59.4595}.
Another interesting phenomena is the ``fermionization'' of strongly-interacting bosons in a one-dimensional gas \cite{Kinoshita2004.Science.305.1125}.
In these examples, the interactions affect the likelihood of finding another particle within a fixed distance., i.e., a spatial correlation.
As a result, tools that provide information regarding the spatial correlation in these systems are extremely insightful.

One of the most ingenious ways of probing spatial correlations due to quantum statistics was demonstrated with metastable helium atoms.
By dropping quantum degenerate Bose and Fermi gases on to a time-resolved position-sensitive MCP detector, the two-body and three-body pair correlation functions of the initial sample were able to be reconstructed \cite{Schellekens2005.Science.310.648, Jeltes2007.Nature.445.402, Hodgman2011.Science.331.1046}.
Although powerful, this method is neither \textit{in situ} nor nondestructive, requiring the samples to be at equilibrium prior to dropping in order for the reconstruction to make sense, thus hindering the possibility of mapping the temporal development of spatial correlations. 

Alternative, less destructive, \textit{in situ} techniques have also been developed and employed but these are generally restricted by the dimensionality of a system or the length scales accessible.
For the shortest length scales (i.e., $R = 0$ contact interactions), inelastic loss due to three-body recombination has been shown to be sensitive to the effects of $g^{\pqty*{3}}\pqty*{R}$ as evidenced by the different loss rates observed in a nondegenerate compared to a degenerate Bose gases \cite{Burt1997.PRL.79.337}. 
For length scales comparable to the sizes of molecular bonds (i.e., $R \sim R_{\text{vdW}} \lesssim \SI{10}{\nm}$), photoassociative spectroscopy (PAS) can be used 
since efficient photoassociation requires the constituent atoms to have a separation distance comparable to the molecular potential (the van der Waals $R_{\text{vdW}}$) \cite{Jones2006.RMP.78.483, Jones2006.RMP.78.1041}.
If the effect can be observed in a two (or one) dimensional lattice structure, quantum gas microscopes (QGM) have the capability of directly detecting the presence or absence of atoms on a particular lattice site and, therefore, are able to directly measure spatial correlations \cite{Bakr2010.Science.329.547, Omran2015.PRL.115.263001, Mazurenko2017.Nature.545.462}.
Since QGMs require an optical lattice, their accessible length scales are generally restricted by the wavelength of light used to form the lattice ($R \gtrsim \SI{266}{\nm}$).
\Cref{fig:correlation_tools} illustrates some of the tools developed for probing the effects and their typical effective length scales.
\begin{figure}[!htbp]
	\centering
	\includegraphics[keepaspectratio, width=5in, height=\textheight]{probing_correlations/correlation_tools/correlation_length_scales.pdf}
	\caption[]{
		\label{fig:correlation_tools}
		A (non-exhaustive) list of lengths scales in typical ultracold gas experiments.
		Black lines show the length scales associated with various interactions and red lines illustrate the effective ranges of various techniques.
		The blue line indicates the length scales probable with ULRRMs for easily accessible ranges $n$.}
\end{figure}

As seen in \cref{fig:correlation_tools}, there is a gap for length scales $\SI{10}{\nm} \lesssim R \lesssim \SI{266}{\nm}$ which can be filled by ULRRMs. 
Analogous to how PAS can be used to probe correlations on the order of $R_{\text{vdW}}$, the formation rate of ULRRMs can also be used to measure spatial correlations. 
The advantage of ULRRMs is that the length scale being probed depends on the size of the parent Rydberg atom, meaning varying interparticle distances can be probed by simply exciting to a different principal quantum number ($n$). 
This tunability gives ULRRMs a huge advantage and can fill the gap between the short-range PAS and the longer-range $R \gtrsim \SI{266}{\nm}$ QGM.

%This chapter contains work that is currently in progress.

\section{Probing of Spatial Correlations with ULRRMs}

In an ideal noninteracting gas, the atoms are expected to exhibit quantum statistical effects in their density distributions depending on whether they are bosonic, fermionic, or classical.
For the two-body case, this effect on the density distribution is typically referred to as the spatial correlation function ${g}^{\pqty*{2}}\pqty*{R}$ and can be written (for the ideal noninteracting case) as *** SOURCE? ***
\begin{equation}
	{g}^{\pqty{2}}\pqty{R}	=	1 + \epsilon {\me}^{-2 \pi R^{2}/{\lambda_{\text{dB}}}^{2}}
\end{equation}
where {$\epsilon = +1$, $-1$, and $0$} for bosons, fermions, and classical cases, respectively, and $\lambda_{\text{dB}} = {h}/{\sqrt{2 \pi m k_{B} T}}$ is the thermal de Broglie wavelength.
For a more detailed discussion, see \cite{Naraschewski1999.PRA.59.4595}. **** OTHER REFERENCES? ****

It was shown in \cite{Whalen2019.PRA.100.011402} that the excitation rate of ULRRMs is sensitive to spatial correlations on length scales of the Rydberg electron orbital radius, i.e., $R \sim R_{n} \approx 2 {n^*}^2$. 
In particular, the work was performed with the most deeply-bound vibrational ($\nu = 0$) dimer state which is spatially well-localized in the outer lobe of the Rydberg wave function as seen in \cref{fig:vary_n-nu0}.
\begin{figure}[!htbp]
	\centering
	\includegraphics[keepaspectratio, width=\textwidth, height=\textheight]{probing_correlations/correlations/vary_n/vary_n-nu0.pdf}
	\caption[]{
		\label{fig:vary_n-nu0}
		Calculated Rydberg molecular potentials of a $\nSLJ{5sns}{3}{S}{1}+\nSLJ{5s^2}{1}{S}{0}$ ULRRM and the $\nu = 0$ radial vibrational wave functions for {$n = 30$, $35$, and $45$}.
		The position of the well-localized $\nu = 0$ vibrational state depends on $n$.}
\end{figure}
By tuning $n$, the location of the $\nu = 0$ state is given by $R_{n} \approx 1.87 {n^*}^{2}$.

The excitation probability of exciting to the $\nu = 0$ ULRRM state can be modeled as an effective Franck-Condon factor of two colliding atoms.
For two free colliding atoms, their wave function can be written as 

*****************

*****************

*****************


The main idea is that the excitation rate of Rydberg molecules depends on an ``effective'' Franck-Condon overlap between the incident (atomic) scattering wave functions and the Rydberg molecular wave functions. 
In particular, the most deeply-bound vibrational ULRRM state ($\nu = 0$) is well-localized with a location that can be related to the classical electron turning point of the parent Rydberg atom.

As seen in  \cref{fig:vary_n-nu0}, different interparticle separations can be probed by varying $n$ of the parent Rydberg atom.
Also, for $\SLJ{}{S}{}$ states, the wave function is spherically symmetric.

\section{Experimental Procedure}

The experimental procedures used for the following experiments are similar to the one presented in \cite{Whalen2019.PRA.100.011402} and so will only be briefly covered.
The production of cold gases of both spin-polarized and unpolarized \Sr{87} begins by loading both \Sr{87} and \Sr{84} in to the magnetic trap by sequentially operating a blue MOT for one isotope before changing to the second isotope, as described in \cref{ssec:magnetic_trap_and_repumping}.
After the atoms are collected in the metastable reservoir, both isotopes are repumped with a pulse of $\SI{481}{\nm}$ light and laser cooling continues with a simultaneous red MOTs for each isotope.
After red MOT cooling, both isotopes are loaded in to the ODT with a temperature of about $\SI{2}{\micro\kelvin}$ at which point the cooling light is extinguished.
At this stage, the cooling sequence for the spin-polarized and unpolarized samples differ slightly.

For the unpolarized sample, the atoms are held in the ODT for a short period of time to allow for free-evaporation before starting forced evaporation.
The spin-polarized sample deviates where, after free-evaporation, a bias magnetic field of about $\SI{7}{\gauss}$ is applied.
Spin-polarization is achieved by optically pumping on the $\nSLJf{5s^2}{1}{S}{0}{{9}/{2}} \rightarrow \nSLJf{5s5p}{3}{P}{1}{{9}/{2}}$ transition, individually driving each ${\Delta}m_{F} = +1$ with $\sigma^{+}$-polarized $\SI{689}{\nm}$ light. 
As seen in \cref{fig:pol_vs_unpol_spectra}, we achieve $\SI{> 90}{\percent}$ population in the $\nSLJfm{5s^2}{1}{S}{0}{{9}/{2}}{{9}/{2}}$ state.
\begin{figure}[!htbp]
	\centering
	\includegraphics[keepaspectratio, width=\textwidth, height=\textheight]{probing_correlations/spin-polarization/producing_spin-polarization/mF_population_vs_duration.pdf}
	\caption[]{
		\label{fig:pol_vs_unpol_spectra}
		**** CHANGE FIGURE TO SHOW UNPOLARIZED VS. POLARIEZD ****
		Checking that linear-linear spectroscopy does not significantly alter our population.}
\end{figure}
Forced evaporation is performed after spin-polarization and the bias magnetic field is ramped down to about $\SI{1}{\gauss}$. 
Prior to Rydberg excitation, a pulse of $\SI{689}{\nm}$ light resonant with the $\nSLJ{5s^2}{1}{S}{0} \rightarrow \nSLJ{5s5p}{3}{P}{1}$ transition in \Sr{84} is applied to remove the bosonic isotope from the ODT.

% \subsection{Spin-Polarization of \Sr{87}}

% In order to realize a fermionic sample, we spin-polarize the \Sr{87} atoms in to the $\nSLJfm{5s^2}{1}{S}{0}{9/2}{9/2}$ state after loading in to the ODT. 
% Once the unpolarized \Sr{87} atoms are loaded in to the ODT, we apply a bias magnetic field of about $\SI{7}{\gauss}$.
% Spin-polarization is achieved by driving the $\nSLJf{5s^2}{1}{S}{0}{9/2} \rightarrow \nSLJf{5s5p}{3}{S}{1}{9/2}$ transition with each $m_F$ state separated by about $\SI{600}{\kHz}$. 
% Using ${\sigma}^{+}$ light, we sequentially drive ${\Delta}m_F=+1$ transitions from $m_F=-{9}/{2}$ to $m_F={7}/{2}$ with $\SI{45}{\ms}$ pulses.
% A last $\SI{500}{\ms}$ ``clean-up'' pulse driving the $m_F={7}/{2} \rightarrow m_{F}^{\prime}={9}/{2}$ helps produce a sample with better than $\SI{90}{\percent}$ population in the $\nSLJfm{5s^2}{1}{S}{0}{9/2}{9/2}$ state. 

% \begin{figure}[!htbp]
	% \centering
	% \includegraphics[keepaspectratio, width=\textwidth, height=\textheight]{probing_correlations/spin-polarization/producing_spin-polarization/mF_population_vs_duration.pdf}
	% \caption[]{
		% \label{fig:lin-lin_vs_loops}
		% Checking that linear-linear spectroscopy does not significantly alter our population.}
% \end{figure}

% Before performing Rydberg excitation, we reduce the bias field to about $\SI{1.2}{\gauss}$. 

% \subsubsection{Evaluation of Spin-Polarization}

% The effectiveness of spin-polarization was evaluated by performing ``linear-linear'' two-photon Rydberg spectroscopy. 
% Crossed-polarized excitation beams travel perpendicular to the bias field with the UV beam driving $\pi$-transitions. 
% In this setup, the $\SI{689}{\nm}$ drives $\nSLJf{5s^2}{1}{S}{0}{9/2} \rightarrow \nSLJf{5s5p}{3}{P}{1}{9/2}$ and the $\SI{320}{\nm}$ drives $\nSLJf{5s5p}{3}{P}{1}{9/2} \rightarrow \nSLJf{5sns}{3}{S}{1}{11/2}$.

% \begin{figure}[!htbp]
	% \centering
	% \includegraphics[keepaspectratio, width=\textwidth, height=\textheight]{probing_correlations/spin-polarization/checking_spin-polarization/lin-lin_vs_loops.pdf}
	% \caption[]{
		% \label{fig:lin-lin_vs_loops}
		% Checking that linear-linear spectroscopy does not significantly alter our population.}
% \end{figure}

With the exception of the $\SI{\approx 1}{\gauss}$ magnetic bias field to maintain spin-polarization (no bias field is applied to the unpolarized sample), both the polarized and unpolarized gases follow the same two-photon Rydberg excitation scheme.
The first photon is at $\SI{689}{\nm}$ with $\sigma^{+}$-polarization and is blue-detuned from the intermediate $\nSLJf{5s5p}{3}{P}{1}{{11}/{2}}$ state by about $\SI{14.8}{\MHz}$.
The second photon is at $\SI{320}{\nm}$ with $\pi$-polarization and is tuned to excite the ULRRMs of the parent $\nSLJf{5sns}{3}{S}{1}{{11}/{2}}$ Rydberg state.
After the Rydberg excitation pulses, a voltage ramp is applied to the in-vacuum electric field plates to ionize any Rydberg atoms/molecules formed and directs them towards the MCP and MCS system for detection and counting. 
During Rydberg excitation, the ODT is turned off to eliminate AC Stark shifts before being turned back on after the field ionization and detection.
We keep our excitation rates low to avoid saturating the MCP.
This excitation-detection procedure is performed $\num{1000}$ times to increase signal-to-noise before a new sample is prepared and the frequency of the $\SI{320}{\nm}$ laser changed.

\section{Extracting the Excitation Rate of $\nu=1$ and $\nu = 2$ ULRRM states}

Spectroscopy was performed for $n = \numrange[range-phrase=-]{31}{41}$.

\begin{figure}[!htbp]
	\centering
	\includegraphics[keepaspectratio, width=\textwidth, height=\textheight]{probing_correlations/excited_dimers/rydberg_excitation_spectra/Fig7.pdf}
	\caption[]{
		\label{fig:excited_ULRRM_spectra}
		**** PLACEHOLDER FIGURE ****.
		Excited ULRRM spectra figure.}
\end{figure}

The integrated Rydberg molecule signal should scale as
\begin{equation}
	S_{n,\nu}	=	\alpha I_{\SI{689}{\nm}} I_{\SI{320}{\nm}} \mathcal{N} \beta_{n} \mathcal{C} \mathcal{F}_{n,\nu}
\end{equation}
where $\alpha$ is the detection efficiency, $I_{\SI{689}{\nm}}$ and $I_{\SI{320}{\nm}}$ are the intensities of the $\SI{689}{\nm}$ and $\SI{320}{\nm}$ excitation lasers, $\mathcal{N}$ is a density-dependent term, $\beta_{n}$ is the two-photon transition matrix element, $\mathcal{C}$ is a Clebsch-Gordan factor, and $\mathcal{F}_{n,\nu}$ is the effective Franck-Condon factor.

\subsection{Density Correction Term}

During data collecting, both the atom number and sample temperature are recorded with resonant absorption imaging on the $\nSLJ{5s^2}{1}{S}{0} \rightarrow \nSLJ{5s5p}{1}{P}{1}$ transition.
All the samples use the same final trapping parameters prior to Rydberg excitation with a vertical trap frequency of $f_{z} \approx \SI{300}{\Hz}$ and radial trap frequency of $f_{r} \approx \SI{130}{\Hz}$.
Knowing these parameters, we can apply a density-dependent correction to the ULRRM signals.

For a thermal gas in a three-dimensional harmonic trap, the density distribution is given by
\begin{equation}
	\rho\pqty{\va{r}}	=	\frac{N}{{\pi}^{3/2} R_{x} R_{y} R_{z}} {\me}^{-{x^2}/{R_x^2}} {\me}^{-{y^2}/{R_y^2}} {\me}^{-{z^2}/{R_z^2}}
\end{equation}
where $N$ is the total atom number and $R_i^2 = {2 k_{B} T}/{m \omega_i^2}$ where $f_{i} = {\omega_i}/{2 \pi}$ is the trap frequency \cite{Pethick.BEC}. 
For the dimers, the total signal should be proportional to $\bqty*{\rho\pqty*{\va{r}}}^{2}$ integrated over all space.
Performing the integration yields
\begin{equation}
	\mathcal{N}_2	=		\int \dd[3]{r} \rho\pqty{\va{r}}
					=		\frac{N^2}{{\pqty{2 \pi}^{3/2}} R_x R_y R_z}
					\propto	\frac{N^2}{T^{3/2}}
\end{equation}
Therefore, we can scale the measured signal by ${N^2}/{T^{3/2}}$ in order to account for the signal dependence on atom number and temperature. 
This correction is applied to all the dimer vibrational excited state data.

\subsection{Extraction of Ratios}

To remove sensitivity to systematics, we examined the ratios of the integral of the {$\nu = 1$ and $2$} ULRRM spectra compared to the $\nu = 0$ state, i.e., $S = \frac{P_{\nu}}{P_{\nu=0}}$.





**********************************


The work presented in \cite{Whalen2019.PRA.100.011402} demonstrated the sensitivity of the excitation rate of dimer $\nu = 0$ vibrational ground-state ULRRMs to $g^{\pqty*{2}}\pqty*{R}$.
An extension of the work is to explore the effects of quantum statistics on the $\nu > 0$ vibrationally excited states.
As seen in \cref{fig:n34_rydberg_molecule_wave_functions_and_spectra}, the $\nu > 0$ states are delocalized and extend towards smaller $R$ than the $\nu = 0$ state and should make them sensitive more sensitive to $g^{\pqty*{2}}\pqty*{R}$. 
Towards this goal, we explored the excitation rates of the $\nu = 1, 2$ states of ULRRMs in spin-polarized and unpolarized samples of \Sr{87}. 

***** ADD MIDDLE *****

Although the results qualitative agreement between theory and experiment, showing that it is possible to extract information about $g^{\pqty*{0}}\pqty*{R}$ from the $\nu > 0$ vibrational states, it was found that the results are very sensitive to initial wave functions. 

\section{Conclusion}

We have measured the photoexcitation rates of the dimer $\nu = 1$ and $\nu = 2$ vibrationally-excited ULRRM states and have shown that they are sensitive to spatial correlations arising from quantum statistics.
In general, there is reasonable agreement between the experimental measurements and theoretical predictions with the spin-polarized data consistently showing weaker excitation compared to the unpolarized gas.
Although weak, there is a small systematic difference between $S_{n} = P_{n,\nu=1,2}/P_{n,\nu=0}$ of the spin-polarized and unpolarized gases.
The presented measurements were very challenging due to long-term drifts in the $\SI{320}{\nm}$ Rydberg laser frequency and to a previously unnoticed beam pointing changes when changing $n$. 
Theoretical calculations are also challenging, as evidenced in \cref{fig:excited_ULRRM_spectra}, where the experimental and theoretical locations of the $\nu = 1$ and $\nu = 2$ ULRRMs lines are slightly shifted.
This suggests that the wave functions of the vibrationally excited state could also be improved.

Knowing these challenges, future work towards stabilizing both the frequency and beam pointing of the Rydberg laser can result in a more accurate spectroscopy.
The updated data can then be passed to theorists so that they can revise the models to more accurately reflect the observed spectra.
Future work should have promising results since it can not only provide a test of theory, but it can also provide a tool for probing spatial correlations on length scales inaccessible with the $\nu = 0$ ULRRM states.

\subsection{Acknowledgments}

We thank Dr. R. G. Hulet for loan of the ODT laser. 