\chapter{Vibrationally-Excited Rydberg Molecules in Cold Gases of \Sr{87}}
\label{ch:correlations}

Many interesting phenomena in quantum systems have a length scale associated with their interaction.
Perhaps one of the most fundamental are the effects of either Bose-Einstein or Fermi-Dirac statistics leading to the bunching or antibunching of particles \cite{Naraschewski1999.PRA.59.4595}.
Another interesting phenomena is the ``fermionization'' of strongly-interacting bosons in a one-dimensional gas \cite{Kinoshita2004.Science.305.1125}.
In these examples, the interactions affect the likelihood of finding another particle within a fixed distance, i.e., they become spatially correlated.
As a result, tools that provide information about the spatial correlations in these systems are extremely insightful.

Perhaps one of the most ingenious ways of probing spatial correlations due to quantum statistics was demonstrated with metastable helium atoms.
By dropping both ultracold cold and quantum degenerate Bose and Fermi gases on to a time-resolved position-sensitive MCP detector, the two-body and three-body pair correlation functions of the initial sample were able to be reconstructed \cite{Schellekens2005.Science.310.648, Jeltes2007.Nature.445.402, Hodgman2011.Science.331.1046}.
Although powerful, this method is neither \textit{in situ} nor nondestructive, requiring the samples to be at equilibrium prior to dropping in order for the reconstruction to make sense, thus hindering the possibility of mapping the temporal development of spatial correlations. 

Alternative, less destructive, \textit{in situ} techniques have also been developed and employed but these are generally restricted by the dimensionality of a system or the length scales accessible.
For the shortest length scales (i.e., $R = 0$ contact interactions), inelastic loss due to three-body recombination has been shown to be sensitive to the effects of $g^{\pqty*{3}}\pqty*{R}$ as evidenced by the different loss rates observed in a nondegenerate compared to a degenerate Bose gases \cite{Burt1997.PRL.79.337}. 
For length scales comparable to the sizes of molecular bonds (i.e., $R \sim R_{\text{vdW}} \lesssim \SI{10}{\nm}$), photoassociative spectroscopy (PAS) can be used 
since efficient photoassociation requires the constituent atoms to have a separation distance comparable to the molecular potential (the van der Waals length $R_{\text{vdW}}$) \cite{Jones2006.RMP.78.483, Jones2006.RMP.78.1041}.
If the effect can be observed in a two (or one) dimensional lattice structure, quantum gas microscopes (QGM) have the capability of directly detecting the presence or absence of atoms on a particular lattice site and, therefore, are able to directly measure spatial correlations \cite{Bakr2010.Science.329.547, Omran2015.PRL.115.263001, Mazurenko2017.Nature.545.462}.
Since QGMs require an optical lattice, their accessible length scales are generally restricted by the wavelength of light used to form the lattice ($R \gtrsim \SI{266}{\nm}$).
\Cref{fig:correlation_tools} illustrates some of the tools and their typical effective length scales.
\begin{figure}[!htbp]
	\centering
	\includegraphics[keepaspectratio, width=5in, height=\textheight]{probing_correlations/correlation_tools/correlation_length_scales.pdf}
	\caption[]{
		\label{fig:correlation_tools}
		A (non-exhaustive) list of lengths scales in typical ultracold gas experiments.
		Black lines show the length scales associated with various interactions and red lines illustrate the effective ranges of various techniques.
		The blue line indicates the length scales probable with ULRRMs for easily accessible ranges of principal quantum numbers ($n$).}
\end{figure}

As seen in \cref{fig:correlation_tools}, there is a gap for length scales $\SI{10}{\nm} \lesssim R \lesssim \SI{266}{\nm}$ which can be explored with ULRRMs. 
Analogous to how PAS can be used to probe correlations on the order of $R_{\text{vdW}}$, the formation rate of ULRRMs can also be used to measure spatial correlations. 
The advantage of ULRRMs is that the length scale being probed depends on the size of the parent Rydberg atom, meaning varying interparticle distances can be explored by simply exciting to a different principal quantum number ($n$). 
This tunability gives ULRRMs a huge advantage and can fill the gap between the short-range PAS and the long-range QGM.

This chapter contains work that is currently in progress. 
******* ADD MORE ??? *******

\section{Probing of Spatial Correlations with ULRRMs}

In an ideal noninteracting gas, the atoms are expected to exhibit quantum statistical effects in their density distributions depending on whether they are bosonic, fermionic, or classical.
For the two-body case, this effect on the density distribution is typically referred to as the spatial correlation function ${g}^{\pqty*{2}}\pqty*{R}$ and can be written (for the ideal noninteracting case) as *** SOURCE? ***
\begin{equation}
	{g}^{\pqty{2}}\pqty{R}	=	1 + \epsilon {\me}^{-2 \pi R^{2}/{\lambda_{\text{dB}}}^{2}}
\end{equation}
where {$\epsilon = +1$, $-1$, and $0$} for bosons, fermions, and classical cases, respectively, and $\lambda_{\text{dB}} = {h}/{\sqrt{2 \pi m k_{B} T}}$ is the thermal de Broglie wavelength.
For a more detailed discussion, see \cite{Naraschewski1999.PRA.59.4595}. **** OTHER REFERENCES? ****

It was shown in \cite{Whalen2019.PRA.100.011402} that the excitation rate of ULRRMs is sensitive to spatial correlations on length scales of the Rydberg electron orbital radius, i.e., $R \sim R_{n} \approx 2 \pqty*{n-\delta}^2$. 
In particular, the work was performed with the most deeply-bound vibrational ($\nu = 0$) dimer state which is spatially well-localized in the outer lobe of the Rydberg electron wave function as seen in \cref{fig:vary_n-nu0}.
\begin{figure}[!htbp]
	\centering
	\includegraphics[keepaspectratio, width=\textwidth, height=\textheight]{probing_correlations/correlations/vary_n/vary_n-nu0.pdf}
	\caption[]{
		\label{fig:vary_n-nu0}
		Calculated radial molecular potentials of a $\nSLJ{5sns}{3}{S}{1}+\nSLJ{5s^2}{1}{S}{0}$ ULRRM and the $\nu = 0$ radial vibrational wave functions for {$n = 30$, $35$, and $45$}.
		As seen in the figure, the position of the well-localized $\nu = 0$ vibrational state depends on $n$.}
\end{figure}
It was this spatially well-localized feature of the $\nu=0$ vibrational state that was demonstrated to be sensitive to the initial spatial correlations in the gas.

The excitation probability of exciting a Rydberg molecule from an ultracold gas can be modeled similarly to photoassociation processes and a schematic is outlined in \cref{fig:ulrrm_photoassociation}.
\begin{figure}[!htbp]
	\centering
	\includegraphics[keepaspectratio, width=4.5in, height=\textheight]{probing_correlations/ulrrm_photoassociation/g2_cartoon_wfn.pdf}
	\caption[]{
		\label{fig:ulrrm_photoassociation}
		Schematic illustration of the excitation process of two free atoms to a vibrational ULRRM state.
		Two colliding free atoms in the $\SLJ{1}{S}{0}-\SLJ{1}{S}{0}$ interaction potential (blue) have the energy-dependent scattering wave function $\ket*{\chi^{E}_{0}}$ (black) with free particle wave vector $k = \sqrt{{2 \mu E}/{\hbar^2}}$ for collision energy $E$ (a thermal average of the initial colliding wave functions is represented by the different $\ket*{\chi^{E}_{0}}$ curves).
		ULRRM potentials for $\SLJ{1}{S}{0} + \nSLJ{5sns}{3}{S}{1}$ states are shown for $n = n, n^{\prime}$ (orange) along with their most deeply-bound $\nu = 0$ wave functions $\ket*{\chi_{n}^{\nu}}$ (green) spatially-localized at $R_{n}^{\nu = 0}$.
		Figure adapted from \cite{Whalen2019.PRA.100.011402}.}
\end{figure}
Starting from a pair of colliding atoms, their scattering wave function can be written as $\ket*{\chi^{E}_{0}}$ where, well-outside the interatomic potential, can be described by free-particle state with wave vector $k = \sqrt{{2 \mu E}/{\hbar^2}}$ in the center-of-mass frame.
(It is this scattering wave function which contains information about the symmetry of the initial states.)
Photoexcitation to a particular ULRRM state with wave function $\ket*{\chi_{n}^{\nu}}$ requires $\mathcal{F}_{n}^{\nu} = \bra*{\chi^{E}_{0}}\ket*{\chi_{n}^{\nu}}$ which can be thought of as a Franck-Condon factor.
Taking a thermal average over the collision energies of the initial two-particle states ($\ev*{\cdots}_{E}$) leads to
\begin{equation}
	\label{eq:effective_fc}
	\ev{\abs{\mathcal{F}_{n}^{\nu}}^2}_{E}
		=	\ev{\abs{\bra{\chi^{E}_{0}}\ket{\chi_{n}^{\nu}}}^{2}}_{E}
		=	\ev{\abs{\int_{0}^{\infty} \dd{R} {R^2} \chi^{E}_{0}\pqty{R} \chi_{n}^{\nu}\pqty{R}}^2}_{E}
\end{equation}

As described in \cite{Whalen2019.PRA.100.011402}, due to the $\nu = 0$ state being well-localized around $R_{n}$, it can be approximated as $\chi_{n}^{\nu=0}\pqty*{R} \approx \delta\pqty*{R - R_n}$.
Evaluating \cref{eq:effective_fc} under this approximation gives\footnote{When evaluating \cref{eq:effective_fc_nu0}, it helps to notice that $\int_{0}^{\infty} \dd{R} {R^2} \chi^{E}_{0}\pqty*{R} A \delta\pqty*{R - R_n} = {R_n^2} \chi^{E}_{0}\pqty*{R_n} A = \chi^{E}_{0}\pqty*{R_n} \int_{0}^{\infty} \dd{R} {R^2} A \delta\pqty*{R-R_n}$.}
\begin{align}
	\label{eq:effective_fc_nu0}
	\ev{\abs{\mathcal{F}_{n}^{\nu=0}}^2}_{E}
		{}&=		\ev{\abs{\int_{0}^{\infty} \dd{R} {R^2} \chi^{E}_{0}\pqty{R} \chi_{n}^{\nu=0}\pqty{R}}^2}_{E}	\notag	\\
		{}&\approx	\ev{\abs{\chi^{E}_{0}\pqty{R_n}}^{2}}_{E} \abs{\int_{0}^{\infty} \dd{R} {R^2} \chi_{n}^{\nu=0}\pqty{R}}^2	\notag	\\
		{}&=		{g^{\pqty{2}}\pqty{R_n}} \abs{\int_{0}^{\infty} \dd{R} {R^2} \chi_{n}^{\nu=0}\pqty{R}}^2
\end{align}
where, in the last line, the relation $\ev*{\abs*{\chi^{E}_{0}\pqty*{R_n}}^{2}}_{E} = g^{\pqty*{2}}\pqty*{R_n}$ with $g^{\pqty*{2}}\pqty*{R_n}$ the pair-correlation function for separation $R_{n}$ was used. 
For the delocalized vibrational states ($\nu > 0$), the approximation $\chi_{n}^{\nu=0}\pqty{R} \approx \delta\pqty*{R_n}$ cannot be made and \cref{eq:effective_fc} must be evaluated with the full $\ket*{\chi_{n}^{\nu}}$ ULRRM wave function.

\section{Experimental Procedure}

The experimental procedures used for the following experiments are similar to the one presented in \cite{Whalen2019.PRA.100.011402} and so will only be briefly covered.
The production of cold gases of both spin-polarized and unpolarized \Sr{87} begins by loading \Sr{87} and \Sr{84} in to the magnetic trap.
As described in \cref{ssec:magnetic_trap_and_repumping}, both isotopes are loaded by sequentially operating a blue MOT on the $\nSLJ{5s^2}{1}{S}{0} \rightarrow \nSLJ{5s5p}{1}{P}{1}$ transition at $\SI{461}{\nm}$ for one isotope before tuning the laser frequency to trap the second isotope.
Laser cooling on the $\SI{461}{\nm}$ transition populates the magnetic trap with atoms following the weak $\nSLJ{5s5p}{1}{P}{1} \rightarrow \nSLJ{5s4d}{1}{D}{2} \rightarrow \nSLJ{5s5p}{3}{P}{2}$ decay path.
After the atoms are collected in the magnetic trap, both isotopes are repumped with a pulse of $\SI{481}{\nm}$ light resonant with the $\nSLJ{5s5p}{3}{P}{2} \rightarrow \nSLJ{5p^2}{3}{P}{2}$ transition and second-stage narrow line laser cooling proceeds simultaneously for both isotopes on the $\nSLJ{5s^2}{1}{S}{0} \rightarrow \nSLJ{5s5p}{3}{P}{1}$ transition at $\SI{689}{\nm}$.
After narrow line cooling, both isotopes are loaded in to the ODT with a temperature of about $\SI{2}{\micro\kelvin}$ at which point all cooling light is extinguished.

Once loaded in to the ODT, sample preparation for the spin-polarized and unpolarized gases differ slightly.
For the unpolarized sample, the atoms are held in the ODT for a short period of time to allow for free-evaporation before starting forced evaporation.
For the spin-polarized sample, an approximately $\SI{7}{\gauss}$ magnetic bias field is ramped on during the free-evaporation hold, providing both a quantization axis and Zeeman splitting the $\nSLJf{5s5p}{3}{P}{1}{{9}/{2}}$ states by approximately $\SI{600}{\kHz}$ per $m_F$. 
Spin-polarization is achieved by optically pumping on the $\nSLJf{5s^2}{1}{S}{0}{{9}/{2}} \rightarrow \nSLJf{5s5p}{3}{P}{1}{{9}/{2}}$ transition, individually addressing each ${\Delta}m_{F} = +1$ transition with $\sigma^{+}$-polarized $\SI{689}{\nm}$ light. 
After spin-polarization, the bias magnetic field is ramped down to $B = B_{\text{pol}} \approx\SI{1}{\gauss}$ to maintain a quantization axis and forced evaporation is performed.

The effectiveness of spin-polarization is evaluated spectroscopically by exciting trapped atoms to a $\nSLJf{5sns}{3}{S}{1}{{11}/{2}}$ Rydberg state in a $B = B_{\text{pol}} \approx\SI{1}{\gauss}$ bias magnetic field.
For this setup, excitation is performed with counterpropagating $\SI{689}{\nm}$ and $\SI{320}{\nm}$ beams both with $\pi$-polarization to drive ${\Delta}m_F = 0$ transitions. 
The $\SI{689}{\nm}$ photon is ${\Delta}/{2\pi} = \SI{14.85+-1}{\MHz}$ blue-detuned of intermediate $\nSLJf{5s5p}{3}{P}{1}{{11}/{2}}$ state and the frequency of the $\SI{320}{\nm}$ laser is scanned.
During excitation, the ODT was turned off to eliminate AC Stark shifts and turned back on to recapture atoms after field ionization and detection.
In order to increase signal, $\pi-\pi$ excitation and measurement procedure is performed $\num{1000}$ times before before a new sample is loaded.
A comparison of excitation in an unpolarized and spin-polarized sample is shown in \cref{fig:n33-check_spin-polarization}.
\begin{figure}[!htbp]
	\centering
	\includegraphics[keepaspectratio, width=5in, height=\textheight]{probing_correlations/spin-polarization/checking_spin-polarization/n33-check_spin-polarization.pdf}
	\caption[]{
		\label{fig:n33-check_spin-polarization}
		Experimental spectra and fit of excitation to the $\nSLJf{5s33s}{3}{S}{1}{{11}/{2}}$ state from both an unpolarized (green) and a spin-polarized (blue) samples.
		A bias magnetic field $B = B_{\text{pol}} \approx\SI{1}{\gauss}$ Zeeman splits the Rydberg state.
		Excitation is performed with $\pi$-polarized counterpropagating $\SI{689}{\nm}$ and $\SI{320}{\nm}$ beams.
		Ideally, only ${\Delta}m_{F} = 0$ transitions are driven with the $\pi$-$\pi$ excitation scheme so only $\num{10}$ peaks should be visible in the unpolarized spectra.
		The small features on the extreme left and right are likely due to imperfect polarization.}
\end{figure}
From the fitted amplitudes, the spin-polarization procedure achieves $\SI{> 90}{\percent}$ of the population in the $\nSLJfm{5s^2}{1}{S}{0}{{9}/{2}}{{9}/{2}}$ state.
It should be noted that the repeated $\pi-\pi$ excitations does cause some population redistribution among the $\nSLJfM{5s^2}{1}{S}{0}{{9}/{2}}{m_F}$ states so it places a lower bound on the effectiveness of spin-polarization.

After forced evaporation, both spin-polarized and unpolarized samples are at about $\SI{900}{\nano\kelvin}$.
Prior to Rydberg excitation, a pulse of $\SI{689}{\nm}$ light resonant with the $\nSLJ{5s^2}{1}{S}{0} \rightarrow \nSLJ{5s5p}{3}{P}{1}$ transition in \Sr{84} is applied to remove the isotope from the ODT.
Due to the large isotope shifts, minimal heating was observed in \Sr{87}.

% \subsection{Spin-Polarization of \Sr{87}}

% In order to realize a fermionic sample, we spin-polarize the \Sr{87} atoms in to the $\nSLJfm{5s^2}{1}{S}{0}{9/2}{9/2}$ state after loading in to the ODT. 
% Once the unpolarized \Sr{87} atoms are loaded in to the ODT, we apply a bias magnetic field of about $\SI{7}{\gauss}$.
% Spin-polarization is achieved by driving the $\nSLJf{5s^2}{1}{S}{0}{9/2} \rightarrow \nSLJf{5s5p}{3}{S}{1}{9/2}$ transition with each $m_F$ state separated by about $\SI{600}{\kHz}$. 
% Using ${\sigma}^{+}$ light, we sequentially drive ${\Delta}m_F=+1$ transitions from $m_F=-{9}/{2}$ to $m_F={7}/{2}$ with $\SI{45}{\ms}$ pulses.
% A last $\SI{500}{\ms}$ ``clean-up'' pulse driving the $m_F={7}/{2} \rightarrow m_{F}^{\prime}={9}/{2}$ helps produce a sample with better than $\SI{90}{\percent}$ population in the $\nSLJfm{5s^2}{1}{S}{0}{9/2}{9/2}$ state. 

% \begin{figure}[!htbp]
	% \centering
	% \includegraphics[keepaspectratio, width=\textwidth, height=\textheight]{probing_correlations/spin-polarization/producing_spin-polarization/mF_population_vs_duration.pdf}
	% \caption[]{
		% \label{fig:lin-lin_vs_loops}
		% Checking that linear-linear spectroscopy does not significantly alter our population.}
% \end{figure}

% Before performing Rydberg excitation, we reduce the bias field to about $\SI{1.2}{\gauss}$. 

% \subsubsection{Evaluation of Spin-Polarization}

% The effectiveness of spin-polarization was evaluated by performing ``linear-linear'' two-photon Rydberg spectroscopy. 
% Crossed-polarized excitation beams travel perpendicular to the bias field with the UV beam driving $\pi$-transitions. 
% In this setup, the $\SI{689}{\nm}$ drives $\nSLJf{5s^2}{1}{S}{0}{9/2} \rightarrow \nSLJf{5s5p}{3}{P}{1}{9/2}$ and the $\SI{320}{\nm}$ drives $\nSLJf{5s5p}{3}{P}{1}{9/2} \rightarrow \nSLJf{5sns}{3}{S}{1}{11/2}$.

% \begin{figure}[!htbp]
	% \centering
	% \includegraphics[keepaspectratio, width=\textwidth, height=\textheight]{probing_correlations/spin-polarization/checking_spin-polarization/lin-lin_vs_loops.pdf}
	% \caption[]{
		% \label{fig:lin-lin_vs_loops}
		% Checking that linear-linear spectroscopy does not significantly alter our population.}
% \end{figure}

With the exception of the $\SI{\approx 1}{\gauss}$ magnetic bias field to maintain spin-polarization (no bias field is applied to the unpolarized sample), both the spin-polarized and unpolarized gases follow the same two-photon Rydberg excitation scheme.
The first photon is at $\SI{689}{\nm}$ with $\sigma^{+}$-polarization and is blue-detuned from the intermediate $\nSLJf{5s5p}{3}{P}{1}{{11}/{2}}$ state by about ${\Delta}/{2 \pi} = \SI{14.85 +- 1}{\MHz}$.
The second photon is at $\SI{320}{\nm}$ with $\pi$-polarization and is tuned to excite ULRRMs of the parent $\nSLJf{5sns}{3}{S}{1}{{11}/{2}}$ Rydberg state.
Excitation powers were kept low to avoid saturating the MCP so signal-to-noise was increased by performing $\num{1000}$ excitation and detection cycles on a sample before a new sample is prepared and the frequency of the $\SI{320}{\nm}$ laser changed.
During Rydberg excitation, the ODT is turned off to eliminate AC Stark shifts before being turned back on after the field ionization and detection.

\section{Spectroscopy of the $\nu=1$ and $\nu = 2$ ULRRM states}

Spectroscopy was performed by exciting ULRRMs of the parent $\nSLJf{5sns}{3}{S}{1}{{11}/{2}}$ Rydberg state for $n = \numrange[range-phrase=-]{31}{41}$.
In the experiment, the photoexcitation signal of the ULRRM dimers can be modeled as
\begin{equation}
	\mathcal{S}_{n}^{\nu}\pqty{f}	=	\alpha I_{\SI{689}{\nm}} I_{\SI{320}{\nm}} \mathcal{N}_2 \beta_{n} \mathcal{C} \ev{\abs{\mathcal{F}_{n}^{\nu}}^2}_{E} L\pqty{f}
\end{equation}
where $\alpha$ is the detection efficiency, $I_{\SI{689}{\nm}}$ and $I_{\SI{320}{\nm}}$ are the intensities of the two-photon excitation lasers, $\beta_n$ is proportional to the square of the reduced two-photon electronic-transition matrix element, $\mathcal{N}_2 \equiv \int \dd[3]{r} \rho\pqty*{R}^2$ is the volume integral of the square of the density distribution, $\mathcal{C}$ is a Clebsch-Gordan factor, and $L\pqty{f}$ is a normalized lineshape function.
To eliminate dependencies on the specific lineshape, the quantity of interest is the integrals of the observed excitation signal
\begin{equation}
	\label{eq:ulrrm_signal}
	\mathcal{S}_{n}^{\nu}
		=	\int \dd{f} \mathcal{S}_{n}^{\nu}\pqty{f}
		=	\alpha I_{\SI{689}{\nm}} I_{\SI{320}{\nm}} \mathcal{N}_2 \beta_{n} \mathcal{C} \ev{\abs{\mathcal{F}_{n}^{\nu}}^2}_{E}
\end{equation}
For a particular $n$, $\beta_{n}$ should remain the same for both spin-polarized and unpolarized samples and the Clebsch-Gordan factor $\mathcal{C}$ can be calculated.
By monitoring the atom number, temperature, $I_{\SI{689}{\nm}}$, and $I_{\SI{320}{\nm}}$, some corrections can be applied to the observed signal to reduce potential systematics.

The first corrections applied to the raw signal involves accounting for any undercounting due to MCP saturation using the model described in \cref{ssec:charged_particle_detection}.
Slow changes in the two-photon excitation rate are easily removed by dividing the signal by $I_{\SI{689}{\nm}}$ and $I_{\SI{320}{\nm}}$ recorded on a DAQ\footnote{We actually record voltages $V_{\SI{689}{\nm}}$ and $V_{\SI{320}{\nm}}$ from monitor photodiodes which are proportional to $I_{\SI{689}{\nm}}$ and $I_{\SI{320}{\nm}}$, respectively.}.
An example two-photon excitation spectra to the $\nSLJf{5s34s}{3}{S}{1}{{11}/{2}}$ parent Rydberg state in both a spin-polarized and an unpolarized \Sr{87} sample is shown in \cref{fig:processing_ULRRM_signal}, demonstrating the effects of removing dependencies on $I_{\SI{689}{\nm}}$, $I_{\SI{320}{\nm}}$, and MCP saturation.
\begin{figure}[!htbp]
	\centering
	\includegraphics[keepaspectratio, width=\textwidth, height=\textheight]{probing_correlations/excited_dimer_signals/20190123-n34/n34-excited_dimer_signal.pdf}
	\caption[]{
		\label{fig:processing_ULRRM_signal}
		Upper figure shows the signals obtained directly from the MCP accumulated from $\num{1000}$ excitation and detection cycles when scanning the $\SI{640}{\nm}$ fiber EOM frequency.
		Lower figure shows the scaled signals after removing the Zeeman shift and accounting for MCP saturation as well as both $I_{\SI{689}{\nm}}$ and $I_{\SI{320}{\nm}}$.
		\Sr{87} atom number was approximately the same for both the spin-polarized and unpolarized samples.
		Data shown is for excitation to the $\nSLJf{5s34s}{3}{S}{1}{{11}/{2}}$ parent state with the theoretically calculated positions of the various vibrational dimer states $\nu$ (lower figure).}
\end{figure}
Fitting is performed on the scaled MCP signals to obtain the integrals of the particular ULRRM line of interest (represented by $\mathcal{S}_{n}^{\nu}$ in \cref{eq:ulrrm_signal}). 
Fitting the spectra in \cref{fig:processing_ULRRM_signal} is shown in \cref{fig:n34_fitted_dimers}.
\begin{figure}[!htbp]
	\centering
	\includegraphics[keepaspectratio, width=\textwidth, height=\textheight]{probing_correlations/excited_dimer_signals/20190123-n34/n34-pol_unpol_fitted.pdf}
	\caption[]{
		\label{fig:n34_fitted_dimers}
		Fitting of the example spin-polarized (upper panel) and unpolarized (lower panel) $\nSLJf{5s34s}{3}{S}{1}{{11}/{2}}$ spectra shown in \cref{fig:processing_ULRRM_signal}.
		Solid lines show the fit results and the shaded regions represent the integrals used to extract $\mathcal{S}_{n}^{\nu}$.}
\end{figure}
Before extracting the ratios, we can still determine the two-body density-dependent correction term $\mathcal{N}_{2}$ to reduce the effects of atom number and temperature variations for the different samples.
To make a comparison between spin-polarized and unpolarized signals, it would also be beneficial to determine the ratio ${\alpha_{\text{pol}}}/{\alpha_{\text{unpol}}}$.

\subsection{Density Correction Term}

During data collecting, both the atom number and sample temperature are recorded with absorption imaging on the $\nSLJ{5s^2}{1}{S}{0} \rightarrow \nSLJ{5s5p}{1}{P}{1}$ transition.
All the samples use the same final ODT trapping parameters prior to Rydberg excitation.
The vertical trap frequency was measured to be $f_{z} \approx \SI{300}{\Hz}$ and radial trap frequency is about $f_{r} \approx \SI{130}{\Hz}$.
Knowing these parameters, the density-dependent $\mathcal{N}_{2}$ factor can be calculated and applied to the ULRRM signals. 

For a thermal gas in a three-dimensional harmonic trap, the density distribution is given by
\begin{equation}
	\rho\pqty{\va{r}}	=	\frac{N}{{\pi}^{3/2} R_{x} R_{y} R_{z}} {\me}^{-{x^2}/{R_x^2}} {\me}^{-{y^2}/{R_y^2}} {\me}^{-{z^2}/{R_z^2}}
\end{equation}
where $N$ is the total atom number and $R_i^2 = {2 k_{B} T}/{m \omega_i^2}$ where $f_{i} = {\omega_i}/{2 \pi}$ is the trap frequency \cite{Pethick.BEC}. 
For the dimers, the total signal should be proportional to $\bqty*{\rho\pqty*{\va{r}}}^{2}$ integrated over all space.
Performing this integration yields
\begin{equation}
	\mathcal{N}_2	=		\int \dd[3]{r} {\rho\pqty{\va{r}}}^2
					=		\frac{N^2}{{\pqty{2 \pi}^{3/2}} R_x R_y R_z}
					\propto	\frac{N^2}{T^{3/2}}
\end{equation}
Therefore, we can scale the measured signal by ${N^2}/{T^{3/2}}$ in order to account for the signal dependence on atom number and temperature. 
Note that, since the same trap was used for both the spin-polarized and unpolarized samples, the dependencies on the trap frequencies drops out.
This correction is applied to all the dimer data.

\subsection{Examining Polarized-Polarized and Unpolarized-Unpolarized Ratios}

In order to eliminate any dependency of $\mathcal{S}_{n}^{\nu}$ on the detection efficiency $\alpha$, spin-polarized signals were compared to spin-polarized signals and unpolarized signals were compared to unpolarized signals within a principal quantum number.
In particular, the ratios of the $\nu = 1,2$ states were compared to the $\nu = 0$ with the ratio defined as
\begin{equation}
	\label{eq:excited_dimer_ratios}
	\frac{\mathcal{S}_{n,x}^{\nu}}{\mathcal{S}_{n,x}^{\nu=0}}
		=	\frac{\alpha_{x} I_{\SI{689}{\nm}} I_{\SI{320}{\nm}} \mathcal{N}_2 \beta_{n} \mathcal{C} \ev{\abs{\mathcal{F}_{n}^{\nu}}^2}_{E}}{\alpha_{x} I_{\SI{689}{\nm}} I_{\SI{320}{\nm}} \mathcal{N}_2 \beta_{n} \mathcal{C} \ev{\abs{\mathcal{F}_{n}^{\nu=0}}^2}_{E}}
		=	\frac{I_{\SI{689}{\nm}} I_{\SI{320}{\nm}} \mathcal{N}_2 \ev{\abs{\mathcal{F}_{n}^{\nu}}^2}_{E}}{I_{\SI{689}{\nm}} I_{\SI{320}{\nm}} \mathcal{N}_2 \ev{\abs{\mathcal{F}_{n}^{\nu=0}}^2}_{E}}
\end{equation}
where $x$ represents the spin-polarized ($\mathcal{S}_{n,\text{pol}}^{\nu}$) or unpolarized ($\mathcal{S}_{n,\text{unpol}}^{\nu}$) signal.
The dependence on $\alpha_{x}$ is eliminated because the data comes from either polarized or unpolarized samples only.
Monitoring the intensities $I_{\SI{689}{\nm}}$ and $I_{\SI{320}{\nm}}$ allows those factors to be scaled out of a particular measurement. 
The recorded atom number and temperature are used to calculate $\mathcal{N}_2$.
Therefore, the ${\ev*{\abs*{\mathcal{F}_{n}^{\nu}}^2}_{E}}/{\ev*{\abs*{\mathcal{F}_{n}^{\nu=0}}^2}_{E}}$ must be the dominant contributing factor to ${\mathcal{S}_{n,x}^{\nu}}/{\mathcal{S}_{n,x}^{\nu=0}}$.

The measured ratios when varying the principal quantum number over $n = \numrange[range-phrase=-]{31}{41}$ is shown in \cref{fig:excited_dimer_ratios}. 
\begin{figure}[!htbp]
	\centering
	\includegraphics[keepaspectratio, width=5in, height=\textheight]{probing_correlations/dimer_ratios/Fig6.pdf}
	\caption[]{
		\label{fig:excited_dimer_ratios}
		Experimentally measured (data points) and theoretically calculated (lines) ratios of the $\nu = 1$ (upper panel) and $\nu = 2$ (lower panel) vibrational states compared to the $\nu = 0$ state of the same $n$ for both spin-polarized (red) and unpolarized (black) samples.}
\end{figure}
The ratios of the $\nu = 1$ vibrational state compared to the $\nu = 0$ state follow the na{\"{i}}ve expectation that they should exhibit additional suppression at lower $n$ due to wave functions extending towards smaller interparticle separations $R$.
The $\nu = 2$ to $\nu = 0$ ratio is counterintuitive and suggests that it is easier to excite the delocalized $\nu = 2$ state at lower $n$ (i.e., smaller interparticle separations) compared to at higher $n$. 

When comparing the spin-polarized to unpolarized measurements, the noisy data makes it difficult to extract quantitative information. 
Qualitatively, the spin-polarized ratios are generally smaller than the unpolarized samples. 
This suggests that quantum statistics could be having an effect on the production of vibrationally excited ULRRM states in addition to the effects on the $\nu = 0$ state.

\subsection{Electron Detection Efficiency Ratio ${\alpha_{\text{pol}}}/{\alpha_{\text{unpol}}}$}

In order to compare the spin-polarized data to the unpolarized data, the electron detection efficiency $\alpha$ needs to be determined.
Although it would be extremely useful to know the exact value of the electron detection efficiency $\alpha$, only the ratio ${\alpha_{\text{pol}}}/{\alpha_{\text{unpol}}}$ is needed since we are interested in how the spin-polarized samples differ from the unpolarized samples.
The ratio ${\alpha_{\text{pol}}}/{\alpha_{\text{unpol}}}$ accounts for the differences in detection efficiency between the spin-polarized sample and unpolarized samples with the former measurements taken in a $B_{\text{pol}} \approx \SI{1}{\gauss}$ bias magnetic field oriented towards the MCP (to maintain spin-polarization) and the latter in no bias field ($B_{\text{unpol}} = \SI{0}{\gauss}$). 
Na{\"{i}}vely, the spin-polarized samples are expected to exhibit a larger signal than the unpolarized sample simply due to the Lorentz force of the magnetic field, oriented towards the MCP, helping guide the electrons towards the detector.
Towards this end, we measured the two-photon excitation signals with \Sr{84} due to its simpler electronic structure in both $B_{\text{pol}}$ and $B_{\text{unpol}}$ configurations.

The measurements with \Sr{84} used the same beam paths and geometry as the measurements in \Sr{87}.
A two-photon excitation following two-photon excitation following the $\nSLJ{5s^2}{1}{S}{0} \rightarrow \nSLJ{5s5p}{3}{P}{1} \rightarrow \nSLJ{5s37s}{3}{S}{1}$ path was used.
The first photon is at $\SI{689}{\nm}$ with $\sigma^+$-polarization and is blue-detuned from the intermediate state by ${\Delta}/{2 \pi} = \SI{14.93+-1}{\MHz}$.
The second photon is at $\SI{320}{\nm}$ with $\pi$-polarization and is tuned such that the two-photon energy is resonant with the Rydberg atom or molecule state.
\Cref{fig:detection_efficiency_vs_B} shows the spectra obtained with and without the bias magnetic field when $\SI{689}{\nm}$ is held fixed while the $\SI{320}{\nm}$ frequency is scanned. 
\begin{figure}[!htbp]
	\centering
	\includegraphics[keepaspectratio, width=5in, height=\textheight]{probing_correlations/detection_efficiency_vs_B/B-field-set_points/detection_efficiency_vs_B.pdf}
	\caption[]{
		\label{fig:detection_efficiency_vs_B}
		The \Sr{84} $\nSLJ{5s37s}{3}{S}{1}$ atomic Rydberg line with $B_{\text{pol}} \approx \SI{1}{\gauss}$ (red circles) and $B_{\text{unpol}} \approx \SI{0}{\gauss}$ (blue squares) bias magnetic fields. 
		The MCP signal was scaled accounting for both the $\SI{689}{\nm}$ and $\SI{320}{\nm}$ powers while the atom number was constant for all the measurements.
		Gaussian fits (lines) were used to extract the integrals, resulting with $\mathcal{S}\pqty*{B_{\text{pol}}} = \num{369+-9}$ (red shaded region) and $\mathcal{S}\pqty*{B_{\text{unpol}}} = \num{278+-7}$ (blue shaded region).}
\end{figure}
As shown in the figure, $\mathcal{S}\pqty*{B_{\text{pol}}} = \num{369+-9}$ and $\mathcal{S}\pqty*{B_{\text{unpol}}} = \num{278+-7}$, their ratio gives ${\alpha\pqty*{B_{\text{pol}}}}/{\alpha\pqty*{B_{\text{unpol}}}} = \num{1.33+-0.05}$.

Before reporting a detection efficiency difference for the $B_{\text{pol}}$ and $ B_{\text{unpol}}$ conditions, there are still a few things to check and consider.
A possible source of systematic error in this determination of the ratio could be due to poor polarization of either (or both) the $\SI{689}{\nm}$ and $\SI{320}{\nm}$ excitation lasers. 
This seems unlikely as there is very little signal at $\Delta = 0$ when the the magnetic field is applied, suggesting that the polarizations are likely reasonably aligned to the quantization axis. 
This also matches our previous measurements checking spin-polarization which found the $\sigma^+$ beam path to have relatively good circular polarization.
The $\SI{320}{\nm}$ path passes through a high-extinction ratio calcite polarizer\footnote{Thorlabs' GLB10 Glan-Laser alpha-BBO calcite polarizer.} that cleans up the polarization.

Another thing to consider is that the Zeeman effect could change the intermediate detuning with the effect scaling as $\Gamma_{\text{sc}} \sim {1}/{\Delta^2}$.
From \cref{fig:detection_efficiency_vs_B}, the observed shift of the $\nSLJ{5s37s}{3}{S}{1}$ state with and without the magnetic field is $\SI{3.326+-0.006}{\MHz}$.
Since the $\nSLJ{5s37s}{3}{S}{1}$ state has $g_J = 2$, this corresponds to $B_{\text{pol}} = \SI{1.1882+-0.0020}{\gauss}$, which agrees with previous calibration measurements using the $\nSLJ{5s^2}{1}{S}{0} \rightarrow \nSLJ{5s5p}{3}{P}{1}$ transition.
Considering the unpolarized intermediate state detuning $\Delta_{\text{unpol}} = \SI{14.9+-1}{\MHz}$, the polarized intermediate state detuning (from the $\nSLJ{5s^2}{1}{S}{0} \rightarrow \nSLJm{5s5p}{3}{P}{1}{1}$) is expected to be reduced to $\Delta_{\text{pol}} = \SI{12.4+-1}{\MHz}$. 
Therefore, the signal in $B_{\text{pol}}$ is expected to be about a factor of ${\Delta_{\text{unpol}}}/{\Delta_{\text{pol}}} = \num{1.44+-0.04}$ larger than the signal in $B_{\text{unpol}}$.

Since the the value of ${\Delta_{\text{unpol}}}/{\Delta_{\text{pol}}}$ is close to the value ${\mathcal{S}\pqty*{B_{\text{pol}}}}/{\mathcal{S}\pqty*{B_{\text{unpol}}}}$, this would suggest that the ``enhanced'' signal in the bias magnetic field is due to the Zeeman shift changing the intermediate state detunings instead of due to the Lorentz force guiding more electrons to the MCP.
Fortuitously, the experimental parameters used to determine ${\alpha_{\text{pol}}}/{\alpha_{\text{unpol}}}$ in \Sr{84} are very similar to the parameters used for the ULRRM measurements in \Sr{87}.
In both measurements, the $\SI{689}{\nm}$ laser was blue-detuned from the intermediate state by approximately $\SI{14.9}{\MHz}$ in no magnetic field (i.e., ${\Delta_{\text{unpol}}}/{2\pi} \approx \SI{14.9}{\MHz}$).
Since the Zeeman shifts of the intermediate states are the same with ${g_J}{m_J} = {3}/{2}$ for the $\nSLJm{5s5p}{3}{P}{1}{1}$ state in \Sr{84} and ${g_F}{m_F} = {3}/{2}$ for the $\nSLJfm{5s5p}{3}{P}{1}{{11}/{2}}{{11}/{2}}$ state in \Sr{87}, we can reasonably expect that the enhancement of spin-polarized signal (with $B = B_{\text{pol}}$) compared to the unpolarized signal (with $B = B_{\text{unpol}}$) to also be given by ${\alpha\pqty*{B_{\text{pol}}}}/{\alpha\pqty*{B_{\text{unpol}}}} = \num{1.33+-0.05}$.

\subsection{Preliminary Comparison of $\mathcal{S}_{n,\text{pol}}^{\nu}$ to $\mathcal{S}_{n,\text{unpol}}^{\nu}$}

A preliminary comparison of the spin-polarized to unpolarized ratios ${\mathcal{S}_{n,\text{pol}}^{\nu}}/{\mathcal{S}_{n,\text{unpol}}^{\nu}}$ for the {$\nu = 0$, $1$, and $2$} vibrational states is shown in \cref{fig:vibrational_pol_unpol_ratio}.
The spin-polarized to unpolarized ratios in the figure include the ${\alpha\pqty*{B_{\text{pol}}}}/{\alpha\pqty*{B_{\text{unpol}}}} = \num{1.33+-0.05}$ factor and the difference in Clebsch-Gordan coupling strength to the Rydberg state.
\begin{figure}[!htbp]
	\centering
	\includegraphics[keepaspectratio, width=5in, height=\textheight]{probing_correlations/dimer_ratios/pol_unpol_ratio.pdf}
	\caption[]{
		\label{fig:vibrational_pol_unpol_ratio}
		Comparison of the ratio ${\mathcal{S}_{n,\text{pol}}^{\nu}}/{\mathcal{S}_{n,\text{unpol}}^{\nu}}$ for {$\nu = 0$, $1$, and $2$} across multiple principal quantum numbers ($n$).
		The spin-polarized to unpolarized ratio is reduced by ${\alpha\pqty*{B_{\text{pol}}}}/{\alpha\pqty*{B_{\text{unpol}}}} = \num{1.33+-0.05}$ and by the differences in the Clebsch-Gordan couplings.}
\end{figure}
With the exception of the data taken at $n = 31$, both the $\nu = 0$ and $\nu = 1$ ratios indicate that the spin-polarized samples exhibit a suppression of the excitation rates compared to the unpolarized samples. 
The $\nu = 2$ ratios do not appear to show a significant deviation and the ratio remains nearly constant. 

From \cite{Whalen2019.PRA.100.011402}, we would expect the $n = 31$ ratio to be further suppressed than for the higher values of $n$ and it is not obvious why this data point does not follow the expected trend.

\section{Conclusion}

We have presented preliminary measurements of the photoexcitation rates of the dimer $\nu = 1$ and $\nu = 2$ vibrationally-excited ULRRM states.
When examining the ratios $\mathcal{S}_{n,\text{pol}}^{\nu}/\mathcal{S}_{n,\text{pol}}^{\nu=0}$ and $\mathcal{S}_{n,\text{unpol}}^{\nu}/\mathcal{S}_{n,\text{unpol}}^{\nu=0}$ shown in \cref{fig:excited_dimer_ratios}, it was observed that there is a weak but consistent systematic reduction of the $\nu=1$ to $\nu=0$ ratio in the spin-polarized sample compared to the unpolarized sample.
This is in agreement with intuition that the less-localized $\nu=1$ vibrational wave function explores smaller interparticle separations which should lead to a suppressed formation rate. 
Conversely, the $\nu=2$ to $\nu=0$ ratio decreases as $n$ increases.

Comparisons between spin-polarized and unpolarized samples is more difficult.
The measurement requires more accurate knowledge of and is sensitive to experimental factors which are not common between the conditions under which data is obtained for the spin-polarized and unpolarized samples. 
\Cref{fig:vibrational_pol_unpol_ratio} does indicate that the $\nu = 1$ polarized-unpolarized ratio follows the general trend of the $\nu = 0$ ratios but slightly reduced, again, in agreement with intuition of the $\nu = 1$ state experiencing additional suppression.
The delocalized $\nu = 2$ state does not appear to show much of a consistent trend.

The presented measurements were very challenging due to long-term drifts in the $\SI{320}{\nm}$ Rydberg laser frequency and to a previously unnoticed changes in beam pointing when tuning to different $n$.
Additional work exploring the vibrational states ULRRM states is worthwhile since they do exhibit more complicated dependencies compared to the $\nu = 0$ state.
Once experimental challenges are overcome, these states can be used to test theoretical calculations of both the ULRRM radial vibrational wave functions and how the effects of quantum statistics affects their production rate. 


% Knowing these challenges, future work towards stabilizing both the frequency and beam pointing of the Rydberg laser can result in a more accurate spectroscopy.
% The updated data can then be passed to theorists so that they can revise the models to more accurately reflect the observed spectra.
% Future work should have promising results since it can not only provide a test of theory, but it can also provide a tool for probing spatial correlations on length scales inaccessible with the $\nu = 0$ ULRRM states.

% \subsection{Considerations for Future Experiments}

% **** We could check spin-polarization with 5s5p 3P1 F=9/2 state instead maybe to check photon polarizations.

% \subsection{Acknowledgments}

% We thank Dr. R. G. Hulet for loan of the ODT laser. 