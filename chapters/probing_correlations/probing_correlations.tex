\chapter{Towards Probing Spatial Correlations with Rydberg Molecules}
\label{ch:correlations}

Many interesting phenomena in quantum systems have a length scale associated with their interaction.
Perhaps one of the most fundamental are the effects of either Bose-Einstein or Fermi-Dirac statistics leading to the bunching or antibunching of particles \cite{Naraschewski1999.PRA.59.4595}.
Another interesting phenomena is the ``fermionization'' of strongly-interacting bosons in a one-dimensional gas \cite{Kinoshita2004.Science.305.1125}.
In these examples, the interactions affect the likelihood of finding another particle separated by a fixed distance., i.e., a spatial correlation.
As a result, tools that provide information regarding the spatial correlation in these systems are extremely insightful.

One of the most ingenious ways of probing spatial correlations due to quantum statistics was demonstrated with metastable helium atoms.
By dropping quantum degenerate Bose and Fermi gases on to a time-resolved position-sensitive MCP detector, the two-body and three-body pair correlation functions of the initial sample were able to be reconstructed \cite{Schellekens2005.Science.310.648, Jeltes2007.Nature.445.402, Hodgman2011.Science.331.1046}.
Although powerful, this method is neither \textit{in situ} nor nondestructive, requiring the samples need to be at equilibrium before the drop in order for a reconstruction to make sense, thus hindering the possibility of mapping the temporal development of spatial correlations. 

Alternative, less destructive, \textit{in situ} techniques have also been developed and employed but these are generally restricted by the dimensionality of a system, the length scales accessible, or both.
For the shortest length scales (i.e., $R = 0$ contact interactions), inelastic loss processes have been demonstrated to be sensitive to three-body spatial correlations \cite{Burt1997.PRL.79.337}. 
For length scales comparable to the sizes of molecular bonds (i.e., $R \sim R_{\text{vdW}} \lesssim \SI{10}{\nm}$), photoassociative spectroscopy (PAS) can be used since these effects take place on the length scale of the van der Waals interaction \cite{Jones2006.RMP.78.483, Jones2006.RMP.78.1041}.
If the effect occurs on a two (or one) dimensional lattice structure, quantum gas microscopes (QGM) have the capability of directly detecting the presence or absence of atoms on a particular lattice site and, therefore, are able to directly measure spatial correlations \cite{Bakr2010.Science.329.547, Omran2015.PRL.115.263001, Mazurenko2017.Nature.545.462}.
Since QGMs require an optical lattice, the length scales they are able to probe are restricted by the optical wavelengths used to form the lattice and typically $R \gtrsim \SI{266}{\nm}$.
\Cref{fig:correlation_tools} illustrates some of the tools developed for probing the effects and their typical ``effective'' length scales.
\begin{figure}[!htbp]
	\centering
	\includesvg[keepaspectratio, width=\textwidth, height=\textheight]{probing_correlations/correlation_tools/probing_correlations.svg}
	\caption[]{
		\label{fig:correlation_tools}
		**** PLACEHOLDER FIGURE ****.
		Various tools for probing \textit{in situ} spatial correlations and their associated length scales.
		Black lines and labels indicate the various length scales in typical ultracold gas experiments. 
		Purple lines illustrate length scales for photoassociative spectroscopy (PAS) and quantum gas microscopes.
		Orange line indicates the length scales probable with ULRRMs.}
\end{figure}

As seen in \cref{fig:correlation_tools}, there is a gap for length scales $\SI{10}{\nm} \lesssim R \lesssim \SI{266}{\nm}$ which can be filled by ULRRMs. 
Analogous to how photoassociation can be used to probe correlations on the order of $R_{\text{vdW}}$, the formation rate of ULRRMs can also be used to measure spatial correlations. 
The advantage of ULRRMs is that the length scale being probed depends on the size of the parent Rydberg atom, meaning varying interparticle distances can be probed by simply exciting to a different principal quantum number $n$. 
This tunability gives ULRRMs a huge advantage and can fill the gap between the short-range PAS and the longer-range $R \gtrsim \SI{266}{\nm}$ QGM.

This chapter contains work that is currently in progress.

\section{Probing of Spatial Correlations with ULRRMs}

The work presented in \cite{Whalen2019.PRA.100.011402} was lead by Joe Whalen so the details will be covered in his upcoming thesis but the ideas are important for understanding the extensions of the work we did. 

In an ideal noninteracting gas, the atoms are expected to exhibit quantum statistical effects in their density distributions depending on whether they are bosonic, fermionic, or classical.
For the two-body case, this effect on the density distribution is typically referred to as ${g}^{\pqty{2}}\pqty{R}$ and can be written (for the ideal noninteracting case) as *** SOURCE? ***
\begin{equation}
	{g}^{\pqty{2}}\pqty{R}	=	1 + \epsilon {\me}^{-2 \pi R^{2}/{\lambda_{\text{dB}}}^{2}}
\end{equation}
where {$\epsilon = +1$, $-1$, and $0$} for bosons, fermions, and classical cases, respectively, and $\lambda_{\text{dB}} = {h}/{\sqrt{2 \pi m k_{B} T}}$ is the thermal de Broglie wavelength.
For a more detailed discussion, see \cite{Naraschewski1999.PRA.59.4595}. **** OTHER REFERENCES? ****

The main idea is that the excitation rate of Rydberg molecules depends on an ``effective'' Franck-Condon overlap between the incident colliding scattering wave functions and the Rydberg molecular wave functions. 
In particular, the most deeply-bound vibrational ULRRM state is well-localized and is related to the classical electron turning point of the parent Rydberg atom (see \cref{fig:vary_n-nu0}). 
\begin{figure}[!htbp]
	\centering
	\includegraphics[keepaspectratio, width=\textwidth, height=\textheight]{probing_correlations/correlations/vary_n/vary_n-nu0.pdf}
	\caption[]{
		\label{fig:vary_n-nu0}
		Calculated Rydberg molecular potentials of a $\nSLJ{5sns}{3}{S}{1}+\nSLJ{5s^2}{1}{S}{0}$ ULLRM and the $\nu = 0$ radial vibrational wave functions for {$n = 30$, $35$, and $45$}.
		The position of the well-localized $\nu = 0$ vibrational state depends on $n$.}
\end{figure}
By varying $n$ of the parent Rydberg atom, different interparticle separations can be probed, i.e. $R \sim {n^*}^{2}$. 
Also, for $\SLJ{}{S}{}$ states, the wave function is spherically symmetric.

\section{Experimental Procedure}

The experimental procedures used for the following experiments are all similar with only minor adjustments depending on the particular isotope(s) being laser cooled and so will only be briefly covered.
For bosonic samples, only \Sr{84} is laser cooled and trapped following the usual blue MOT and red MOT laser cooling stages that results in samples of about $\SI{2}{\micro\kelvin}$ loaded in to the ODT.
After loading, all laser cooling light is extinguished and the atoms are held for a short period of time to allow free-evaporation to occur before forced evaporation is performed by lowering the power of the ODT until the desired temperature is reached.
We confirm that the sample is still thermal and not quantum degenerate by absorption images after a short time-of-flight drop.

Fermionic and classical samples are represented by spin-polarized and unpolarized gases of \Sr{87}, respectively. 
In both cases, we sequentially load both \Sr{87} and \Sr{84} in to the magnetic trap, as described in \cref{ssec:magnetic_trap_and_repumping}, in preparation for sympathetic cooling.
Once both isotopes are loaded, the $\SI{461}{\nm}$ light is extinguished and the $\SI{481}{\nm}$ repumping process returns both isotopes to the ground state. 
Dual red MOTs are then operated to laser cool both \Sr{87} and \Sr{84} isotopes simultaneously, achieving temperatures of about $\SI{2}{\micro\kelvin}$ for loading in to the ODT.
For the spin-polarized sample, an approximately $\SI{7}{\gauss}$ bias magnetic field along the $x$-axis is applied to provide a quantization axis.
Spin-polarization is achieved by optically pumping, sequentially driving ${\Delta}m_{F} = +1$ transitions on the $\nSLJf{5s^2}{1}{S}{0}{{9}/{2}} \rightarrow \nSLJf{5s5p}{3}{P}{1}{{9}/{2}}$.
This results in *** $\SI{>90}{\percent}$ *** of the atoms populating the $\nSLJfm{5s^2}{1}{S}{0}{{9}/{2}}{{9}/{2}}$ state stretched state.
For the unpolarized sample, no bias magnetic field is applied and no optical pumping occurs. 
At this point, both the spin-polarized and unpolarized samples undergo a short hold before forced evaporation commences which continues until the the desired temperature is achieved.
Before moving on to spectroscopy, the \Sr{84} atoms in the ODT are removed with a pulse of light resonant on the $\nSLJ{5s^2}{1}{S}{0} \rightarrow \nSLJ{5s5p}{3}{P}{1}$ transition.

For all samples, a two-photon transition is used to produce ULRRMs.
For bosonic samples, the first photon is $\SI{689}{\nm}$ detuned from the intermediate $\nSLJ{5s5p}{3}{P}{1}$ by **** Xxx MHz **** and the second $\SI{320}{\nm}$ photon is resonant with a particular Rydberg molecular state. 
For the fermionic and classical samples, the first leg is a $\SI{689}{\nm}$ photon detuned from the intermediate $\nSLJf{5s5p}{3}{P}{1}{{11}/{2}}$ state by **** XXX MHz ****. 
The second $\SI{320}{\nm}$ photon then drives states to the $\nSLJf{5sns}{3}{S}{1}{{11}/{2}}$ Rydberg state. 


\subsection{Spin-Polarization of \Sr{87}}

In order to realize a fermionic sample, we spin-polarize the \Sr{87} atoms in to the $\nSLJfm{5s^2}{1}{S}{0}{9/2}{9/2}$ state after loading in to the ODT. 
Once the unpolarized \Sr{87} atoms are loaded in to the ODT, we apply a bias magnetic field of about $\SI{7}{\gauss}$.
Spin-polarization is achieved by driving the $\nSLJf{5s^2}{1}{S}{0}{9/2} \rightarrow \nSLJf{5s5p}{3}{S}{1}{9/2}$ transition with each $m_F$ state separated by about $\SI{600}{\kHz}$. 
Using ${\sigma}^{+}$ light, we sequentially drive ${\Delta}m_F=+1$ transitions from $m_F=-{9}/{2}$ to $m_F={7}/{2}$ with $\SI{45}{\ms}$ pulses.
A last $\SI{500}{\ms}$ ``clean-up'' pulse driving the $m_F={7}/{2} \rightarrow m_{F}^{\prime}={9}/{2}$ helps produce a sample with better than $\SI{90}{\percent}$ population in the $\nSLJfm{5s^2}{1}{S}{0}{9/2}{9/2}$ state. 

\begin{figure}[!htbp]
	\centering
	\includegraphics[keepaspectratio, width=\textwidth, height=\textheight]{probing_correlations/spin-polarization/producing_spin-polarization/mF_population_vs_duration.pdf}
	\caption[]{
		\label{fig:lin-lin_vs_loops}
		Checking that linear-linear spectroscopy does not significantly alter our population.}
\end{figure}

Before performing Rydberg excitation, we reduce the bias field to about $\SI{1.2}{\gauss}$. 

\subsubsection{Evaluation of Spin-Polarization}

The effectiveness of spin-polarization was evaluated by performing ``linear-linear'' two-photon Rydberg spectroscopy. 
Crossed-polarized excitation beams travel perpendicular to the bias field with the UV beam driving $\pi$-transitions. 
In this setup, the $\SI{689}{\nm}$ drives $\nSLJf{5s^2}{1}{S}{0}{9/2} \rightarrow \nSLJf{5s5p}{3}{P}{1}{9/2}$ and the $\SI{320}{\nm}$ drives $\nSLJf{5s5p}{3}{P}{1}{9/2} \rightarrow \nSLJf{5sns}{3}{S}{1}{11/2}$.

\begin{figure}[!htbp]
	\centering
	\includegraphics[keepaspectratio, width=\textwidth, height=\textheight]{probing_correlations/spin-polarization/checking_spin-polarization/lin-lin_vs_loops.pdf}
	\caption[]{
		\label{fig:lin-lin_vs_loops}
		Checking that linear-linear spectroscopy does not significantly alter our population.}
\end{figure}

\subsection{Rydberg Excitation}

Rydberg excitation is performed with a two-photon transition 

\section{Analysis}

The integrated Rydberg molecule signal should scale as
\begin{equation}
	S_{n}	=	\alpha I_{\SI{689}{\nm}} I_{\SI{320}{\nm}} \mathcal{N} \mathcal{C} \mathcal{F}_{n}
\end{equation}

\subsection{Density Correction Term}

During data taking, we record both the sample temperature and atom number. 
In additional to knowing our trap frequencies, we are able to perform a correction for the density-dependent excitation rate. 
For a thermal gas in a three-dimensional harmonic trap, the density distribution is given by \cite{Pethick.BEC}
\begin{equation}
	\rho\pqty{\va{r}}	=	\frac{N}{{\pi}^{3/2} R_{x} R_{y} R_{z}} {\me}^{-{x^2}/{R_x^2}} {\me}^{-{y^2}/{R_y^2}} {\me}^{-{z^2}/{R_z^2}}
\end{equation}
with $R_i^2 = {2 k_{B} T}/{m \omega_i^2}$ where $f_{i} = {\omega_i}/{2 \pi}$ is the trap frequency. 

For the dimers, the signal is proportional to square of the density distribution integrated over all space meaning
\begin{equation}
	\mathcal{N}_2	=		\int \dd[3]{r} \rho\pqty{\va{r}}
					=		\frac{N^2}{{\pqty{2 \pi}^{3/2}} R_x R_y R_z}
					\propto	\frac{N^2}{T^{3/2}}
\end{equation}
Therefore, we can scale the measured signal by ${N^2}/{T^{3/2}}$ in order to reduce the sensitivity to the sample density and temperature. 
This correction is applied to all the dimer vibrational excited state data.

\section{Creation of Vibrationally-Excited Rydberg Molecules in Spin-Polarized and Unpolarized Ultracold Gases of \Sr{87}}

The work presented in \cite{Whalen2019.PRA.100.011402} demonstrated the sensitivity of the excitation rate of dimer $\nu = 0$ vibrational ground-state ULRRMs to $g^{\pqty*{2}}\pqty*{R}$.
An extension of the work is to explore the effects of quantum statistics on the $\nu > 0$ vibrationally excited states.
As seen in \cref{fig:n34_rydberg_molecule_wave_functions_and_spectra}, the $\nu > 0$ states are delocalized and extend towards smaller $R$ than the $\nu = 0$ state and should make them sensitive more sensitive to $g^{\pqty*{2}}\pqty*{R}$. 
Towards this goal, we explored the excitation rates of the $\nu = 1, 2$ states of ULRRMs in spin-polarized and unpolarized samples of \Sr{87}. 

***** ADD MIDDLE *****

Although the results qualitative agreement between theory and experiment, showing that it is possible to extract information about $g^{\pqty*{0}}\pqty*{R}$ from the $\nu > 0$ vibrational states, it was found that the results are very sensitive to initial wave functions. 


\section{Progress Towards Measuring ${g}^{\pqty{3}}\pqty{R}$}

Building on the work presented in \cite{Whalen2019.PRA.100.011402}, we are currently making progress towards measuring non-local three-body correlations with the trimer Rydberg molecules. 

\section{Conclusion}

We have shown ....