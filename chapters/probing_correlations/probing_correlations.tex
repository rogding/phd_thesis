\chapter{Towards Probing Spatial Correlations with Rydberg Molecules}

This chapter contains work that was either recently completed or are currently in progress. 

Several previous experiments were able to probe correlations in Bose and Fermi gases. 
One example is the measurement of three-body recombination rates, which occurs when three atoms spatially coincide, in both a BEC and thermal gas where the BEC exhibited a slower three-body recombination rate compared to the thermal (non-condensed) sample \cite{Burt1997.PRL.79.337}. 
Other experiments using metastable helium were able to reconstruct the three-dimensional pair separations in the cloud to determine both two-body and three-body correlations \cite{Schellekens2005.Science.310.648, Jeltes2007.Nature.445.402, Hodgman2011.Science.331.1046}.


\section{Rydberg Molecules as an \textit{In Situ} Probe of Spatial Correlations}

The work presented in \cite{Whalen2019.PRA.100.011402} was lead by Joe Whalen so the details will be covered in his upcoming thesis but the ideas are important for understanding the extensions of the work we did. 

The main idea is that the excitation rate of Rydberg molecules depends on an ``effective'' Franck-Condon overlap between the incident colliding scattering wave functions and the Rydberg molecular wave functions. 

**********




**********

For a more detailed discussion, see \cite{Naraschewski1999.PRA.59.4595}.



In an ideal noninteracting gas, the atoms are expected to exhibit quantum statistical effects in their density distributions depending on whether they are classical, bosons, or fermions. 
For the two-body case, this effect on the density distribution is typically referred to as ${g}^{\pqty{2}}\pqty{R}$ and can be written (for the ideal noninteracting case) as *** SOURCE ***
\begin{equation}
	{g}^{\pqty{2}}\pqty{R}	=	1 + \epsilon {\me}^{-2 \pi R^{2}/{\lambda_{\text{dB}}}^{2}}
\end{equation}
where 
\begin{equation}
	\epsilon	=	\begin{cases}
						+1	&	\text{boson}		\\
						0	&	\text{classical}	\\
						-1	&	\text{fermion}
					\end{cases}
\end{equation}


\section{Experimental Procedure}

The experimental procedure used for these experiments are all very similar with the goal of producing representative samples of bosonic, fermionic, and classical gases. 
Our boson sample consists of \Sr{84} evaporatively cooled to about ** \SI{700}{\nano\kelvin} **. 

The fermion sample is a gas of spin-polarized \Sr{87} atoms such that they are all in the stretched $\nSLJfm{5s^2}{1}{S}{0}{9/2}{9/2}$ state.

A ``classical'' gas is represented by an unpolarized gas of \Sr{87} (i.e., we did not perform spin-polarization after loading the optical dipole trap with \Sr{87}). 
We had trouble producing cold gases of \Sr{87} so we loaded both \Sr{84} and \Sr{87} in order to enhance the evaporative cooling efficiency in the dipole trap following the methodology above for multi-isotope trapping. 



\subsection{Spin-Polarization of \Sr{87}}

In order to realize the ``ideal'' fermionic sample, we spin-polarize the \Sr{87} atoms in to the $\nSLJfm{5s^2}{1}{S}{0}{9/2}{9/2}$ state after loading in to the ODT. 
Once the unpolarized atoms are loaded in to the ODT, we apply a bias magnetic field of about 
Spin-polarization is achieved by driving the $\nSLJf{5s^2}{1}{S}{0}{9/2} \rightarrow \nSLJf{5s5p}{3}{S}{1}{9/2}$ transition

\subsection{Density Correction Term}

During data taking, we record the sample temperature and atom number. 
In additional to knowing our trap frequencies, we are able to perform a correction for the density-dependent excitation rate. 

\section{Creation of Vibrationally-Excited Rydberg Molecules in Spin-Polarized and Unpolarized Ultracold Gases of \Sr{87}}

\section{Progress Towards Measuring ${g}^{\pqty{3}}\pqty{R}$}

Building on the work in \cite{Whalen2019.PRA.100.011402}