\chapter{Conclusion}
\label{ch:conclusion}

Hopefully this thesis has demonstrated that strontium is an exciting system for studying long-range interactions enabled by Rydberg states.
Starting from the atom itself, as detailed in \cref{ch:introduction}, strontium has a variety of properties not available in other atomic species. 
These include a variety of stable bosonic and fermionic isotopes, narrow and ultranarrow transitions for both laser cooling and for Rydberg excitations, and many accessible Rydberg states due to having two valence electrons with varying interactions strengths and anisotropies.

\Cref{ch:experiment} described the various techniques employed for producing cold and ultracold gases of strontium.
In particular, details relevant for the current strontium Rydberg experiment were covered. 
Over the past few years, the strontium Rydberg experiment has grown from barely being able to sustain a blue MOT (\cref{fig:early_rydberg_apparatus}) to an experiment capable of producing and detecting the effects of quantum statistics (\cref{fig:current_rydberg_apparatus}). 
\begin{figure}[!htbp]
	\centering
	\includegraphics[keepaspectratio, width=\textwidth, height=\textheight]{conclusion/rydberg_apparatus/_MG_2724.JPG}
	\caption[]{
		\label{fig:early_rydberg_apparatus}
		Picture of the Rydberg apparatus taken on 2014/05/29 just before we started preparing to bake the vacuum system.
		We had just observed our first blue MOT the day before. 
		**I'm not sure why the $\SI{461}{\nm}$ beams appear pink; it's probably due to the camera's sensor saturating and bleeding in to the red sensor.**}
\end{figure}
\begin{figure}[!htbp]
	\centering
	\includegraphics[keepaspectratio, width=\textwidth, height=\textheight]{conclusion/rydberg_apparatus/IMG_20180920_162752.jpg}
	\caption[]{
		\label{fig:current_rydberg_apparatus}
		The **current** Rydberg apparatus after **xxx years** of development.
		Picture was taken on **2018/09/20**.}
\end{figure}

\Cref{ch:fermion_spectroscopy} explored the effects of hyperfine interactions on the Rydberg states of \Sr{87}.
Compared to the relatively well-understood structure of \Sr{88} which lacks nuclear spin, the $I = {9}/{2}$ nuclear spin in \Sr{87} leads to strong hyperfine interactions which mixes the $LS$-coupled Rydberg states.
Although the spectra is complicated, it does open the possibility of exciting both singlet and triplet Rydberg states via the $\nSLJ{5s5p}{3}{P}{1}$ state.

We concluded in \cref{ch:correlations} with current efforts towards exploring the effects of quantum statistics on the excitation rates of ULRRMs. 
Explorations of the vibrationally excited $\nu = 1$ state suggest that, while they are sensitive to the effects of quantum statistics, it is difficult to extract quantitative information without accurate knowledge of the ULRRM wave functions. 
We also presented data towards probing three-body correlations with the well-localited $\nu = 0$ vibrational state, an extension of the work presented in \cite{Whalen2019.PRA.100.011402}.

Still, we have only explored a niche portion of what can be explored with strontium Rydberg systems. 
For example, it was recently demonstrated that the weak binding mechanism of Rydberg molecules was used to modify the scattering length much like magnetic Feshbach resonances \cite{Sandor2017.PRA.96.032719, Thomas2018.NatComm.9.2238}. 
\Sr{88} is a particularly attractive candidate as $a_{s} \approx \SI{0}{\bohr}$. 

As for future work, there are many avenues currently left wide open. 
Perhaps one of the most exciting involves dressing ground-state atoms with strongly-interacting Rydberg character \cite{Johnson2010.PRA.82.033412, Honer2010.PRL.105.160404}. 
If the loss and decoherence processes can be overcome \cite{Balewski2014.NJP.16.063012, DeSalvo2016.PRA.93.022709, Aman2016.PRA.93.043425, Goldschmidt2016.PRL.116.113001, Boulier2017.PRA.96.053409}.
One exciting possibility is the realization of a three-dimensional soliton which has currently eluded observation \cite{Maucher2011.PRL.106.170401}. 
\Sr{88} is an ideal candidate as it produces relatively small BECs.

Another avenue could include pursuing Rydberg dressing to create entangled states for improved performance of optical atomic clocks \cite{Gil2014.PRL.112.103601}. 
Work towards this goal are actively being pursued and spin-squeezing was recently demonstrated with potassium \cite{Arias2019.PRL.122.053601}.

\section{Future Directions}

OFR with Rydberg molecules.

Driving Rydberg molecule - Rydberg molecule transitions.

Rotational blockade with ULRRMs and quantum statistics (Sous paper). 

Progress towards g3 measurements.