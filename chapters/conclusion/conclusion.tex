\chapter{Summary}
\label{ch:summary}

Hopefully this thesis has demonstrated that strontium is an exciting system for studying long-range interactions enabled by Rydberg atoms.
Starting from the atom itself, as detailed in \cref{ch:introduction}, strontium has a variety of properties not available in other atomic species. 
These include a variety of stable bosonic and fermionic isotopes, narrow and ultra-narrow transitions for both laser cooling and for Rydberg excitation, and many accessible Rydberg states due to having two valence electrons with varying interactions strengths and anisotropies.

\Cref{ch:experiment} described the various techniques employed for producing cold and ultracold gases of strontium.
In particular, details relevant for the current strontium Rydberg experiment were covered. 
Over the past few years, the strontium Rydberg experiment has grown from barely being able to sustain a blue MOT (\cref{fig:early_rydberg_apparatus}) to an experiment capable of the reaching ultracold necessary for creating and detecting Rydberg atoms and molecules (\cref{fig:current_rydberg_apparatus}). 
\begin{figure}[!htbp]
	\centering
	\includegraphics[keepaspectratio, width=\textwidth, height=\textheight]{conclusion/rydberg_apparatus/_MG_2724.JPG}
	\caption[]{
		\label{fig:early_rydberg_apparatus}
		Picture of the Rydberg apparatus taken on 2014/05/29 just before we started preparing to bake the vacuum system.
		We had observed our first blue MOT the previous day before. 
		The pink appearance of the $\SI{461}{\nm}$ beams are likely due to the camera's sensor saturating.}
\end{figure}
\begin{figure}[!htbp]
	\centering
	\includegraphics[keepaspectratio, width=\textwidth, height=\textheight]{conclusion/rydberg_apparatus/IMG_20180920_162752.jpg}
	\caption[]{
		\label{fig:current_rydberg_apparatus}
		A more recent picture of the Rydberg apparatus taken on 2018/09/20.
		The table got a lot more crowded with many additions to the system.}
\end{figure}

\Cref{ch:fermion_spectroscopy} explored the effects of hyperfine interactions on the Rydberg states of \Sr{87}.
Compared to the relatively well-understood structure of \Sr{88}, the $I = {9}/{2}$ nuclear spin in \Sr{87} leads to strong hyperfine interactions which mix the Rydberg states normally well-described by $LS$-coupling.
Working with theory collaborators, we were able to develop an understanding of the hyperfine Rydberg states that were essential for producing the first fermionic ULRRMs in a gas of \Sr{87}.

We concluded in \cref{ch:correlations} with current efforts exploring the effects of quantum statistics on the excitation of ULRRMs. 
Experiments involving the vibrationally excited $\nu = 1$ states suggest that, while they are sensitive to the effects of quantum statistics, it is difficult to extract quantitative information without accurate knowledge of the ULRRM wave functions. 
We also presented data towards probing two-body correlations with the delocalized {$\nu = 1$ and $2$} vibrational state, an extension of the work presented in \cite{Whalen2019.PRA.100.011402}.

\section{Future Directions}

This section contains various ideas for future work with strontium Rydberg atoms and molecules.
Many of the experiments listed below are achievable with the current Rydberg apparatus or may require some straightforward upgrades.

\subsection{Probing Effects of $N$-Body Spatial Correlations on Formation Rates of ULRRMs}

One of the most straightforward directions for future exploration is to extend the work presented in \cite{Whalen2019.PRA.100.011402} beyond probing two-body spatial correlations to higher-order $N$-body effects. 
That work demonstrated that the production rate of the well-localized $\nu = 0$ vibrational dimer ULRRM states are sensitive to the two-body spatial correlations arising from quantum statistics of the initial gas.
By exciting to different principal quantum numbers, which changes the location of the $\nu = 0$ state, a measurement of $g^\pqty*{2}\pqty*{R}$ can be extracted.
A similar experiment can be carried out by exciting to a $N$-mer ULRRM state which should be sensitive to $g^\pqty*{N}\pqty*{R}$ spatial correlations. 

Considering our explorations of the vibrationally excited ULRRM states, it would make sense to continue to use the well-localized $\nu = 0$ state to eliminate the additional complexity associated with the delocalized $\nu > 0$ states. 
Obtaining sufficient signal-to-noise of $N$-mer state will be challenging considering the dimer signal is about $10^{-1}$ weaker than the atomic line and so the trimers are expected to be about $10^{-2}$ weaker.
Nevertheless, work towards this goal is currently underway with promising measurements in a Fermi gas (spin-polarized \Sr{87}), a Bose gas (\Sr{84}), and a nearly classical gas (unpolarized \Sr{87}).

\subsection{High-Resolution Spectroscopy of ULRRMs}

Improvements to the $\SI{320}{\nm}$ laser systems used to produce Rydberg atoms and molecules could lead to the observation of very weak effects.
In the current work, it was assumed that the embedded neutral atom does not significantly modify the Rydberg electron wave function but it is not clear at what point this approximation becomes invalid.
Under the current approximation, the binding energies of an $N$-mer state is simply $N$ times the binding energy of the dimer state (e.g., a trimer $\nu = 0$ has the same binding energy as two dimer $\nu = 0$ states). 
Deviations from these expected binding energies could be a signature that the approximation is no longer valid. 

Highly accurate measurements of the binding energies could be used to improve theoretical calculations of the ULRRM wave functions. 
During experiments involving the excited ULRRM states, it was found that although the theory predicts the binding energies of the $\nu = 0$ state well, discrepancies exist in the predicted energies of the $\nu = 1$ and $\nu = 2$ states. 
Measurements of the binding energies could potentially be used to constrain the model potential and the resulting wave functions.

Another possibility is the observation of effects of angular momentum of ULRRMs, previously not considered, on their formation in Bose and Fermi gases. 
For fermions in particular, they are expected to fill the various vibrational ULRRM states following shell structure as Pauli statistics exclude fermions from occupying the same state (a ``rotational blockade''`) whereas no suppression should exist for bosons \cite{Sous2019.arXiv.1907.07685}. 






% \subsection{Rydberg Optical Feshbach Resonance}

% Magnetic Feshbach resonances (MFRs) have become an extremely valuable tool where a magnetic field is used to tune the $s$-wave interactions of atoms with magnetically-sensitive ground states.
% In MFRs, the magnetic field tunes the energy of a molecular bound state to be (nearly) resonant with the energy of two colliding atoms so that coupling occurs between the free-atom and molecular states which leads to a modification of the $s$-wave scattering properties \cite{Chin2010.RMP.82.1225}.
% Since not all atoms have magnetically-sensitive ground states (e.g., strontium), optical Feshbach resonances (OFRs) have also been explored where an optical field couples ground-state atoms to a bound molecular state \cite{Blatt2011.PRL.107.073202, Yan2013.PRL.110.123201, Nicholson2015.PRA.92.022709}.
% Unfortunately, OFRs have been found to be susceptible to losses due to scattering of the coupling light. 

% An alternative to coupling to the bound molecular state is to couple to a ULRRM state instead \cite{Sandor2017.PRA.96.032719, Thomas2018.NatComm.9.2238}.
% The advantage of coupling to a ULRRM state is that the coupling should be insensitive 

\subsection{Wave Packet Dynamics with ULRRMs}

Due to the ability to drive Rydberg to Rydberg transitions with microwave radiation, one could study how the ULRRM vibrational states evolve when the parent Rydberg atom is changed from one state to another. 
Since changing the Rydberg state changes the molecular potential experienced by the ground-state atom(s) embedded in the Rydberg electron wave function and therefore the vibrational states, one could consider the microwaves as projecting the initial wave function in to a different potential landscape.
For example, starting from the well-localized $\nu = 0$ ULRRM state with the parent Rydberg atom in the $\nSLJ{5sns}{3}{S}{1}$ state, microwave radiation can excite the parent Rydberg atom to a $\nSLJ{5s{n^\prime}s}{3}{S}{1}$ state which would then project the initially well-localized $\nu = 0$ state at on to the ULRRM vibrational wave functions of the $\nSLJ{5s{n^\prime}s}{3}{S}{1}$ state. 

Another possibility is to excite the parent Rydberg atom to a state with different symmetry. 
For example, microwave excitation could be used to drive the $\nu = 0$ ULRRM state of the initial $\nSLJ{5sns}{3}{S}{1}$ state to $\nSLJ{5snp}{3}{P}{0,1,2}$ which should drastically alter the allowed states since the Rydberg electron wave function now has an angular dependence. 

\subsection{Rydberg Dressing}

One of the most exciting areas of research involves modification of interaction between ground-state atoms by coupling to highly-excited Rydberg states \cite{Johnson2010.PRA.82.033412, Honer2010.PRL.105.160404, Balewski2014.NJP.16.063012, Gaul2016.PRL.116.243001, Helmrich2016.JPB.49.03LT02}.
The idea is that the coupling process ``dresses'' the ground-state atoms by mixing in a fraction of the strongly-interacting Rydberg states while mitigating the challenges of actually producing Rydberg atoms (e.g., decoherence and reduced sample lifetimes).
Rydberg dressing has been proposed to be a potential route towards the observation of a three-dimensional soliton \cite{Maucher2011.PRL.106.170401}, the realization of a supersolid phase \cite{Pupillo2010.PRL.104.223002}, and for enhanced metrology \cite{Gil2014.PRL.112.103601, Arias2019.PRL.122.053601}. 

Although Rydberg dressing has been experimentally demonstrated with two atoms in optical tweezers \cite{Jau2015.NatPhys.12.71} and on a two-dimensional lattice \cite{Zeiher2016.NatPhys.12.1095}, experimental realization of Rydberg dressing in a bulk gas has been difficult. 
Due to the distribution of atoms in a bulk gas, it was observed that Rydberg dressing in a bulk gas leads to a sudden onset of decoherence \cite{DeSalvo2016.PRA.93.022709, Aman2016.PRA.93.043425, Goldschmidt2016.PRL.116.113001} with the effects being attributed to the production of ``contaminant'' Rydberg atoms which leads to avalanche dephasing since it was highly likely to find an atom at the correct separation for resonant interactions \cite{Boulier2017.PRA.96.053409}. 
If these challenges can be overcome, Rydberg dressing is an promising method for incorporating tunable long-range interactions without the need for controling exotic atomic species or molecules. 