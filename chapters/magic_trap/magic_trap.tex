\chapter{Magic Trap}

This project ended up being more of a technical exploration where we wanted to see if we could make \Sr{88} Bose–Einstein condensates (BECs) quickly.
Due to the near zero {\Sr{88}-\Sr{88}} scattering length ({$a_{88-88}\SI{-1.4(6)}{\bohr}$} \cite{Mickelson2005.PRL.95.223002}), making \Sr{88} BECs typically requires sympathetic cooling with another isotope \cite{Mickelson2010.PRA.81.051601, Stellmer2013.PRA.87.013611}. 
But due to the wide range of strontium scattering lengths, a trap optimized for one isotope is unlikely to work well for another isotope (e.g., \Sr{88} needs to be collected at high-density whereas \Sr{86} needs to be kept at low density). 

This necessitates the two isotopes being trapped in two different trapping geometries. 
Since optical trapping scales as {$\flatfrac{\Gamma}{\Delta}$}, one method for doing so is to take advantage of the {$\nSLJ{5s^2}{1}{S}{0} - \nSLJ{5s5p}{3}{P}{1}$} intercombination line which posses a narrow linewidth ({$\flatfrac{\Gamma}{2\pi} = \SI{7.5}{\kHz}$}) and large isotope shifts ({$\flatfrac{\Delta_{\text{isotope}}}{2\pi} \sim \SI{100}{\MHz}$}).

\section{Evaporative Cooling Strontium}

Although there have been some recent developments towards producing quantum degenerate gases without evaporative cooling which take advantage of narrow atomic transitions \cite{Stellmer2013.PRL.110.263003} or using Raman sideband cooling \cite{Hu2017.Science.aan5614}, the majority of the experiments predominately perform (forced) evaporative cooling to reach quantum degeneracy. 
Evaporative cooling works by removing the hot ``tail'' of the atomic velocity distribution after which collisions between atoms rethermalize to a lower temperature. 
Due to the requirements for collisions, the feasibility of evaporative cooling is highly dependent on the scattering length of the atoms. 
\Cref{tab:sr_scattering_lengths_compact} gives the various scattering lengths for strontium.

\begin{table}[h]
	\caption{
		\label{tab:sr_scattering_lengths_compact}
			{$s$-wave} scattering lengths of strontium from \cite{Stein2010.EPJD.57.171} given in \si{\bohr}.
			See \cref{tab:sr_scattering_lengths_long} for a detailed table.}
	\centering
	\begin{tabular}{@{}ccccc@{}}
		\toprule
		Isotope	& \Sr{88}			& \Sr{87}			& \Sr{86}			& \Sr{84}			\\
		\midrule
		\Sr{88}	& \num{-2.00(27)}	& \num{54.819(92)}	& \num{97.374(69)}	& \num{1658(54)}	\\
		\Sr{87}	& \num{54.819(92)}	& \num{96.198(68)}	& \num{162.25(21)}	& \num{-57.61(61)}	\\
		\Sr{86}	& \num{97.374(69)}	& \num{162.25(21)}	& \num{798(12)}		& \num{ 31.65(14)}	\\
		\Sr{84}	& \num{1658(54)}	& \num{-57.61(61)}	& \num{31.65(14)}	& \num{122.762(92)}	\\
		\bottomrule
	\end{tabular}
\end{table}

The {\Sr{88}-\Sr{88}} scattering length is approximately zero ({$a_{88-88}\SI{-1.4(6)}{\bohr}$} \cite{Mickelson2005.PRL.95.223002}) which means the atoms do not thermalize as the optical trap depth is lowered.

The first BECs of \Sr{88} were obtained by sympathetic cooling with \Sr{87} \cite{Mickelson2010.PRA.81.051601}.
The {$a_{88-87} = \SI{55}{\bohr}$} is respectable but it seems like optimum evaporative cooling occurs when the scattering length is around \SI{100}{\bohr} (e.g., \Sr{84}, {\textsuperscript{87}Rb} \cite{Anderson1995.Science.269.5221.198}, etc. ** CHECK !! **)

\subsection{Choosing \Sr{86} as the Coolant}

\section{Trap Design}

The trap geometry also influences how evaporative cooling proceeds due to the density dependences of the elastic and inelastic collision rates where elastic collisions redistribute the energy and results in thermalization whereas inelastic collisions lead to losses (** Add more details!! **).

In order to form BECs of \Sr{86}, evaporative was performed at low density in a large-volume trap \cite{Stellmer2010.82.041602, Stellmer2013.PRA.87.013611}.
On the other hand, \Sr{88} BECs were achieved through sympathetic cooling with \Sr{87} in a tight trapping geometry. 
Therefore, we want the different isotopes to experience different confining potentials:
-\Sr{86} only experiences a large-volume trap.
-\Sr{88} and \Sr{87} experiences a tight-confining potential.

The magic dimple would only provide transverse confinement with the sample temperature predominately set by the vertical trap depth of the horizontal \SI{1064}{\nm} sheet trap. 

\subsection{Lock-In Power Control}

Due to the requirements of having both the trap and cancel beams be perfectly mode-matched with the same polarizations, we decided to use lock-in detection to actively stabilize power of the two beams. 
This was accomplished by chopping the trap and cancel beams at two different frequencies, \SI{\sim500}{\kHz} and \SI{\sim200}{\kHz}, both significantly higher than the trap frequencies of the \SI{1064}{\nm} ODT and the predicted trap frequencies of the dimple.



\section{Producing an \Sr{84} BEC}

With some concerns of heating due to off-resonant scattering of the \SI{689}{\nm} beams, we first tested the dimple with just the trap beam (still being chopped at \SI{500}{\kHz}) in order to see how well we're able to evaporate with the dimple on.

Blah blah blah.

Blah blah blah.

In this configuration, we were able to produce an \Sr{84} BEC in just the sheet trap as well as in the dimple following the same intensity ramp of the sheet. 
(** Show images of evaporation in just sheet vs. evaporation in sheet + dimple **)

\section{Sympathetically cooling \Sr{88} with \Sr{86}}

Sympathetically cooling of \Sr{88} with \Sr{86} was first reported by \citeauthor{Poli2005.PRA.71.061403} but they had difficulty reaching quantum degeneracy.


\section{Preliminary Conclusions and Future Directions}

We don't currently have the capability to conclusively determine whether a doubly-degenerate BEC was formed. Blah, blah, blah.