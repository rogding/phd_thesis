\chapter{Isotope-Dependent Optical Trapping}

This project ended up being more of a technical exploration where we wanted to see if we could make \Sr{88} Bose–Einstein condensates (BECs) quickly.
Due to the near zero {\Sr{88}-\Sr{88}} scattering length ({$a_{88-88}\SI{-1.4(6)}{\bohr}$} \cite{Mickelson2005.PRL.95.223002}), making \Sr{88} BECs typically requires sympathetic cooling with another isotope \cite{Mickelson2010.PRA.81.051601, Stellmer2013.PRA.87.013611}. 
But due to the wide range of strontium scattering lengths, a trap optimized for one isotope is unlikely to work well for another isotope (e.g., \Sr{88} needs to be collected at high-density whereas \Sr{86} needs to be kept at low density). 

This necessitates the two isotopes being trapped in two different trapping geometries. 
Since optical trapping scales as {$\flatfrac{\Gamma}{\Delta}$}, one method for doing so is to take advantage of the {$\nSLJ{5s^2}{1}{S}{0} - \nSLJ{5s5p}{3}{P}{1}$} intercombination line which posses a narrow linewidth ({$\flatfrac{\Gamma}{2\pi} = \SI{7.5}{\kHz}$}) and large isotope shifts ({$\flatfrac{\Delta_{\text{isotope}}}{2\pi} \sim \SI{100}{\MHz}$}).

\section{Challenges in Producing Quantum Degenerate Gases of \Sr{88} and \Sr{87}}

Although there have been some recent developments towards producing quantum degenerate gases without evaporative cooling which take advantage of narrow atomic transitions \cite{Stellmer2013.PRL.110.263003} or using Raman sideband cooling \cite{Hu2017.Science.aan5614}, the majority of the experiments predominately perform (forced) evaporative cooling to reach quantum degeneracy. 
Evaporative cooling works by removing the hot ``tail'' of the atomic velocity distribution after which collisions between atoms rethermalize to a lower temperature. 
Due to the requirements for collisions, the feasibility of evaporative cooling is highly dependent on the scattering length of the atoms. 
\Cref{tab:sr_scattering_lengths_compact} gives the various scattering lengths for strontium.

Being able to quickly produce \Sr{88} BECs are nearly ``perfect'' non-interacting Bosons with no nuclear spin. 
Early attempts at making \Sr{88} BECs failed due to the near zero {$s$-wave} scattering length and it was difficult to evaporative cool this isotope (** REFERENCES? **).
The first BECs of \Sr{88} were obtained by sympathetic cooling with \Sr{87} \cite{Mickelson2010.PRA.81.051601}.
The {$a_{88-87} = \SI{55}{\bohr}$} is respectable but it seems like optimum evaporative cooling occurs when the scattering length is around \SI{100}{\bohr} (e.g., \Sr{84}, {\textsuperscript{87}Rb} \cite{Anderson1995.Science.269.5221.198}, etc. ** CHECK !! **)
Considering that \Sr{88} BECs are typically very small (** ADD REFERENCE AND/OR CALCULATION **) due to the slightly negative {$s$-wave} scattering length, there's interest in studying how the Rydberg electron couples to a BEC \cite{Balewski2013.Nature.12592} and, as $n$ increase, ion-BEC interactions \cite{Kleinbach2018.PRL.120.193401} without needing electrodes for an ion trap.
The small \Sr{88} BEC is almost non-interacting also makes it a great candidate for observing three-dimensional solitons \cite{Maucher2011.PRL.106.170401} as it should be easier to Rydberg dress the entire BEC and there's less mean field interaction energy.

There's also interest in degenerate Fermi gases (DFG) of \Sr{87} due to its large nuclear spin ($I=\flatfrac{9}{2}$) for qubit registers \cite{Gorshkov2009.PRL.102.110503} and quantum simulation/{$SU\left(N\right)$} magnetism \cite{Gorshkov2010.NPhys.1535, Hazzard2012.PRA.85.041604, Pagano2014.NPhys.2878, Hofrichter2016.PRX.6.021030, Ozawa2018.PRL.121.225303} (** CHECK SOURCES/READ A FEW OF THEM **).

\begin{table}[h]
	\caption{
		\label{tab:sr_scattering_lengths_compact}
			{$s$-wave} scattering lengths of strontium from \cite{Stein2010.EPJD.57.171} given in \si{\bohr}.
			See \cref{tab:sr_scattering_lengths_long} for a detailed table.}
	\centering
	\begin{tabular}{@{}ccccc@{}}
		\toprule
		Isotope	& \Sr{88}			& \Sr{87}			& \Sr{86}			& \Sr{84}			\\
		\midrule
		\Sr{88}	& \num{-2.00(27)}	& \num{54.819(92)}	& \num{97.374(69)}	& \num{1658(54)}	\\
		\Sr{87}	& \num{54.819(92)}	& \num{96.198(68)}	& \num{162.25(21)}	& \num{-57.61(61)}	\\
		\Sr{86}	& \num{97.374(69)}	& \num{162.25(21)}	& \num{798(12)}		& \num{ 31.65(14)}	\\
		\Sr{84}	& \num{1658(54)}	& \num{-57.61(61)}	& \num{31.65(14)}	& \num{122.762(92)}	\\
		\bottomrule
	\end{tabular}
\end{table}

The {\Sr{88}-\Sr{88}} scattering length is approximately zero ({$a_{88-88}\SI{-1.4(6)}{\bohr}$} \cite{Mickelson2005.PRL.95.223002}) which means the atoms do not thermalize as the optical trap depth is lowered.

Since \Sr{87} is a fermion, a spin-polarized gas stops colliding (and hence thermalizing) once Pauli blockade sets in (** ELABORATE ADD DETAILS **). 

Early attempts at making \Sr{86} BECs were done in relatively tight optical traps (** CHECK \cite{Ferrari2006.PRA.73.023408} **).
Bose-Einstein condensation of \Sr{86} requires evaporation to occur at low density due to the large $a_{86-86}=\SI{798(12)}{\bohr}$ {$s$-wave} scattering length \cite{Stellmer2010.82.041602, Stellmer2013.PRA.87.013611}. 

\section{Isotope-Dependent Optical Trapping}

As previously mentioned, making \Sr{88} BECs requires sympathetic cooling (** ADD SOURCES **) due to $a_{88-88}=\SI{-2(1)}{\bohr}$. 
Making an \Sr{87} DFG is possible due to its ten spin states \cite{DeSalvo2010.PRL.105.030402} but producing a spin-polarized DFG also requires sympathetic cooling as identical fermions no longer collide \cite{Stellmer2013.PRA.87.013611} (** PROBABLY INCLUDE OTHER SOURCES AS WELL **).
Although having abysmal abundance, \Sr{84} has a nice {$s$-wave} scattering length of $a_{84-84}=\SI{122.762(92)}{\bohr}$ and is generally quite easy to make \Sr{84} BECs using the magnetic trap to capture enough atoms.

It's here that we note \Sr{86} has relatively nice {$s$-wave} scattering lengths with \Sr{88} ($a_{88-86}=\SI{97.374(69)}{\bohr}$) and \Sr{87} ($a_{87-86}=\SI{162.25(21)}{\bohr}$).
Coupled with it's plentiful abundance, \Sr{86} looks to be an ideal cooling for sympathetic cooling if the challenges of the large $a_{86-86}=\SI{798(12)}{\bohr}$ can be overcome.

Since the potential scales as $\flatfrac{\Gamma}{\Delta}$ whereas the off-resonant scattering scales as $\left(\flatfrac{\Gamma}{\Delta}\right)^2$, the idea here is to take advantage of the large isotope shift of the $\nSLJ{5s^2}{1}{S}{0} \rightarrow \nSLJ{5s5p}{3}{P}{1}$ ($\Delta \sim \SI{100}{\MHz}$) compared to the transition linewidth $\flatfrac{\Gamma}{2\pi} = \SI{7.5}{\kHz}$.

\subsection{Trap Design}

The trap geometry also influences how evaporative cooling proceeds due to the density dependences of the elastic and inelastic collision rates where elastic collisions redistribute the energy and results in thermalization whereas inelastic collisions lead to losses (** Add more details!! **).

In order to form BECs of \Sr{86}, evaporative was performed at low density in a large-volume trap \cite{Stellmer2010.82.041602, Stellmer2013.PRA.87.013611}.
On the other hand, \Sr{88} BECs were achieved through sympathetic cooling with \Sr{87} in a tight trapping geometry. 
Therefore, we want the different isotopes to experience different confining potentials:
-\Sr{86} only experiences a large-volume trap.
-\Sr{88} and \Sr{87} experiences a tight-confining potential.

\begin{figure}[h]
	\centering
	\includesvg[keepaspectratio, width=4in, height=\textheight]{magic_trap/dimple_setup/dimple_and_sheet_geometry.svg}
	\caption{
		\label{fig:sheet_and_dimple_geometry}
		Geometry of the large-volume \SI{1064}{\nm} horizontal ``sheet'' trap and the tight vertical \SI{689}{\nm} ``dimple''.}
\end{figure}

The magic dimple would only provide transverse confinement with the sample temperature predominately set by the vertical trap depth of the horizontal \SI{1064}{\nm} sheet trap. 

The ``trap'' beam is generated by a GHz AOM used to injection lock the \Sr{87} trap slave. 
The ``cancel'' beam is generated from a slave injection locked closer to resonance. 

*** Add more details? ***
Based on our experiences with $\SI{60}{\um} \times \SI{60}{\um}$ crossed-beam \SI{1064}{\nm} ODT and the designs of other groups \cite{Stellmer2013.PRA.87.013611}, we decided to try starting with a $\SI{60}{\um} \times \SI{60}{\um}$ circular dimple propagating nearly vertically.
This waist size is relatively easy to generate with reasonably focal length lenses allowing us to place the setup somewhere convenient. 
\begin{figure}[h]
	\centering
	\includesvg[keepaspectratio, width=4in, height=\textheight]{magic_trap/dimple_setup/dimple_profile.svg}
	\caption{
		\label{fig:magic_dimple_profile}
		The dimple is relatively well described by a nearly-circular Gaussian beam with waists $w_{0,x}=\SI{64.9+-1.6}{\um}$ and $w_{0,y}=\SI{59.5+-1.1}{\um}$ located at $z_{0,x}=\SI{27.17+-0.07}{\cm}$ and $z_{0,y}=\SI{26.98+-0.07}{\cm}$.}
\end{figure}

The ``trap'' light is generated from RS3 injection locked at \SI{-1323.44}{\MHz} from the $\nSLJ{5s^2}{1}{S}{0} \rightarrow \nSLJ{5s5p}{3}{P}{1}$ in \Sr{88}, putting it about \SI{-82}{\MHz} detuned of the $\nSLJf{5s^2}{1}{S}{0}{9/2} \rightarrow \nSLJf{5s5p}{3}{P}{1}{11/2}$ transition needed for the operating an \Sr{87} red MOT. 
Since we needed an additional AOM to provide the amplitude modulation for lock-in power control, an additional \SI{-85}{\MHz} was applied, placing the final detuning of the ``trap'' beam at \SI{-1408.44}{\MHz}. 
The ``cancel'' light is generated from RS1 or RS2, both of which are injection locked at \SI{-82}{\MHz} and further shifted by AOMs at \SI{+110}{\MHz} and \SI{-120}{\MHz} to a final detuning around \SI{-92}{\MHz} (** Did we purposefully slightly bias it to be closer to \Sr{86} resonance instead of directly in between? **).
Although I didn't notice significant losses the frequencies for both the ``trap'' and ``cancel'' beams can be chosen to avoid photoassociation lines \cite{Zelevinsky2006.PRL.96.203201, Borkowski2014.PRA.90.032713, Reschovsky2018.arXiv.1808.06507}.

Now the beam size and ``trap'' and ``cancel'' frequencies are chosen, we can calculate the expected trap depths based on the power of each beam.

\subsubsection{Polarizability}

** I'm not sure where I should put this section. **

** Talk about polarizability calculation here ! I still need to work through the derivation... **

Assuming that the AC Stark shift is dominated by the \SI{689}{\nm} transition, we make the two-level approximation where the potential experimenced by the atoms can be approximated by \cite{Grimm1999.arXiv.9902072}
\begin{align}
	U\qty(\vb{r})		&{}={}	-\frac{3 \pi c^2}{2 \omega_{0}^{3}} \qty(\frac{\Gamma}{\omega_{0}-\omega} + \frac{\Gamma}{\omega_{0}+\omega}) I\qty(\vb{r})	\\
	\Gamma\qty(\vb{r})	&{}={}	\frac{3 \pi c^2}{2 \hbar \omega_{0}^{3}} \qty(\frac{\omega}{\omega_{0}})^{2} \qty(\frac{\Gamma}{\omega_{0}-\omega} ++ \frac{\Gamma}{\omega_{0} + \omega})^{2} I\qty(\vb{r})
\end{align}
where $U\qty(\vb{r})$ is the potential and $\Gamma\qty(\vb{r})$ is the scattering rate (see appendix xxx for details... ** I still need to work through the derivation... **).

** Include note about the ``real'' polarizability accounting for other states (i.e., not in the two-level approximation)? See \cite{Boyd2007.PhD} **

\subsubsection{Lock-In Power Control}

\begin{figure}[h]
	\centering
	\includesvg[keepaspectratio, width=\textwidth, height=\textheight]{magic_trap/dimple_setup/dimple_power_control.svg}
	\caption{
		\label{fig:dimple_setup}
		Setup of the \SI{689}{\nm} dimple showing how the ``trap'' and ``cancel'' beams were chopped to facilitate lock-in power control.}
\end{figure}

Due to the requirements of having both the trap and cancel beams be perfectly mode-matched with the same polarizations, we decided to use lock-in detection to actively stabilize power of the two beams. 
This was accomplished by chopping the trap and cancel beams at two different frequencies, \SI{\sim500}{\kHz} and \SI{\sim200}{\kHz}, both significantly higher than the trap frequencies of the \SI{1064}{\nm} ODT and the predicted trap frequencies of the dimple.

\section{Single-Isotope Testing with \Sr{84}}

With some concerns of heating due to off-resonant scattering of the \SI{689}{\nm} ``trap'' and ``cancel'' beams, we first tested the dimple with just the ``trap'' beam (still being chopped at \SI{500}{\kHz}) in order to see how well we're able to evaporate with the dimple on.
We decided to work with \Sr{84} because it's easy to evaporate to quantum degeneracy.
\begin{figure}[h]
	\centering
	\includesvg[keepaspectratio, width=3in, height=\textheight]{magic_trap/dimple_84Sr/dimple-84Sr_lifetime.svg}
	\caption{
		\label{fig:dimple-84Sr_lifetime}
		Lifetime of \Sr{84} in dimple varying the (time-averaged) $P_\text{trap}$.
		Horizontal shaded region represents the lifetime of atoms in just the sheet trap.}
\end{figure}

The \SI{1064}{\nm} sheet trap provides tight confinement along gravity but loose confinement transversely. 
With xxx power in the ``trap'' beam, the transverse trap depth should be able to increase the transverse confinement to about xxx nK. 

In this configuration, we were able to produce an \Sr{84} BEC in just the sheet trap as well as in the dimple following the same intensity ramp of the sheet. 
(** Show images of evaporation in just sheet vs. evaporation in sheet + dimple **)

\subsection{Producing an \Sr{84} BEC in the Dimple}

\begin{figure}[h]
	\centering
	\includesvg[keepaspectratio, width=\textwidth, height=\textheight]{magic_trap/dimple_84Sr/dimple_84Sr_BEC.svg}
	\caption{
		\label{fig:dimple_84Sr_BEC}
		Production of an \Sr{84} BEC with and without the dimple in the sheet trap.
		Absorption image taken after a \SI{40}{\ms} time-of-flight.}
\end{figure}

\begin{figure}[h]
	\centering
	\includesvg[keepaspectratio, width=\textwidth, height=\textheight]{magic_trap/dimple_84Sr/dimple_84Sr_results.svg}
	\caption{
		\label{fig:dimple_84Sr_results}
		Results of evaporating the sheet trap with and without the \SI{689}{\nm} dimple.}
\end{figure}

\section{Sympathetically Cooling \Sr{88} with \Sr{86}}

Sympathetically cooling of \Sr{88} with \Sr{86} was first reported by \citeauthor{Poli2005.PRA.71.061403} but they had difficulty reaching quantum degeneracy.

\begin{figure}[h]
	\centering
	\includesvg[keepaspectratio, width=\textwidth, height=\textheight]{magic_trap/dimple_88Sr-86Sr/dimple_88Sr-86Sr_results.svg}
	\caption{
		\label{fig:dimple_88Sr-86Sr_results}
		Results varying the evaporation time while maintaining the same final sheet trap depth (** CHECK!! **).}
\end{figure}

Possible observation of double-degeneracy with BECs of \Sr{88} and \Sr{86}. 

\begin{figure}[h]
	\centering
	\includesvg[keepaspectratio, width=\textwidth, height=\textheight]{magic_trap/dimple_88Sr-86Sr/possible_88Sr_86Sr_BECs.svg}
	\caption{
		\label{fig:possible_88Sr_86Sr_BECs}
		Possible observation of double-degeneracy of \Sr{88} and \Sr{86} forming BECs.
		Imaged after \SI{40}{\ms} drops and averaging 20 images.
		\Sr{86} is possibly not a BEC but expanding due to hydrodynmaics (** ASK TOM **).
		Our imaging system didn't have good enough resolution to confirm without further investigations.
		(** CHECK!! **)When imaging \Sr{88}, \Sr{86} was removed with a resonant \SI{689}{\nm} beam before imaging and vice-versa for imaging \Sr{86} to remove background atoms.}
\end{figure}

\section{Proposal Sympathetically Cooling \Sr{87} with \Sr{86}}

Using circularly polarized dimple beams should suppress off-resonant scattering off the intermediate state by coupling F=9/2 to F=11/2 in a spin-polarized gas.  

Somewhat similar have been previously proposed for trapping spin-polarized \Rb{85} \cite{Corwin1999.PRL.83.1311, Corwin1999.PhD} (** CHECCK !!**) or producing a spin-dependent optical lattice \cite{Mandel2003.PRL.91.010407, McKay2010.NJP.12.055013, Gadway2010.PRL.105.045303, Reimann2009.MA, Pertot2011.PhD, Gadway2012.PhD} (** CHECK!! **)

\section{Preliminary Conclusions and Future Directions}

We don't currently have the capability to conclusively determine whether a doubly-degenerate BEC was formed. Blah, blah, blah.

*****************

Strontium is a particularly strong candidate for studying ultracold and quantum degenerate gases interacting with ions.
A previous study used a combined ion trap for a \Ybion{174}{+} ion and an ODT for a \Rb{87} BEC but the difficulties with such setups is that the ion is typically very hot compared to the BEC ($T_{\text{ion}} \sim \SI{1}{\kelvin}$) \cite{Zipkes2010.Nature.464.388}.
Lower collision energies were achieved by exciting Rydberg atoms out of a BEC with the orbit of the Rydberg electron being outside the Thomas-Fermi radius of the BEC \cite{Kleinbach2018.PRL.120.193401}.
They were able to observe the reduction of the width of the Rydbreg excitation spectrum in a BEC which indicates that the electron is no longer interacting with the BEC atoms. 
A challenge with that experiment is that \Rb{87} has a $p$-wave scattering resonance which complicates the Rydberg electron-atom interaction at lower $n$ \cite{Butscher2011.JPB.44.184004, Schlagmuller2016.PRL.116.053001}.
Strontium was shown to not have a $p$-wave scattering resonance with Rydberg molecules having significantly longer lifetimes than in rubidium \cite{Camargo2016.PRA.93.022702} (although the lifetimes of Rydberg molecules in a BEC are also quickly destroyed in a BEC \cite{Whalen2017.PRA.96.042702}).