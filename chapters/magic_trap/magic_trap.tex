\chapter{Isotope-Dependent Optical Trapping}

This project ended up being more of a technical exploration where we wanted to see if we could make \Sr{88} Bose–Einstein condensates (BECs) quickly.
Due to the near zero {\Sr{88}-\Sr{88}} scattering length ({$a_{88-88}\SI{-1.4(6)}{\bohr}$} \cite{Mickelson2005.PRL.95.223002}), making \Sr{88} BECs typically requires sympathetic cooling with another isotope \cite{Mickelson2010.PRA.81.051601, Stellmer2013.PRA.87.013611}. 
But due to the wide range of strontium scattering lengths, a trap optimized for one isotope is unlikely to work well for another isotope (e.g., \Sr{88} needs to be collected at high-density whereas \Sr{86} needs to be kept at low density). 

This necessitates the two isotopes being trapped in two different trapping geometries. 
Since optical trapping scales as {$\flatfrac{\Gamma}{\Delta}$}, one method for doing so is to take advantage of the {$\nSLJ{5s^2}{1}{S}{0} - \nSLJ{5s5p}{3}{P}{1}$} intercombination line which posses a narrow linewidth ({$\flatfrac{\Gamma}{2\pi} = \SI{7.5}{\kHz}$}) and large isotope shifts ({$\flatfrac{\Delta_{\text{isotope}}}{2\pi} \sim \SI{100}{\MHz}$}).

\section{Challenges in Producing Quantum Degenerate Gases of \Sr{88} and \Sr{87}}

Although there have been some recent developments towards producing quantum degenerate gases without evaporative cooling which take advantage of narrow atomic transitions \cite{Stellmer2013.PRL.110.263003} or using Raman sideband cooling \cite{Hu2017.Science.aan5614}, the majority of the experiments predominately perform (forced) evaporative cooling to reach quantum degeneracy. 
Evaporative cooling works by removing the hot ``tail'' of the atomic velocity distribution after which collisions between atoms rethermalize to a lower temperature. 
Due to the requirements for collisions, the feasibility of evaporative cooling is highly dependent on the scattering length of the atoms. 
\Cref{tab:sr_scattering_lengths_compact} gives the various scattering lengths for strontium.

Being able to quickly produce \Sr{88} BECs are nearly ``perfect'' non-interacting Bosons with no nuclear spin. 
Early attempts at making \Sr{88} BECs failed due to the near zero {$s$-wave} scattering length and it was difficult to evaporative cool this isotope (** REFERENCES? **).
The first BECs of \Sr{88} were obtained by sympathetic cooling with \Sr{87} \cite{Mickelson2010.PRA.81.051601}.
The {$a_{88-87} = \SI{55}{\bohr}$} is respectable but it seems like optimum evaporative cooling occurs when the scattering length is around \SI{100}{\bohr} (e.g., \Sr{84}, {\textsuperscript{87}Rb} \cite{Anderson1995.Science.269.5221.198}, etc. ** CHECK !! **)
Considering that \Sr{88} BECs are typically very small (** ADD REFERENCE AND/OR CALCULATION **) due to the slightly negative {$s$-wave} scattering length, there's interest in studying how the Rydberg electron couples to a BEC \cite{Balewski2013.Nature.12592} and, as $n$ increase, ion-BEC interactions \cite{Kleinbach2018.PRL.120.193401} without needing electrodes for an ion trap.
The small \Sr{88} BEC is almost non-interacting also makes it a great candidate for observing three-dimensional solitons \cite{Maucher2011.PRL.106.170401} as it should be easier to Rydberg dress the entire BEC and there's less mean field interaction energy.

There's also interest in degenerate Fermi gases (DFG) of \Sr{87} due to its large nuclear spin ($I=\flatfrac{9}{2}$) for qubit registers \cite{Gorshkov2009.PRL.102.110503} and quantum simulation/{$SU\left(N\right)$} magnetism \cite{Gorshkov2010.NPhys.1535, Hazzard2012.PRA.85.041604, Pagano2014.NPhys.2878, Hofrichter2016.PRX.6.021030, Ozawa2018.PRL.121.225303} (** CHECK SOURCES/READ A FEW OF THEM **).


\begin{table}[h]
	\caption{
		\label{tab:sr_scattering_lengths_compact}
			{$s$-wave} scattering lengths of strontium from \cite{Stein2010.EPJD.57.171} given in \si{\bohr}.
			See \cref{tab:sr_scattering_lengths_long} for a detailed table.}
	\centering
	\begin{tabular}{@{}ccccc@{}}
		\toprule
		Isotope	& \Sr{88}			& \Sr{87}			& \Sr{86}			& \Sr{84}			\\
		\midrule
		\Sr{88}	& \num{-2.00(27)}	& \num{54.819(92)}	& \num{97.374(69)}	& \num{1658(54)}	\\
		\Sr{87}	& \num{54.819(92)}	& \num{96.198(68)}	& \num{162.25(21)}	& \num{-57.61(61)}	\\
		\Sr{86}	& \num{97.374(69)}	& \num{162.25(21)}	& \num{798(12)}		& \num{ 31.65(14)}	\\
		\Sr{84}	& \num{1658(54)}	& \num{-57.61(61)}	& \num{31.65(14)}	& \num{122.762(92)}	\\
		\bottomrule
	\end{tabular}
\end{table}

The {\Sr{88}-\Sr{88}} scattering length is approximately zero ({$a_{88-88}\SI{-1.4(6)}{\bohr}$} \cite{Mickelson2005.PRL.95.223002}) which means the atoms do not thermalize as the optical trap depth is lowered.

Since \Sr{87} is a fermion, a spin-polarized gas stops colliding (and hence thermalizing) once Pauli blockade sets in (** ELABORATE ADD DETAILS **). 

Early attempts at making \Sr{86} BECs were done in relatively tight optical traps (** CHECK \cite{Ferrari2006.PRA.73.023408} **).
Bose-Einstein condensation of \Sr{86} requires evaporation to occur at low density due to the large $a_{86-86}=\SI{798(12)}{\bohr}$ {$s$-wave} scattering length \cite{Stellmer2010.82.041602, Stellmer2013.PRA.87.013611}. 

\section{Isotope-Dependent Optical Trapping}

As previously mentioned, making \Sr{88} BECs requires sympathetic cooling (** ADD SOURCES **) due to $a_{88-88}=\SI{-2(1)}{\bohr}$. 
Making an \Sr{87} DFG is possible due to its ten spin states \cite{DeSalvo2010.PRL.105.030402} but producing a spin-polarized DFG also requires sympathetic cooling as identical fermions no longer collide \cite{Stellmer2013.PRA.87.013611} (** PROBABLY INCLUDE OTHER SOURCES AS WELL **).
Although having abysmal abundance, \Sr{84} has a nice {$s$-wave} scattering length of $a_{84-84}=\SI{122.762(92)}{\bohr}$ and is generally quite easy to make \Sr{84} BECs using the magnetic trap to capture enough atoms.

It's here that we note \Sr{86} has relatively nice {$s$-wave} scattering lengths with \Sr{88} ($a_{88-86}=\SI{97.374(69)}{\bohr}$) and \Sr{87} ($a_{87-86}=\SI{162.25(21)}{\bohr}$).
Coupled with it's plentiful abundance, \Sr{86} looks to be an ideal cooling for sympathetic cooling if the challenges of the large $a_{86-86}=\SI{798(12)}{\bohr}$ can be overcome.

\subsection{Trap Design}

The trap geometry also influences how evaporative cooling proceeds due to the density dependences of the elastic and inelastic collision rates where elastic collisions redistribute the energy and results in thermalization whereas inelastic collisions lead to losses (** Add more details!! **).

In order to form BECs of \Sr{86}, evaporative was performed at low density in a large-volume trap \cite{Stellmer2010.82.041602, Stellmer2013.PRA.87.013611}.
On the other hand, \Sr{88} BECs were achieved through sympathetic cooling with \Sr{87} in a tight trapping geometry. 
Therefore, we want the different isotopes to experience different confining potentials:
-\Sr{86} only experiences a large-volume trap.
-\Sr{88} and \Sr{87} experiences a tight-confining potential.

\begin{figure}[h]
	\centering
	\includesvg[keepaspectratio, width=4in, height=\textheight]{magic_trap/dimple_setup/dimple_and_sheet_geometry.svg}
	\caption{
		\label{fig:sheet_and_dimple_geometry}
		Geometry of the large-volume \SI{1064}{\nm} horizontal ``sheet'' trap and the tight vertical \SI{689}{\nm} ``dimple''.}
\end{figure}

The magic dimple would only provide transverse confinement with the sample temperature predominately set by the vertical trap depth of the horizontal \SI{1064}{\nm} sheet trap. 

The ``trap'' beam is generated by a GHz AOM used to injection lock the \Sr{87} trap slave. 
The ``cancel'' beam is generated from a slave injection locked closer to resonance. 

\begin{figure}[h]
	\centering
	\includesvg[keepaspectratio, width=\textwidth, height=\textheight]{magic_trap/dimple_setup/dimple_power_control.svg}
	\caption{
		\label{fig:dimple_setup}
		Setup of the \SI{689}{\nm} dimple showing how the ``trap'' and ``cancel'' beams were chopped to facilitate lock-in power control.}
\end{figure}

\subsection{Lock-In Power Control}

Due to the requirements of having both the trap and cancel beams be perfectly mode-matched with the same polarizations, we decided to use lock-in detection to actively stabilize power of the two beams. 
This was accomplished by chopping the trap and cancel beams at two different frequencies, \SI{\sim500}{\kHz} and \SI{\sim200}{\kHz}, both significantly higher than the trap frequencies of the \SI{1064}{\nm} ODT and the predicted trap frequencies of the dimple.

\section{Producing an \Sr{84} BEC}

With some concerns of heating due to off-resonant scattering of the \SI{689}{\nm} beams, we first tested the dimple with just the trap beam (still being chopped at \SI{500}{\kHz}) in order to see how well we're able to evaporate with the dimple on.

With just these beam parameters, the hypothetical trap frequencies are xxx Hz, yyy Hz, and zzz Hz. 

Blah blah blah.

Blah blah blah.

\begin{figure}[h]
	\centering
	\includesvg[keepaspectratio, width=\textwidth, height=\textheight]{magic_trap/dimple_84Sr/dimple_84Sr_BEC.svg}
	\caption{
		\label{fig:dimple_84Sr_BEC}
		Production of an \Sr{84} BEC with and without the dimple in the sheet trap.
		Absorption image taken after a \SI{40}{\ms} time-of-flight.}
\end{figure}

In this configuration, we were able to produce an \Sr{84} BEC in just the sheet trap as well as in the dimple following the same intensity ramp of the sheet. 
(** Show images of evaporation in just sheet vs. evaporation in sheet + dimple **)

\section{Sympathetically cooling \Sr{88} with \Sr{86}}

Sympathetically cooling of \Sr{88} with \Sr{86} was first reported by \citeauthor{Poli2005.PRA.71.061403} but they had difficulty reaching quantum degeneracy.

\begin{figure}[h]
	\centering
	\includesvg[keepaspectratio, width=\textwidth, height=\textheight]{magic_trap/dimple_88Sr-86Sr/possible_88Sr_86Sr_BECs.svg}
	\caption{
		\label{fig:possible_88Sr_86Sr_BECs}
		Possible observation of double-degeneracy of \Sr{88} and \Sr{86} forming BECs.
		Imaged after \SI{40}{\ms} drops and averaging 20 images.
		\Sr{86} is possibly not a BEC but expanding due to hydrodynmaics (** ASK TOM **).
		Our imaging system didn't have good enough resolution to confirm without further investigations.
		(** CHECK!! **)When imaging \Sr{88}, \Sr{86} was removed with a resonant \SI{689}{\nm} beam before imaging and vice-versa for imaging \Sr{86} to remove background atoms.}
\end{figure}

\section{Preliminary Conclusions and Future Directions}

We don't currently have the capability to conclusively determine whether a doubly-degenerate BEC was formed. Blah, blah, blah.