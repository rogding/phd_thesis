This dissertation describes studies of ultra-long-range Rydberg molecule (ULRRM) formation in \Sr{87} as tool for probing spatial correlations in ultracold gases.
Spatial correlations in ultracold gases can arise from the particles obeying quantum statistics, either Bose-Einstein (BE) or Fermi-Dirac (FD), which leads bosons to exhibit ``bunching'' behavior whereas fermions exhibit ``antibunching'' behavior.
Strontium is an ideal candidate with both bosonic (\Sr{88}, \Sr{86}, and \Sr{84}) and fermionic (\Sr{87}) isotopes which have all been laser cooled to quantum degeneracy. 
Due to the large $I = {9}/{2}$ nuclear spin, \Sr{87} is of particular interest because it can be used to represent both an ideal Fermi gas and an approximation to a classical gas.

Unlike traditional molecular binding mechanisms, ULRRMs are comprised of one or more nearby ground-state atoms bound to a Rydberg atom by the weak Rydberg electron-neutral atom scattering.
Therefore, the production of ULRRMs requires knowing the energies of the parent Rydberg state. 
The bosons lack nuclear spin ($I = 0$) which leads to a particularly simple and well-understood Rydberg spectra whereas the nuclear spin of \Sr{87} results in large hyperfine interactions, greatly complicating its excitation spectra.
In order to understand the hyperfine Rydberg states in \Sr{87}, we performed two-photon spectroscopy to measure and identify the $\SLJ{3}{S}{}$ and $\SLJ{3}{D}{}$ hyperfine Rydberg states for $30 \lesssim n \lesssim 99$.
Working with theory collaborators, we were able to develop an understanding of how the hyperfine interaction affects the Rydberg states and were able to extract revised quantum defects.

Having identified the Rydberg states in \Sr{87}, measurements are currently underway which seeks to explore the effects of quantum statistics on the formation rate of ULRRMs.
Since the bond length of ULRRMs is related to the size of the parent Rydberg atom, tuning the principal quantum number $n$ changes the interparticle separations being probed.
We explore the effects of quantum statistics in ultracold gases of spin-polarized of \Sr{87} which are expected to exhibit an $n$-dependent suppression of the ULRRM excitation rate compared to an unpolarized gas due to FD statistics. 
Although most of the current work involves dimer ULRRMs (one ground-state atom bound to a Rydberg atom), we also describe current progress towards generalizing ULRRMs as a tool for probing spatial correlations with measurements of the formation rate of trimer ULRRMs (two ground-state atoms bound to a Rydberg atom) which are sensitive to three-body spatial correlations. 