This dissertation describes spectroscopic studies of \Sr{87} Rydberg states and its application towards the production of ultra-long-range Rydberg molecules (ULRRMs). 
Most previous spectroscopic studies of strontium were performed with the bosonic isotope \Sr{88} which has $I = 0$, resulting in a relatively simple and well-understood Rydberg excitation spectra. 
In contrast, \Sr{87} has a large $I = {9}/{2}$ nuclear spin that leads to hyperfine interactions, greatly complicating its Rydberg excitation spectra.
In order to understand the Rydberg states in \Sr{87}, we performed two-photon spectroscopy to measure and identify the $\nSLJ{5sns}{3}{S}{}$ and $\nSLJ{5snd}{3}{D}{}$ hyperfine Rydberg states for $30 \lesssim n \lesssim 99$.
Working with theory collaborators, we were able to develop an understanding of how the hyperfine interaction affects the Rydberg states and extracted revised quantum defects.

Using this knowledge of the \Sr{87} hyperfine Rydberg states, we were able to produce the first ULRRMs in a fermionic gas.
Unlike traditional molecular binding mechanisms, ULRRMs are comprised of one or more nearby ground-state atoms bound to a Rydberg atom by the weak Rydberg electron-neutral atom scattering.
Therefore, production of ULRRMs is dependent on both the principal quantum number ($n$) of the parent Rydberg atom and the initial spatial distribution of atoms. 
Since ultracold gases are expected to exhibit effects of either Bose-Einstein or Fermi-Dirac statistics, their effects on the spatial distribution should be reflected in the excitation rates of ULRRMs.
Towards this end, we present current progress exploring the effects of quantum statistics on the formation of ULRRMs in ultracold gases of \Sr{87}.