This dissertation describes the development of an apparatus to undertake spectroscopic studies of \Sr{87} Rydberg states and their application in the production of ultra-long-range Rydberg molecules (ULRRMs). 
Most previous spectroscopic studies of strontium have been performed with the bosonic isotope \Sr{88} which has no nuclear spin ($I = 0$), resulting in a relatively simple and well-understood Rydberg excitation spectrum. 
In contrast, fermionic \Sr{87} has a large nuclear spin ($I = {9}/{2}$) that leads to strong hyperfine interactions which greatly complicate the Rydberg excitation spectrum.
In order to understand the Rydberg states in \Sr{87}, two-photon spectroscopy was performed to measure and identify the $\nSLJ{5sns}{3}{S}{}$ and $\nSLJ{5snd}{3}{D}{}$ hyperfine Rydberg states for $30 \lesssim n \lesssim 99$.
Working with theory collaborators, a detailed understanding of how the hyperfine interaction affects the Rydberg levels is developed and used to extract revised quantum defects.

The detailed understanding of the \Sr{87} hyperfine Rydberg structure was then utilized to produce the first ULRRMs in a fermionic gas.
Unlike traditional molecular binding mechanisms, ULRRMs comprise of one or more ground-state atoms embedded in the electron cloud of a Rydberg atom with the entire system bound together through the weak Rydberg electron-neutral atom scattering. 
Therefore, production of ULRRMs is dependent on both the principal quantum number ($n$) of the parent Rydberg atom and the initial spatial distribution of atoms. 
At low temperatures, the effects of quantum statistics becomes important and result in bunching (bosons) and antibunching (fermions).
These differences in spatial distributions can influence the excitation rates of ULRRMs. 
Current progress in exploring the role of quantum statistics in the excitation of ULRRMs using cold, dense strontium gases is described, with an emphasis on \Sr{87}, and how such measurements can be used to extract the pair correlation function $g^{\pqty*{2}}\pqty*{R}$.