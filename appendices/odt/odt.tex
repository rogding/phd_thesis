\chapter{Optical dipole traps}
%\label{ap:SrProperties}

TO DO:

Go through Gaussian beams section and use widetilde command for complex values/functions. 

\section{Gaussian beams}

Since we almost exclusively work with Gaussian laser beams in the lab, I found it instructive to derive the intensity for an elliptical Gaussian beam.
The derivation below mostly follows from the one presented in \cite{sie_1986}.

%\subsection{Paraxial wave equation}

We start with the wave equation for the electric field $\bm{E} = \widetilde{E}(x,y,z,t) \hat{e}$ where $\hat{e}$ is the unit polarization vector, $\widetilde{E}$ is the complex amplitude, and $c=\lambda f = \omega/k$
\begin{align}
	\label{eq:wave_eq}
	\frac{1}{c^2} \partial_{tt}\widetilde{E}
		= \nabla^2\widetilde{E}
		= \partial_{xx}\widetilde{E} + \partial_{yy}\widetilde{E} + \partial_{zz}\widetilde{E}
\end{align}
Choosing the $z$-axis as the direction of propagation, the time dependence can be removed by assuming $\widetilde{E}(x,y,z,t) = \widetilde{u}(x,y,z)\exp\left[-\iu (k z - \omega t)\right]$
\begin{align}
	\label{eq:paraxial_wave_eq}
	0 = \partial_{xx}\widetilde{u} + \partial_{yy}\widetilde{u} + \partial_{zz}\widetilde{u} -\iu 2 k \partial_{z}\widetilde{u} \approx \partial_{xx}u + \partial_{yy}\widetilde{u} -\iu 2 k \partial_{z}\widetilde{u}
\end{align}
Here, the paraxial approximation $\abs{\partial_{zz}\widetilde{u}} \ll \abs{2 k \partial_{z}\widetilde{u}}, \abs{\partial_{xx}\widetilde{u}}, \abs{\partial_{yy}\widetilde{u}m}$ was made which assumes the envelope of the electric field varies ``slowly'' along $z$.
\cref{eq:paraxial_wave_eq} is generally referred to as the ``paraxial wave equation''.

%\subsubsection{Separating in rectangular coordinates}

For a beam with symmetry along the $x$- and $y$-axis, the paraxial wave equation can be further separated by writing $\widetilde{u}(x,y,z) = \widetilde{u}_{x}(x,z) \widetilde{u}_{y}(y,z)$ where $\widetilde{u}_{x}$ ($\widetilde{u}_{y}$) represents the envelope along the $x$-axis ($y$-axis).
Inserting this in to \cref{eq:paraxial_wave_eq} leads to
\begin{align}
	\label{eq:sep_paraxial_wave_eq}
	0 = u_{x}\left(\partial_{yy}u_{y} - \iu 2 k \partial_{z}u_{y}\right) + u_{y}\left(\partial_{xx}u_{x} - \iu 2 k \partial_{z}u_{x}\right)
\end{align}
Requiring $\widetilde{u}_{x}, \widetilde{u}_{y} \neq 0$ avoids the trivial solution meaning the terms in parentheses must be zero. 
We only need to solve for $\widetilde{u}_{x}$ since the solution for $\widetilde{u}_{y}$ will be similar.

\subsection{Solving $0 = \partial_{xx}\widetilde{u}_{x} - \iu 2 k \partial_{z}\widetilde{u}_{x}$}

Using the trial solution $\widetilde{u}_{x}(x,z) = \widetilde{A}(z) \exp\left[-\iu k \frac{x^2}{2\widetilde{q}(z)}\right]$, where $\widetilde{A}(z)$ and $\widetilde{q}(z)$ can be complex, and collecting terms in powers of $x$
\begin{align}
	0 = -\iu \widetilde{q} \left(\widetilde{A} + 2 q \widetilde{A}^\prime\right) + k \widetilde{A} \left(\widetilde{q}^\prime - 1\right) x^2
\end{align}
Requiring $\widetilde{A}(z), \widetilde{q}(z) \neq 0$ means the terms in the parentheses must be zero
\begin{align}
	0 &= \widetilde{q}^\prime - 1 								\label{eq:q_prime}	\\
	0 &= \widetilde{A} + 2 \widetilde{q} \widetilde{A}^\prime	\label{eq:A_prime}
\end{align}
With the condition $\widetilde{q}(0) = \widetilde{q}_0$, \cref{eq:q_prime} is easily solved
\begin{align}
	\label{eq:q_sol}
	\frac{d\widetilde{q}}{dz} = 1 \implies \widetilde{q}\left(z\right) = z + \widetilde{q}_0
\end{align}
Plugging \cref{eq:q_sol} into \cref{eq:A_prime} and taking $\widetilde{A}(0) = \widetilde{A}_0$
\begin{align}
	\frac{d\widetilde{A}}{dz} = -\frac{\widetilde{A}}{2\widetilde{q}}
		= -\frac{\widetilde{A}}{2\left(\widetilde{q}_0 + z\right)} \implies \widetilde{A}(z)
		= \widetilde{A}_0 \sqrt{\frac{\widetilde{q}_0}{\widetilde{q}_0 + z}}
		= \widetilde{A}_0 \sqrt{\frac{\widetilde{q}_0}{\widetilde{q}(z)}}
\end{align}

%\subsubsection{Complex beam parameter $\widetilde{q}(z)$}

The complex beam parameter $\widetilde{q}(z)$ can be related the real-valued radius of curvature $R(z)$ and spot size $w(z)$ by the definition $\frac{1}{\widetilde{q}(z)} \equiv \frac{1}{R(z)} - \iu \frac{\lambda}{\pi w^2(z)}$.
At the beam waist, taken to be at $z=0$, $R(0) \rightarrow \infty$ so we can relate $\widetilde{q}(0) = \widetilde{q}_0$ to $w(0) = w_0$, 
\begin{align}
	\frac{1}{\widetilde{q_0}} = -\iu \frac{\lambda}{\pi w_0^2} \implies \widetilde{q_0} = \iu \frac{\pi w_0^2}{\lambda} = \iu z_R
\end{align}
where ${z_R \equiv \frac{\pi w_0^2}{\lambda}}$ is the Rayleigh length.
From this, it follows 
\begin{align}
	\frac{1}{z + \iu z_R}
		&= \frac{1}{R} - \iu \frac{\lambda}{\pi w^2}	\nonumber \\
	z + \iu z_R	
		&= \left(\frac{\pi w^2}{\lambda}\right)^2 \frac{R}{R^2 + \left(\frac{\pi w^2}{\lambda}\right)^2} 
			+ \iu \left(\frac{\pi w^2}{\lambda}\right) \frac{R^2}{R^2 + \left(\frac{\pi w^2}{\lambda}\right)^2}
\end{align}
Equating the real and imaginary parts gives
\begin{align}
	R(z)	&= z + \frac{z_R^2}{z}	\\
	w^2(z)	&= w_0^2 \left[1+\left(\frac{z}{z_R}\right)^2\right]
\end{align}

%\subsubsection{The (unnormalized) solution}

Putting all the parts together gives the (unnormalized) solution
\begin{align}
	\widetilde{u}_{x}(x,z) = \widetilde{A}_0 \frac{1}{\sqrt{1 - \iu \frac{z}{z_R}}} \exp\left[-\iu k \frac{x^2}{2}\left(\frac{1}{R(z)} - \iu\frac{\lambda}{\pi w^2(z)}\right)\right]
\end{align}
To normalize, we follow the usual procedure to find $|\widetilde{A}_0|$
\begin{align}
	1	&= \int_{-\infty}^{\infty} \widetilde{u}_{x}^{*}\widetilde{u}_{x} \,dx \nonumber \\
		&= \abs{\widetilde{A}_0}^2 \frac{1}{\sqrt{1 + \left(\frac{z}{z_R}\right)^2}} \int_{-\infty}^{\infty} \exp\left[-\frac{2 x^2}{w^2(z)}\right] \,dx \nonumber \\
		&= \abs{\widetilde{A}_0}^2 \frac{1}{\sqrt{1 + \left(\frac{z}{z_R}\right)^2}} \sqrt{\frac{\pi}{2}} w(z) \nonumber \\
		&= \abs{\widetilde{A}_0}^2 \sqrt{\frac{\pi}{2}} w_0
\end{align}
Therefore the solution for the $x$-axis is
\begin{align}
	\abs{\widetilde{A}_0}^2 = \sqrt{\frac{2}{\pi}}\frac{1}{w_0} \implies \abs{\widetilde{A}_0} = \pm \left(\frac{2}{\pi}\right)^{1/4} \frac{1}{\sqrt{w_0}}
\end{align}
Taking the positive value for $\abs{\widetilde{A}_0}$ gives the normalized solution for the envelope along the $x$-axis
\begin{align}
	\widetilde{u}_{x}(x,z)
		&= \left(\frac{2}{\pi}\right)^{1/4} \frac{1}{\sqrt{w_0\left(1 - \iu \frac{z}{z_R}\right)}} \exp\left[-\iu k \frac{x^2}{2}\left(\frac{1}{R(z)} - \iu\frac{\lambda}{\pi w^2(z)}\right)\right]
\end{align}

\subsection{Normalized solution}

(**the parts below get messy and need to be worked on.)

A general solution $\widetilde{u}(x,y,z)$ with $x$-waist ($y$-waist) of $w_{x,0}$ ($w_{y,0}$) at $z_{x,0}$ ($z_{y,0}$)
\begin{align}
	\label{eq:E_sol}
	\widetilde{u}(x,y,z)
		=& \widetilde{u}_x(x-x_0,z) \widetilde{u}_y(y-y_0,z) \exp\left[-\iu (k z - \omega t)\right] \nonumber \\
		=& \left(\frac{2}{\pi}\right)^{1/2}
			\frac{1}{\sqrt{w_{x,0}\left(1 - \iu \frac{z-z_{x,0}}{z_{x,R}}\right)}} \frac{1}{\sqrt{w_{y,0}\left(1 - \iu \frac{z-z_{y,0}}{z_{y,R}}\right)}} \nonumber \\
		& \times \exp\left[-\iu k \frac{x^2}{2}\left(\frac{1}{R(z-z_{x,0})} - \iu\frac{\lambda}{\pi w^2(z-z_{x,0})}\right)\right] \nonumber \\
		& \times \exp\left[-\iu (k {z-z_{x,0}} - \omega t)\right] \nonumber \\
\end{align}


**************************

Interesting discussions about astigmatic Gaussian beams are discussed in \cite{ark_1969, kws_2013}. 