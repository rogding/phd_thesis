\chapter{Optical dipole traps}
%\label{ap:SrProperties}

TO DO:

Go through Gaussian beams section and use widetilde command for complex values/functions. 

\section{Gaussian beams}

Since we almost exclusively work with Gaussian laser beams, it will be instructive to derive the intensity for a Gaussian beam.
The derivation presented below for the intensity distribution of a Gaussian beams mostly follows from the one presented in \cite{sie_1986} which itself references \cite{kol_1966}. 

\subsection{Paraxial wave equation}

Starting from the wave equation for the electric field ${\bm{E} = \widetilde{E}(x,y,z,t) \hat{e}}$ for unit polarization vector $\hat{e}$ and complex electric field amplitude $\widetilde{E}$
\begin{align}
	\label{eq:wave_eq}
	\frac{1}{c^2} \partial_{tt}\widetilde{E}
		= \nabla^2\widetilde{E}
		= \partial_{xx}\widetilde{E} + \partial_{yy}\widetilde{E} + \partial_{zz}\widetilde{E}
\end{align}
where ${c=\lambda f = \omega/k}$.
Choosing the $z$-axis as the direction of propagation and assuming ${\widetilde{E}(x,y,z,t) = \widetilde{u}(x,y,z)\exp\left[-\iu (k z - \omega t)\right]}$, \cref{eq:wave_eq} can be written as
\begin{align}
	\label{eq:paraxial_wave_eq}
	0 = 
		\partial_{xx}\widetilde{u} + \partial_{yy}\widetilde{u} + \partial_{zz}\widetilde{u} -\iu 2 k \partial_{z}\widetilde{u} \approx \partial_{xx}u + \partial_{yy}\widetilde{u} -\iu 2 k \partial_{z}\widetilde{u}
\end{align}
where the paraxial approximation ${\abs{\partial_{zz}\widetilde{u}} \ll \abs{2 k \partial_{z}\widetilde{u}}, \abs{\partial_{xx}\widetilde{u}}, \abs{\partial_{yy}\widetilde{u}m}}$ was made which assumes the envelope of the electric field varies ``slowly'' with respect to $z$.
\cref{eq:paraxial_wave_eq} is generally referred to as the ``paraxial wave equation''.

\subsubsection{Separating in rectangular coordinates}

Assuming the symmetry of the beam is along the $x$- and $y$-axis, \cref{eq:paraxial_wave_eq} can be further separated. 
Writing ${\widetilde{u}(x,y,z)} = \widetilde{u}_{x}(x,z) \widetilde{u}_{y}(y,z)$, with $\widetilde{u}_{x}$ ($\widetilde{u}_{y}$) representing the envelope along the $x$-axis ($y$-axis), and inserting in to \cref{eq:paraxial_wave_eq} leads to
\begin{align}
	\label{eq:sep_paraxial_wave_eq}
	0 = u_{x}\left(\partial_{yy}u_{y} - \iu 2 k \partial_{z}u_{y}\right) + u_{y}\left(\partial_{xx}u_{x} - \iu 2 k \partial_{z}u_{x}\right)
\end{align}
Requiring ${\widetilde{u}_{m} \neq 0}$ for ${m \in \left\{x,y\right\}}$ avoids the trivial solution meaning the terms in parenthesis must be zero. 
Without loss of generality, we only need to solve the case ${0 = \partial_{xx}\widetilde{u}_{x} - \iu 2 k \partial_{z}\widetilde{u}_{x}}$.

\subsection{Solving ${0 = \partial_{xx}\widetilde{u}_{x} - \iu 2 k \partial_{z}\widetilde{u}_{x}}$}

Assuming the trial solution ${\widetilde{u}_{x}(x,z) = \widetilde{A}(z) \exp\left[-\iu k \frac{x^2}{2\widetilde{q}(z)}\right]}$, where $\widetilde{A}(z)$ and $\widetilde{q}(z)$ can be complex, and collecting terms in powers of $x$
\begin{align}
	0 = -\iu \widetilde{q} \left(\widetilde{A} + 2 q \widetilde{A}^\prime\right) + k x^2 \widetilde{A} \left(\widetilde{q}^\prime - 1\right)
\end{align}
Requiring ${\widetilde{A}(z), \widetilde{q}(z) \neq 0}$ means the terms in the parentheses must be zero
\begin{align}
	0 &= \widetilde{q}^\prime - 1 								\label{eq:q_prime}	\\
	0 &= \widetilde{A} + 2 \widetilde{q} \widetilde{A}^\prime	\label{eq:A_prime}
\end{align}

With the condition ${\widetilde{q}(0) = \widetilde{q}_0}$, \cref{eq:q_prime} is easily solved
\begin{align}
	\frac{d\widetilde{q}}{dz} = 1 \implies \widetilde{q}\left(z\right) = \widetilde{q}_0 + z
\end{align}

Knowing $\widetilde{q}(z)$ and taking ${\widetilde{A}(0) = \widetilde{A}_0}$, 
\begin{align}
	\frac{d\widetilde{A}}{dz} 
		= -\frac{\widetilde{A}}{2\widetilde{q}}
		= -\frac{\widetilde{A}}{2\left(\widetilde{q}_0 + z\right)} \implies \widetilde{A}(z)
		= \widetilde{A}_0 \sqrt{\frac{\widetilde{q}_0}{\widetilde{q}_0 + z}}
		= \widetilde{A}_0 \sqrt{\frac{\widetilde{q}_0}{\widetilde{q}(z)}}
\end{align}

\subsubsection{Complex beam parameter $\widetilde{q}(z)$}

The complex beam parameter $\widetilde{q}(z)$ can be related the real-valued radius of curvature $R(z)$ and beam waist $w(z)$ by the definition ${\frac{1}{\widetilde{q}(z)} \equiv \frac{1}{R(z)} - \iu \frac{\lambda}{\pi w^2(z)}}$.
At the beam waist $z=0$, we take $R(0)$ to be infinite and can relate ${\widetilde{q}(0) = \widetilde{q}_0}$ to ${w(0) = w_0}$, 
\begin{align}
	\frac{1}{\widetilde{q_0}} = -\iu \frac{\lambda}{\pi w_0^2} \implies \widetilde{q_0} = \iu \frac{\pi w_0^2}{\lambda} = \iu z_R
\end{align}
where ${z_R \equiv \frac{\pi w_0^2}{\lambda}}$ is the Rayleigh length.
From this, it follows 
\begin{align}
	\frac{1}{z + \iu z_R}	&= \frac{1}{R} - \iu \frac{\lambda}{\pi w^2}	\nonumber \\
	z + \iu z_R				&= \frac{\pi^2 R w^4}{\pi^2 w^4 + \lambda^2 R^2} + \iu\frac{\pi \lambda R^2 w^2}{\pi^2 w^4 + \lambda^2 R^2}
\end{align}
Equating the real and imaginary parts gives
\begin{align}
	R(z)	&= z + \frac{z_R^2}{z}	\\
	w^2(z)	&= w_0^2 \left[1+\left(\frac{z}{z_R}\right)^2\right]
\end{align}

\subsubsection{The (unnormalized) solution}

Putting all the parts together gives
\begin{align}
	\widetilde{u}_{x}(x,z) = \widetilde{A}_0 \frac{1}{\sqrt{1 - \iu \frac{z}{z_R}}} \exp\left[-\iu k \frac{x^2}{2}\left(\frac{1}{R(z)} - \iu\frac{\lambda}{\pi w^2(z)}\right)\right]
\end{align}
Normalization follows the usual procedure,
\begin{align}
	\int_{-\infty}^{\infty} \widetilde{u}_{x}^{*}\widetilde{u}_{x} \,dx
		&= \abs{\widetilde{A}_0}^2 \frac{1}{\sqrt{1 + \left(\frac{z}{z_R}\right)^2}} \int_{-\infty}^{\infty} \exp\left[-\frac{2 x^2}{w^2(z)}\right] \,dx \nonumber \\
		&= \abs{\widetilde{A}_0}^2 \frac{1}{\sqrt{1 + \left(\frac{z}{z_R}\right)^2}} \sqrt{\frac{\pi}{2}} w(z) \nonumber \\
		&= \abs{\widetilde{A}_0}^2 \sqrt{\frac{\pi}{2}} w_0
\end{align}
Therefore,
\begin{align}
	\abs{\widetilde{A}_0}^2 = \sqrt{\frac{2}{\pi}}\frac{1}{w_0}
\end{align}
As a result, 
\begin{align}
	\widetilde{u}_{x}(x,z)
		&= \pm \left(\frac{2}{\pi}\right)^{1/4} \frac{1}{\sqrt{w_0\left(1 - \iu \frac{z}{z_R}\right)}} \exp\left[-\iu k \frac{x^2}{2}\left(\frac{1}{R(z)} - \iu\frac{\lambda}{\pi w^2(z)}\right)\right]
\end{align}


**************************

Interesting discussions about astigmatic Gaussian beams are discussed in \cite{ark_1969, kws_2013}. 