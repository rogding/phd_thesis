\chapter{Optical Dipole Traps}

Put ODT details here.

\section{Optical Dipole Potential}

**** The derivation below loosely follows that found in \cite{Steck.QuantumAtomOptics, Ludlow2008.PhD} ****

For a two-level atom with ground (excited) state $\ket{g}$ ($\ket{e}$) with energy $E_{g} = \hbar \omega_{g}$ ($E_{e} = \hbar \omega_{e}$), the atomic Hamiltonian $H_{A}$ can be written as
\begin{equation}
	H_{A}	=	\hbar
				\begin{pmatrix}
					\omega_{e}	&	0			\\
					0			&	\omega_{g}
				\end{pmatrix}
\end{equation}
** Somehow introduce $\vb{d}\cdot\vb{E} \rightarrow \Omega \exp(-\iu \omega_{L} t)$ **
Introducing a laser at frequency $\omega$ which couples $\ket{g}$ and $\ket{e}$ and can be represented by
\begin{equation}
	H_{AF}	=	\frac{\hbar}{2}
				\begin{pmatrix}
					0							&	\Omega \cos(\omega t)	\\
					\Omega^{*} \cos(\omega t)	&	0
				\end{pmatrix}
\end{equation}
Using the unitary transformation
\begin{equation}
	U	=	\begin{pmatrix}
				\me^{\iu \omega t}	&	0	\\
				0					&	1
			\end{pmatrix}
\end{equation}
the total (transformed) Hamiltonian for the system can be written as
\begin{equation}
	\widetilde{H}	=		\hbar
							\begin{pmatrix}
								-\Delta												&	\frac{\Omega}{2} \qty(1 + \me^{\iu 2 \omega t})	\\
								\frac{\Omega^{*}}{2} \qty(1 + \me^{-\iu 2 \omega t})	&	\omega_{g}
							\end{pmatrix}
					\approx	\hbar
							\begin{pmatrix}
								-\Delta					&	\frac{\Omega}{2}	\\
								\frac{\Omega^{*}}{2}	&	\omega_{g}
							\end{pmatrix}
\end{equation}
where $\Delta \equiv \omega - \omega_{e}$ and the rotating wave approximation (RWA) was made ** clarify why/how the RWA is done **.

For $\Delta \gg \Omega$, ..... continue deriving...