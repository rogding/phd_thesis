\chapter{Optical Dipole Traps}

Put ODT details here.

\section{Optical Dipole Potential}

**** The derivation below loosely follows that found in \cite{Steck.QuantumAtomOptics, Ludlow2008.PhD} ****

For a two-level atom with ground (excited) state $\ket{g}$ ($\ket{e}$) and energy $E_{g} = \hbar \omega_{g}$ ($E_{e} = \hbar \omega_{e}$), the atomic Hamiltonian $H_{A}$ can be written as
\begin{equation}
	H_{A}	=	\hbar
				\begin{pmatrix}
					\omega_{e}	&	0			\\
					0			&	\omega_{g}
				\end{pmatrix}
\end{equation}
** Somehow introduce $\vb{d}\cdot\vb{E} \rightarrow \Omega \exp(-\iu \omega_{L} t)$ **
Introducing a laser at frequency $\omega$ which couples $\ket{g}$ and $\ket{e}$ and can be represented by
\begin{equation}
	H_{AF}	=	\frac{\hbar}{2}
				\begin{pmatrix}
					0							&	\Omega \cos(\omega t)	\\
					\Omega^{*} \cos(\omega t)	&	0
				\end{pmatrix}
\end{equation}
Using the unitary transformation
\begin{equation}
	U	=	\begin{pmatrix}
				\me^{\iu \omega t}	&	0	\\
				0					&	1
			\end{pmatrix}
\end{equation}
the total (transformed) Hamiltonian for the system can be written as
\begin{equation}
	\widetilde{H}	=		\hbar
							\begin{pmatrix}
								-\Delta												&	\frac{\Omega}{2} \qty(1 + \me^{\iu 2 \omega t})	\\
								\frac{\Omega^{*}}{2} \qty(1 + \me^{-\iu 2 \omega t})	&	\omega_{g}
							\end{pmatrix}
					\approx	\hbar
							\begin{pmatrix}
								-\Delta					&	\frac{\Omega}{2}	\\
								\frac{\Omega^{*}}{2}	&	\omega_{g}
							\end{pmatrix}
\end{equation}
where $\Delta \equiv \omega - \omega_{e}$ and the rotating wave approximation (RWA) was made ** clarify why/how the RWA is done **.

For $\Delta \gg \Omega$, ..... continue deriving...

********************

********************

********************

The potential of atoms in an optical trap can be written as
\begin{equation}
	U	= \hbar \Delta\omega
		= -\hbar \frac{E^{2}_{0}}{4} \frac{\abs{d}^{2}}{\hbar^{2}} \qty(\frac{1}{\omega_{0} - \omega_{L}} + \frac{1}{\omega_{0} + \omega_{L}})
		= -\frac{E^{2}_{0}}{4 \hbar} \qty(\frac{2 \abs{d}^{2}}{\hbar} \frac{\omega_{0}}{\omega^{2}_{0}-\omega^{2}_{L}})
\end{equation}
where terms generally excluded by the rotating wave approximation were kept.
The term in parenthesis is the dipole polarizability $\alpha$.

A simple way of generalizing to a multi-level atom is to treat each excited level separately and sum their shift contributions together\footnote{A ``real'' way to do this calculation would be to diagnolize the full interaction matrix for all the atomic levels coupled by the laser.}. 
Labeling the ground state $\ket{g}$ and the excited states coupled by the optical field with $\ket{e}$, the AC polarizability can be written as
\begin{equation}
	\alpha_{g} = \frac{2}{\hbar} \sum_{e} \abs{d}^{2}_{ge} \frac{\omega_{ge}}{\omega^{2}_{ge}-\omega^{2}_{L}}
\end{equation}

Since the energy shift depends on the square of the electric field, this is easily related to the intensity with $I = c \epsilon_{0} E^{2}_{0}/2$.
Optical trapping occurs when there is a spatially varying intensity $I\qty(r)$.

\subsection{A tale of three lasers}

Our first ODT laser was a Coherent Verdi IR - essentially a \SI{1064}{\nm} version of the more common \SI{532}{\nm} variant. 
That worked ok until it died due to an electrical surge (?). 
At which point we moved to using a Nufern NuAmp \SI{50}{\W} fiber amplifier laser.
