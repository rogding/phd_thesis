\chapter{Absorption imaging of strontium}

Our main diagnostic tool is absorption imaging where the atoms are illuminated with resonant light and an image of their shadow is recorded. 
From these images, the main properties we can extract are the atom number and the temperature. 
Therefore it's important to know how the electronic structure of strontium affects how many photons are scattered. 
All our absorption images are done using the strong $\nSLJ{5s^2}{1}{S}{0} \rightarrow \nSLJ{5s5p}{1}{P}{1}$ transition at \SI{461}{\nm}. 

The calculation of the scattering cross section was originally presented in \cite{mickelson2010.phd} using a rate equation model. 
\citeauthor{barker2016.phd} took it a step further and calculated the scattering cross section using the optical Bloch equations and found it to be effectively the same as using rate equations. 

\section{Imaging \Sr{84}, \Sr{86}, and \Sr{88}}

The bosonic isotopes have no nuclear spin ($I=0$) so their absorption cross section is relatively simple. 

The resonant absorption cross section for a single polarization is given by \cite{steck_qao_notes}
\begin{equation*}
	\sigma_0 = \frac{\lambda_0^2}{2 \pi} = \SI{3.38036E-14}{\square\m\per\s}
\end{equation*}
where $\lambda_0 = \SI{460.862}{\nm}$. ** Use cross section with matrix elements. **

*** ****



\section{Imaging \Sr{87}}

Due to the hyperfine structure, the scattering cross section for \Sr{87} is not as straightforward as for the bosons since as the atoms scatter light, they can be optically pumped between different $m_F$ states which have different coupling strengths to the excited $m_F^\prime$ levels. 

The calculation of the \Sr{87} scattering cross section was originally presented in \cite{mickelson2010.phd} using a rate equation model. 
\citeauthor{barker2016.phd} took it a step further and calculated the scattering cross section using the optical Bloch equations and found it to be effectively the same as using rate equations. 