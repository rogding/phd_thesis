\chapter{Tips and Tricks}
\label{ap:tips_n_tricks}

Various tips and tricks I picked up over the years. 

\section{Working with (high-power) fibers}

Notes about fibered ODT setup.

Having a fiber inspection scope is super helpful\footnote{We recently acquired a F1MS200U from Fiber Instrument Sales and FS201-PM from Thorlabs.}

\section{Fiber injection lock slave lasers}

We found that using a fiber to couple injection light to a slave laser greatly increased the flexibility and ease-of-use of the slave laser. Fibering the injection light greatly shortens the path lengths compared to free-space injection locking. The fiber also enables easier optimization of the injection lock by using the light back-coupled through the fiber (exiting the ``input'' end) to optimize alignment to the diode. 

\section{Scanning the UV laser}
\label{ap:scanning_uv_laser}

**Talk about how we used to flip-flop sidebands to patch together the $n=98-99$ scan but then the new synthesizer allows us to continue scanning once we hop over the transfer cavity crossing.**

Our current transfer cavity has an FSR of about **\SI{300}{\MHz}** which means the sidebands cross every **\SI{150}{\MHz}**.

\subsection{Switching sidebands to increase scan range}

**Talk about how we used to flip-flop sidebands to patch together the $n=98-99$ scan but then the new synthesizer allows us to continue scanning once we hop over the transfer cavity crossing.**

\subsection{``Continuous'' scanning of the transfer cavity}

**Using the new synthesizer, blah, blah, blah, we no longer have to switch EOM sidebands every FSR/2. 
If we want to scan a few GHz (limited by the synthesizer and RF amplifier ranges) we just need to hop over sideband crossings.**