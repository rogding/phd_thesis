\chapter{Tips and Tricks}
\label{ap:tips_n_tricks}

This section contains a collection of random tips and tricks I picked up during the my time working on the experiment.
The idea for this section came from coming across one in \cite{Senaratne2018.PhD}.
Hopefully something in here will be helpful for future students if they run in to similar problems. 

\section{Working with (high-power) fibers}

Notes about fibered ODT setup.

Having a fiber inspection scope is super helpful\footnote{We recently acquired a F1MS200U from Fiber Instrument Sales and FS201-PM from Thorlabs.}

\section{Fiber injection lock slave lasers}

We found that using a fiber to couple injection light to a slave laser greatly increased the flexibility and ease-of-use of the slave laser. Fibering the injection light greatly shortens the path lengths compared to free-space injection locking. The fiber also enables easier optimization of the injection lock by using the light back-coupled through the fiber (exiting the ``input'' end) to optimize alignment to the diode. 

\section{Scanning the UV laser}
\label{ap:scanning_uv_laser}

**Talk about how we used to flip-flop sidebands to patch together the $n=98-99$ scan but then the new synthesizer allows us to continue scanning once we hop over the transfer cavity crossing.**

Our current transfer cavity has an FSR of about **\SI{300}{\MHz}** which means the sidebands cross every **\SI{150}{\MHz}**.

\subsection{Switching sidebands to increase scan range}

**Talk about how we used to flip-flop sidebands to patch together the $n=98-99$ scan but then the new synthesizer allows us to continue scanning once we hop over the transfer cavity crossing.**

\subsection{``Continuous'' scanning of the transfer cavity}

**Using the new synthesizer, blah, blah, blah, we no longer have to switch EOM sidebands every FSR/2. 
If we want to scan a few GHz (limited by the synthesizer and RF amplifier ranges) we just need to hop over sideband crossings.**

\section{Winding coils on aluminum U-channels}

I learned from my undergraduate days that winding magnetic field coils on aluminum U-channels without doing something about the corners very likely leads to short circuits since the wires are typically thinly coated with an insulating layer so that they can be wound as tightly as possible.
When designing the trim coils for this experiment, I purposefully left gaps between the right-angle U-channels so that I could later round them off. 
I first filled in the gaps with J-B Weld HighHeat epoxy putty and rounded them off before adding a few layers of polyimide (i.e., Kapton) tape as extra insurance. 

\begin{figure}[htbp]
	\centering
	\includesvg[keepaspectratio, width=\textwidth]{tips_n_tricks/winding_coils/winding_coils.svg}
	\caption{\label{fig:winding_coils}Winding coils on aluminum U-channel forms. (Left) The bare aluminum U-channel forms with the rounded corners. (Right) After adding extra electrical insulation with polyimide tape on all sides of U-channel which come in contact with the wire.}
\end{figure}

\section{Hard Drive Shutters and Drivers}

We've tried a variety of shutters but found it difficult to beat the durability and cost-effectiveness of using old hard drives as shutters. 
The shutters are current-driven devices and we've found success following the circuit in \cite{Scholten2007.RSI.78.026101}.
We modified that circuit to be able to use either the Texas Instruments LMD18200 driver or the IXYS IXDF604PI which provides comparable capabilities to the LMD18200 but at a fraction of the cost.

\section{Don't wait until the end to start writing your thesis}

This paper accurately describes what it's like to write a thesis \cite{Upper1974.JABA.7.497}. 