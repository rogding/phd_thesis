\chapter{Derivations}

Random derivations are presented below: partly for completeness and partly because I don't want to have to re-derive them in the future.

\section{Gaussian Beams}
\label{sec:gaussian_beams}

Since we work (almost) exclusively with Gaussian laser beams in the laboratory, I found it instructive to derive the intensity for an astigmatic, elliptical Gaussian beam.
The derivation presented below mostly follows from ones presented in \cite{Siegman1986, Kogelnik1966.AO.5.1550, Arnaud1969.AO.8.1687}.

\subsubsection{Paraxial Wave Equation}

For an arbitrary electric field $\vb{E} = \widetilde{E}\qty(x,y,z,t) \vu{e}$, where $\vu{e}$ is the unit polarization vector and $\widetilde{E}$ is the complex amplitude, the wave equation gives
\begin{equation}
	\label{eq:wave_eq}
	\frac{1}{c^2} \partial_{tt}\widetilde{E}
		=	\laplacian \widetilde{E}
		=	\partial_{xx}\widetilde{E} + \partial_{yy}\widetilde{E} + \partial_{zz}\widetilde{E}
\end{equation}
As usual, $c=\lambda f=\omega/k$. 
Choosing the $z$-axis as the direction of propagation, the time dependence can be removed by assuming $\widetilde{E}\qty(x,y,z,t) = \widetilde{u}\qty(x,y,z)\exp[-\iu \qty(k z - \omega t)]$
\begin{equation}
	\label{eq:paraxial_wave_eq}
	0	=		\partial_{xx}\widetilde{u} + \partial_{yy}\widetilde{u} + \partial_{zz}\widetilde{u} -\iu 2 k \partial_{z}\widetilde{u}
		\approx	\partial_{xx}\widetilde{u} + \partial_{yy}\widetilde{u} -\iu 2 k \partial_{z}\widetilde{u}
\end{equation}
Here, the paraxial approximation $\abs{\partial_{zz}\widetilde{u}} \ll \abs{2 k \partial_{z}\widetilde{u}}, \abs{\partial_{xx}\widetilde{u}}, \abs{\partial_{yy}\widetilde{u}}$ was made which assumes the envelope of the electric field $\widetilde{u}\qty(x,y,z)$ varies ``slowly'' along $z$\footnote{\cref{eq:paraxial_wave_eq} is generally referred to as the ``paraxial wave equation''.}.

\subsubsection{Working in Rectangular Coordinates}

Choosing the rectangular coordinate axes such that the beam is symmetric about the $x$-axis and $y$-axis, the paraxial wave equation can be further separated by writing $\widetilde{u}\qty(x,y,z) = \widetilde{u}_{x}\qty(x,z) \widetilde{u}_{y}\qty(y,z)$ where $\widetilde{u}_{i}\qty(x_{i},z)$ represent the electric field envelopes along the $x_{i}$-axis. 
Inserting this in to \cref{eq:paraxial_wave_eq} leads to
\begin{equation}
	\label{eq:transverse_paraxial_wave_eq}
	0	=	\widetilde{u}_{x}\qty(\partial_{yy}\widetilde{u}_{y} - \iu 2 k \partial_{z}\widetilde{u}_{y})
			+ \widetilde{u}_{y}\qty(\partial_{xx}\widetilde{u}_{x} - \iu 2 k \partial_{z}\widetilde{u}_{x})
\end{equation}
Requiring $\widetilde{u}_{x}\qty(x,z), \widetilde{u}_{y}\qty(y,z) \neq 0$ to avoid the trivial solution means the terms in parentheses must vanish.
Since the terms in the parentheses both have the same form, we only need to find the solution for one axis coordinate with the other solution obtained by analogy.

\subsubsection{Solving $0 = \partial_{xx}\widetilde{u}_{x} - \iu 2 k \partial_{z}\widetilde{u}_{x}$}

Inserting the trial solution $\widetilde{u}_{x}\qty(x,z) = \widetilde{A}\qty(z) \exp[-\iu k {x^2}/{2\widetilde{q}\qty(z)}]$, where $\widetilde{A}\qty(z)$ and $\widetilde{q}\qty(z)$ can be complex, in to \cref{eq:transverse_paraxial_wave_eq} and collecting terms in powers of $x$ leads to
\begin{equation}
	0	=	-\iu \widetilde{q} \qty(\widetilde{A} + 2 q \dv{\widetilde{A}}{z})
			+ k \widetilde{A} \qty(\dv{\widetilde{q}}{z} - 1) x^2
\end{equation}
Requiring $\widetilde{A}\qty(z), \widetilde{q}\qty(z) \neq 0$ means the terms in the parentheses must be zero.
Using the conditions $\widetilde{q}\qty(z_{0}) = \widetilde{q}_{0}$ and $\widetilde{A}\qty(z_{0}) = \widetilde{A}_{0}$, 
\begin{equation}
	\label{eq:q_prime}
	0	=			\dv{\widetilde{q}}{z} - 1
		\implies	\widetilde{q}\qty(z) = z - z_{0} + \widetilde{q}_{0}
\end{equation}
\begin{equation}
	\label{eq:A_prime}
	0	=			\widetilde{A} + 2 \widetilde{q} \dv{\widetilde{A}}{z}
		\implies	\widetilde{A}\qty(z) = \widetilde{A}_{0} \sqrt{\frac{\widetilde{q}_{0}}{\widetilde{q}\qty(z)}}
\end{equation}

$\widetilde{q}\qty(z)$ is the complex beam parameter and can be related the (real-valued) radius of curvature $R\qty(z)$ and spot size $w\qty(z)$ by the definition $1/\widetilde{q}\qty(z) \equiv 1/R\qty(z) - \iu \lambda/\pi w^{2}\qty(z)$.
At $z=z_{0}$, the beam is at a waist $w\qty(z_{0}) = w_0$ with $R\qty(z_{0}) \rightarrow \infty$, allowing us to relate $\widetilde{q}_{0}$ to $w_{0}$, 
\begin{equation}
	\frac{1}{\widetilde{q}\qty(z_{0})} = -\iu \frac{\lambda}{\pi w^{2}\qty(z_{0})} \implies \widetilde{q_0} = \iu \frac{\pi w_0^2}{\lambda} = \iu z_R
\end{equation}
where $z_R \equiv \pi w_{0}^{2}/\lambda$ is the Rayleigh length.
From this, it follows 
\begin{equation}
	\frac{1}{\widetilde{q}\qty(z)}	=	\frac{1}{z - z_{0} + \iu z_{R}}	= \frac{1}{R\qty(z)} -\iu \frac{\lambda}{\pi w^{2}\qty(z)}
\end{equation}
Equating the real and imaginary parts gives
\begin{align}
	R\qty(z)		&{}={}	\qty(z-z_{0}) \qty[1 + \frac{z_R^2}{\qty(z-z_{0})^2}]	\\
	w^{2}\qty(z)	&{}={}	w_{0}^{2} \qty[1+\frac{\qty(z-z_0)^2}{z_R^2}]
\end{align}

Putting all the parts together gives the (unnormalized) solution
\begin{align}
	\widetilde{u}_{x}\qty(x,z)
		=	\widetilde{A}_0 \frac{1}{\sqrt{1 - \iu \frac{z-z_0}{z_R}}} \exp[-\iu k \frac{x^2}{2}\qty(\frac{1}{R\qty(z)} - \iu\frac{\lambda}{\pi w^{2}\qty(z)})]
\end{align}
To normalize, we follow the usual procedure for finding $\abs{\widetilde{A}_0}^{2}$
\begin{equation}
	1	=	\int_{-\infty}^{\infty} \widetilde{u}_{x}^{*}\qty(x,z) \widetilde{u}_{x}\qty(x,z) \dd{x}
		=	\abs{\widetilde{A}_0}^2 \frac{w_0}{w\qty(z)} \int_{-\infty}^{\infty} \exp[-\frac{2 x^2}{w^2\qty(z)}] \dd{x}
		=	\abs{\widetilde{A}_0}^2 \sqrt{\frac{\pi}{2}} w_0
\end{equation}
With $\abs{\widetilde{A}_0}^2 = \sqrt{2/\pi}/w_0$, the solution for $\abs{\widetilde{u}_{x}\qty(x,z)}$ is
\begin{equation}
	\abs{\widetilde{u}_{x}\qty(x,z)}^2	=	\sqrt{\frac{2}{\pi}} \frac{1}{w\qty(z)} \exp[-\frac{2 x^2}{w^2\qty(z)}]
\end{equation}

\subsubsection{Complete Solution}

*** Need a smoother way to work in power $P$ ***

Now that we know the normalized solutions $\abs{\widetilde{u}_{i}\qty(i,z)}^2$, we can write the intensity $I\qty(x,y,z) = \abs{\vb{E}\qty(x,y,z)}^2 = \abs{\widetilde{u}_{x}\qty(x,z)}^2 \abs{\widetilde{u}_{y}\qty(y,z)}^2$ for a beam with total power $P$ as
\begin{equation}
	I\qty(x,y,z)	=	\frac{2 P}{\pi} \frac{1}{w_{x}\qty(z) w_{y}\qty(z)} \exp[-\frac{2 x^2}{w^{2}_{x}\qty(z)} - \frac{2 y^2}{w^{2}_{y}\qty(z)}]
\end{equation}
where 
\begin{align}
	w_{i}^{2}\qty(z)	&{}={}	w_{0,i}^{2} \qty[1 + \frac{\qty(z-z_{0,i})^2}{z_{R,i}^{2}}]	\\
	z_{R,i}				&{}={}	\frac{\pi w_{0,i}^{2}}{\lambda}
\end{align}
for the $i$-axis beam waist radius $w_{0,i}^{2}$ and location $z_{0,i}$.

\subsection{Circular Gaussian Beam}

For a circular ($w_{0,x} = w_{0,y} = w_{0}$), non-astigmatic ($z_{0,x} = z_{0,y} = z_{0}$) Gaussian beam, $w_{x}\qty(z) = w_{y}\qty(z) = w\qty(z)$ so the intensity can be written as
\begin{equation}
	I\qty(x,y,z)
		=	\frac{2 P}{\pi} \frac{1}{w^{2}\qty(z)} \exp[-\frac{2 \qty(x^2 + y^2)}{w^{2}\qty(z)}]
		=	I_{0} \qty(\frac{w_{0}}{w\qty(z)})^{2} \exp[-\frac{2 \qty(x^2 + y^2)}{w^{2}\qty(z)}]
\end{equation}
for peak intensity $I_{0} = 2 P/\pi w_{0}^{2}$.
Changing to cylindrical coordinates with $r^2 = x^2 + y^2$ gives the more familiar intensity distribution
\begin{equation}
	I\qty(r,z)	=	I_{0} \qty(\frac{w_{0}}{w\qty(z)})^{2} \exp[-\frac{2 r^2}{w^{2}\qty(z)}]
\end{equation}

\section{Unitary transformation}

Following the derivation in \cite{Steck.QuantumAtomOptics}, we derive how to obtain the transformed Hamiltonian for states related by the unitary transformation $U$ where $\ket*{\widetilde{\psi}} = U \ket*{\psi}$.
Starting from the Schr\"{o}dinger equation, 
\begin{equation}
	\iu \hbar \ket*{\dot{\psi}}
		=			H \ket*{\psi}
		\implies	\iu \hbar \pdv{t}\qty(U^\dag \ket*{\widetilde{\psi}})	=	H \qty(U^\dag \ket*{\widetilde{\psi}})
\end{equation}
Working through the algebra, we obtain the transformed Hamiltonian $\widetilde{H}$
\begin{align}
	\iu \hbar \qty(\dot{U}^{\dag} \ket*{\widetilde{\psi}} + U^{\dag} \ket*{\dot{\widetilde{\psi}}})
			&{}={}	H U^{\dag} \ket*{\widetilde{\psi}}	\notag	\\
	\iu \hbar U^{\dag} \ket*{\dot{\widetilde{\psi}}}
			&{}={}	H U^{\dag} \ket*{\widetilde{\psi}} - \iu \hbar \dot{U}^{\dag} \ket*{\widetilde{\psi}}	\notag	\\
	\iu \hbar \ket*{\dot{\widetilde{\psi}}}
			&{}={}	U H U^{\dag} \ket*{\widetilde{\psi}} - \iu \hbar U \dot{U}^{\dag} \ket*{\widetilde{\psi}}	\notag	\\
	\iu \hbar \ket*{\dot{\widetilde{\psi}}}
			&{}={}	\widetilde{H} \ket*{\widetilde{\psi}}
\end{align}
where $\widetilde{H} = U H U^{\dag} - \iu \hbar U \dot{U}^{\dag} = U H U^{\dag} + \iu \hbar \dot{U} U^{\dag}$.
The last part was obtained using $\partial_{t} \qty(U^{\dag} U) = \partial_{t} \qty(1) = 0$.

\section{Hyperfine Mixing Matrix Elements}
\label{sec:hyperfine_mixing_derivation}

This section contains an extended derivation of the hyperfine mixing presented in xxx. 
The derivation presented below was originally done by Shuhei Yoshida.

From xxxxx, we can approximate the hyperfine interaction of singly-excited Rydberg states (where the inner electron is in the $5s$ state) as
\begin{equation}
	\hat{V}_\text{hf}	\simeq	a_{ms} \hat{\bm{s}}_{\text{in}} \cdot \hat{\bm{I}}
\end{equation}
Expanding $\hat{\bm{s}}_\text{in}$ and $\hat{\bm{I}}$ in terms of ladder operators $\hat{J}_{\pm}=\hat{J}_{x} \pm \hat{J}_{y}$ and $\hat{J}_{\pm}\ket{J,m_J}=\sqrt{J\left(J+1\right)-m_J\left(m_J\pm1\right)} \ket{J,m_J\pm1}$ gives
\begin{equation}
	\hat{V}_\text{hf} \simeq \frac{a_{5s}}{2} \left(\hat{s}_{\text{in},+}\hat{I}_{-} + \hat{s}_{\text{in},-}\hat{I}_{+} + 2\hat{s}_{\text{in},z}\hat{I}_{z}\right)
\end{equation}

\subsubsection{Singly Excited Singlet and Triplet States}

Before going to the derivations, it will be helpful to write out the singlet and triplet spin wave functions for two electron systems.
In the following, I use the notation $\ket{\psi_1;\psi_2}=\ket{\psi_1}\ket{\psi_2}$ for electron 1 and electron 2.
Denoting $\ket{+}$ for a spin-up electron and $\ket{-}$ for a spin-down electron, the singlet and triplet wave functions $\ket{S,m_S}_{S}$ are (see e.g., \cite{Sakurai2010})
\begin{align}
	\ket{0,0}_{S}	&{}={}	\frac{1}{\sqrt{2}} \left(\ket{+;-} - \ket{-;+}\right)	\\
	\ket{1,1}_{S}	&{}={}	\ket{+;+}											\\
	\ket{1,0}_{S}	&{}={}	\frac{1}{\sqrt{2}} \left(\ket{+;-} + \ket{-;+}\right)	\\
	\ket{1,-1}_{S}	&{}={}	\ket{-;-}
\end{align}

For a singly excited Rydberg states $\ket{\nSLJM{msnl}{1}{L}{J}{m_J}}_{J}$ with the ``inner'' electron in the $\ket{ms}$ state and the Rydberg electron in the $\ket{nl}$ state, the symmetric ($+$) and antisymmetric ($-$) orbital wave functions can be written as $\ket{L,m_L}_{L\pm}=\frac{1}{\sqrt{2}}\left(\ket{ms;nl,m_L}\pm\ket{nl;ms,m_L}\right)$.
Therefore the combined orbital and spin wave functions for singlet and triplet states are
\begin{align}
	\ket{\nSLJM{msnl}{1}{L}{J}{m_J}}_{J}	&{}={}	\frac{1}{\sqrt{2}} \left(\ket{ms;nl,m_L}+\ket{nl;ms,m_L}\right) \ket{0,0}_{S}	\notag	\\
											&{}={}	\ket{L,m_L=m_J}_{L+} \ket{0,0}_{S}	\\
	\ket{\nSLJM{msnl}{3}{L}{J}{m_J}}_{J}	&{}={}	\sum_{m_L} \sum_{m_S} \CG{L,m_L}{1,m_S}{J,m_J} \frac{1}{\sqrt{2}}\left(\ket{ms;nl,m_L}-\ket{nl;ms,m_L}\right) \ket{1,m_S}_{S}	\notag	\\
											&{}={}	\sum_{m_S} \CG{L,m_L}{1,m_S}{J,m_J} \ket{L,m_L=m_J-m_S}_{L-} \ket{1,m_S}_{S}
\end{align}
with the Clebsch–Gordan coefficients defined as $\CG{J_1,M_1}{J_2,M_2}{J,M}=\braket{J_1,M_1;J_2,M_2}{J,M}$.

Working in the total angular momentum basis $\bm{F}=\bm{J}+\bm{I}$, we can combine the $\ket{J,m_J}_{J}$ states with $\ket{I,m_I}_{I}$ to get $\ket{F,m_F}_{F}$ basis states
\begin{align}
	\ket{\nSLJFM{msnl}{2S+1}{L}{J}{F}{m_F}}_{F}
		&{}={}	\sum_{m_J} \sum_{m_I} \CG{J,m_J}{I,m_I}{F,m_F} \ket{\nSLJM{msnl}{2S+1}{L}{J}{m_J}}_{J} \ket{I,m_I}_{I}	\notag	\\
		&{}={}	\sum_{m_J} \CG{J,m_J}{I,m_I}{F,m_F} \ket{\nSLJM{msnl}{2S+1}{L}{J}{m_J}}_{J} \ket{I,m_I=m_F-m_J}_{I}
\end{align}

\subsection{Singly excited $S$ states}

For singly excited $\ket{\nSLJFM{n}{1}{S}{0}{F}{m_F}}_{F}$ and $\ket{\nSLJFM{n}{3}{S}{1}{F}{m_F}}_{F}$ states, the four basis states are
\begin{align}
	\ket{\nSLJFM{n}{1}{S}{0}{F=I}{m_F}}_{F}
		&{}={}	\ket{\nSLJm{n}{1}{S}{0}{0}}_{J} \ket{I,m_I=m_F-m_J}_{I}	\\
	\ket{\nSLJm{msns}{3}{S}{1}{1}}	&{}={}	\ket{1,0}_{L-} \ket{1,1}_{S}	\\
	\ket{\nSLJm{msns}{3}{S}{1}{0}}	&{}={}	\ket{1,0}_{L-} \ket{1,0}_{S}	\\
	\ket{\nSLJm{msns}{3}{S}{1}{-1}}	&{}={}	\ket{1,0}_{L-} \ket{1,-1}_{S}
\end{align}




*******************

*******************

*******************

Working in the total angular momentum basis $\bm{F}=\bm{J}+\bm{I}=\bm{S}+\bm{I}$\footnote{Since $L=0$ for $S$~states.}, we can use Clebsch–Gordan coefficients\footnote{I use the following notation for Clebsch–Gordan coefficients: $\CG{J_1,M_1}{J_2,M_2}{J,M}=\braket{J_1,M_1;J_2,M_2}{J,M}$.} to expand $\ket{F,m_F}$ into the $\ket{J,m_J}\ket{I,m_I}$ basis
\begin{align}
	\ket{\SLJfM{1}{S}{0}{I}{m_F}}	&{}={}	\ket{\SLJM{1}{S}{0}{m_J=0}} \ket{I,m_I=m_F}													\\
	\ket{\SLJfM{3}{S}{1}{I+1}{m_F}}	&{}={}	\sum_{m_J} \sum_{m_I} \CG{J=1,m_J}{I,m_I}{F=I+1,m_F} \ket{\SLJM{3}{S}{1}{m_J}} \ket{I,m_I}	\notag \\
									&{}={}	\sum_{m_J} \CG{J=1,m_J}{I,m_I}{F=I+1,m_F} \ket{\SLJM{3}{S}{1}{m_J}} \ket{I,m_I=m_F-m_J}		\\
	\ket{\SLJfM{3}{S}{1}{I}{m_F}}	&{}={}	\sum_{m_J} \sum_{m_I} \CG{J=1,m_J}{I,m_I}{F=I,m_F} \ket{\SLJM{3}{S}{1}{m_J}} \ket{I,m_I}	\notag \\
									&{}={}	\sum_{m_J} \CG{J=1,m_J}{I,m_I}{F=I,m_F} \ket{\SLJM{3}{S}{1}{m_J}} \ket{I,m_I=m_F-m_J}		\\
	\ket{\SLJfM{3}{S}{1}{I-1}{m_F}}	&{}={}	\sum_{m_J} \sum_{m_I} \CG{J=1,m_J}{I,m_I}{F=I-1,m_F} \ket{\SLJM{3}{S}{1}{m_J}} \ket{I,m_I}	\notag \\
									&{}={}	\sum_{m_J} \CG{J=1,m_J}{I,m_I}{F=I-1,m_F} \ket{\SLJM{3}{S}{1}{m_J}} \ket{I,m_I=m_F-m_J}
\end{align}
** Clarify the following sentence: ** Since $F$ is a good quantum number of $\hat{V}_\text{hf}$, it only mixes states of the same $F$ and $m_F$.


We can expand $\ket{J,m_J}$ in to $\ket{L,m_L}\ket{S,m_S}=\ket{0,0}\ket{S,m_S}$



*******************************

Calculating the matrix elements of the hyperfine operator.
For $S$ states, the pure singlet and triplet states can be written as
\begin{align}
	& \ket{n\SLJm{1}{S}{0}{0}}^{0} = \frac{1}{2} \left(\ket{5s;ns} + \ket{ns;5s}\right) \left(\ket{\uparrow;\downarrow} - \ket{\downarrow;\uparrow}\right) \\
	& \ket{n\SLJm{3}{S}{1}{1}}^{0} = \frac{1}{\sqrt{2}} \left(\ket{5s;ns} - \ket{ns;5s}\right) \ket{\uparrow;\uparrow}	\\
	& \ket{n\SLJm{3}{S}{1}{0}}^{0} = \frac{1}{2} \left(\ket{5s;ns} - \ket{ns;5s}\right) \left(\ket{\uparrow;\downarrow} + \ket{\downarrow;\uparrow}\right)	\\
	& \ket{n\SLJm{3}{S}{1}{-1}}^{0} = \frac{1}{\sqrt{2}} \left(\ket{5s;ns} - \ket{ns;5s}\right) \ket{\downarrow;\downarrow}
\end{align}
To obtain include the nuclear spin, the above states are multiplied by $\ket{I,m_I}$.

** CHECK BELOW **
\begin{align}
	& \hat{s}_{\text{in},+}\ket{n\SLJm{1}{S}{0}{0}}^0	=	-\frac{1}{\sqrt{2}} \ket{n\SLJm{3}{S}{1}{1}}^{0}												\\
	& \hat{s}_{\text{in},-}\ket{n\SLJm{1}{S}{0}{0}}^0	=	\frac{1}{\sqrt{2}} \ket{n\SLJm{3}{S}{1}{-1}}^{0}												\\
	& \hat{s}_{\text{in},z}\ket{n\SLJm{1}{S}{0}{0}}^0	=	\frac{1}{2} \ket{n\SLJm{3}{S}{1}{0}}^{0}														\\
	& \hat{s}_{\text{in},+}\ket{n\SLJm{3}{S}{1}{1}}^0	=	0																								\\
	& \hat{s}_{\text{in},-}\ket{n\SLJm{3}{S}{1}{1}}^0	=	-\frac{1}{\sqrt{2}} \left(\ket{n\SLJm{1}{S}{0}{0}}^{0} - \ket{n\SLJm{3}{S}{1}{0}}^{0}\right)	\\
	& \hat{s}_{\text{in},z}\ket{n\SLJm{3}{S}{1}{1}}^0	=	\frac{1}{2} \ket{n\SLJm{3}{S}{1}{1}}^{0}														\\
	& \hat{s}_{\text{in},+}\ket{n\SLJm{3}{S}{1}{0}}^0	=	\frac{1}{\sqrt{2}} \ket{n\SLJm{3}{S}{1}{1}}^{0}													\\
	& \hat{s}_{\text{in},-}\ket{n\SLJm{3}{S}{1}{0}}^0	=	\frac{1}{\sqrt{2}} \ket{n\SLJm{3}{S}{1}{-1}}^{0}												\\
	& \hat{s}_{\text{in},z}\ket{n\SLJm{3}{S}{1}{0}}^0	=	\frac{1}{2} \ket{n\SLJm{1}{S}{0}{0}}^{0}														\\
	& \hat{s}_{\text{in},+}\ket{n\SLJm{3}{S}{1}{-1}}^0	=	\frac{1}{\sqrt{2}} \left(\ket{n\SLJm{1}{S}{0}{0}}^{0} + \ket{n\SLJm{3}{S}{1}{0}}^{0}\right)		\\
	& \hat{s}_{\text{in},-}\ket{n\SLJm{3}{S}{1}{-1}}^0	=	0																								\\
	& \hat{s}_{\text{in},z}\ket{n\SLJm{3}{S}{1}{-1}}^0	=	-\frac{1}{2} \ket{n\SLJm{3}{S}{1}{-1}}^{0}
\end{align}

For the $\hat{I}$ operators, 
\begin{align}
	& \hat{I}_{+} \ket{I,m_I}	=	\sqrt{I\left(I+1\right) - m_I \left(m_I+1\right)} \ket{I,m_I+1}	\\
	& \hat{I}_{-} \ket{I,m_I}	=	\sqrt{I\left(I+1\right) - m_I \left(m_I-1\right)} \ket{I,m_I-1}	\\
	& \hat{I}_{z} \ket{I,m_I}	=	m_I \ket{I,m_I}
\end{align}

Now that we know all the relevant states, we can calculate the effect of $\hat{V}_\text{hf}$ on each state. 
** For simplicity, we used the stretched states $\ket{I=F, m_I=F}$ (provide a better argument?) **
\begin{align}
	& \hat{V}_\text{hf}\ket{n\SLJm{1}{S}{0}{0}}^{0}		=	\frac{a_{5s}}{2}	\\
	& \hat{V}_\text{hf}\ket{n\SLJm{3}{S}{1}{1}}^{0}		=	\frac{a_{5s}}{2}	\\
	& \hat{V}_\text{hf}\ket{n\SLJm{3}{S}{1}{0}}^{0}		=	\frac{a_{5s}}{2}	\\
	& \hat{V}_\text{hf}\ket{n\SLJm{3}{S}{1}{-1}}^{0}	=	\frac{a_{5s}}{2} 
\end{align}

*******************

Therefore the total Hamiltonian can be written as
\begin{align}
	\hat{H}	{}={}&	\begin{pmatrix}
						E\qty(\SLJf{3}{S}{1}{I+1})	&	0							&	0							&	0	\\ 
						0							&	E\qty(\SLJf{1}{S}{0}{I})	&	0							&	0	\\ 
						0							&	0							&	E\qty(\SLJf{3}{S}{1}{I})	&	0	\\ 
						0							&	0							&	0							&	E\qty(\SLJf{3}{S}{1}{I-1})
					\end{pmatrix}	\\
			&		+ \frac{1}{2} a_{5s}
					\begin{pmatrix}
						0	&	0							&	0							&	0	\\ 
						0	&	0							&	\sqrt{I\left(I+1\right)}	&	0	\\ 
						0	&	\sqrt{I\left(I+1\right)}	&	-1							&	0	\\ 
						0	&	0							&	0							&	0
					\end{pmatrix}
\end{align}

\subsection{$D$-states}

Move details of derivation here? 

The derivations below are done for the stretched states $m_F = F$ but, due to the rotational symmetry of $\hat{V}_\text{hf}$, there should be no dependence on $m_F$.
Working through how the $\hat{s}_{\text{in}}$ operator acts on the $\ket{J, m_J}_{J}$ states gives
** WORK THROUGH FOR D-STATES **
\begin{align}
	& \hat{s}_{\text{in},+}\ket{\SLJm{1}{D}{2}{0}}^0	=	-\frac{1}{\sqrt{2}} \ket{n\SLJm{3}{S}{1}{1}}^{0}												\\
	& \hat{s}_{\text{in},-}\ket{\SLJm{1}{D}{2}{0}}^0	=	\frac{1}{\sqrt{2}} \ket{n\SLJm{3}{S}{1}{-1}}^{0}												\\
	& \hat{s}_{\text{in},z}\ket{\SLJm{1}{D}{2}{0}}^0	=	\frac{1}{2} \ket{n\SLJm{3}{S}{1}{0}}^{0}
\end{align}
\begin{align}
	& \hat{s}_{\text{in},+}\ket{\SLJm{3}{D}{1}{1}}^0	=	0																								\\
	& \hat{s}_{\text{in},-}\ket{\SLJm{3}{D}{1}{1}}^0	=	-\frac{1}{\sqrt{2}} \left(\ket{n\SLJm{1}{S}{0}{0}}^{0} - \ket{n\SLJm{3}{S}{1}{0}}^{0}\right)	\\
	& \hat{s}_{\text{in},z}\ket{\SLJm{3}{D}{1}{1}}^0	=	\frac{1}{2} \ket{n\SLJm{3}{S}{1}{1}}^{0}
\end{align}
\begin{align}
	& \hat{s}_{\text{in},+}\ket{\SLJm{3}{D}{2}{0}}^0	=	\frac{1}{\sqrt{2}} \ket{n\SLJm{3}{S}{1}{1}}^{0}													\\
	& \hat{s}_{\text{in},-}\ket{\SLJm{3}{D}{2}{0}}^0	=	\frac{1}{\sqrt{2}} \ket{n\SLJm{3}{S}{1}{-1}}^{0}												\\
	& \hat{s}_{\text{in},z}\ket{\SLJm{3}{D}{2}{0}}^0	=	\frac{1}{2} \ket{n\SLJm{1}{S}{0}{0}}^{0}
\end{align}
\begin{align}
	& \hat{s}_{\text{in},+}\ket{\SLJm{3}{D}{3}{-1}}^0	=	\frac{1}{\sqrt{2}} \left(\ket{n\SLJm{1}{S}{0}{0}}^{0} + \ket{n\SLJm{3}{S}{1}{0}}^{0}\right)		\\
	& \hat{s}_{\text{in},-}\ket{\SLJm{3}{D}{3}{-1}}^0	=	0																								\\
	& \hat{s}_{\text{in},z}\ket{\SLJm{3}{D}{3}{-1}}^0	=	-\frac{1}{2} \ket{n\SLJm{3}{S}{1}{-1}}^{0}
\end{align}
\begin{align}
	& \hat{s}_{\text{in},+}\ket{\SLJm{3}{D}{3}{-1}}^0	=	\frac{1}{\sqrt{2}} \left(\ket{n\SLJm{1}{S}{0}{0}}^{0} + \ket{n\SLJm{3}{S}{1}{0}}^{0}\right)		\\
	& \hat{s}_{\text{in},-}\ket{\SLJm{3}{D}{3}{-1}}^0	=	0																								\\
	& \hat{s}_{\text{in},z}\ket{\SLJm{3}{D}{3}{-1}}^0	=	-\frac{1}{2} \ket{n\SLJm{3}{S}{1}{-1}}^{0}
\end{align}
** WORK THROUGH FOR D-STATES **

\subsubsection{$F=I+3$}

For the stretched state $\ket{F=I+3, m_F = I+3}$, there is only one state with $m_L = 2$ and $m_S = 1$ which satisfies $m_J = 3$ since $m_I = I$
\begin{equation}
	\ket{F=I+3, m_F = I+3}	=	\ket{\SLJm{3}{D}{3}{3}}_{J} \ket{I, I}_{I}
\end{equation}
Therefore
\begin{equation}
	\hat{V}_\text{hf} \ket{F=I+3, m_F = I+3}	=	\frac{1}{2} a_{5s} I \ket{F=I+3, m_F = I+3}
\end{equation}

\subsubsection{$F=I+2$}

For $F=I+2$, there are three states which can produce the state $\ket{F = I+2, m_F = I+2}$
\begin{align}
	\ket{\SLJf{1}{D}{2}{m_F=I+2}}	&{}={}	\ket{\SLJm{1}{D}{2}{2}}_{J} \ket{I, I}_{I}	\\
	\ket{\SLJf{3}{D}{2}{m_F=I+2}}	&{}={}	\ket{\SLJm{3}{D}{2}{2}}_{J} \ket{I, I}_{I}	\\
	\ket{\SLJf{3}{D}{3}{m_F=I+2}}	&{}={}	\sum_{m_J} \ket{\SLJm{3}{D}{2}{2}}_{J} \ket{I, I}_{I}
\end{align}
since $\ket{\SLJ{3}{D}{1}}$ cannot produce an $m_F=I+2$ state. 



\subsubsection{$F=I+1$}




\subsubsection{$F=I$}



**************************************************

**************************************************

**************************************************

\section{Density distributions}

Section about thermal, bose, and fermi gas density distributions in a trap. 

\subsection{Harmonic approximation}

**When the atoms explore a small volume of the trap (i.e., they are much colder than the trap depth), we can approximate the Gaussian laser potential as being harmonic.**

In this case, the trapping potential can be written as
\begin{equation}
	V\left(x,y,z\right)
		= \frac{1}{2} m \left(\omega_x^2 x^2 + \omega_y^2 y^2 + \omega_z^2 z^2\right)
\end{equation}
where $\omega_i$ for $i \in \left\{x,y,z\right\}$ are the trap frequencies along their respective directions.

\subsubsection{Thermal gas}

We assume the trapped thermal gas follows a Maxwell-Boltzmann distribution where each particle is specified by its position $\bm{r}$ and velocity $\bm{v}$. 
The partition function for this system is 
\begin{equation}
	Z
		= \int\limits_{\text{all space}}^{} \dd[3]{r} \int\limits_{\text{all velocities}}^{} \dd[3]{v} \exp(-\beta \epsilon)
		= \frac{8 \pi^3}{m^3 \omega_x \omega_y \omega_z \beta^3}
		= \frac{8 \pi^3}{m^3 \bar{\omega}^3 \beta^3}
\end{equation}
where $\beta = \flatfrac{1}{k_B T}$ and $\bar{\omega} \equiv \left(\omega_x \omega_y \omega_z\right)^{1/3}$.
Therefore, the distribution is given by $f\left(\bm{r}, \bm{v}\right) = \flatfrac{\exp(-\beta \epsilon)}{Z}$. 
Integrating out the velocities gives the spatial density
\begin{equation}
	\rho\left(\bm{r}\right)
		= \int f\left(\bm{r},\bm{v}\right) \dd[3]{v}
		= \left(\frac{\beta m}{2 \pi}\right)^{3/2} \bar{\omega}^3 \exp[-\frac{\beta}{2} m \left(\omega_x^2 x^2 + \omega_y^2 y^2 + \omega_z^2 z^2\right)]
\end{equation}
Changing coordinates with $x^\prime = \omega_x x$, $y^\prime = \omega_y y$, $z^\prime = \omega_z z$ gives
\begin{equation}
	\rho\left(\bm{r}^\prime\right)
		= \left(\frac{\beta m}{2 \pi}\right)^{3/2} \bar{\omega}^3 \exp[-\frac{\beta}{2} m \left({x^\prime}^2 + {y^\prime}^2 + {z^\prime}^2\right)]
		= \left(\frac{\beta m}{2 \pi}\right)^{3/2} \bar{\omega}^3 \exp[-\frac{\beta}{2} m {r^\prime}^2]
\end{equation}
**need to keep track of differential volume element in prime coordinates**.
Expressing $r$ in terms of $\rho$ gives
\begin{equation}
	r = 
\end{equation}


*********************


For a trapped thermal gas, we assume the system follows a Maxwell-Boltzmann distribution. For a particle at position $\bm{r}$ and velocity $\bm{v}$, it's energy is given by $\epsilon = \frac{1}{2} m v^2 + V\left(r\right)$, therefore
\begin{align}
	f\left(\bm{r}, \bm{v}\right) 
		&= \frac{\exp(-\beta \epsilon)}{\sum_{\bm{r}} \sum_{\bm{v}} \exp(-\beta \epsilon)}	\nonumber	\\
		&= \frac{\exp(-\beta \epsilon)}{\int \dd[3]{r} \int \dd[3]{v} \exp(-\beta \epsilon)}	\nonumber	\\
		&= \frac{\exp(-\beta \epsilon)}{\int \dd[3]{r} \int \dd[3]{v} \exp(-\beta \epsilon)}	\nonumber	\\
\end{align}
where we assumed **(energy $>>$ harmonic oscillator spacing)**. 