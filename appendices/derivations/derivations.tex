\chapter{Derivations}

Random derivations I needed for my thesis so I included them here. 

\section{Gaussian Beams}

Since we almost exclusively work with Gaussian laser beams in the lab, I found it instructive to derive the intensity for an elliptical Gaussian beam.
The derivation below mostly follows from the one presented in \cite{siegman1986}.

%\subsection{Paraxial wave equation}

We start with the wave equation for the electric field $\bm{E} = \widetilde{E}(x,y,z,t) \hat{e}$ where $\hat{e}$ is the unit polarization vector, $\widetilde{E}$ is the complex amplitude, and $c=\lambda f = \omega/k$
\begin{align}
	\label{eq:wave_eq}
	\frac{1}{c^2} \partial_{tt}\widetilde{E}
		= \nabla^2\widetilde{E}
		= \partial_{xx}\widetilde{E} + \partial_{yy}\widetilde{E} + \partial_{zz}\widetilde{E}
\end{align}
Choosing the $z$-axis as the direction of propagation, the time dependence can be removed by assuming $\widetilde{E}(x,y,z,t) = \widetilde{u}(x,y,z)\exp\left[-\iu (k z - \omega t)\right]$
\begin{align}
	\label{eq:paraxial_wave_eq}
	0 = \partial_{xx}\widetilde{u} + \partial_{yy}\widetilde{u} + \partial_{zz}\widetilde{u} -\iu 2 k \partial_{z}\widetilde{u} \approx \partial_{xx}u + \partial_{yy}\widetilde{u} -\iu 2 k \partial_{z}\widetilde{u}
\end{align}
Here, the paraxial approximation $\abs{\partial_{zz}\widetilde{u}} \ll \abs{2 k \partial_{z}\widetilde{u}}, \abs{\partial_{xx}\widetilde{u}}, \abs{\partial_{yy}\widetilde{u}m}$ was made which assumes the envelope of the electric field varies ``slowly'' along $z$.
\cref{eq:paraxial_wave_eq} is generally referred to as the ``paraxial wave equation''.

%\subsubsection{Separating in rectangular coordinates}

For a beam with symmetry along the $x$- and $y$-axis, the paraxial wave equation can be further separated by writing $\widetilde{u}(x,y,z) = \widetilde{u}_{x}(x,z) \widetilde{u}_{y}(y,z)$ where $\widetilde{u}_{x}$ ($\widetilde{u}_{y}$) represents the envelope along the $x$-axis ($y$-axis).
Inserting this in to \cref{eq:paraxial_wave_eq} leads to
\begin{align}
	\label{eq:sep_paraxial_wave_eq}
	0 = u_{x}\left(\partial_{yy}u_{y} - \iu 2 k \partial_{z}u_{y}\right) + u_{y}\left(\partial_{xx}u_{x} - \iu 2 k \partial_{z}u_{x}\right)
\end{align}
Requiring $\widetilde{u}_{x}, \widetilde{u}_{y} \neq 0$ avoids the trivial solution meaning the terms in parentheses must be zero. 
We only need to solve for $\widetilde{u}_{x}$ since the solution for $\widetilde{u}_{y}$ will be similar.

\subsection{Solving $0 = \partial_{xx}\widetilde{u}_{x} - \iu 2 k \partial_{z}\widetilde{u}_{x}$}

Using the trial solution $\widetilde{u}_{x}(x,z) = \widetilde{A}(z) \exp\left[-\iu k \frac{x^2}{2\widetilde{q}(z)}\right]$, where $\widetilde{A}(z)$ and $\widetilde{q}(z)$ can be complex, and collecting terms in powers of $x$
\begin{align}
	0 = -\iu \widetilde{q} \left(\widetilde{A} + 2 q \widetilde{A}^\prime\right) + k \widetilde{A} \left(\widetilde{q}^\prime - 1\right) x^2
\end{align}
Requiring $\widetilde{A}(z), \widetilde{q}(z) \neq 0$ means the terms in the parentheses must be zero
\begin{align}
	0 &= \widetilde{q}^\prime - 1 								\label{eq:q_prime}	\\
	0 &= \widetilde{A} + 2 \widetilde{q} \widetilde{A}^\prime	\label{eq:A_prime}
\end{align}
With the condition $\widetilde{q}(0) = \widetilde{q}_0$, \cref{eq:q_prime} is easily solved
\begin{align}
	\label{eq:q_sol}
	\frac{d\widetilde{q}}{dz} = 1 \implies \widetilde{q}\left(z\right) = z + \widetilde{q}_0
\end{align}
Plugging \cref{eq:q_sol} into \cref{eq:A_prime} and taking $\widetilde{A}(0) = \widetilde{A}_0$
\begin{align}
	\frac{d\widetilde{A}}{dz} = -\frac{\widetilde{A}}{2\widetilde{q}}
		= -\frac{\widetilde{A}}{2\left(\widetilde{q}_0 + z\right)} \implies \widetilde{A}(z)
		= \widetilde{A}_0 \sqrt{\frac{\widetilde{q}_0}{\widetilde{q}_0 + z}}
		= \widetilde{A}_0 \sqrt{\frac{\widetilde{q}_0}{\widetilde{q}(z)}}
\end{align}

%\subsubsection{Complex beam parameter $\widetilde{q}(z)$}

The complex beam parameter $\widetilde{q}(z)$ can be related the real-valued radius of curvature $R(z)$ and spot size $w(z)$ by the definition $\frac{1}{\widetilde{q}(z)} \equiv \frac{1}{R(z)} - \iu \frac{\lambda}{\pi w^2(z)}$.
At the beam waist, taken to be at $z=0$, $R(0) \rightarrow \infty$ so we can relate $\widetilde{q}(0) = \widetilde{q}_0$ to $w(0) = w_0$, 
\begin{align}
	\frac{1}{\widetilde{q_0}} = -\iu \frac{\lambda}{\pi w_0^2} \implies \widetilde{q_0} = \iu \frac{\pi w_0^2}{\lambda} = \iu z_R
\end{align}
where ${z_R \equiv \frac{\pi w_0^2}{\lambda}}$ is the Rayleigh length.
From this, it follows 
\begin{align}
	\frac{1}{z + \iu z_R}
		&= \frac{1}{R} - \iu \frac{\lambda}{\pi w^2}	\nonumber \\
	z + \iu z_R	
		&= \left(\frac{\pi w^2}{\lambda}\right)^2 \frac{R}{R^2 + \left(\frac{\pi w^2}{\lambda}\right)^2} 
			+ \iu \left(\frac{\pi w^2}{\lambda}\right) \frac{R^2}{R^2 + \left(\frac{\pi w^2}{\lambda}\right)^2}
\end{align}
Equating the real and imaginary parts gives
\begin{align}
	R(z)	&= z + \frac{z_R^2}{z}	\\
	w^2(z)	&= w_0^2 \left[1+\left(\frac{z}{z_R}\right)^2\right]
\end{align}

%\subsubsection{The (unnormalized) solution}

Putting all the parts together gives the (unnormalized) solution
\begin{align}
	\widetilde{u}_{x}(x,z) = \widetilde{A}_0 \frac{1}{\sqrt{1 - \iu \frac{z}{z_R}}} \exp\left[-\iu k \frac{x^2}{2}\left(\frac{1}{R(z)} - \iu\frac{\lambda}{\pi w^2(z)}\right)\right]
\end{align}
To normalize, we follow the usual procedure to find $|\widetilde{A}_0|$
\begin{align}
	1	&= \int_{-\infty}^{\infty} \widetilde{u}_{x}^{*}\widetilde{u}_{x} \,dx \nonumber \\
		&= \abs{\widetilde{A}_0}^2 \frac{1}{\sqrt{1 + \left(\frac{z}{z_R}\right)^2}} \int_{-\infty}^{\infty} \exp\left[-\frac{2 x^2}{w^2(z)}\right] \,dx \nonumber \\
		&= \abs{\widetilde{A}_0}^2 \frac{1}{\sqrt{1 + \left(\frac{z}{z_R}\right)^2}} \sqrt{\frac{\pi}{2}} w(z) \nonumber \\
		&= \abs{\widetilde{A}_0}^2 \sqrt{\frac{\pi}{2}} w_0
\end{align}
Therefore the solution for the $x$-axis is
\begin{align}
	\abs{\widetilde{A}_0}^2 = \sqrt{\frac{2}{\pi}}\frac{1}{w_0} \implies \abs{\widetilde{A}_0} = \pm \left(\frac{2}{\pi}\right)^{1/4} \frac{1}{\sqrt{w_0}}
\end{align}
Taking the positive value for $\abs{\widetilde{A}_0}$ gives the normalized solution for the envelope along the $x$-axis
\begin{align}
	\widetilde{u}_{x}(x,z)
		&= \left(\frac{2}{\pi}\right)^{1/4} \frac{1}{\sqrt{w_0\left(1 - \iu \frac{z}{z_R}\right)}} \exp\left[-\iu k \frac{x^2}{2}\left(\frac{1}{R(z)} - \iu\frac{\lambda}{\pi w^2(z)}\right)\right]
\end{align}

\subsection{Normalized solution}

(**the parts below get messy and need to be worked on.)

A general solution $\widetilde{u}(x,y,z)$ with $x$-waist ($y$-waist) of $w_{x,0}$ ($w_{y,0}$) at $z_{x,0}$ ($z_{y,0}$)
\begin{align}
	\label{eq:E_sol}
	\widetilde{u}(x,y,z)
		=& \widetilde{u}_x(x-x_0,z) \widetilde{u}_y(y-y_0,z) \exp\left[-\iu (k z - \omega t)\right] \nonumber \\
		=& \left(\frac{2}{\pi}\right)^{1/2}
			\frac{1}{\sqrt{w_{x,0}\left(1 - \iu \frac{z-z_{x,0}}{z_{x,R}}\right)}} \frac{1}{\sqrt{w_{y,0}\left(1 - \iu \frac{z-z_{y,0}}{z_{y,R}}\right)}} \nonumber \\
		& \times \exp\left[-\iu k \frac{x^2}{2}\left(\frac{1}{R(z-z_{x,0})} - \iu\frac{\lambda}{\pi w^2(z-z_{x,0})}\right)\right] \nonumber \\
		& \times \exp\left[-\iu (k {z-z_{x,0}} - \omega t)\right] \nonumber \\
\end{align}


**************************

Interesting discussions about astigmatic Gaussian beams are discussed in \cite{arn1969.AO.8.1687, koc2013.AO.52.6030}. 

\section{Unitary transformation}

For a unitary transformation $U$ where $\ket*{\psi} = U \ket*{\widetilde{\psi}}$, the Schr\"{o}dinger equation becomes
\begin{equation}
	\iu \hbar \ket*{\dot{\psi}} = H \ket*{\psi}
		\rightarrow \iu \hbar \pdv{t} \left(U \ket*{\widetilde{\psi}}\right) = H \left(U \ket*{\widetilde{\psi}}\right)
\end{equation}
From here, it's pretty easy to obtain the transformed Hamiltonian $\widetilde{H}$
\begin{align}
	\iu \hbar \left(\dot{U} \ket*{\widetilde{\psi}} + U \ket*{\dot{\widetilde{\psi}}}\right) &= H U \ket*{\widetilde{\psi}}	\nonumber \\
	\iu \hbar U \ket*{\dot{\widetilde{\psi}}} &= H U \ket*{\widetilde{\psi}} - \iu \hbar \dot{U} \ket*{\widetilde{\psi}}	\nonumber \\
	\iu \hbar \ket*{\dot{\widetilde{\psi}}} &= \left(U^\dag H U - \iu \hbar U^\dag \dot{U}\right) \ket*{\widetilde{\psi}}	\nonumber \\
	\iu \hbar \ket*{\dot{\widetilde{\psi}}} &= \widetilde{H} \ket*{\widetilde{\psi}}
\end{align}
where $\widetilde{H} = U^\dag H U - \iu \hbar U^\dag \dot{U}$ for the transformed state $\ket*{\widetilde{\psi}} = U^\dag \ket*{\psi}$.

TO DO:

Go through Gaussian beams section and use widetilde command for complex values/functions. 

\section{Hyperfine Mixing}

**Derivation of the hyperfine mixing coefficients.**

Expanding $\hat{\bm{s}}_\text{in}$ and $\hat{\bm{I}}$ in terms of ladder operators gives
\begin{equation}
	\hat{V}_\text{hf} \simeq \frac{a_{5s}}{2} \left(\hat{s}_{\text{in},+}\hat{I}_{-} + \hat{s}_{\text{in},-}\hat{I}_{+} + 2\hat{s}_{\text{in},z}\hat{I}_{z}\right)
\end{equation}
with the usual ladder operators $\hat{J}_{\pm}=\hat{J}_{x} \pm \hat{J}_{y}$ and $\hat{J}_{\pm}\ket{J,m_J}=\sqrt{J\left(J+1\right)-m_J\left(m_J\pm1\right)} \ket{J,m_J\pm1}$.

\subsection{{$5sns$-series} $S$~states}

Working in the total angular momentum basis $\bm{F}=\bm{J}+\bm{I}=\bm{S}+\bm{I}$\footnote{Since $L=0$ for $S$~states.}, we can express the pure hyperfine $S$~states as
\begin{align}
	& \ket{\left(n\SLJ{1}{S}{0}\right)F=I,m_F}		=	\sum_{m_J} \sum_{m_I} \CG{0,0}{I,m_I}{I,m_F} \ket{0,0} \ket{I,m_I}	= \ket{n\SLJ{1}{S}{0}} \ket{I,m_F}	\\
	& \ket{\left(n\SLJ{3}{S}{1}\right)F=I+1,m_F}	=	\sum \CG{J=1,m_J=1}{I,m_I}{F=I+1,m_F} \ket{J=1,m_J} \ket{I+1,m_I}					= \ket{n\SLJ{1}{S}{0},m_J}	\\
	& \ket{\left(n\SLJ{3}{S}{1}\right)F=I,m_F}		=	\sum \CG{J=1,m_J=0}{I,m_I}{F=I,m_F}	\ket{J=1,m_J} \ket{I,m_I} \\
	& \ket{\left(n\SLJ{3}{S}{1}\right)F=I-1,m_F}	=	\sum \CG{J=1,m_J=-1}{I,m_I}{F=I-1,m_F} \ket{J=1,m_J} \ket{I-1,m_I}
\end{align}



***************************************


Since the hyperfine interaction couples $\bm{J}=\bm{L}+\bm{S}=\bm{S}$\footnote{Since $L=0$ for $S$~states.} and $\bm{I}$, we work with the total angular momentum basis $\bm{F}=\bm{J}+\bm{I}$.
Using Clebsch–Gordan coefficients $\CG{J_1,M_1}{J_2,M_2}{J,M}=\braket{J_1,M_1;J_2,M_2}{J,M}$, we can decompose the $\ket{F,m_F}$ states



***************************************


Starting with the with pure $\nSLJ{5sns}{1}{S}{0} = \ket{\nSLJ{n}{1}{S}{0}}$ and $\nSLJ{5sns}{3}{S}{1} = \ket{\nSLJ{n}{3}{S}{1}}$ singly-excited $S$~states, the nuclear spin can be incorporated 


*******************************


Using Clebsch–Gordan coefficients $\CG{J_1,M_1}{J_2,M_2}{J,M}=\braket{J_1,M_1;J_2,M_2}{J,M}$, we can decompose the $\ket{F,m_F}$ states
\begin{align}
	& \ket{\left(n\SLJ{1}{S}{0}\right)F=I,m_F}		=	\sum_{m_J} \sum_{m_I} \CG{0,0}{I,m_I}{I,m_F} \ket{0,0} \ket{I,m_I}	= \ket{n\SLJ{1}{S}{0}} \ket{I,m_F}	\\
	& \ket{\left(n\SLJ{3}{S}{1}\right)F=I+1,m_F}	=	\sum \CG{J=1,m_J=1}{I,m_I}{F=I+1,m_F} \ket{J=1,m_J} \ket{I+1,m_I}					= \ket{n\SLJ{1}{S}{0},m_J}	\\
	& \ket{\left(n\SLJ{3}{S}{1}\right)F=I,m_F}		=	\sum \CG{J=1,m_J=0}{I,m_I}{F=I,m_F}	\ket{J=1,m_J} \ket{I,m_I} \\
	& \ket{\left(n\SLJ{3}{S}{1}\right)F=I-1,m_F}	=	\sum \CG{J=1,m_J=-1}{I,m_I}{F=I-1,m_F} \ket{J=1,m_J} \ket{I-1,m_I}
\end{align}







Calculating the matrix elements of the hyperfine operator.
For $S$ states, the pure singlet and triplet states can be written as
\begin{align}
	& \ket{n\SLJm{1}{S}{0}{0}}^{0} = \frac{1}{2} \left(\ket{5s;ns} + \ket{ns;5s}\right) \left(\ket{\uparrow;\downarrow} - \ket{\downarrow;\uparrow}\right) \\
	& \ket{n\SLJm{3}{S}{1}{1}}^{0} = \frac{1}{\sqrt{2}} \left(\ket{5s;ns} - \ket{ns;5s}\right) \ket{\uparrow;\uparrow}	\\
	& \ket{n\SLJm{3}{S}{1}{0}}^{0} = \frac{1}{2} \left(\ket{5s;ns} - \ket{ns;5s}\right) \left(\ket{\uparrow;\downarrow} + \ket{\downarrow;\uparrow}\right)	\\
	& \ket{n\SLJm{3}{S}{1}{-1}}^{0} = \frac{1}{\sqrt{2}} \left(\ket{5s;ns} - \ket{ns;5s}\right) \ket{\downarrow;\downarrow}
\end{align}
To obtain include the nuclear spin, the above states are multiplied by $\ket{I,m_I}$.

** CHECK BELOW **
\begin{align}
	& \hat{s}_{\text{in},+}\ket{n\SLJm{1}{S}{0}{0}}^0	=	-\frac{1}{\sqrt{2}} \ket{n\SLJm{3}{S}{1}{1}}^{0}												\\
	& \hat{s}_{\text{in},-}\ket{n\SLJm{1}{S}{0}{0}}^0	=	\frac{1}{\sqrt{2}} \ket{n\SLJm{3}{S}{1}{-1}}^{0}												\\
	& \hat{s}_{\text{in},z}\ket{n\SLJm{1}{S}{0}{0}}^0	=	\frac{1}{2} \ket{n\SLJm{3}{S}{1}{0}}^{0}														\\
	& \hat{s}_{\text{in},+}\ket{n\SLJm{3}{S}{1}{1}}^0	=	0																								\\
	& \hat{s}_{\text{in},-}\ket{n\SLJm{3}{S}{1}{1}}^0	=	-\frac{1}{\sqrt{2}} \left(\ket{n\SLJm{1}{S}{0}{0}}^{0} - \ket{n\SLJm{3}{S}{1}{0}}^{0}\right)	\\
	& \hat{s}_{\text{in},z}\ket{n\SLJm{3}{S}{1}{1}}^0	=	\frac{1}{2} \ket{n\SLJm{3}{S}{1}{1}}^{0}														\\
	& \hat{s}_{\text{in},+}\ket{n\SLJm{3}{S}{1}{0}}^0	=	\frac{1}{\sqrt{2}} \ket{n\SLJm{3}{S}{1}{1}}^{0}													\\
	& \hat{s}_{\text{in},-}\ket{n\SLJm{3}{S}{1}{0}}^0	=	\frac{1}{\sqrt{2}} \ket{n\SLJm{3}{S}{1}{-1}}^{0}												\\
	& \hat{s}_{\text{in},z}\ket{n\SLJm{3}{S}{1}{0}}^0	=	\frac{1}{2} \ket{n\SLJm{1}{S}{0}{0}}^{0}														\\
	& \hat{s}_{\text{in},+}\ket{n\SLJm{3}{S}{1}{-1}}^0	=	\frac{1}{\sqrt{2}} \left(\ket{n\SLJm{1}{S}{0}{0}}^{0} + \ket{n\SLJm{3}{S}{1}{0}}^{0}\right)		\\
	& \hat{s}_{\text{in},-}\ket{n\SLJm{3}{S}{1}{-1}}^0	=	0																								\\
	& \hat{s}_{\text{in},z}\ket{n\SLJm{3}{S}{1}{-1}}^0	=	-\frac{1}{2} \ket{n\SLJm{3}{S}{1}{-1}}^{0}
\end{align}

For the $\hat{I}$ operators, 
\begin{align}
	& \hat{I}_{+} \ket{I,m_I}	=	\sqrt{I\left(I+1\right) - m_I \left(m_I+1\right)} \ket{I,m_I+1}	\\
	& \hat{I}_{-} \ket{I,m_I}	=	\sqrt{I\left(I+1\right) - m_I \left(m_I-1\right)} \ket{I,m_I-1}	\\
	& \hat{I}_{z} \ket{I,m_I}	=	m_I \ket{I,m_I}
\end{align}

Now that we know all the relevant states, we can calculate the effect of $\hat{V}_\text{hf}$ on each state. 
** For simplicity, we used the stretched states $\ket{I=F, m_I=F}$ (provide a better argument?) **
\begin{align}
	& \hat{V}_\text{hf}\ket{n\SLJm{1}{S}{0}{0}}^{0}		=	\frac{a_{5s}}{2}	\\
	& \hat{V}_\text{hf}\ket{n\SLJm{3}{S}{1}{1}}^{0}		=	\frac{a_{5s}}{2}	\\
	& \hat{V}_\text{hf}\ket{n\SLJm{3}{S}{1}{0}}^{0}		=	\frac{a_{5s}}{2}	\\
	& \hat{V}_\text{hf}\ket{n\SLJm{3}{S}{1}{-1}}^{0}	=	\frac{a_{5s}}{2} 
\end{align}















\subsection{$D$-states}

Move details of derivation here?


**************************************************

**************************************************

**************************************************

\section{Density distributions}

Section about thermal, bose, and fermi gas density distributions in a trap. 

\subsection{Harmonic approximation}

**When the atoms explore a small volume of the trap (i.e., they are much colder than the trap depth), we can approximate the Gaussian laser potential as being harmonic.**

In this case, the trapping potential can be written as
\begin{equation}
	V\left(x,y,z\right)
		= \frac{1}{2} m \left(\omega_x^2 x^2 + \omega_y^2 y^2 + \omega_z^2 z^2\right)
\end{equation}
where $\omega_i$ for $i \in \left\{x,y,z\right\}$ are the trap frequencies along their respective directions.

\subsubsection{Thermal gas}

We assume the trapped thermal gas follows a Maxwell-Boltzmann distribution where each particle is specified by its position $\bm{r}$ and velocity $\bm{v}$. 
The partition function for this system is 
\begin{equation}
	Z
		= \int\limits_{\text{all space}}^{} \dd[3]{r} \int\limits_{\text{all velocities}}^{} \dd[3]{v} \exp(-\beta \epsilon)
		= \frac{8 \pi^3}{m^3 \omega_x \omega_y \omega_z \beta^3}
		= \frac{8 \pi^3}{m^3 \bar{\omega}^3 \beta^3}
\end{equation}
where $\beta = \flatfrac{1}{k_B T}$ and $\bar{\omega} \equiv \left(\omega_x \omega_y \omega_z\right)^{1/3}$.
Therefore, the distribution is given by $f\left(\bm{r}, \bm{v}\right) = \flatfrac{\exp(-\beta \epsilon)}{Z}$. 
Integrating out the velocities gives the spatial density
\begin{equation}
	\rho\left(\bm{r}\right)
		= \int f\left(\bm{r},\bm{v}\right) \dd[3]{v}
		= \left(\frac{\beta m}{2 \pi}\right)^{3/2} \bar{\omega}^3 \exp[-\frac{\beta}{2} m \left(\omega_x^2 x^2 + \omega_y^2 y^2 + \omega_z^2 z^2\right)]
\end{equation}
Changing coordinates with $x^\prime = \omega_x x$, $y^\prime = \omega_y y$, $z^\prime = \omega_z z$ gives
\begin{equation}
	\rho\left(\bm{r}^\prime\right)
		= \left(\frac{\beta m}{2 \pi}\right)^{3/2} \bar{\omega}^3 \exp[-\frac{\beta}{2} m \left({x^\prime}^2 + {y^\prime}^2 + {z^\prime}^2\right)]
		= \left(\frac{\beta m}{2 \pi}\right)^{3/2} \bar{\omega}^3 \exp[-\frac{\beta}{2} m {r^\prime}^2]
\end{equation}
**need to keep track of differential volume element in prime coordinates**.
Expressing $r$ in terms of $\rho$ gives
\begin{equation}
	r = 
\end{equation}


*********************


For a trapped thermal gas, we assume the system follows a Maxwell-Boltzmann distribution. For a particle at position $\bm{r}$ and velocity $\bm{v}$, it's energy is given by $\epsilon = \frac{1}{2} m v^2 + V\left(r\right)$, therefore
\begin{align}
	f\left(\bm{r}, \bm{v}\right) 
		&= \frac{\exp(-\beta \epsilon)}{\sum_{\bm{r}} \sum_{\bm{v}} \exp(-\beta \epsilon)}	\nonumber	\\
		&= \frac{\exp(-\beta \epsilon)}{\int \dd[3]{r} \int \dd[3]{v} \exp(-\beta \epsilon)}	\nonumber	\\
		&= \frac{\exp(-\beta \epsilon)}{\int \dd[3]{r} \int \dd[3]{v} \exp(-\beta \epsilon)}	\nonumber	\\
\end{align}
where we assumed **(energy $>>$ harmonic oscillator spacing)**. 